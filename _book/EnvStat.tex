\documentclass[]{book}
\usepackage{lmodern}
\usepackage{amssymb,amsmath}
\usepackage{ifxetex,ifluatex}
\usepackage{fixltx2e} % provides \textsubscript
\ifnum 0\ifxetex 1\fi\ifluatex 1\fi=0 % if pdftex
  \usepackage[T1]{fontenc}
  \usepackage[utf8]{inputenc}
\else % if luatex or xelatex
  \ifxetex
    \usepackage{mathspec}
  \else
    \usepackage{fontspec}
  \fi
  \defaultfontfeatures{Ligatures=TeX,Scale=MatchLowercase}
\fi
% use upquote if available, for straight quotes in verbatim environments
\IfFileExists{upquote.sty}{\usepackage{upquote}}{}
% use microtype if available
\IfFileExists{microtype.sty}{%
\usepackage{microtype}
\UseMicrotypeSet[protrusion]{basicmath} % disable protrusion for tt fonts
}{}
\usepackage[margin=1in]{geometry}
\usepackage{hyperref}
\PassOptionsToPackage{usenames,dvipsnames}{color} % color is loaded by hyperref
\hypersetup{unicode=true,
            pdftitle={Statistika Lingkungan Menggunakan R},
            pdfauthor={Moh. Rosidi},
            colorlinks=true,
            linkcolor=Maroon,
            citecolor=Blue,
            urlcolor=Blue,
            breaklinks=true}
\urlstyle{same}  % don't use monospace font for urls
\usepackage{natbib}
\bibliographystyle{apalike}
\usepackage{color}
\usepackage{fancyvrb}
\newcommand{\VerbBar}{|}
\newcommand{\VERB}{\Verb[commandchars=\\\{\}]}
\DefineVerbatimEnvironment{Highlighting}{Verbatim}{commandchars=\\\{\}}
% Add ',fontsize=\small' for more characters per line
\usepackage{framed}
\definecolor{shadecolor}{RGB}{248,248,248}
\newenvironment{Shaded}{\begin{snugshade}}{\end{snugshade}}
\newcommand{\KeywordTok}[1]{\textcolor[rgb]{0.13,0.29,0.53}{\textbf{#1}}}
\newcommand{\DataTypeTok}[1]{\textcolor[rgb]{0.13,0.29,0.53}{#1}}
\newcommand{\DecValTok}[1]{\textcolor[rgb]{0.00,0.00,0.81}{#1}}
\newcommand{\BaseNTok}[1]{\textcolor[rgb]{0.00,0.00,0.81}{#1}}
\newcommand{\FloatTok}[1]{\textcolor[rgb]{0.00,0.00,0.81}{#1}}
\newcommand{\ConstantTok}[1]{\textcolor[rgb]{0.00,0.00,0.00}{#1}}
\newcommand{\CharTok}[1]{\textcolor[rgb]{0.31,0.60,0.02}{#1}}
\newcommand{\SpecialCharTok}[1]{\textcolor[rgb]{0.00,0.00,0.00}{#1}}
\newcommand{\StringTok}[1]{\textcolor[rgb]{0.31,0.60,0.02}{#1}}
\newcommand{\VerbatimStringTok}[1]{\textcolor[rgb]{0.31,0.60,0.02}{#1}}
\newcommand{\SpecialStringTok}[1]{\textcolor[rgb]{0.31,0.60,0.02}{#1}}
\newcommand{\ImportTok}[1]{#1}
\newcommand{\CommentTok}[1]{\textcolor[rgb]{0.56,0.35,0.01}{\textit{#1}}}
\newcommand{\DocumentationTok}[1]{\textcolor[rgb]{0.56,0.35,0.01}{\textbf{\textit{#1}}}}
\newcommand{\AnnotationTok}[1]{\textcolor[rgb]{0.56,0.35,0.01}{\textbf{\textit{#1}}}}
\newcommand{\CommentVarTok}[1]{\textcolor[rgb]{0.56,0.35,0.01}{\textbf{\textit{#1}}}}
\newcommand{\OtherTok}[1]{\textcolor[rgb]{0.56,0.35,0.01}{#1}}
\newcommand{\FunctionTok}[1]{\textcolor[rgb]{0.00,0.00,0.00}{#1}}
\newcommand{\VariableTok}[1]{\textcolor[rgb]{0.00,0.00,0.00}{#1}}
\newcommand{\ControlFlowTok}[1]{\textcolor[rgb]{0.13,0.29,0.53}{\textbf{#1}}}
\newcommand{\OperatorTok}[1]{\textcolor[rgb]{0.81,0.36,0.00}{\textbf{#1}}}
\newcommand{\BuiltInTok}[1]{#1}
\newcommand{\ExtensionTok}[1]{#1}
\newcommand{\PreprocessorTok}[1]{\textcolor[rgb]{0.56,0.35,0.01}{\textit{#1}}}
\newcommand{\AttributeTok}[1]{\textcolor[rgb]{0.77,0.63,0.00}{#1}}
\newcommand{\RegionMarkerTok}[1]{#1}
\newcommand{\InformationTok}[1]{\textcolor[rgb]{0.56,0.35,0.01}{\textbf{\textit{#1}}}}
\newcommand{\WarningTok}[1]{\textcolor[rgb]{0.56,0.35,0.01}{\textbf{\textit{#1}}}}
\newcommand{\AlertTok}[1]{\textcolor[rgb]{0.94,0.16,0.16}{#1}}
\newcommand{\ErrorTok}[1]{\textcolor[rgb]{0.64,0.00,0.00}{\textbf{#1}}}
\newcommand{\NormalTok}[1]{#1}
\usepackage{longtable,booktabs}
\usepackage{graphicx,grffile}
\makeatletter
\def\maxwidth{\ifdim\Gin@nat@width>\linewidth\linewidth\else\Gin@nat@width\fi}
\def\maxheight{\ifdim\Gin@nat@height>\textheight\textheight\else\Gin@nat@height\fi}
\makeatother
% Scale images if necessary, so that they will not overflow the page
% margins by default, and it is still possible to overwrite the defaults
% using explicit options in \includegraphics[width, height, ...]{}
\setkeys{Gin}{width=\maxwidth,height=\maxheight,keepaspectratio}
\IfFileExists{parskip.sty}{%
\usepackage{parskip}
}{% else
\setlength{\parindent}{0pt}
\setlength{\parskip}{6pt plus 2pt minus 1pt}
}
\setlength{\emergencystretch}{3em}  % prevent overfull lines
\providecommand{\tightlist}{%
  \setlength{\itemsep}{0pt}\setlength{\parskip}{0pt}}
\setcounter{secnumdepth}{5}
% Redefines (sub)paragraphs to behave more like sections
\ifx\paragraph\undefined\else
\let\oldparagraph\paragraph
\renewcommand{\paragraph}[1]{\oldparagraph{#1}\mbox{}}
\fi
\ifx\subparagraph\undefined\else
\let\oldsubparagraph\subparagraph
\renewcommand{\subparagraph}[1]{\oldsubparagraph{#1}\mbox{}}
\fi

%%% Use protect on footnotes to avoid problems with footnotes in titles
\let\rmarkdownfootnote\footnote%
\def\footnote{\protect\rmarkdownfootnote}

%%% Change title format to be more compact
\usepackage{titling}

% Create subtitle command for use in maketitle
\providecommand{\subtitle}[1]{
  \posttitle{
    \begin{center}\large#1\end{center}
    }
}

\setlength{\droptitle}{-2em}

  \title{Statistika Lingkungan Menggunakan R}
    \pretitle{\vspace{\droptitle}\centering\huge}
  \posttitle{\par}
    \author{Moh. Rosidi}
    \preauthor{\centering\large\emph}
  \postauthor{\par}
      \predate{\centering\large\emph}
  \postdate{\par}
    \date{2019-04-11}

\usepackage{booktabs}

\begin{document}
\maketitle

{
\hypersetup{linkcolor=black}
\setcounter{tocdepth}{1}
\tableofcontents
}
\listoftables
\listoffigures
\chapter*{Pengantar}\label{pengantar}
\addcontentsline{toc}{chapter}{Pengantar}

Buku ini menyajikan penerapan program \texttt{R} dalam
\texttt{Statistika\ Lingkungan}. Buku ini akan disajikan secara ringkas
menggunakan sejumlah contoh kasus yang relevan dalam bidang lingkungan.

Penulis berharap buku ini dapat menjadi referensi sumber terbuka bagi
mahasiswa yang ingin menggunakan \texttt{R} untuk kegiatan analisa data.
Sehingga dapat mengurangi ketergantungan pada penggunaan aplikasi yang
berlisensi.

\part*{Bahasa Pemrograman R}\label{part-bahasa-pemrograman-r}
\addcontentsline{toc}{part}{Bahasa Pemrograman R}

\chapter{Mengenal Bahasa R}\label{mengenal-bahasa-r}

Dewasa ini tersedia banyak sekali \emph{software} yang dapat digunakan
untuk membantu kita dalam melakukan analisa data. \emph{software} yang
digunakan dapat berupa \emph{software} berbayar atau gratis.

\texttt{R} merupakan merupakan salah satu \emph{software} gratis yang
sangat populer di Indonesia. Kemudahan penggunaan serta banyaknya
besarnya dukungan komunitas membuat \texttt{R} menjadi salah satu bahasa
pemrograman paling populer di dunia.

Paket yang disediakan untuk analisis statistika juga sangat lengkap dan
terus bertambah setiap saat. Hal ini membuat \texttt{R} banyak digunakan
oleh para analis data.

Pada \emph{chapter} ini penulis akan memperkenalkan kepada pembaca
mengenai bahasa pemrograman \texttt{R}. Mulai dari sejarah, cara
instalasi sampai dengan bagaimana kita memanfaatkan fitur dasar bantuan
untuk menggali lebih jauh tentang fungsi-fungsi \texttt{R}.

\section{Sejarah R}\label{sejarah-r}

\texttt{R} Merupakan bahasa yang digunakan dalam komputasi
\textbf{statistik} yang pertama kali dikembangkan oleh \textbf{Ross
Ihaka} dan \textbf{Robert Gentlement} di University of Auckland New
Zealand yang merupakan akronim dari nama depan kedua pembuatnya. Sebelum
\texttt{R} dikenal ada \texttt{S} yang dikembangkan oleh \textbf{John
Chambers} dan rekan-rekan dari \textbf{Bell Laboratories} yang memiliki
fungsi yang sama untuk komputasi statistik. Hal yang membedakan antara
keduanya adalah \texttt{R} merupakan sistem komputasi yang bersifat
gratis.Logo \texttt{R} dapat dilihat pada Gambar \ref{fig:Logo}.

\begin{figure}

{\centering \includegraphics[width=0.4\linewidth]{r-icon} 

}

\caption{Logo R.}\label{fig:Logo}
\end{figure}

\texttt{R} dapat dibilang merupakan aplikasi sistem \textbf{statistik}
yang kaya. Hal ini disebabkan banyak sekali paket yang dikembangkan oleh
pengembang dan komunitas untuk keperluan analisa statistik seperti
\emph{linear regression}, \emph{clustering}, \emph{statistical test},
dll. Selain itu, \texttt{R} juga dapat ditambahkan paket-paket lain yang
dapat meningkatkan fiturnya.

Sebagai sebuah bahasa pemrograman yang banyak digunakan untuk keperluan
analisa data, \texttt{R} dapat dioperasikan pada berbagai sistem operasi
pada komputer. Adapun sistem operasi yang didukung antara lain:
\texttt{UNIX}, \texttt{Linux}, \texttt{Windows}, dan \texttt{MacOS}.

\section{Fitur dan Karakteristik R}\label{fitur-dan-karakteristik-r}

\texttt{R} memiliki karakteristik yang berbeda dengan bahasa pemrograman
lain seperti \texttt{C++},\texttt{python}, dll. \texttt{R} memiliki
aturan/sintaks yang berbeda dengan bahasa pemrograman yang lain yang
membuatnya memiliki ciri khas tersendiri dibanding bahasa pemrograman
yang lain.

Beberapa ciri dan fitur pada \texttt{R} antara lain:

\begin{enumerate}
\def\labelenumi{\arabic{enumi}.}
\tightlist
\item
  \textbf{Bahasa \texttt{R} bersifat case sensitif}. maksudnya adalah
  dalam proses input \texttt{R} huruf besar dan kecil sangat
  diperhatikan. Sebagai contoh kita ingin melihat apakah objek A dan B
  pada sintaks berikut:
\end{enumerate}

\begin{Shaded}
\begin{Highlighting}[]
\NormalTok{A <-}\StringTok{ "Andi"}
\NormalTok{B <-}\StringTok{ "andi"}

\CommentTok{# cek kedua objek A dan B}
\NormalTok{A }\OperatorTok{==}\StringTok{ }\NormalTok{B}
\end{Highlighting}
\end{Shaded}

\begin{verbatim}
## [1] FALSE
\end{verbatim}

\begin{Shaded}
\begin{Highlighting}[]
\CommentTok{# Kesimpulan : Kedua objek berbeda}
\end{Highlighting}
\end{Shaded}

\begin{enumerate}
\def\labelenumi{\arabic{enumi}.}
\setcounter{enumi}{1}
\tightlist
\item
  \textbf{Segala sesuatu yang ada pada program \texttt{R} akan diangap
  sebagai objek}. konsep objek ini sama dengan bahasa pemrograma
  berbasis objek yang lain seperti \texttt{Java}, \texttt{C++},
  \texttt{python}, dll.Perbedaannya adalah bahasa \texttt{R} relatif
  lebih sederhana dibandingkan bahasa pemrograman berbasis obejk yang
  lain.
\item
  \textbf{interpreted language atau script}. Bahasa \texttt{R}
  memungkinkan pengguna untuk melakukan kerja pada \texttt{R} tanpa
  perlu kompilasi kode program menjadi bahasa mesin.
\item
  Mendukung proses \textbf{loop}, \textbf{decision making}, dan
  menyediakan berbagai jenis \textbf{operstor} (aritmatika, logika,
  dll).
\item
  \textbf{Mendukung export dan import berbagai format file},
  seperti:TXT, CSV, XLS, dll.
\item
  \textbf{Mudah ditingkatkan melalui penambahan fungsi atau paket}.
  Penambahan paket dapat dilakukan secara online melalui
  \href{https://cran.r-project.org/}{CRAN} atau melalui sumber seperti
  \href{https://github.com/}{github}.
\item
  \textbf{Menyedikan berbagai fungsi untuk keperluan visualisasi data}.
  Visualisasi data pada \texttt{R} dapat menggunakan paket bawaan atau
  paket lain seperti \texttt{ggplo2},\texttt{ggvis}, dll.
\end{enumerate}

\section{Kelebihan dan Kekurangan R}\label{kelebihan-dan-kekurangan-r}

Selain karena \texttt{R} dapat digunakan secara gratis terdapat
\textbf{kelebihan} lain yang ditawarkan, antara lain:

\begin{enumerate}
\def\labelenumi{\arabic{enumi}.}
\tightlist
\item
  \textbf{Protability}. Penggunaan software dapat digunakan kapanpun
  tanpa terikat oleh masa berakhirnya lisensi.
\item
  \textbf{Multiplatform}. \texttt{R} bersifat \emph{Multiplatform
  Operating Systems}, dimana \emph{software} \texttt{R} lebih kompatibel
  dibanding \emph{software} statistika lainnya. Hal in berdampak pada
  kemudahan dalam penyesuaian jika pengguna harus berpindah sistem
  operasi karena \texttt{R} baik pada sistem operasi seperti
  \texttt{windows} akan sama pengoperasiannya dengan yang ada di
  \texttt{Linux} (paket yang digunakan sama).
\item
  \textbf{General} dan \textbf{Cutting-edge}. Berbagai metode statistik
  baik metode klasik maupun baru telah diprogram kedalam \texttt{R}.
  Dengan demikian \emph{software} ini dapat digunakan untuk analisis
  statistika dengan pendekatan klasik dan pendekatan modern.
\item
  \textbf{Programable}. Pengguna dapat memprogram metode baru atau
  mengembangakan modifikasi dari analisis statistika yang telah ada pada
  sistem \texttt{R}.
\item
  \textbf{Berbasis analisis matriks}. Bahasa \texttt{R} sangat baik
  digunakan untuk \emph{programming} dengan basis matriks.
\item
  Fasiltas grafik yang lengkap.
\end{enumerate}

Adapun kekurangan dari \texttt{R} antara lain:

\begin{enumerate}
\def\labelenumi{\arabic{enumi}.}
\tightlist
\item
  \textbf{Point and Click GUI}. Interaksi utama dengan \texttt{R}
  bersifat \emph{CLI} (\emph{Command Line Interface}), walaupun saat ini
  telah dikembangkan paket yang memungkinkan kita berinteraksi dengan
  \texttt{R} menggunakan \emph{GUI} (\emph{Graphical User Interface})
  sederhana menggunakan paket \texttt{R-Commander} yang memiliki fungsi
  yang terbatas. \texttt{R-\ Commander} sendiri merupakan \emph{GUI}
  yang diciptakan dengan tujuan untuk keperluan pengajaran sehingga
  analisis statistik yang disediakan adalah yang klasik. Meskipun
  terbatas paket ini berguna jika kita membutuhkan analisis statistik
  sederhana dengan cara yang simpel.
\item
  \textbf{Missing statistical function}. Meskipun analisis statistika
  dalam \texttt{R} sudah cukup lengkap, namun tidak semua metode
  statistika telah diimplementasikan ke dalam \texttt{R}. Namun karena
  \texttt{R} merupakan \emph{lingua franca} untuk keperluan komputasi
  statistika modern staan ini, dapat dikatakan ketersediaan fungsi
  tambahan dalam bentuk paket hanya masalah waktu saja.
\end{enumerate}

\section{RStudio}\label{rstudio}

Aplikasi \texttt{R} pada dasarnya berbasis teks atau \emph{command line}
sehingga pengguna harus mengetikkan perintah-perintah tertentu dan harus
hapal perintah-perintahnya. Setidaknya jika kita ingin melakukan
kegiatan analisa data menggunakan \texttt{R} kita harus selalu siap
dengan perintah-perintah yang hendak digunakan sehingga buku manual
menjadi sesuatu yang wajib adasaat berkeja dengan \texttt{R}.

Kondisi ini sering kali membingunkan bagi pengguna pemula maupun
pengguna mahir yang sudah terbiasa dengan aplikasi statistik lain
seperti SAS, SPSS, Minitab, dll. Alasan itulah yang menyebabkan
pengembang \texttt{R} membuat berbagai \emph{frontend} untuk \texttt{R}
yang berguna untuk memudahkan dalam pengoperasian \texttt{R}.

\texttt{RStudio} merupakan salah satu bentuk \emph{frontend} \texttt{R}
yang cukup populer dan nyaman digunakan. Selain nyaman digunakan,
\texttt{RStudio} memungkinkan kita melakukan penulisan laporan
menggunakan \texttt{Rmarkdown} atau \texttt{RNotebook} serta membuat
berbagai bentuk project seperti shyni, dll. Pada \texttt{R} studio juga
memungkinkan kita mengatur \emph{working directory} tanpa perlu
mengetikkan sintaks pada Commander, yang diperlukan hanya memilihnya di
menu \texttt{RStudio}. Selain itu, kita juga dapat meng-import file
berisikan data tanpa perlu mengetikkan pada Commander dengan cara
memilih pada menu \texttt{Environment}.

\section{Menginstall R dan RStudio}\label{menginstall-r-dan-rstudio}

Pada tutorial ini hanya akan dijelaskan bagaimana menginstal \texttt{R}
dan \texttt{RStudio} pada sistem operasi \texttt{windows}. Sebelum
memulai menginstal sebaiknya pembaca mengunduh terlebih dahulu
\emph{installer} \href{https://cran.r-project.org/bin/windows/base/}{R}
dan \href{https://www.rstudio.com/products/rstudio/download/}{RStudio}.

\begin{enumerate}
\def\labelenumi{\arabic{enumi}.}
\tightlist
\item
  Jalankan proses pemasangan dengan meng-klik \emph{installer} aplikasi
  \texttt{R} dan \texttt{RStudio}.
\item
  Ikuti langkah proses pemasangan aplikasi yang ditampilkan dengan klik
  \texttt{OK} atau \texttt{Next}.
\item
  Apabila pemasangan telah dilakukan, jalankan aplikasi yang telah
  terpasang untuk menguji jika aplikasi telah berjalan dengan baik.
\end{enumerate}

Jendela aplikasi yang telah terpasang ditampilkan pada Gambar
\ref{fig:jendela-R} dan Gambar \ref{fig:jendela-RStudio}.

\begin{figure}

{\centering \includegraphics[width=0.8\linewidth]{jendela_r} 

}

\caption{Jendela R.}\label{fig:jendela-R}
\end{figure}

\begin{figure}

{\centering \includegraphics[width=0.8\linewidth]{jendela_rstudio} 

}

\caption{Jendela RStudio.}\label{fig:jendela-RStudio}
\end{figure}

\begin{quote}
\textbf{Note: } Sebaiknya install \texttt{R} terlebih dahulu sebelum
\texttt{RStudio}
\end{quote}

\section{Working Directory}\label{working-directory}

Setiap pengguna akan bekerja pada tempat khusus yang disebut sebagai
\emph{working directory}. \emph{working directory} merupakan sebuah
folder dimana \texttt{R} akan membaca dan menyimpan file kerja kita.
Pada pengguna \texttt{windows}, \emph{working directory} secara default
pada saat pertama kali menginstall \texttt{R} terletak pada folder
\texttt{c:\textbackslash{}\textbackslash{}Document}.

\subsection{Mengubah Lokasi Working
Directory}\label{mengubah-lokasi-working-directory}

Kita dapat mengubah lokasi \emph{working directory} berdasarkan lokasi
yang kita inginkan, misalnya letak data yang akan kita olah tidak ada
pada folder default atau kita ingin pekerjaan kita terkait \texttt{R}
dapat berlangsung pada satu folder khusus.

Berikut adalah cara mengubah \emph{working directory} pada \texttt{R}.

\begin{enumerate}
\def\labelenumi{\arabic{enumi}.}
\tightlist
\item
  Buatlah folder pada drive (kita bisa membuat folder pada selain drive
  c) dan namai dengan nama yang kalian inginkan. Pada tutorial ini
  penulis menggunakan nama folder \texttt{R}.
\item
  Jika pengguna menggunakan \texttt{RStudio}, pada menu \texttt{RStudio}
  pilih \textbf{Session \textgreater{} Set Working Directory
  \textgreater{} Chooses Directory}. Proses tersebut ditampilkan pada
  Gambar \ref{fig:working}
\item
  Pilih folder yang telah dibuat pada step 1 sebagai *working directory.
\end{enumerate}

\begin{quote}
\textbf{Note: } Data atau file yang hendak dibaca selama proses kerja
pada \texttt{R} harus selalu diletakkan pada working directory. Jika
tidak maka data atau file tidak akan terbaca.
\end{quote}

Untuk mengecek apakah proses perubahan telah terjadi, kita dapat
mengeceknya dengan menjalankan perintah berikut untuk melihat lokasi
\emph{working directory} kita yang baru.

\begin{Shaded}
\begin{Highlighting}[]
\KeywordTok{getwd}\NormalTok{()}
\end{Highlighting}
\end{Shaded}

\begin{figure}

{\centering \includegraphics[width=0.8\linewidth]{working} 

}

\caption{Mengubah working directory.}\label{fig:working}
\end{figure}

Selain itu kita dapat mengubah \emph{working directory} menggunakan
perintah berikut:

\begin{Shaded}
\begin{Highlighting}[]
\CommentTok{# Ubah working directori pada folder R}
\KeywordTok{setwd}\NormalTok{(}\StringTok{"/Documents/R"}\NormalTok{)}
\end{Highlighting}
\end{Shaded}

\begin{quote}
\textbf{Note: } Pada proses pengisian lokasi folder pastikan pemisah
pada lokasi folder menggunakan tanda ``/'' bukan ``"
\end{quote}

\subsection{Mengubah Lokasi Working Directory
Default}\label{mengubah-lokasi-working-directory-default}

Pada proses yang telah penulis jelaskan sebelumnya. Proses perubahan
\emph{working directory} hanya berlaku pada saat pekerjaan tersebut
dilakukan. Setelah pekerjaan selesai dan kita menjalankan kembali
\texttt{R} maka \emph{working directory} akan kembali secara default
pada working directory lama.

Untuk membuat lokasi default \emph{working directory} pindah, kita dapat
melakukannya dengan memilih pada menu: \textbf{Tools \textgreater{}
Global options \textgreater{} pada ``General'' klik pada ``Browse'' dan
pilih lokasi working directory yang diinginkan}. Proses tersebut
ditampilkan pada Gambar \ref{fig:default}

\begin{figure}

{\centering \includegraphics[width=0.8\linewidth]{default} 

}

\caption{Merubah working directory melalui Global options.}\label{fig:default}
\end{figure}

\section{Fasilitas Help}\label{fasilitas-help}

Agar dapat menggunakan \texttt{R} dengan secara lebih baik, pengetahuan
untuk mengakses fasilitas \emph{help} in cukup penting untuk
disampaikan. Adapun cara yang dapat digunakan adalah sebagai berikut.

\subsection{Mencari Help dari Suatu Perintah
Tertentu}\label{mencari-help-dari-suatu-perintah-tertentu}

Untuk memperoleh bantuan terkait suatu perintah tertentu kita dapat
menggunakan fungsi \texttt{help()}. Secara umum format yang digunakan
adalah sebagai berikut:

\begin{Shaded}
\begin{Highlighting}[]
\KeywordTok{help}\NormalTok{(nama_perintah)}
\end{Highlighting}
\end{Shaded}

atau dapat juga menggunakan tanda tanya (?) pada awal
\texttt{nama\_perintah} seperti berikut:

\begin{Shaded}
\begin{Highlighting}[]
\NormalTok{?nama_perintah}
\end{Highlighting}
\end{Shaded}

Misalkan kita kebingungan terkait bagaimana cara menuliskan perintah
untuk menghitung rata-rata suatu vektor. Kita dapat mengetikkan perintah
berikut untuk mengakses fasilitas \emph{help}.

\begin{Shaded}
\begin{Highlighting}[]
\KeywordTok{help}\NormalTok{(mean)}

\CommentTok{#atau}
\NormalTok{?mean}
\end{Highlighting}
\end{Shaded}

Perintah tersebut akan memunculkan hasil berupa dokumentasi yang
ditampilkan pada Gambar \ref{fig:meandoc}.

\begin{figure}

{\centering \includegraphics[width=0.5\linewidth]{meandoc} 

}

\caption{Jendela help dokumentasi fungsi mean().}\label{fig:meandoc}
\end{figure}

Keterangan pada jendela pada Gambar \ref{fig:meandoc} adalah sebagia
berikut:

\begin{enumerate}
\def\labelenumi{\arabic{enumi}.}
\tightlist
\item
  Pada bagian jendela kiri atas jendela \emph{help}, diberikan
  keterangan nama dari perintah yang sedang ditampilkan.
\item
  Selanjutnya, pada bagian atas dokumen, ditampilkan infomasi terkait
  nama perintah, dan nama \emph{library} yang memuat perintah tersebut.
  Pada gambar diatas informasi terkait perintah dan nama \emph{library}
  ditunjukkan pada teks \texttt{mean\ \{base\}} yang menunjukkan
  perintah \texttt{mean()} pada paket (\emph{library}) \emph{base}
  (paket bawaan \texttt{R}).
\item
  Setiap jendela \emph{help} dari suatu perintah tertentu selanjutnya
  akan memuat bagian-bagian berikut:
\end{enumerate}

\begin{itemize}
\tightlist
\item
  \emph{Title}
\item
  \emph{Description} : deskripsi singkat tentang perintah.
\item
  \emph{Usage} : menampilkan sintaks perintah untuk penggunaan perintah
  tersebut.
\item
  \emph{Arguments} : keterangan mengenai \emph{argument/input}yang
  diperlukan pada perintah tersebut.
\item
  \emph{Details} : keterangan lebih lengkap lengkap tentang perintah
  tersebut.
\item
  \emph{Value} : keterangan tentang \emph{output} suatu perintah dapat
  diperoleh pada bagian ini.
\item
  \emph{Author(s)} : memberikan keterangan tentang \emph{Author} dari
  perintah tersebut.
\item
  \emph{References} : seringkali referensi yang dapat digunakan untuk
  memperoleh keterangan lebih lanjut terhadap suatu perintah ditampilkan
  pada bagian ini.
\item
  \emph{See also}: bagian ini berisikan daftar perintah/fungsi yang
  berhubungan erat dengan perintah tersebut.
\item
  \emph{Example} : berisikan contoh-contoh penggunaan perintah tersebut.
\end{itemize}

Kita juga dapat melihat contoh penggunaan dari perintah tersebut. Untuk
melakukannya kita dapat menggunakan fungsi \texttt{example()}. Fungsi
tersebut akan menampilkan contoh kode penerapan dari fungsi yang kita
inginkan. Secara sederhana fungsi tersebut dapat dituliskan sebagai
berikut:

\begin{Shaded}
\begin{Highlighting}[]
\KeywordTok{example}\NormalTok{(nama_perintah)}
\end{Highlighting}
\end{Shaded}

Untuk mengetahui contoh kode fungsi \texttt{mean()}, ketikkan sintaks
berikut:

\begin{Shaded}
\begin{Highlighting}[]
\KeywordTok{example}\NormalTok{(mean)}
\end{Highlighting}
\end{Shaded}

\begin{verbatim}
## 
## mean> x <- c(0:10, 50)
## 
## mean> xm <- mean(x)
## 
## mean> c(xm, mean(x, trim = 0.10))
## [1] 8.75 5.50
\end{verbatim}

kita juga dapat mencoba kode yang dihasilkan pada console \texttt{R}.
Berikut adalah contoh penerapannya:

\begin{Shaded}
\begin{Highlighting}[]
\CommentTok{# Menghitung rata-rata bilangan 1 sampai 10 dan 50}
\CommentTok{# membuat vektor}
\NormalTok{x <-}\StringTok{ }\KeywordTok{c}\NormalTok{(}\DecValTok{0}\OperatorTok{:}\DecValTok{10}\NormalTok{, }\DecValTok{50}\NormalTok{)}

\CommentTok{# Print}
\NormalTok{x}
\end{Highlighting}
\end{Shaded}

\begin{verbatim}
##  [1]  0  1  2  3  4  5  6  7  8  9 10 50
\end{verbatim}

\begin{Shaded}
\begin{Highlighting}[]
\CommentTok{# mean}
\KeywordTok{mean}\NormalTok{(x)}
\end{Highlighting}
\end{Shaded}

\begin{verbatim}
## [1] 8.75
\end{verbatim}

Pembaca dapat mencoba melakukanya sendiri dengan mengganti nilai yang
telah ada serta mencoba contoh kode yang lain.

\subsection{General Help}\label{general-help}

Kita juga dapat membaca beberapa dokumen manual yang ada pada
\texttt{R}. Untuk melakukannya jalankan perintah berikut:

\begin{Shaded}
\begin{Highlighting}[]
\KeywordTok{help.start}\NormalTok{()}
\end{Highlighting}
\end{Shaded}

Output yang dihasilkan berupa link pada sejumlah dokumen yang dapat kita
klik. Tampilan halaman yang dihasilkan disajikan pada Gambar
\ref{fig:generalhelp}.

\begin{figure}

{\centering \includegraphics[width=0.5\linewidth]{generalhelp} 

}

\caption{Jendela general help dokumentasi fungsi mean().}\label{fig:generalhelp}
\end{figure}

\subsection{Fasilitas Help Lainnya}\label{fasilitas-help-lainnya}

Selain yang telah penulis sebutkan sebelumnya. Kita juga dapat
memanfaatkan fasilitas \emph{help} lainnya melalui fungsi
\texttt{apropos()} dan \texttt{help.search()}.

\texttt{apropos\ ()}: mengembalikan daftar objek, berisi pola yang
pembaca cari, dengan pencocokan sebagian. Ini berguna ketika pembaca
tidak ingat persis nama fungsi yang akan digunakan. Berikut adalah
contoh ketika penulis ingin mengetahui fungsi yang digunakan untuk
menghitung median.

\begin{Shaded}
\begin{Highlighting}[]
\KeywordTok{apropos}\NormalTok{(}\StringTok{"med"}\NormalTok{)}
\end{Highlighting}
\end{Shaded}

\begin{verbatim}
##  [1] "decmedian"       "elNamed"        
##  [3] "elNamed<-"       "interp.median"  
##  [5] "median"          "median.default" 
##  [7] "median_hilow"    "mediate"        
##  [9] "mediate.diagram" "medpolish"      
## [11] "runmed"
\end{verbatim}

\emph{List} yang dihasilkan berupa fungsi-fungsi yang memiliki elemen
kata ``med''. Berdasarkan pencaria tersebut penulis dapat mencoba
menggunakan fungsi ``median'' untuk menghitung median.

\texttt{help.search\ ()} (sebagai alternatif ??): mencari dokumentasi
yang cocok dengan karakter yang diberikan dengan cara yang berbeda. Ini
mengembalikan daftar fungsi yang mengandung istilah yang pembaca cari
dengan deskripsi singkat dari fungsi.

Berikut adalah contoh penerapan dari fungsi tersebut:

\begin{Shaded}
\begin{Highlighting}[]
\KeywordTok{help.search}\NormalTok{(}\StringTok{"mean"}\NormalTok{)}

\CommentTok{# atau}
\NormalTok{??mean}
\end{Highlighting}
\end{Shaded}

\emph{Output} yang dihasilkan akan tampak seperti pada Gambar
\ref{fig:helpsearch}.

\begin{figure}

{\centering \includegraphics[width=0.5\linewidth]{helpsearch} 

}

\caption{Jendela help search dokumentasi fungsi mean().}\label{fig:helpsearch}
\end{figure}

\section{Referensi}\label{referensi}

\begin{enumerate}
\def\labelenumi{\arabic{enumi}.}
\tightlist
\item
  Primartha, R. 2018. \textbf{Belajar Machine Learning Teori dan
  Praktik}. Penerbit Informatika : Bandung
\item
  Rosadi,D. 2016. \textbf{Analisis Statistika dengan R}. Gadjah Mada
  University Press: Yogyakarta
\item
  STHDA. Running RStudio and Setting Up Your Working Directory - Easy R
  Programming
  .\url{http://www.sthda.com/english/wiki/running-rstudio-and-setting-up-your-working-directory-easy-r-programming\#set-your-working-directory}
\item
  STDHA. \textbf{Getting Help With Functions In R Programming}.
  \url{http://www.sthda.com/english/wiki/getting-help-with-functions-in-r-programming}
  .
\item
  Venables, W.N. Smith D.M. and R Core Team. 2018. \textbf{An
  Introduction to R}. R Manuals.
\end{enumerate}

\chapter{Sintaks Bahasa R}\label{sintaks-bahasa-r}

Pada \emph{chapter} ini penulis hendak mengajak pembaca lebih familiar
dengan sintaks atau perintah yang ada pada \texttt{R}. Pembaca akan
mempelajari penggunaan operator dalam melakukan operasi pengolahan data
pada \texttt{R}, jenis data yang ada pada \texttt{R}, sampai dengan
bagaimana kita melakukan proses \emph{decision making} menggunakan
\texttt{R}.

\section{Operator Aritmatika}\label{operator-aritmatika}

Proses perhitungan akan ditangani oleh fungsi khusus. \texttt{R} akan
memahami urutannya secara benar. Kecuali kita secara eksplisit
menetapkan yang lain. Sebagai contoh jalankan sintaks berikut:

\begin{Shaded}
\begin{Highlighting}[]
\DecValTok{2}\OperatorTok{+}\DecValTok{4}\OperatorTok{*}\DecValTok{2}
\end{Highlighting}
\end{Shaded}

\begin{verbatim}
## [1] 10
\end{verbatim}

Bandingkan dengan sintaks berikut:

\begin{Shaded}
\begin{Highlighting}[]
\NormalTok{(}\DecValTok{2}\OperatorTok{+}\DecValTok{4}\NormalTok{)}\OperatorTok{*}\DecValTok{2}
\end{Highlighting}
\end{Shaded}

\begin{verbatim}
## [1] 12
\end{verbatim}

\begin{quote}
\texttt{R} dapat digunakan sebagai kalkulator
\end{quote}

Berdasarkan kedua hasil tersebut dapat disimpulkan bahwa ketika kita
tidak menetapkan urutan perhitungan menggunakan tanda kurung, \texttt{R}
akan secara otomatis akan menghitung terlebih dahulu perkalian atau
pembangian.

Operator aritmatika yang disediakan \texttt{R} disajikan pada Tabel
\ref{tab:oparitmatika}:

\begin{longtable}[]{@{}ll@{}}
\toprule
\begin{minipage}[b]{0.15\columnwidth}\raggedright\strut
\textbf{Simbol}\strut
\end{minipage} & \begin{minipage}[b]{0.79\columnwidth}\raggedright\strut
\textbf{Keterangan}\strut
\end{minipage}\tabularnewline
\midrule
\endhead
\begin{minipage}[t]{0.15\columnwidth}\raggedright\strut
+\strut
\end{minipage} & \begin{minipage}[t]{0.79\columnwidth}\raggedright\strut
\emph{Addition}, untuk operasi penjumlahan\strut
\end{minipage}\tabularnewline
\begin{minipage}[t]{0.15\columnwidth}\raggedright\strut
-\strut
\end{minipage} & \begin{minipage}[t]{0.79\columnwidth}\raggedright\strut
\emph{Substraction}, untuk operasi pengurangan\strut
\end{minipage}\tabularnewline
\begin{minipage}[t]{0.15\columnwidth}\raggedright\strut
*\strut
\end{minipage} & \begin{minipage}[t]{0.79\columnwidth}\raggedright\strut
\emph{Multiplication}, untuk operasi pembagian\strut
\end{minipage}\tabularnewline
\begin{minipage}[t]{0.15\columnwidth}\raggedright\strut
/\strut
\end{minipage} & \begin{minipage}[t]{0.79\columnwidth}\raggedright\strut
\emph{Division}, untuk operasi pembagian\strut
\end{minipage}\tabularnewline
\begin{minipage}[t]{0.15\columnwidth}\raggedright\strut
\^{}\strut
\end{minipage} & \begin{minipage}[t]{0.79\columnwidth}\raggedright\strut
\emph{Eksponentiation}, untuk operasi pemangkatan\strut
\end{minipage}\tabularnewline
\begin{minipage}[t]{0.15\columnwidth}\raggedright\strut
\%\%\strut
\end{minipage} & \begin{minipage}[t]{0.79\columnwidth}\raggedright\strut
\emph{Modulus}, Untuk mencari sisa pembagian\strut
\end{minipage}\tabularnewline
\begin{minipage}[t]{0.15\columnwidth}\raggedright\strut
\%/\%\strut
\end{minipage} & \begin{minipage}[t]{0.79\columnwidth}\raggedright\strut
\emph{Integer}, Untuk mencari bilangan bulat hasil pembagian saja dan
tanpa sisa pembagian\strut
\end{minipage}\tabularnewline
\bottomrule
\end{longtable}

*: \label{tab:oparitmatika} Operator Aritmatika \texttt{R}.**

Untuk lebih memahaminya berikut contoh sintaks penerapan operator
tersebut.

\begin{Shaded}
\begin{Highlighting}[]
\CommentTok{# Addition}
\DecValTok{5}\OperatorTok{+}\DecValTok{3}
\end{Highlighting}
\end{Shaded}

\begin{verbatim}
## [1] 8
\end{verbatim}

\begin{Shaded}
\begin{Highlighting}[]
\CommentTok{# Substraction}
\DecValTok{5}\OperatorTok{-}\DecValTok{3}
\end{Highlighting}
\end{Shaded}

\begin{verbatim}
## [1] 2
\end{verbatim}

\begin{Shaded}
\begin{Highlighting}[]
\CommentTok{# Multiplication}
\DecValTok{5}\OperatorTok{*}\DecValTok{3}
\end{Highlighting}
\end{Shaded}

\begin{verbatim}
## [1] 15
\end{verbatim}

\begin{Shaded}
\begin{Highlighting}[]
\CommentTok{# Division}
\DecValTok{5}\OperatorTok{/}\DecValTok{3}
\end{Highlighting}
\end{Shaded}

\begin{verbatim}
## [1] 1.667
\end{verbatim}

\begin{Shaded}
\begin{Highlighting}[]
\CommentTok{# Eksponetiation}
\DecValTok{5}\OperatorTok{^}\DecValTok{3}
\end{Highlighting}
\end{Shaded}

\begin{verbatim}
## [1] 125
\end{verbatim}

\begin{Shaded}
\begin{Highlighting}[]
\CommentTok{# Modulus}
\DecValTok{5}\OperatorTok\DecValTok{3}
\end{Highlighting}
\end{Shaded}

\begin{verbatim}
## [1] 2
\end{verbatim}

\begin{Shaded}
\begin{Highlighting}[]
\CommentTok{# Integer}
\DecValTok{5}\OperatorTok\DecValTok{3}
\end{Highlighting}
\end{Shaded}

\begin{verbatim}
## [1] 1
\end{verbatim}

\begin{quote}
\emph{Note: } Pada \texttt{R} tanda \texttt{\#} berfungsi menambahkan
keterangan untuk menjelaskan sebuah sintaks pada \texttt{R}.
\end{quote}

\section{Fungsi Aritmetik}\label{fungsi-aritmetik}

Selain fungsi operator aritmetik, pada \texttt{R} juga telah tersedia
fungsi aritmetik yang lain seperti logaritmik, ekponensial,
trigonometri, dll.

\begin{enumerate}
\def\labelenumi{\arabic{enumi}.}
\tightlist
\item
  Logaritma dan eksponensial
\end{enumerate}

Untuk contoh fungsi logaritmik dan eksponensial jalankan sintaks
berikut:

\begin{Shaded}
\begin{Highlighting}[]
\KeywordTok{log2}\NormalTok{(}\DecValTok{8}\NormalTok{) }\CommentTok{# logaritma basis 2 untuk 8}
\end{Highlighting}
\end{Shaded}

\begin{verbatim}
## [1] 3
\end{verbatim}

\begin{Shaded}
\begin{Highlighting}[]
\KeywordTok{log10}\NormalTok{(}\DecValTok{8}\NormalTok{) }\CommentTok{# logaritma basis 10 untuk 8}
\end{Highlighting}
\end{Shaded}

\begin{verbatim}
## [1] 0.9031
\end{verbatim}

\begin{Shaded}
\begin{Highlighting}[]
\KeywordTok{exp}\NormalTok{(}\DecValTok{8}\NormalTok{) }\CommentTok{# eksponensial 8}
\end{Highlighting}
\end{Shaded}

\begin{verbatim}
## [1] 2981
\end{verbatim}

\begin{enumerate}
\def\labelenumi{\arabic{enumi}.}
\setcounter{enumi}{1}
\tightlist
\item
  Fungsi trigonometri
\end{enumerate}

fungsi trigonometri yang ditampilkan seperti sin,cos, tan, dll.

\begin{Shaded}
\begin{Highlighting}[]
\KeywordTok{cos}\NormalTok{(x) }\CommentTok{# cos x}
\KeywordTok{sin}\NormalTok{(x) }\CommentTok{# Sin x}
\KeywordTok{tan}\NormalTok{(x) }\CommentTok{# Tan x}
\KeywordTok{acos}\NormalTok{(x) }\CommentTok{# arc-cos x}
\KeywordTok{asin}\NormalTok{(x) }\CommentTok{# arc-sin x}
\KeywordTok{atan}\NormalTok{(x) }\CommentTok{#arc-tan x}
\end{Highlighting}
\end{Shaded}

\begin{quote}
\textbf{Note: } x dalam fungsi trigonometri memiliki satuan radian
\end{quote}

Berikut adalah salah satu contoh penggunaannya:

\begin{Shaded}
\begin{Highlighting}[]
\KeywordTok{cos}\NormalTok{(pi)}
\end{Highlighting}
\end{Shaded}

\begin{verbatim}
## [1] -1
\end{verbatim}

\begin{enumerate}
\def\labelenumi{\arabic{enumi}.}
\setcounter{enumi}{2}
\tightlist
\item
  Fungsi matematik lainnya
\end{enumerate}

Fungsi lainnya yang dapat digunakan adalah fungsi absolut, akar kuadrat,
dll. Berikut adalah contoh sintaks penggunaan fungsi absolut dan akar
kuadrat.

\begin{Shaded}
\begin{Highlighting}[]
\KeywordTok{abs}\NormalTok{(}\OperatorTok{-}\DecValTok{2}\NormalTok{) }\CommentTok{# nilai absolut -2}
\end{Highlighting}
\end{Shaded}

\begin{verbatim}
## [1] 2
\end{verbatim}

\begin{Shaded}
\begin{Highlighting}[]
\KeywordTok{sqrt}\NormalTok{(}\DecValTok{4}\NormalTok{) }\CommentTok{# akar kuadrat 4}
\end{Highlighting}
\end{Shaded}

\begin{verbatim}
## [1] 2
\end{verbatim}

\section{Operator Relasi}\label{operator-relasi}

Operator relasi digunakan untuk membandingkan satu objek dengan objek
lainnya. Operator yang disediakan \texttt{R} disajikan pada Tabel
\ref{tab:oprelasi}.

\begin{longtable}[]{@{}ll@{}}
\caption{\label{tab:oprelasi} Operator Relasi \texttt{R}.}\tabularnewline
\toprule
\textbf{Simbol} & \textbf{Keterangan}\tabularnewline
\midrule
\endfirsthead
\toprule
\textbf{Simbol} & \textbf{Keterangan}\tabularnewline
\midrule
\endhead
``\textgreater{}'' & Lebih besar dari\tabularnewline
``\textless{}'' & Lebih Kecil dari\tabularnewline
``=='' & Sama dengan\tabularnewline
``\textgreater{}='' & Lebih besar sama dengan\tabularnewline
``\textless{}='' & Lebih kecil sama dengan\tabularnewline
``!='' & Tidak sama dengan\tabularnewline
\bottomrule
\end{longtable}

Berikut adalah penerapan operator pada tabel tersebut:

\begin{Shaded}
\begin{Highlighting}[]
\NormalTok{x <-}\StringTok{ }\DecValTok{34}
\NormalTok{y <-}\StringTok{ }\DecValTok{35}

\CommentTok{# Operator >}
\NormalTok{x }\OperatorTok{>}\StringTok{ }\NormalTok{y}
\end{Highlighting}
\end{Shaded}

\begin{verbatim}
## [1] FALSE
\end{verbatim}

\begin{Shaded}
\begin{Highlighting}[]
\CommentTok{# Operator <}
\NormalTok{x }\OperatorTok{<}\StringTok{ }\NormalTok{y}
\end{Highlighting}
\end{Shaded}

\begin{verbatim}
## [1] TRUE
\end{verbatim}

\begin{Shaded}
\begin{Highlighting}[]
\CommentTok{# operator ==}
\NormalTok{x }\OperatorTok{==}\StringTok{ }\NormalTok{y}
\end{Highlighting}
\end{Shaded}

\begin{verbatim}
## [1] FALSE
\end{verbatim}

\begin{Shaded}
\begin{Highlighting}[]
\CommentTok{# Operator >=}
\NormalTok{x }\OperatorTok{>=}\StringTok{ }\NormalTok{y}
\end{Highlighting}
\end{Shaded}

\begin{verbatim}
## [1] FALSE
\end{verbatim}

\begin{Shaded}
\begin{Highlighting}[]
\CommentTok{# Operator <=}
\NormalTok{x }\OperatorTok{<=}\StringTok{ }\NormalTok{y}
\end{Highlighting}
\end{Shaded}

\begin{verbatim}
## [1] TRUE
\end{verbatim}

\begin{Shaded}
\begin{Highlighting}[]
\CommentTok{# Operator !=}
\NormalTok{x }\OperatorTok{!=}\StringTok{ }\NormalTok{y}
\end{Highlighting}
\end{Shaded}

\begin{verbatim}
## [1] TRUE
\end{verbatim}

\section{Operator Logika}\label{operator-logika}

Operator logika hanya berlaku pada vektor dengan tipe logical, numeric,
atau complex. Semua angka bernilai 1 akan dianggap bernilai logika
\texttt{TRUE}. Operator logika yang disediakan \texttt{R} dapat dilihat
pada Tabel \ref{tab:oplogika}.

\begin{longtable}[]{@{}ll@{}}
\caption{\label{tab:oplogika} Operator logika \texttt{R}.}\tabularnewline
\toprule
\textbf{Simbol} & \textbf{Keterangan}\tabularnewline
\midrule
\endfirsthead
\toprule
\textbf{Simbol} & \textbf{Keterangan}\tabularnewline
\midrule
\endhead
\&\& & Operator logika AND\tabularnewline
&\tabularnewline
! & Opeartor logika NOT\tabularnewline
\& & Operator logika AND element wise\tabularnewline
& Operator logika OR element wise\tabularnewline
\bottomrule
\end{longtable}

Penerapannya terdapat pada sintaks berikut:

\begin{Shaded}
\begin{Highlighting}[]
\NormalTok{v <-}\StringTok{ }\KeywordTok{c}\NormalTok{(}\OtherTok{TRUE}\NormalTok{,}\OtherTok{TRUE}\NormalTok{, }\OtherTok{FALSE}\NormalTok{)}
\NormalTok{t <-}\StringTok{ }\KeywordTok{c}\NormalTok{(}\OtherTok{FALSE}\NormalTok{,}\OtherTok{FALSE}\NormalTok{,}\OtherTok{FALSE}\NormalTok{)}

\CommentTok{# Operator &&}
\KeywordTok{print}\NormalTok{(v}\OperatorTok{&&}\NormalTok{t)}
\end{Highlighting}
\end{Shaded}

\begin{verbatim}
## [1] FALSE
\end{verbatim}

\begin{Shaded}
\begin{Highlighting}[]
\CommentTok{# Operator ||}
\KeywordTok{print}\NormalTok{(v}\OperatorTok{||}\NormalTok{t)}
\end{Highlighting}
\end{Shaded}

\begin{verbatim}
## [1] TRUE
\end{verbatim}

\begin{Shaded}
\begin{Highlighting}[]
\CommentTok{# Operator !}
\KeywordTok{print}\NormalTok{(}\OperatorTok{!}\NormalTok{v)}
\end{Highlighting}
\end{Shaded}

\begin{verbatim}
## [1] FALSE FALSE  TRUE
\end{verbatim}

\begin{Shaded}
\begin{Highlighting}[]
\CommentTok{# operator &}
\KeywordTok{print}\NormalTok{(v}\OperatorTok{&}\NormalTok{t)}
\end{Highlighting}
\end{Shaded}

\begin{verbatim}
## [1] FALSE FALSE FALSE
\end{verbatim}

\begin{Shaded}
\begin{Highlighting}[]
\CommentTok{# Operator |}
\KeywordTok{print}\NormalTok{(v}\OperatorTok{|}\NormalTok{t)}
\end{Highlighting}
\end{Shaded}

\begin{verbatim}
## [1]  TRUE  TRUE FALSE
\end{verbatim}

\begin{quote}
\textbf{Note: }

operator \& dan \textbar{} akan mengecek logika tiap elemen pada vektor
secara berpesangan (sesuai urutan dari kiri ke kanan).

Operator \%\% dan \textbar{}\textbar{} hanya mengecek dari kiri ke kanan
pada observasi pertama. Misal saat menggunakan \&\& jika observasi
pertama TRUE maka observasi pertama pada vektor lainnya akan dicek,
namun jika observasi pertama FALSE maka proses akan segera dihentikan
dan menghasilkan FALSE.
\end{quote}

\section{Memasukkan Nilai Kedalam
Variabel}\label{memasukkan-nilai-kedalam-variabel}

Variabel pada \texttt{R} dapat digunakan untuk menyimpan nilai. Sebagai
contoh jalankan sintaks berikut:

\begin{Shaded}
\begin{Highlighting}[]
\CommentTok{# Harga sebuah lemon adalah 500 rupiah}
\NormalTok{lemon <-}\StringTok{ }\DecValTok{500}

\CommentTok{# Atau}
\DecValTok{500}\NormalTok{ ->}\StringTok{ }\NormalTok{lemon}

\CommentTok{# dapat juga menggunakan tanda "="}
\NormalTok{lemon =}\StringTok{ }\DecValTok{500}
\end{Highlighting}
\end{Shaded}

\begin{quote}
\textbf{Note: }

\begin{enumerate}
\def\labelenumi{\arabic{enumi}.}
\item
  \texttt{R} memungkinkan penggunaan \textless{}-,-\textgreater{}, atau
  = sebagai perintah pengisi nilai variabel
\item
  \texttt{R} bersifat \emph{case-sensitive}. Maksudnya adalah variabel
  Lemon tidak sama dengan lemon (Besar kecil huruf berpengaruh)
\end{enumerate}
\end{quote}

Untuk mengetahui nilai dari objek \texttt{lemon} kita dapat menggunakan
fungsi \texttt{print()} atau mengetikkan nama objeknya secara langsung.

\begin{Shaded}
\begin{Highlighting}[]
\CommentTok{# Menggunakan fungsi print()}
\KeywordTok{print}\NormalTok{(lemon)}
\end{Highlighting}
\end{Shaded}

\begin{verbatim}
## [1] 500
\end{verbatim}

\begin{Shaded}
\begin{Highlighting}[]
\CommentTok{# Atau}
\NormalTok{lemon}
\end{Highlighting}
\end{Shaded}

\begin{verbatim}
## [1] 500
\end{verbatim}

\texttt{R} akan menyimpan variabel \texttt{lemon} sebagai objek pada
memori. Sehingga kita dapat melakukan operasi terhadap objek tersebut
seperti mengalikannya atau menjumlahkannya dengan bilangan lain. Sebagai
contoh jalankan sintaks berikut:

\begin{Shaded}
\begin{Highlighting}[]
\CommentTok{# Operasi perkalian terhadap objek lemon}
\DecValTok{5}\OperatorTok{*}\NormalTok{lemon}
\end{Highlighting}
\end{Shaded}

\begin{verbatim}
## [1] 2500
\end{verbatim}

Kita dapat juga mengubah nilai dari objek \texttt{lemon} dengan cara
menginput nilai baru terhadap objek yang sama. \texttt{R} secara
otomatis akan menggatikan nilai sebelumnya. Untuk lebih memahaminya
jalankan sintaks berikut:

\begin{Shaded}
\begin{Highlighting}[]
\NormalTok{lemon <-}\StringTok{ }\DecValTok{1000}

\CommentTok{# Print lemon}
\KeywordTok{print}\NormalTok{(lemon)}
\end{Highlighting}
\end{Shaded}

\begin{verbatim}
## [1] 1000
\end{verbatim}

Untuk lebih memahaminya berikut adalah sintaks untuk menghitung volume
suatu objek.

\begin{Shaded}
\begin{Highlighting}[]
\CommentTok{# Dimensi objek}
\NormalTok{panjang <-}\StringTok{ }\DecValTok{10}
\NormalTok{lebar <-}\StringTok{ }\DecValTok{5}
\NormalTok{tinggi <-}\StringTok{ }\DecValTok{5}

\CommentTok{# Menghitung volume}
\NormalTok{volume <-}\StringTok{ }\NormalTok{panjang}\OperatorTok{*}\NormalTok{lebar}\OperatorTok{*}\NormalTok{tinggi}

\CommentTok{# Print objek volume}
\KeywordTok{print}\NormalTok{(volume)}
\end{Highlighting}
\end{Shaded}

\begin{verbatim}
## [1] 250
\end{verbatim}

Untuk mengetahui objek apa saja yang telah kita buat sepanjang artikel
ini kita dapang menggunakan fungsi \texttt{ls()}.

\begin{Shaded}
\begin{Highlighting}[]
\KeywordTok{ls}\NormalTok{()}
\end{Highlighting}
\end{Shaded}

\begin{verbatim}
##  [1] "A"         "B"         "img1_path" "lebar"    
##  [5] "lemon"     "panjang"   "t"         "tinggi"   
##  [9] "v"         "volume"    "x"         "xm"       
## [13] "y"
\end{verbatim}

\begin{quote}
Kumpulan objek yang telah tersimpan dalam memori disebut sebagai
\textbf{workspace}
\end{quote}

Untuk menghapus objek pada memori kita dapat menggunakan fungsi
\texttt{rm()}. Pada sintaks berikut penulis hendak menghapus objek
\texttt{lemon} dan \texttt{volume}.

\begin{Shaded}
\begin{Highlighting}[]
\CommentTok{# Menghapus objek lemon dan volume}
\KeywordTok{rm}\NormalTok{(lemon, volume)}

\CommentTok{# Tampilkan kembali objek yang tersisa}
\KeywordTok{ls}\NormalTok{()}
\end{Highlighting}
\end{Shaded}

\begin{verbatim}
##  [1] "A"         "B"         "img1_path" "lebar"    
##  [5] "panjang"   "t"         "tinggi"    "v"        
##  [9] "x"         "xm"        "y"
\end{verbatim}

\begin{quote}
\textbf{Note: } Setiap variabel atau objek yang dibuat akan menempati
sejumlah memori pada komputer sehingga jika kita bekerja dengan jumlah
data yang banyak pastikan kita menghapus seluruh objek pada memori
sebelum memulai kerja.
\end{quote}

\section{Tipe Data}\label{tipe-data}

Data pada \texttt{R} dapat dikelompokan berdasarkan beberapa tipe. Tipe
data pada \texttt{R} disajikan pada Tabel \ref{tab:tipedata}.

\begin{longtable}[]{@{}lll@{}}
\caption{\label{tab:tipedata} Tipe Data \texttt{R}.}\tabularnewline
\toprule
\textbf{Tipe Data} & \textbf{Contoh} &
\textbf{Keterangan}\tabularnewline
\midrule
\endfirsthead
\toprule
\textbf{Tipe Data} & \textbf{Contoh} &
\textbf{Keterangan}\tabularnewline
\midrule
\endhead
Logical & TRUE, FALSE & Nilai Boolean\tabularnewline
Numeric & 12.3, 5, 999 & Segala jenis angka\tabularnewline
Integer & 23L, 97L, 3L & Bilangan integer (bilangan
bulat)\tabularnewline
Complex & 2i, 3i, 9i & Bilangan kompleks\tabularnewline
Character & `a', ``b'', ``123'' & Karakter dan string\tabularnewline
Raw & Identik dengan ``hello'' & Segala jenis data yang disimpan sebagai
raw bytes\tabularnewline
\bottomrule
\end{longtable}

Sintaks berikut adalah contoh dari tipe data pada \texttt{R}. Untuk
mengetahui tipa data suatu objek kita dapat menggunakan perintah
\texttt{class()}

\begin{Shaded}
\begin{Highlighting}[]
\CommentTok{# Logical}
\NormalTok{apel <-}\StringTok{ }\OtherTok{TRUE}
\KeywordTok{class}\NormalTok{(apel)}
\end{Highlighting}
\end{Shaded}

\begin{verbatim}
## [1] "logical"
\end{verbatim}

\begin{Shaded}
\begin{Highlighting}[]
\CommentTok{# Numeric}
\NormalTok{x <-}\StringTok{ }\FloatTok{2.3}
\KeywordTok{class}\NormalTok{(x)}
\end{Highlighting}
\end{Shaded}

\begin{verbatim}
## [1] "numeric"
\end{verbatim}

\begin{Shaded}
\begin{Highlighting}[]
\CommentTok{# Integer}
\NormalTok{y <-}\StringTok{ }\NormalTok{2L}
\KeywordTok{class}\NormalTok{(y)}
\end{Highlighting}
\end{Shaded}

\begin{verbatim}
## [1] "integer"
\end{verbatim}

\begin{Shaded}
\begin{Highlighting}[]
\CommentTok{# Compleks}
\NormalTok{z <-}\StringTok{ }\DecValTok{5}\OperatorTok{+}\NormalTok{2i}
\KeywordTok{class}\NormalTok{(z)}
\end{Highlighting}
\end{Shaded}

\begin{verbatim}
## [1] "complex"
\end{verbatim}

\begin{Shaded}
\begin{Highlighting}[]
\CommentTok{# string}
\NormalTok{w <-}\StringTok{ "saya"}
\KeywordTok{class}\NormalTok{(w)}
\end{Highlighting}
\end{Shaded}

\begin{verbatim}
## [1] "character"
\end{verbatim}

\begin{Shaded}
\begin{Highlighting}[]
\CommentTok{# Raw}
\NormalTok{xy <-}\StringTok{ }\KeywordTok{charToRaw}\NormalTok{(}\StringTok{"hello world"}\NormalTok{)}
\KeywordTok{class}\NormalTok{(xy)}
\end{Highlighting}
\end{Shaded}

\begin{verbatim}
## [1] "raw"
\end{verbatim}

Keenam jenis data tersebut disebut sebagai tipe data atomik. Hal ini
disebabkan karena hanya dapat menangani satu tipe data saja. Misalnya
hanya numeric atau hanya integer.

Selain menggunakan fungsi \texttt{class()}, kita dapat pula menggunakan
fungsi \texttt{is\_numeric()}, \texttt{is.character()},
\texttt{is.logical()}, dan sebagainya berdasarkan jenis data apa yang
ingin kita cek. Berbeda dengan fungsi \texttt{class()}, ouput yang
dihasilkan pada fungsi seperti \texttt{is\_numeric()} adalah nilai
Boolean sehingga fungsi ini hanya digunakan untuk mengecek apakah jenis
data pada objek sama seperti yang kita pikirkan. Sebagai contoh
disajikan pada sintaks berikut:

\begin{Shaded}
\begin{Highlighting}[]
\NormalTok{data <-}\StringTok{ }\DecValTok{25}

\CommentTok{# Cek apakah objek berisi data numerik}
\KeywordTok{is.numeric}\NormalTok{(data)}
\end{Highlighting}
\end{Shaded}

\begin{verbatim}
## [1] TRUE
\end{verbatim}

\begin{Shaded}
\begin{Highlighting}[]
\CommentTok{# Cek apakah objek adalah karakter}
\KeywordTok{is.character}\NormalTok{(data)}
\end{Highlighting}
\end{Shaded}

\begin{verbatim}
## [1] FALSE
\end{verbatim}

Kita juga dapat mengubah jenis data menjadi jenis lainnya seperti
integer menjadi numerik atau sebaliknya. Fungsi yang digunakan adalah
\texttt{as.numeric()} jika ingin mengubah suatu jenis data menjadi
numerik. Fungsi lainnya juga dapat digunakan sesuai dengan kita ingin
mengubah jenis data objek menjadi jenis data lainnya.

\begin{Shaded}
\begin{Highlighting}[]
\CommentTok{# Integer}
\NormalTok{apel <-}\StringTok{ }\NormalTok{2L}

\CommentTok{# Ubah menjadi numerik}
\KeywordTok{as.numeric}\NormalTok{(apel)}
\end{Highlighting}
\end{Shaded}

\begin{verbatim}
## [1] 2
\end{verbatim}

\begin{Shaded}
\begin{Highlighting}[]
\CommentTok{# Cek}
\KeywordTok{is.numeric}\NormalTok{(apel)}
\end{Highlighting}
\end{Shaded}

\begin{verbatim}
## [1] TRUE
\end{verbatim}

\begin{Shaded}
\begin{Highlighting}[]
\CommentTok{# Logical}
\NormalTok{nangka <-}\StringTok{ }\OtherTok{TRUE}

\CommentTok{# Ubah logical menjadi numeric}
\KeywordTok{as.numeric}\NormalTok{(nangka)}
\end{Highlighting}
\end{Shaded}

\begin{verbatim}
## [1] 1
\end{verbatim}

\begin{Shaded}
\begin{Highlighting}[]
\CommentTok{# Karakter}
\NormalTok{minum <-}\StringTok{ "minum"}

\CommentTok{# ubah karakter menjadi numerik}
\KeywordTok{as.numeric}\NormalTok{(minum)}
\end{Highlighting}
\end{Shaded}

\begin{verbatim}
## Warning: NAs introduced by coercion
\end{verbatim}

\begin{verbatim}
## [1] NA
\end{verbatim}

\begin{quote}
\textbf{Note: } Konversi karakter menjadi numerik akan menghasilkan
output NA (\emph{not available}). \texttt{R} tidak mengetahui bagaimana
cara merubah karakter menjadi bentuk numerik.
\end{quote}

Berdasarkan Tabel 2, vektor karakter dapat dibuat menggunakan tanda
kurung baik \emph{double quote} (``'') maupun \emph{single quote}
('').Jika pada teks yang kita tuliskan mengandung \emph{quote} maka kita
harus menghentikannya menggunakan tanda ( ~). Sbegai contoh kita ingin
menuliskan `\textbf{My friend's name is ``Adi''}, pada sintaks akan
dituliskan:

\begin{Shaded}
\begin{Highlighting}[]
\StringTok{'My friend\textbackslash{}`s name is "Adi"'}
\end{Highlighting}
\end{Shaded}

\begin{verbatim}
## [1] "My friend`s name is \"Adi\""
\end{verbatim}

\begin{Shaded}
\begin{Highlighting}[]
\CommentTok{# Atau}

\StringTok{"My friend's name }\CharTok{\textbackslash{}"}\StringTok{Adi}\CharTok{\textbackslash{}"}\StringTok{"}
\end{Highlighting}
\end{Shaded}

\begin{verbatim}
## [1] "My friend's name \"Adi\""
\end{verbatim}

\section{Vektor}\label{vektor}

Vektor merupakan kombinasi berbagai nilai (numerik, karakter, logical,
dan sebagainya berdasarkan jenis input data) pada objek yang sma. Pada
contoh kasus berikut, pembaca akan memiliki sesuai jenis data input
yaitu\textbf{vektor numerik}, \textbf{vector karakter}, \textbf{vektor
logical}, dll.

\subsection{Membuat vektor}\label{membuat-vektor}

Vektor dibuat dengan menggunakan fungsi \texttt{c()}(concatenate)
seperti yang disajikan pada sintaks berikut:

\begin{Shaded}
\begin{Highlighting}[]
\CommentTok{# membuat vektor numerik}
\NormalTok{x <-}\StringTok{ }\KeywordTok{c}\NormalTok{(}\DecValTok{3}\NormalTok{,}\FloatTok{3.5}\NormalTok{,}\DecValTok{4}\NormalTok{,}\DecValTok{7}\NormalTok{)}
\NormalTok{x }\CommentTok{# print vektor}
\end{Highlighting}
\end{Shaded}

\begin{verbatim}
## [1] 3.0 3.5 4.0 7.0
\end{verbatim}

\begin{Shaded}
\begin{Highlighting}[]
\CommentTok{# membuat vektor karakter}
\NormalTok{y <-}\StringTok{ }\KeywordTok{c}\NormalTok{(}\StringTok{"Apel"}\NormalTok{, }\StringTok{"Jeruk"}\NormalTok{, }\StringTok{"Rambutan"}\NormalTok{, }\StringTok{"Salak"}\NormalTok{)}
\NormalTok{y }\CommentTok{# print vektor}
\end{Highlighting}
\end{Shaded}

\begin{verbatim}
## [1] "Apel"     "Jeruk"    "Rambutan" "Salak"
\end{verbatim}

\begin{Shaded}
\begin{Highlighting}[]
\CommentTok{# membuat vektor logical}
\NormalTok{t <-}\StringTok{ }\KeywordTok{c}\NormalTok{(}\StringTok{"TRUE"}\NormalTok{, }\StringTok{"FALSE"}\NormalTok{, }\StringTok{"TRUE"}\NormalTok{)}
\NormalTok{t }\CommentTok{# print vektor}
\end{Highlighting}
\end{Shaded}

\begin{verbatim}
## [1] "TRUE"  "FALSE" "TRUE"
\end{verbatim}

selain menginput nilai pada vektor, kita juga dapat memberi nama nilai
setiap vektor menggunakan fungsi \texttt{names()}.

\begin{Shaded}
\begin{Highlighting}[]
\CommentTok{# Membuat vektor jumlah buah yang dibeli}
\NormalTok{Jumlah <-}\StringTok{ }\KeywordTok{c}\NormalTok{(}\DecValTok{5}\NormalTok{,}\DecValTok{5}\NormalTok{,}\DecValTok{6}\NormalTok{,}\DecValTok{7}\NormalTok{)}
\KeywordTok{names}\NormalTok{(Jumlah) <-}\StringTok{ }\KeywordTok{c}\NormalTok{(}\StringTok{"Apel"}\NormalTok{, }\StringTok{"Jeruk"}\NormalTok{, }\StringTok{"Rambutan"}\NormalTok{, }\StringTok{"Salak"}\NormalTok{)}

\CommentTok{# Atau}
\NormalTok{Jumlah <-}\StringTok{ }\KeywordTok{c}\NormalTok{(}\DataTypeTok{Apel=}\DecValTok{5}\NormalTok{, }\DataTypeTok{Jeruk=}\DecValTok{5}\NormalTok{, }\DataTypeTok{Rambutan=}\DecValTok{6}\NormalTok{, }\DataTypeTok{Salak=}\DecValTok{7}\NormalTok{)}

\CommentTok{# Print}
\NormalTok{Jumlah}
\end{Highlighting}
\end{Shaded}

\begin{verbatim}
##     Apel    Jeruk Rambutan    Salak 
##        5        5        6        7
\end{verbatim}

\begin{quote}
\textbf{Note: } Vektor hanya dapat memuat satu buah jenis data. Vektor
hanya dapat mengandung jenis data numerik saja, karakter saja, dll.
\end{quote}

Untuk menentukan panjang sebuah vektor kita dapat menggunakan fungsi
\texttt{lenght()}.

\begin{Shaded}
\begin{Highlighting}[]
\KeywordTok{length}\NormalTok{(Jumlah)}
\end{Highlighting}
\end{Shaded}

\begin{verbatim}
## [1] 4
\end{verbatim}

\subsection{Missing Values}\label{missing-values}

Seringkali nilai pada vektor kita tidak lengkap atau terdapat nilai yang
hilang (\emph{missing value}) pada vektor. \emph{Missing value} pada
\texttt{R} dilambangkan oleh \texttt{NA}(\emph{not available}). Berikut
adalah contoh vektor dengan \emph{missing value}.

\begin{Shaded}
\begin{Highlighting}[]
\NormalTok{Jumlah <-}\StringTok{ }\KeywordTok{c}\NormalTok{(}\DataTypeTok{Apel=}\DecValTok{5}\NormalTok{, }\DataTypeTok{Jeruk=}\OtherTok{NA}\NormalTok{, }\DataTypeTok{Rambutan=}\DecValTok{6}\NormalTok{, }\DataTypeTok{Salak=}\DecValTok{7}\NormalTok{)}
\end{Highlighting}
\end{Shaded}

Untuk mengecek apakah dalam objek terdapat \emph{missing value} dapat
menggunakan fungsi \texttt{is.na()}. ouput dari fungsi tersebut adalah
nilai Boolean. Jika terdapat \emph{Missing value}, maka output yang
dihasilkan akan memberikan nilai \texttt{TRUE}.

\begin{Shaded}
\begin{Highlighting}[]
\KeywordTok{is.na}\NormalTok{(Jumlah)}
\end{Highlighting}
\end{Shaded}

\begin{verbatim}
##     Apel    Jeruk Rambutan    Salak 
##    FALSE     TRUE    FALSE    FALSE
\end{verbatim}

\begin{quote}
\textbf{Note: }

Selain NA terdapat NaN (\emph{not a number}) sebagai \emph{missing
value8}. Nilai tersebut muncul ketika fungsi matematika yang digunakan
pada proses perhitungan tidak bekerja sebagaimana mestinya. Contoh: 0/0
= NaN

\texttt{is.na()} juga akan menghasilkan nilai \texttt{TRUE} pada NaN.
Untuk membedakannya dengan NA dapat digunakan fungsi \texttt{is.nan()}.
\end{quote}

\subsection{Subset Pada Vektor}\label{subset-pada-vektor}

\emph{Subseting vector} terdiri atas tiga jenis, yaitu: \emph{positive
indexing}, \emph{Negative Indexing}, dan .

\begin{itemize}
\tightlist
\item
  \textbf{Positive indexing}: memilih elemen vektor berdasarkan
  posisinya (indeks) dalam kurung siku.
\end{itemize}

\begin{Shaded}
\begin{Highlighting}[]
\CommentTok{# Subset vektor pada urutan kedua}
\NormalTok{Jumlah[}\DecValTok{2}\NormalTok{]}
\end{Highlighting}
\end{Shaded}

\begin{verbatim}
## Jeruk 
##    NA
\end{verbatim}

\begin{Shaded}
\begin{Highlighting}[]
\CommentTok{# Subset vektor pada urutan 2 dan 4}
\NormalTok{Jumlah[}\KeywordTok{c}\NormalTok{(}\DecValTok{2}\NormalTok{, }\DecValTok{4}\NormalTok{)]}
\end{Highlighting}
\end{Shaded}

\begin{verbatim}
## Jeruk Salak 
##    NA     7
\end{verbatim}

Selain melalui urutan (indeks), kita juga dapat melakukan subset
berdasarkan nama elemen vektornya.

\begin{Shaded}
\begin{Highlighting}[]
\NormalTok{Jumlah[}\StringTok{"Jeruk"}\NormalTok{]}
\end{Highlighting}
\end{Shaded}

\begin{verbatim}
## Jeruk 
##    NA
\end{verbatim}

\begin{quote}
\textbf{Note: } Indeks pada \texttt{R} dimulai dari 1. Sehingga kolom
atau elemen pertama vektor dimulai dari {[}1{]}
\end{quote}

\begin{itemize}
\tightlist
\item
  \textbf{Negative indexing}: mengecualikan (\emph{exclude}) elemen
  vektor.
\end{itemize}

\begin{Shaded}
\begin{Highlighting}[]
\CommentTok{# mengecualikan elemen vektor 2 dan 4}
\NormalTok{Jumlah[}\OperatorTok{-}\KeywordTok{c}\NormalTok{(}\DecValTok{2}\NormalTok{,}\DecValTok{4}\NormalTok{)]}
\end{Highlighting}
\end{Shaded}

\begin{verbatim}
##     Apel Rambutan 
##        5        6
\end{verbatim}

\begin{Shaded}
\begin{Highlighting}[]
\CommentTok{# mengecualikan elemen vektor 1 sampai 3}
\NormalTok{Jumlah[}\OperatorTok{-}\KeywordTok{c}\NormalTok{(}\DecValTok{1}\OperatorTok{:}\DecValTok{3}\NormalTok{)]}
\end{Highlighting}
\end{Shaded}

\begin{verbatim}
## Salak 
##     7
\end{verbatim}

\begin{itemize}
\tightlist
\item
  \textbf{Subset berdasarkan vektor logical}: Hanya, elemen-elemen yang
  nilai yang bersesuaian dalam vektor pemilihan bernilai TRUE, akan
  disimpan dalam subset.
\end{itemize}

\begin{quote}
\textbf{Note: } panjang vektor yang digunakan untuk subset harus sama.
\end{quote}

\begin{Shaded}
\begin{Highlighting}[]
\NormalTok{Jumlah <-}\StringTok{ }\KeywordTok{c}\NormalTok{(}\DataTypeTok{Apel=}\DecValTok{5}\NormalTok{, }\DataTypeTok{Jeruk=}\OtherTok{NA}\NormalTok{, }\DataTypeTok{Rambutan=}\DecValTok{6}\NormalTok{, }\DataTypeTok{Salak=}\DecValTok{7}\NormalTok{)}

\CommentTok{# selecting vector}
\NormalTok{merah <-}\StringTok{ }\KeywordTok{c}\NormalTok{(}\OtherTok{TRUE}\NormalTok{, }\OtherTok{FALSE}\NormalTok{, }\OtherTok{TRUE}\NormalTok{, }\OtherTok{FALSE}\NormalTok{)}

\CommentTok{# Subset}
\NormalTok{Jumlah[merah}\OperatorTok{==}\OtherTok{TRUE}\NormalTok{]}
\end{Highlighting}
\end{Shaded}

\begin{verbatim}
##     Apel Rambutan 
##        5        6
\end{verbatim}

\begin{Shaded}
\begin{Highlighting}[]
\CommentTok{# Subset untuk elemen vektor bukan missing value}
\NormalTok{Jumlah[}\OperatorTok{!}\KeywordTok{is.na}\NormalTok{(Jumlah)]}
\end{Highlighting}
\end{Shaded}

\begin{verbatim}
##     Apel Rambutan    Salak 
##        5        6        7
\end{verbatim}

\subsection{Perhitungan Menggunakan
Vektor}\label{perhitungan-menggunakan-vektor}

Jika pembaca melakukan operasi dengan vektor, operasi akan diterapkan ke
setiap elemen vektor. Contoh disediakan pada sintaks di bawah ini:

\begin{Shaded}
\begin{Highlighting}[]
\NormalTok{pendapatan <-}\StringTok{ }\KeywordTok{c}\NormalTok{(}\DecValTok{2000}\NormalTok{, }\DecValTok{1800}\NormalTok{, }\DecValTok{2500}\NormalTok{, }\DecValTok{3000}\NormalTok{)}
\KeywordTok{names}\NormalTok{(pendapatan) <-}\StringTok{ }\KeywordTok{c}\NormalTok{(}\StringTok{"Andi"}\NormalTok{, }\StringTok{"Joni"}\NormalTok{, }\StringTok{"Lina"}\NormalTok{, }\StringTok{"Rani"}\NormalTok{)}
\NormalTok{pendapatan}
\end{Highlighting}
\end{Shaded}

\begin{verbatim}
## Andi Joni Lina Rani 
## 2000 1800 2500 3000
\end{verbatim}

\begin{Shaded}
\begin{Highlighting}[]
\CommentTok{# Kalikan pendapatan dengan 3}
\NormalTok{pendapatan}\OperatorTok{*}\DecValTok{3}
\end{Highlighting}
\end{Shaded}

\begin{verbatim}
## Andi Joni Lina Rani 
## 6000 5400 7500 9000
\end{verbatim}

Seperti yang dapat dilihat, \texttt{R} mengalikan setiap elemen dengan
bilangan pengali.

Kita juga dapat mengalikan vektor dengan vektor lainnya.Contohnya
disajikan pada sintaks berikut:

\begin{Shaded}
\begin{Highlighting}[]
\CommentTok{# membuat vektor dengan panjang sama dengan dengan vektor pendapatan}
\NormalTok{coefs <-}\StringTok{ }\KeywordTok{c}\NormalTok{(}\DecValTok{2}\NormalTok{, }\FloatTok{1.5}\NormalTok{, }\DecValTok{1}\NormalTok{, }\DecValTok{3}\NormalTok{)}

\CommentTok{# Mengalikan pendapatan dengan vektor coefs}
\NormalTok{pendapatan}\OperatorTok{*}\NormalTok{coefs}
\end{Highlighting}
\end{Shaded}

\begin{verbatim}
## Andi Joni Lina Rani 
## 4000 2700 2500 9000
\end{verbatim}

Berdasarkan sintaks tersebut dapat terlihat bahwa operasi matematik
terhadap masing-masing vektor dapat berlangsung jika panjang vektornya
sama.

Berikut adalah fungsi lain yang dapat digunakan pada operasi matematika
vektor.

\begin{Shaded}
\begin{Highlighting}[]
\KeywordTok{max}\NormalTok{(x) }\CommentTok{# memperoleh nilai maksimum x}
\KeywordTok{min}\NormalTok{(x) }\CommentTok{# memperoleh nilai minimum x}
\KeywordTok{range}\NormalTok{(x) }\CommentTok{# memperoleh range vektor x}
\KeywordTok{length}\NormalTok{(x) }\CommentTok{# memperoleh jumlah elemen vektor x}
\KeywordTok{sum}\NormalTok{(x) }\CommentTok{# memperoleh total penjumlahan elemen vektor x}
\KeywordTok{prod}\NormalTok{(x) }\CommentTok{# memeperoleh produk elemen vektor x}
\KeywordTok{mean}\NormalTok{(x) }\CommentTok{# memperoleh nilai rata-rata seluruh elemen vektor x}
\KeywordTok{sd}\NormalTok{(x) }\CommentTok{# standar deviasi vektor x}
\KeywordTok{var}\NormalTok{(x) }\CommentTok{# varian vektor x}
\KeywordTok{sort}\NormalTok{(x) }\CommentTok{# mengurutkan elemen vektor x dari yang terbesar}
\end{Highlighting}
\end{Shaded}

Contoh penggunaan fungsi tersebut disajikan beberapa pada sintaks
berikut:

\begin{Shaded}
\begin{Highlighting}[]
\CommentTok{# Menghitung range pendapatan}
\KeywordTok{range}\NormalTok{(pendapatan)}
\end{Highlighting}
\end{Shaded}

\begin{verbatim}
## [1] 1800 3000
\end{verbatim}

\begin{Shaded}
\begin{Highlighting}[]
\CommentTok{# menghitung rata-rata dan standar deviasi pendapatan}
\KeywordTok{mean}\NormalTok{(pendapatan)}
\end{Highlighting}
\end{Shaded}

\begin{verbatim}
## [1] 2325
\end{verbatim}

\begin{Shaded}
\begin{Highlighting}[]
\KeywordTok{sd}\NormalTok{(pendapatan)}
\end{Highlighting}
\end{Shaded}

\begin{verbatim}
## [1] 537.7
\end{verbatim}

\section{Matriks}\label{matriks}

Matriks seperti Excel sheet yang berisi banyak baris dan kolom (kumpulan
bebrapa vektor). Matriks digunakan untuk menggabungkan vektor dengan
tipe yang sama, yang bisa berupa numerik, karakter, atau logis. Matriks
digunakan untuk menyimpan tabel data dalam R. Baris-baris matriks pada
umumnya adalah individu / pengamatan dan kolom adalah variabel.

\subsection{Membuat matriks}\label{membuat-matriks}

Untuk membuat matriks kita dapat menggunakan fungsi \texttt{cbind()}
atau \texttt{rbind()}. Berikut adalah contoh sintaks untuk membuat
matriks.

\begin{Shaded}
\begin{Highlighting}[]
\CommentTok{# membuat vektor numerik}
\NormalTok{col1 <-}\StringTok{ }\KeywordTok{c}\NormalTok{(}\DecValTok{5}\NormalTok{, }\DecValTok{6}\NormalTok{, }\DecValTok{7}\NormalTok{, }\DecValTok{8}\NormalTok{, }\DecValTok{9}\NormalTok{)}
\NormalTok{col2 <-}\StringTok{ }\KeywordTok{c}\NormalTok{(}\DecValTok{2}\NormalTok{, }\DecValTok{4}\NormalTok{, }\DecValTok{5}\NormalTok{, }\DecValTok{9}\NormalTok{, }\DecValTok{8}\NormalTok{)}
\NormalTok{col3 <-}\StringTok{ }\KeywordTok{c}\NormalTok{(}\DecValTok{7}\NormalTok{, }\DecValTok{3}\NormalTok{, }\DecValTok{4}\NormalTok{, }\DecValTok{8}\NormalTok{, }\DecValTok{7}\NormalTok{)}

\CommentTok{# menggabungkan vektor berdasarkan kolom}
\NormalTok{my_data <-}\StringTok{ }\KeywordTok{cbind}\NormalTok{(col1, col2, col3)}
\NormalTok{my_data}
\end{Highlighting}
\end{Shaded}

\begin{verbatim}
##      col1 col2 col3
## [1,]    5    2    7
## [2,]    6    4    3
## [3,]    7    5    4
## [4,]    8    9    8
## [5,]    9    8    7
\end{verbatim}

\begin{Shaded}
\begin{Highlighting}[]
\CommentTok{# Mengubah atau menambahkan nama baris}
\KeywordTok{rownames}\NormalTok{(my_data) <-}\StringTok{ }\KeywordTok{c}\NormalTok{(}\StringTok{"row1"}\NormalTok{, }\StringTok{"row2"}\NormalTok{, }\StringTok{"row3"}\NormalTok{, }\StringTok{"row4"}\NormalTok{, }\StringTok{"row5"}\NormalTok{)}
\NormalTok{my_data}
\end{Highlighting}
\end{Shaded}

\begin{verbatim}
##      col1 col2 col3
## row1    5    2    7
## row2    6    4    3
## row3    7    5    4
## row4    8    9    8
## row5    9    8    7
\end{verbatim}

\begin{quote}
\textbf{Note: }

\begin{itemize}
\tightlist
\item
  \textbf{cbind()}: menggabungkan objek \texttt{R} berdasarkan kolom
\item
  \textbf{rbind()}: menggabungkan objek \texttt{R} berdasarkan baris
\item
  \textbf{rownames()}: mengambil atau menetapkan nama-nama baris dari
  objek seperti-matriks
\item
  \textbf{colnames()}: mengambil atau menetapkan nama-nama kolom dari
  objek seperti-matriks
\end{itemize}
\end{quote}

Kita dapat melakukan tranpose (merotasi matriks sehingga kolom menjadi
baris dan sebaliknya) menggunakan fungsi \texttt{t()}. Berikut adalah
contoh penerapannya:

\begin{Shaded}
\begin{Highlighting}[]
\KeywordTok{t}\NormalTok{(my_data)}
\end{Highlighting}
\end{Shaded}

\begin{verbatim}
##      row1 row2 row3 row4 row5
## col1    5    6    7    8    9
## col2    2    4    5    9    8
## col3    7    3    4    8    7
\end{verbatim}

Selain melalui pembentukan sejumlah objek vektor, kita juga dapat
membuat matriks menggunakan fungsi \texttt{matrix()}. Secara sederhana
fungsi tersebut dapat dituliskan sebagai berikut:

\begin{Shaded}
\begin{Highlighting}[]
\KeywordTok{matrix}\NormalTok{(}\DataTypeTok{data =} \OtherTok{NA}\NormalTok{, }\DataTypeTok{nrow =} \DecValTok{1}\NormalTok{, }\DataTypeTok{ncol =} \DecValTok{1}\NormalTok{, }\DataTypeTok{byrow =} \OtherTok{FALSE}\NormalTok{,}
       \DataTypeTok{dimnames =} \OtherTok{NULL}\NormalTok{)}
\end{Highlighting}
\end{Shaded}

\begin{quote}
\textbf{Note: }

\begin{itemize}
\tightlist
\item
  \textbf{data}: vektor data opsional
\item
  \textbf{nrow}, \textbf{ncol}: jumlah baris dan kolom yang diinginkan,
  masing-masing.
\item
  \textbf{byrow}: nilai logis. Jika FALSE (default) matriks diisi oleh
  kolom, jika tidak, matriks diisi oleh baris.
\item
  \textbf{dimnames}: Daftar dua vektor yang memberikan nama baris dan
  kolom masing-masing.
\end{itemize}
\end{quote}

Dalam kode \texttt{R} di bawah ini, data input memiliki panjang 6. Kita
ingin membuat matriks dengan dua kolom. Kita tidak perlu menentukan
jumlah baris (di sini \texttt{nrow\ =\ 3}). \texttt{R} akan menyimpulkan
ini secara otomatis. Matriks diisi kolom demi kolom saat argumen
\texttt{byrow\ =\ FALSE}. Jika kita ingin mengisi matriks dengan baris,
gunakan \texttt{byrow\ =\ TRUE}. Berikut adalah contoh pembuatan matriks
menggunakan fungsi \texttt{matrix()}.

\begin{Shaded}
\begin{Highlighting}[]
\NormalTok{data <-}\StringTok{ }\KeywordTok{matrix}\NormalTok{(}
           \DataTypeTok{data =} \KeywordTok{c}\NormalTok{(}\DecValTok{1}\NormalTok{,}\DecValTok{2}\NormalTok{,}\DecValTok{3}\NormalTok{, }\DecValTok{11}\NormalTok{,}\DecValTok{12}\NormalTok{,}\DecValTok{13}\NormalTok{), }
           \DataTypeTok{nrow =} \DecValTok{2}\NormalTok{, }\DataTypeTok{byrow =} \OtherTok{TRUE}\NormalTok{,}
           \DataTypeTok{dimnames =} \KeywordTok{list}\NormalTok{(}\KeywordTok{c}\NormalTok{(}\StringTok{"row1"}\NormalTok{, }\StringTok{"row2"}\NormalTok{), }\KeywordTok{c}\NormalTok{(}\StringTok{"C.1"}\NormalTok{, }\StringTok{"C.2"}\NormalTok{, }\StringTok{"C.3"}\NormalTok{))}
\NormalTok{           )}
\NormalTok{data}
\end{Highlighting}
\end{Shaded}

\begin{verbatim}
##      C.1 C.2 C.3
## row1   1   2   3
## row2  11  12  13
\end{verbatim}

Untuk mengetahui dimensi dari suatu matriks, kita dapat menggunakan
fungsi \texttt{ncol()} untuk mengetahui jumlah kolom matriks dan
\texttt{nrow()} untuk mengetahui jumlah baris pada matriks. Berikut
adalah contoh penerapannya:

\begin{Shaded}
\begin{Highlighting}[]
\CommentTok{# mengetahui jumlah kolom}
\KeywordTok{ncol}\NormalTok{(my_data)}
\end{Highlighting}
\end{Shaded}

\begin{verbatim}
## [1] 3
\end{verbatim}

\begin{Shaded}
\begin{Highlighting}[]
\CommentTok{# mengetahui jumlah baris}
\KeywordTok{nrow}\NormalTok{(my_data)}
\end{Highlighting}
\end{Shaded}

\begin{verbatim}
## [1] 5
\end{verbatim}

Jika ingin memperoleh ringkasan terkait dimensi matriks kita juga dapat
mengunakan fungsi \texttt{dim()} untuk mengetahui jumlah baris dan kolom
matriks. Berikut adalah contoh penerapannya:

\begin{Shaded}
\begin{Highlighting}[]
\KeywordTok{dim}\NormalTok{(my_data) }\CommentTok{# jumlah baris dan kolom}
\end{Highlighting}
\end{Shaded}

\begin{verbatim}
## [1] 5 3
\end{verbatim}

\subsection{Subset Pada Matriks}\label{subset-pada-matriks}

Sama dengan vektor, subset juga dapat dilakukan pada matriks. Bedanya
subset dilakukan berdasarkan baris dan kolom pada matriks.

\begin{itemize}
\tightlist
\item
  \textbf{Memilih baris/kolom} berdasarkan pengindeksan positif
\end{itemize}

baris atau kolom dapat diseleksi menggunakan format
\texttt{data{[}row,\ col{]}}. Cara selesi ini sama dengan vektor,
bedanya kita harus menetukan baris dan kolom dari data yang akan kita
pilih. Berikut adalah contoh penerapannya:

\begin{Shaded}
\begin{Highlighting}[]
\CommentTok{# Pilih baris ke-2}
\NormalTok{my_data[}\DecValTok{2}\NormalTok{,]}
\end{Highlighting}
\end{Shaded}

\begin{verbatim}
## col1 col2 col3 
##    6    4    3
\end{verbatim}

\begin{Shaded}
\begin{Highlighting}[]
\CommentTok{# Pilih baris 2 sampai 4}
\NormalTok{my_data[}\DecValTok{2}\OperatorTok{:}\DecValTok{4}\NormalTok{,]}
\end{Highlighting}
\end{Shaded}

\begin{verbatim}
##      col1 col2 col3
## row2    6    4    3
## row3    7    5    4
## row4    8    9    8
\end{verbatim}

\begin{Shaded}
\begin{Highlighting}[]
\CommentTok{# Pilih baris 2 dan 4}
\NormalTok{my_data[}\KeywordTok{c}\NormalTok{(}\DecValTok{2}\NormalTok{,}\DecValTok{4}\NormalTok{),]}
\end{Highlighting}
\end{Shaded}

\begin{verbatim}
##      col1 col2 col3
## row2    6    4    3
## row4    8    9    8
\end{verbatim}

\begin{Shaded}
\begin{Highlighting}[]
\CommentTok{# Pilih baris 2 dan kolom 3}
\NormalTok{my_data[}\DecValTok{2}\NormalTok{, }\DecValTok{3}\NormalTok{]}
\end{Highlighting}
\end{Shaded}

\begin{verbatim}
## [1] 3
\end{verbatim}

\begin{itemize}
\tightlist
\item
  \textbf{Pilih berdasarkan nama baris/kolom}
\end{itemize}

Berikut adalah contoh subset berdasarkan nama baris atau kolom.

\begin{Shaded}
\begin{Highlighting}[]
\CommentTok{# Pilih baris 1 dan kolom 3}
\NormalTok{my_data[}\StringTok{"row1"}\NormalTok{,}\StringTok{"col3"}\NormalTok{]}
\end{Highlighting}
\end{Shaded}

\begin{verbatim}
## [1] 7
\end{verbatim}

\begin{Shaded}
\begin{Highlighting}[]
\CommentTok{# Pilih baris 1 sampai 4 dan kolom 3}
\NormalTok{baris <-}\StringTok{ }\KeywordTok{c}\NormalTok{(}\StringTok{"row1"}\NormalTok{,}\StringTok{"row2"}\NormalTok{,}\StringTok{"row3"}\NormalTok{)}
\NormalTok{my_data[baris, }\StringTok{"col3"}\NormalTok{]}
\end{Highlighting}
\end{Shaded}

\begin{verbatim}
## row1 row2 row3 
##    7    3    4
\end{verbatim}

\begin{itemize}
\tightlist
\item
  \textbf{Kecualikan baris/kolom} dengan pengindeksan negatif
\end{itemize}

Sama seperti vektor pengecualian data dapat dilakukan di matriks
menggunakan pengindeksan negatif. Berikut cara melakukannya:

\begin{Shaded}
\begin{Highlighting}[]
\CommentTok{# Kecualikan baris 2 dan 3 serta kolom 3}
\NormalTok{my_data[}\OperatorTok{-}\KeywordTok{c}\NormalTok{(}\DecValTok{2}\NormalTok{,}\DecValTok{3}\NormalTok{), }\OperatorTok{-}\DecValTok{3}\NormalTok{]}
\end{Highlighting}
\end{Shaded}

\begin{verbatim}
##      col1 col2
## row1    5    2
## row4    8    9
## row5    9    8
\end{verbatim}

\begin{itemize}
\tightlist
\item
  \textbf{Pilihan dengan logik}
\end{itemize}

Dalam kode \texttt{R} di bawah ini, misalkan kita ingin hanya menyimpan
baris di mana col3\textgreater{} = 4:

\begin{Shaded}
\begin{Highlighting}[]
\NormalTok{col3 <-}\StringTok{ }\NormalTok{my_data[, }\StringTok{"col3"}\NormalTok{]}
\NormalTok{my_data[col3 }\OperatorTok{>=}\StringTok{ }\DecValTok{4}\NormalTok{, ]}
\end{Highlighting}
\end{Shaded}

\begin{verbatim}
##      col1 col2 col3
## row1    5    2    7
## row3    7    5    4
## row4    8    9    8
## row5    9    8    7
\end{verbatim}

\subsection{Perhitungan Menggunakan
Matriks}\label{perhitungan-menggunakan-matriks}

\_ Kita juga dapat melakukan operasi matematika pada matriks. Pada
operasi matematika pada matriks proses yang terjadi bisa lebih kompleks
dibanding pada vektor, dimana kita dapat melakukan operasi untuk
memperoleh gambaran data pada tiap kolom atau baris.

Berikut adalah contoh operasi matematika sederhana pada matriks:

\begin{Shaded}
\begin{Highlighting}[]
\CommentTok{# mengalikan masing-masing elemen matriks dengan 2}
\NormalTok{my_data}\OperatorTok{*}\DecValTok{2}
\end{Highlighting}
\end{Shaded}

\begin{verbatim}
##      col1 col2 col3
## row1   10    4   14
## row2   12    8    6
## row3   14   10    8
## row4   16   18   16
## row5   18   16   14
\end{verbatim}

\begin{Shaded}
\begin{Highlighting}[]
\CommentTok{# memperoleh nilai log basis 2 pada masing-masing elemen matriks}
\KeywordTok{log2}\NormalTok{(my_data)}
\end{Highlighting}
\end{Shaded}

\begin{verbatim}
##       col1  col2  col3
## row1 2.322 1.000 2.807
## row2 2.585 2.000 1.585
## row3 2.807 2.322 2.000
## row4 3.000 3.170 3.000
## row5 3.170 3.000 2.807
\end{verbatim}

Seperti yang telah penulis jelaskan sebelumnya, kita juga dapat
melakukan operasi matematika untuk memperoleh hasil penjumlahan elemen
pada tiap baris atau kolom dengan menggunakan fungsi \texttt{rowSums()}
untuk baris dan \texttt{colSums()} untuk kolom.

\begin{Shaded}
\begin{Highlighting}[]
\CommentTok{# Total pada tiap kolom}
\KeywordTok{colSums}\NormalTok{(my_data)}
\end{Highlighting}
\end{Shaded}

\begin{verbatim}
## col1 col2 col3 
##   35   28   29
\end{verbatim}

\begin{Shaded}
\begin{Highlighting}[]
\CommentTok{# Total pada tiap baris}
\KeywordTok{rowSums}\NormalTok{(my_data)}
\end{Highlighting}
\end{Shaded}

\begin{verbatim}
## row1 row2 row3 row4 row5 
##   14   13   16   25   24
\end{verbatim}

Jika kita tertarik untuk mencari nilai rata-rata tiap baris arau kolom
kita juga dapat menggunakan fungsi \texttt{rowMeans()} atau
\texttt{colMeans()}. Berikut adalah contoh penerapannya:

\begin{Shaded}
\begin{Highlighting}[]
\CommentTok{# Rata-rata tiap baris}
\KeywordTok{rowMeans}\NormalTok{(my_data)}
\end{Highlighting}
\end{Shaded}

\begin{verbatim}
##  row1  row2  row3  row4  row5 
## 4.667 4.333 5.333 8.333 8.000
\end{verbatim}

\begin{Shaded}
\begin{Highlighting}[]
\CommentTok{# Rata-rata tiap kolom}
\KeywordTok{colMeans}\NormalTok{(my_data)}
\end{Highlighting}
\end{Shaded}

\begin{verbatim}
## col1 col2 col3 
##  7.0  5.6  5.8
\end{verbatim}

Kita juga dapat melakukan perhitungan statistika lainnya menggunakan
fungsi \texttt{apply()}. Berikut adalah format sederhananya:

\begin{Shaded}
\begin{Highlighting}[]
\KeywordTok{apply}\NormalTok{(x, MARGIN, FUN)}
\end{Highlighting}
\end{Shaded}

\begin{quote}
\textbf{Note: }

\begin{itemize}
\tightlist
\item
  x : data matriks
\item
  MARGIN : Nilai yang dapat digunakan adalah 1 (untuk operasi pada
  baris) dan 2 (untuk operasi pada kolom)
\item
  FUN : fungsi yang diterapkan pada baris atau kolom
\end{itemize}
\end{quote}

untuk mengetahui fungsi (\texttt{FUN}) apa saja yang dapat diterapkan
pada fungsi \texttt{apply()} jalankan sintaks bantuan berikut:

\begin{Shaded}
\begin{Highlighting}[]
\KeywordTok{help}\NormalTok{(apply)}
\end{Highlighting}
\end{Shaded}

Berikut adalah contoh penerapannya:

\begin{Shaded}
\begin{Highlighting}[]
\CommentTok{# Rata-rata pada tiap baris}
\KeywordTok{apply}\NormalTok{(my_data, }\DecValTok{1}\NormalTok{, mean)}
\end{Highlighting}
\end{Shaded}

\begin{verbatim}
##  row1  row2  row3  row4  row5 
## 4.667 4.333 5.333 8.333 8.000
\end{verbatim}

\begin{Shaded}
\begin{Highlighting}[]
\CommentTok{# Median pada tiap kolom}
\KeywordTok{apply}\NormalTok{(my_data, }\DecValTok{2}\NormalTok{, median)}
\end{Highlighting}
\end{Shaded}

\begin{verbatim}
## col1 col2 col3 
##    7    5    7
\end{verbatim}

\section{Faktor}\label{faktor}

Dalam bahasa \texttt{R} , faktor merupakan verktor dengan level. Level
disimpan sebagai \texttt{R} Character. Jika kita menggunakan SPSS maka
factor ini akan sama dengan jenis data numerik atau ordinal.

Faktor merepresentasikan kategori atau grup pada data. Untuk membuat
faktor pada \texttt{R}, kita dapat menggunakan fungsi \texttt{factor()}.

\subsection{Membuat Variabel Faktor}\label{membuat-variabel-faktor}

Berikut adalah contoh sintaks pembuatan variabel faktor.

\begin{Shaded}
\begin{Highlighting}[]
\CommentTok{# membuat variabel faktor}
\NormalTok{faktor <-}\StringTok{ }\KeywordTok{factor}\NormalTok{(}\KeywordTok{c}\NormalTok{(}\DecValTok{1}\NormalTok{,}\DecValTok{2}\NormalTok{,}\DecValTok{1}\NormalTok{,}\DecValTok{2}\NormalTok{))}
\NormalTok{faktor}
\end{Highlighting}
\end{Shaded}

\begin{verbatim}
## [1] 1 2 1 2
## Levels: 1 2
\end{verbatim}

Pada sintaks tersebut objek faktor terdiri atas dua buah kategori atau
pada \texttt{R} disebut sebagai \textbf{factor levels}. Kita dapat
mengecek factor levels menggunakan fungsi \texttt{levels()}.

\begin{Shaded}
\begin{Highlighting}[]
\KeywordTok{levels}\NormalTok{(faktor)}
\end{Highlighting}
\end{Shaded}

\begin{verbatim}
## [1] "1" "2"
\end{verbatim}

Kita juga dapat memberikan label atau mengubah level pada faktor.
Berikut adalah contoh bagaimana kita melakukannya:

\begin{Shaded}
\begin{Highlighting}[]
\CommentTok{# Ubah level}
\KeywordTok{levels}\NormalTok{(faktor) <-}\StringTok{ }\KeywordTok{c}\NormalTok{(}\StringTok{"baik"}\NormalTok{,}\StringTok{"tidak_baik"}\NormalTok{)}
\NormalTok{faktor}
\end{Highlighting}
\end{Shaded}

\begin{verbatim}
## [1] baik       tidak_baik baik       tidak_baik
## Levels: baik tidak_baik
\end{verbatim}

\begin{Shaded}
\begin{Highlighting}[]
\CommentTok{# Ubah urutan level}
\NormalTok{faktor <-}\StringTok{ }\KeywordTok{factor}\NormalTok{(faktor,}
                 \DataTypeTok{levels =} \KeywordTok{c}\NormalTok{(}\StringTok{"tidak_baik"}\NormalTok{,}\StringTok{"baik"}\NormalTok{))}
\NormalTok{faktor}
\end{Highlighting}
\end{Shaded}

\begin{verbatim}
## [1] baik       tidak_baik baik       tidak_baik
## Levels: tidak_baik baik
\end{verbatim}

\begin{quote}
\textbf{Note: }

\begin{itemize}
\tightlist
\item
  Fungsi \texttt{is.factor()} dapat digunakan untuk mengecek apakah
  sebuah variabel adalah faktor. Hasil yang dimunculkan dapat berupa
  TRUE (jika faktor) atau FALSE (jika bukan)
\item
  Fungsi \texttt{as.factor()} dapat digunakan untuk merubah sebuah
  variabel menjadi faktor.
\end{itemize}
\end{quote}

\begin{Shaded}
\begin{Highlighting}[]
\CommentTok{# Cek jika objek faktor adalah faktor}
\KeywordTok{is.factor}\NormalTok{(faktor)}
\end{Highlighting}
\end{Shaded}

\begin{verbatim}
## [1] TRUE
\end{verbatim}

\begin{Shaded}
\begin{Highlighting}[]
\CommentTok{# Cek jika objek Jumlah adalah faktor}
\KeywordTok{is.factor}\NormalTok{(Jumlah)}
\end{Highlighting}
\end{Shaded}

\begin{verbatim}
## [1] FALSE
\end{verbatim}

\begin{Shaded}
\begin{Highlighting}[]
\CommentTok{# Ubah objek Jumlah menjadi faktor}
\KeywordTok{as.factor}\NormalTok{(Jumlah)}
\end{Highlighting}
\end{Shaded}

\begin{verbatim}
##     Apel    Jeruk Rambutan    Salak 
##        5     <NA>        6        7 
## Levels: 5 6 7
\end{verbatim}

\subsection{Perhitungan Menggunakan
Faktor}\label{perhitungan-menggunakan-faktor}

Jika kita ingin mengetahui jumlah masing-masing observasi pada
masing-masing faktor, kita dapat menggunakan fungsi \texttt{summary()}.
Berikut adalah contoh penerapannya:

\begin{Shaded}
\begin{Highlighting}[]
\KeywordTok{summary}\NormalTok{(faktor)}
\end{Highlighting}
\end{Shaded}

\begin{verbatim}
## tidak_baik       baik 
##          2          2
\end{verbatim}

Pada contoh perhitungan menggunakan vektor kita telah membuat objek
\texttt{pendapatan}. Pada objek tersebut kita ingin menghitung nilai
rata-rata pendapatan berdasarkan objek faktor. Untuk melakukannya kita
dapat menggunakan fungsi \texttt{tapply()}.

\begin{Shaded}
\begin{Highlighting}[]
\NormalTok{pendapatan}
\end{Highlighting}
\end{Shaded}

\begin{verbatim}
## Andi Joni Lina Rani 
## 2000 1800 2500 3000
\end{verbatim}

\begin{Shaded}
\begin{Highlighting}[]
\NormalTok{faktor}
\end{Highlighting}
\end{Shaded}

\begin{verbatim}
## [1] baik       tidak_baik baik       tidak_baik
## Levels: tidak_baik baik
\end{verbatim}

\begin{Shaded}
\begin{Highlighting}[]
\CommentTok{# Rata-rata pendapatan dan simpan sebagai objek dengan nama:}
\CommentTok{# mean_pendapatan}
\NormalTok{mean_pendapatan <-}\StringTok{ }\KeywordTok{tapply}\NormalTok{(pendapatan, faktor, mean)}
\NormalTok{mean_pendapatan}
\end{Highlighting}
\end{Shaded}

\begin{verbatim}
## tidak_baik       baik 
##       2400       2250
\end{verbatim}

\begin{Shaded}
\begin{Highlighting}[]
\CommentTok{# Hitung ukuran/panjang masing-masing grup}
\KeywordTok{tapply}\NormalTok{(pendapatan, faktor, length)}
\end{Highlighting}
\end{Shaded}

\begin{verbatim}
## tidak_baik       baik 
##          2          2
\end{verbatim}

Untuk mengetahui jumlah masing-masing observasi masing-masing factor
levels kita juga dapat menggunakan fungsi \texttt{table()}. Fungsi
tersebut akan membuat frekuensi tabel pada masing-masing factor levels
atau yang dikenal sebagai \emph{contingency table}.

\begin{Shaded}
\begin{Highlighting}[]
\KeywordTok{table}\NormalTok{(faktor)}
\end{Highlighting}
\end{Shaded}

\begin{verbatim}
## faktor
## tidak_baik       baik 
##          2          2
\end{verbatim}

\begin{Shaded}
\begin{Highlighting}[]
\CommentTok{# Cross-tabulation antara}
\CommentTok{# faktor dan pendapatan}
\KeywordTok{table}\NormalTok{(pendapatan, faktor)}
\end{Highlighting}
\end{Shaded}

\begin{verbatim}
##           faktor
## pendapatan tidak_baik baik
##       1800          1    0
##       2000          0    1
##       2500          0    1
##       3000          1    0
\end{verbatim}

\section{Data Frames}\label{data-frames}

Data frame merupakan kumpulan vektor dengan panjang sama atau dapat pula
dikatan sebagai matriks yang memiliki kolom dengan jenis data yang
berbeda-beda (numerik, karakter, logical). Pada data frame terdapat
baris dan kolom. Baris disebut sebagai observasi, sedangkan kolom
disebut sebagai variabel. Sehingga dapat dikatakan bahwa setiap
observasi akan memiliki satu atau beberapa variabel.

\subsection{Membuat Data Frame}\label{membuat-data-frame}

Data frame dapat dibuat menggunakan fungsi \texttt{data.frame()}.
Berikut adalah contoh cara membuat data frame:

\begin{Shaded}
\begin{Highlighting}[]
\CommentTok{# Membuat data frame}
\NormalTok{nama <-}\StringTok{ }\KeywordTok{c}\NormalTok{(}\StringTok{"Andi"}\NormalTok{,}\StringTok{"Rizal"}\NormalTok{,}\StringTok{"Ani"}\NormalTok{,}\StringTok{"Ina"}\NormalTok{)}
\NormalTok{pendapatan <-}\StringTok{ }\KeywordTok{c}\NormalTok{(}\DecValTok{1000}\NormalTok{, }\DecValTok{2000}\NormalTok{, }\DecValTok{3500}\NormalTok{, }\DecValTok{500}\NormalTok{)}
\NormalTok{tinggi <-}\StringTok{ }\KeywordTok{c}\NormalTok{(}\DecValTok{160}\NormalTok{, }\DecValTok{155}\NormalTok{, }\DecValTok{170}\NormalTok{, }\DecValTok{146}\NormalTok{)}
\NormalTok{usia <-}\StringTok{ }\KeywordTok{c}\NormalTok{(}\DecValTok{35}\NormalTok{, }\DecValTok{40}\NormalTok{, }\DecValTok{25}\NormalTok{, }\DecValTok{27}\NormalTok{)}
\NormalTok{menikah <-}\StringTok{ }\KeywordTok{c}\NormalTok{(}\OtherTok{TRUE}\NormalTok{, }\OtherTok{FALSE}\NormalTok{, }\OtherTok{TRUE}\NormalTok{, }\OtherTok{TRUE}\NormalTok{)}

\NormalTok{data_teman <-}\StringTok{ }\KeywordTok{data.frame}\NormalTok{(}\DataTypeTok{nama =}\NormalTok{ nama,}
                         \DataTypeTok{gaji =}\NormalTok{ pendapatan,}
                         \DataTypeTok{tinggi =}\NormalTok{ tinggi,}
                         \DataTypeTok{menikah =}\NormalTok{ menikah)}

\NormalTok{data_teman}
\end{Highlighting}
\end{Shaded}

\begin{verbatim}
##    nama gaji tinggi menikah
## 1  Andi 1000    160    TRUE
## 2 Rizal 2000    155   FALSE
## 3   Ani 3500    170    TRUE
## 4   Ina  500    146    TRUE
\end{verbatim}

Untuk mengecek apakah objek \texttt{data\_teman} merupakan data frame,
kita dapat menggunakan fungsi \texttt{is.data.frame()}. Jika hasilnya
TRUE, maka objek tersebut adalah data frame. Berikut adalah contoh
penerapannya:

\begin{Shaded}
\begin{Highlighting}[]
\KeywordTok{is.data.frame}\NormalTok{(data_teman)}
\end{Highlighting}
\end{Shaded}

\begin{verbatim}
## [1] TRUE
\end{verbatim}

\begin{quote}
\textbf{Note: } untuk konversi objek menjadi data frame, kita dapat
menjalankan fungsi \texttt{as.data.frame()}.
\end{quote}

\subsection{Subset Pada Data Frame}\label{subset-pada-data-frame}

Subset pada data frame sebenarnya tidak berbeda dengan subset pada
matriks. Bedanya adalah kita juga bisa melakukan subset langsung
terhadap nama variabel menggunakan dollar sign. Untuk lebih memahaminya
berikut adalah jenis subset pada data frame.

\begin{itemize}
\tightlist
\item
  \textbf{Pengindeksan positif} menggunakan nama dan lokasi.
\end{itemize}

\begin{Shaded}
\begin{Highlighting}[]
\CommentTok{# Subset menggunakan dollar sign}
\NormalTok{data_teman}\OperatorTok{$}\NormalTok{nama}
\end{Highlighting}
\end{Shaded}

\begin{verbatim}
## [1] Andi  Rizal Ani   Ina  
## Levels: Andi Ani Ina Rizal
\end{verbatim}

\begin{Shaded}
\begin{Highlighting}[]
\CommentTok{# atau }
\NormalTok{data_teman[, }\StringTok{"nama"}\NormalTok{]}
\end{Highlighting}
\end{Shaded}

\begin{verbatim}
## [1] Andi  Rizal Ani   Ina  
## Levels: Andi Ani Ina Rizal
\end{verbatim}

\begin{Shaded}
\begin{Highlighting}[]
\CommentTok{# subset baris 1 sampai 3 serta kolom 1 dan 3}
\NormalTok{data_teman[}\DecValTok{1}\OperatorTok{:}\DecValTok{3}\NormalTok{, }\KeywordTok{c}\NormalTok{(}\DecValTok{1}\NormalTok{,}\DecValTok{3}\NormalTok{)]}
\end{Highlighting}
\end{Shaded}

\begin{verbatim}
##    nama tinggi
## 1  Andi    160
## 2 Rizal    155
## 3   Ani    170
\end{verbatim}

\begin{itemize}
\tightlist
\item
  \textbf{Pengindeksan negatif}
\end{itemize}

\begin{Shaded}
\begin{Highlighting}[]
\CommentTok{# Kecualikan kolom nama}
\NormalTok{data_teman[,}\OperatorTok{-}\DecValTok{1}\NormalTok{]}
\end{Highlighting}
\end{Shaded}

\begin{verbatim}
##   gaji tinggi menikah
## 1 1000    160    TRUE
## 2 2000    155   FALSE
## 3 3500    170    TRUE
## 4  500    146    TRUE
\end{verbatim}

\begin{itemize}
\tightlist
\item
  \textbf{Pengideksan berdasarkan karakteristik}
\end{itemize}

Kita ingin memilih data dengan kriteria teman yang telah menikah

\begin{Shaded}
\begin{Highlighting}[]
\NormalTok{data_teman[data_teman}\OperatorTok{$}\NormalTok{menikah}\OperatorTok{==}\OtherTok{TRUE}\NormalTok{, ]}
\end{Highlighting}
\end{Shaded}

\begin{verbatim}
##   nama gaji tinggi menikah
## 1 Andi 1000    160    TRUE
## 3  Ani 3500    170    TRUE
## 4  Ina  500    146    TRUE
\end{verbatim}

\begin{Shaded}
\begin{Highlighting}[]
\CommentTok{# Tampilkan hanya kolom nama dan gaji untuk yang telah menikah}
\NormalTok{data_teman[data_teman}\OperatorTok{$}\NormalTok{menikah}\OperatorTok{==}\OtherTok{TRUE}\NormalTok{, }\DecValTok{1}\OperatorTok{:}\DecValTok{2}\NormalTok{]}
\end{Highlighting}
\end{Shaded}

\begin{verbatim}
##   nama gaji
## 1 Andi 1000
## 3  Ani 3500
## 4  Ina  500
\end{verbatim}

kita juga dapat menggunakan fungsi \texttt{subset()} agar lebih mudah.
Berikut adalah contoh penerapannya:

\begin{Shaded}
\begin{Highlighting}[]
\CommentTok{# subset terhadap teman yang berusia >=30 tahun}
\KeywordTok{subset}\NormalTok{(data_teman, usia}\OperatorTok{>=}\DecValTok{30}\NormalTok{)}
\end{Highlighting}
\end{Shaded}

\begin{verbatim}
##    nama gaji tinggi menikah
## 1  Andi 1000    160    TRUE
## 2 Rizal 2000    155   FALSE
\end{verbatim}

Opsi lain adalah menggunakan fungsi \texttt{attach()} dan
\texttt{detach()}. Fungsi \texttt{attach()} mengambil data frame dan
membuat kolomnya dapat diakses hanya dengan memberikan nama mereka.

\begin{Shaded}
\begin{Highlighting}[]
\CommentTok{# attach data frame}
\KeywordTok{attach}\NormalTok{(data_teman)}
\end{Highlighting}
\end{Shaded}

\begin{verbatim}
## The following objects are masked _by_ .GlobalEnv:
## 
##     menikah, nama, tinggi
\end{verbatim}

\begin{Shaded}
\begin{Highlighting}[]
\CommentTok{# ==== memulai data manipulation ====}
\NormalTok{data_teman[usia}\OperatorTok{>=}\DecValTok{30}\NormalTok{]}
\end{Highlighting}
\end{Shaded}

\begin{verbatim}
##    nama gaji
## 1  Andi 1000
## 2 Rizal 2000
## 3   Ani 3500
## 4   Ina  500
\end{verbatim}

\begin{Shaded}
\begin{Highlighting}[]
\CommentTok{# ==== mengakhiri data manipulation ====}
\CommentTok{# detach data frame}

\KeywordTok{detach}\NormalTok{(data_teman)}
\end{Highlighting}
\end{Shaded}

\subsection{Memperluas Data Frame}\label{memperluas-data-frame}

Kita dapat juga memperluas data frame dengan cara menambahkan variabel
atau kolombaru pada data frame. Pada contoh kali ini penulis akan
menambahkan kolom pendidikan terakhir pada objek \texttt{data\_teman}.
Berikut adalah sintaks yang digunakan.

\begin{Shaded}
\begin{Highlighting}[]
\CommentTok{# membuat vektor pendidikan}
\NormalTok{pendidikan <-}\StringTok{ }\KeywordTok{c}\NormalTok{(}\StringTok{"S1"}\NormalTok{,}\StringTok{"S2"}\NormalTok{,}\StringTok{"D3"}\NormalTok{,}\StringTok{"D1"}\NormalTok{)}

\CommentTok{# menambahkan variabel pendidikan pada data frame}
\NormalTok{data_teman}\OperatorTok{$}\NormalTok{pendidikan <-}\StringTok{ }\NormalTok{pendidikan}
\end{Highlighting}
\end{Shaded}

\begin{Shaded}
\begin{Highlighting}[]
\CommentTok{# atau}
\KeywordTok{cbind}\NormalTok{(data_teman, }\DataTypeTok{pendidikan=}\NormalTok{pendidikan)}
\end{Highlighting}
\end{Shaded}

\subsection{Perhitungan Pada Data
Frame}\label{perhitungan-pada-data-frame}

Perhitungan pada variabel numerik data frame pada dasarnya sama dengan
perhitungan pada matriks. kita dapat menggunakan fungsi
\texttt{rowSums()}, \texttt{colSums()}, \texttt{rowMeans()} dan
\texttt{apply()}. Proses perhitungan dan manipulasi pada data frame akan
dibahas pada sesi yang lain secara lebih detail.

\section{List}\label{list}

List adalah kumpulan objek yang diurutkan, yang dapat berupa vektor,
matriks, data frame, dll. Dengan kata lain, daftar dapat berisi semua
jenis objek \texttt{R}.

\subsection{Membuat List}\label{membuat-list}

List dapat dibuat menggunakan fungsi \texttt{list()}. Berikut disajikan
contoh sebuah list sebuah keluarga:

\begin{Shaded}
\begin{Highlighting}[]
\CommentTok{# Membuat list keluarga}
\NormalTok{keluarga <-}\StringTok{ }\KeywordTok{list}\NormalTok{(}
  \DataTypeTok{ayah =} \StringTok{"Budi"}\NormalTok{,}
  \DataTypeTok{usia_ayah =} \DecValTok{48}\NormalTok{,}
  \DataTypeTok{ibu  =} \StringTok{"Ani"}\NormalTok{,}
  \DataTypeTok{usia_ibu =} \StringTok{"47"}\NormalTok{,}
  \DataTypeTok{anak =} \KeywordTok{c}\NormalTok{(}\StringTok{"Andi"}\NormalTok{, }\StringTok{"Adi"}\NormalTok{),}
  \DataTypeTok{usia_anak =} \KeywordTok{c}\NormalTok{(}\DecValTok{15}\NormalTok{,}\DecValTok{10}\NormalTok{)}
\NormalTok{  )}

\CommentTok{# Print}
\NormalTok{keluarga}
\end{Highlighting}
\end{Shaded}

\begin{verbatim}
## $ayah
## [1] "Budi"
## 
## $usia_ayah
## [1] 48
## 
## $ibu
## [1] "Ani"
## 
## $usia_ibu
## [1] "47"
## 
## $anak
## [1] "Andi" "Adi" 
## 
## $usia_anak
## [1] 15 10
\end{verbatim}

\begin{Shaded}
\begin{Highlighting}[]
\CommentTok{# Nama elemen dalam list}
\KeywordTok{names}\NormalTok{(keluarga)}
\end{Highlighting}
\end{Shaded}

\begin{verbatim}
## [1] "ayah"      "usia_ayah" "ibu"       "usia_ibu" 
## [5] "anak"      "usia_anak"
\end{verbatim}

\begin{Shaded}
\begin{Highlighting}[]
\CommentTok{# Jumlah elemen pada list}
\KeywordTok{length}\NormalTok{(keluarga)}
\end{Highlighting}
\end{Shaded}

\begin{verbatim}
## [1] 6
\end{verbatim}

\subsection{Subset List}\label{subset-list}

Kita dapat memilih sebuah elemen pada list dengan menggunakan nama
elemen atau indeks dari elemen tersebut. Berikut adalah contoh
penerapannya:

\begin{Shaded}
\begin{Highlighting}[]
\CommentTok{# Subset berdasarkan nama}
\CommentTok{# mengambil elemen usia_ayah}
\NormalTok{keluarga}\OperatorTok{$}\NormalTok{usia_ayah}
\end{Highlighting}
\end{Shaded}

\begin{verbatim}
## [1] 48
\end{verbatim}

\begin{Shaded}
\begin{Highlighting}[]
\CommentTok{# Atau}
\NormalTok{keluarga[[}\StringTok{"usia_ayah"}\NormalTok{]]}
\end{Highlighting}
\end{Shaded}

\begin{verbatim}
## [1] 48
\end{verbatim}

\begin{Shaded}
\begin{Highlighting}[]
\CommentTok{# Subset berdasarkan indeks}
\NormalTok{keluarga[[}\DecValTok{2}\NormalTok{]]}
\end{Highlighting}
\end{Shaded}

\begin{verbatim}
## [1] 48
\end{verbatim}

\begin{Shaded}
\begin{Highlighting}[]
\CommentTok{# subset elemen pertama pada keluarga[[5]]}
\NormalTok{keluarga[[}\DecValTok{5}\NormalTok{]][}\DecValTok{1}\NormalTok{]}
\end{Highlighting}
\end{Shaded}

\begin{verbatim}
## [1] "Andi"
\end{verbatim}

\subsection{Memperluas List}\label{memperluas-list}

Kita juga dapat menambahkan elemen pada list yang telah kita buat. Pada
contoh list sebelumnya penulis akan menambahkan elemen keluarga yang
lain seperti berikut:

\begin{Shaded}
\begin{Highlighting}[]
\CommentTok{# Menambahkan kakek dan nenek pada list}
\NormalTok{keluarga}\OperatorTok{$}\NormalTok{kakek <-}\StringTok{ "Suprapto"}
\NormalTok{keluarga}\OperatorTok{$}\NormalTok{nenek <-}\StringTok{ "Sri"}

\CommentTok{# Print}
\NormalTok{keluarga}
\end{Highlighting}
\end{Shaded}

\begin{verbatim}
## $ayah
## [1] "Budi"
## 
## $usia_ayah
## [1] 48
## 
## $ibu
## [1] "Ani"
## 
## $usia_ibu
## [1] "47"
## 
## $anak
## [1] "Andi" "Adi" 
## 
## $usia_anak
## [1] 15 10
## 
## $kakek
## [1] "Suprapto"
## 
## $nenek
## [1] "Sri"
\end{verbatim}

Kita juga dapat menggabungkan beberapa list menjadi satu. Berikut adalah
format sederhana bagaimana cara menggabungkan beberapa list menjadi
satu:

\begin{Shaded}
\begin{Highlighting}[]
\NormalTok{list_baru <-}\StringTok{ }\KeywordTok{c}\NormalTok{(list_a, list_b, list_c, ...)}
\end{Highlighting}
\end{Shaded}

\section{Loop}\label{loop}

\emph{Loop} merupakan kode program yang berulang-ulang. \emph{Loop}
berguna saat kita ingin melakukan sebuah perintah yang perlu dijalankan
berulang-ulang seperti melakukan perhitungan maupaun melakukan
visualisasi terhadap banyak variabel secara serentak. Hal ini tentu saja
membantu kita karena kita tidak perlu menulis sejumlah sintaks yang
berulang-ulang. Kita hanya perlu mengatur \emph{statement} berdasarkan
hasil yang kita harapkan.

Pada \texttt{R} bentuk \emph{loop} dapat bermacam-macam (``\emph{for
loop}'',``\emph{while loop}'', dll). \texttt{R} menyederhanakan bentuk
\emph{loop} ini dengan menyediakan sejumlah fungsi seperti
\texttt{apply()},\texttt{tapply()}, dll. Sehingga \texttt{loop} jarang
sekali muncul dalam kode \texttt{R}. Sehingga \texttt{R} sering disebut
sebagai \emph{loopless loop}.

Meski \emph{loop} jarang muncul bukan berarti kita tidak akan
melakukannya. Terkadang saat kita melakukan komputasi statistik atau
matematik dan belum terdapat paket yang mendukung proses tersebut,
sering kali kita akan membuat sintaks sendiri berdasarkan algoritma
metode tersebut. Pada algoritma tersebut sering pula terdapat
\emph{loop} yang diperlukan selama proses perhitungan. Secara sederhana
diagram umum loop ditampilkan pada Gambar \ref{fig:loop}

\begin{figure}

{\centering \includegraphics[width=0.4\linewidth]{skema_loop} 

}

\caption{Diagram umum loop (sumber: Primartha, 2018).}\label{fig:loop}
\end{figure}

\subsection{For Loop}\label{for-loop}

Mengulangi sebuah \emph{statement} atau sekelompok \emph{statement}
sebanyak nilai yang ditentukan di awal. Jadi operasi akan terus
dilakukan sampai dengan jumlah yang telah ditetapkan di awal atau dengan
kata lain tes kondisi (Jika jumlah pengulangan telah cukup) hanya akan
dilakukan di akhir. Secara sederhana bentuk dari \emph{for loop} dapat
dituliskan sebagai berikut:

\begin{Shaded}
\begin{Highlighting}[]
\ControlFlowTok{for}\NormalTok{ (value }\ControlFlowTok{in}\NormalTok{ vector)\{}
\NormalTok{  statements}
\NormalTok{\}}
\end{Highlighting}
\end{Shaded}

Berikut adalah contoh sintaks penerapan \emph{for loop}:

\begin{Shaded}
\begin{Highlighting}[]
\CommentTok{# Membuat vektor numerik}
\NormalTok{vektor <-}\StringTok{ }\KeywordTok{c}\NormalTok{(}\DecValTok{1}\OperatorTok{:}\DecValTok{5}\NormalTok{)}

\CommentTok{# loop }
\ControlFlowTok{for}\NormalTok{(i }\ControlFlowTok{in}\NormalTok{ vektor)\{}
  \KeywordTok{print}\NormalTok{(i)}
\NormalTok{\}}
\end{Highlighting}
\end{Shaded}

\begin{verbatim}
## [1] 1
## [1] 2
## [1] 3
## [1] 4
## [1] 5
\end{verbatim}

\emph{Loop} akan dimulai dari blok \emph{statement for} sampai dengan
\texttt{print(i)}. Berdasarkan \emph{loop} pada contoh tersebut,
\emph{loop} hanya dilakukan sebanyak 5 kali sesuai dengan jumlah vektor
yang ada.

\subsection{While Loop}\label{while-loop}

\emph{While loop} merupakan loop yang digunakan ketika kita telah
menetapkan \emph{stop condition} sebelumnya. Blok \emph{statement}/kode
yang sama akan terus dijalankan sampai \emph{stop condition} ini
tercapai. \emph{Stop condition} akan di cek sebelum melakukan proses
\emph{loop}. Berikut adalah pola dari \emph{while loop} dapat dituliskan
sebagai berikut:

\begin{Shaded}
\begin{Highlighting}[]
\ControlFlowTok{while}\NormalTok{ (test_expression)\{}
\NormalTok{  statement}
\NormalTok{\}}
\end{Highlighting}
\end{Shaded}

Berikut adalah contoh penerapan dari \emph{while loop}:

\begin{Shaded}
\begin{Highlighting}[]
\NormalTok{coba <-}\StringTok{ }\KeywordTok{c}\NormalTok{(}\StringTok{"Contoh"}\NormalTok{)}
\NormalTok{counter <-}\StringTok{ }\DecValTok{1}

\CommentTok{# loop}
\ControlFlowTok{while}\NormalTok{ (counter}\OperatorTok{<}\DecValTok{5}\NormalTok{)\{}
  \CommentTok{# print vektor}
  \KeywordTok{print}\NormalTok{(coba)}
  \CommentTok{# tambahkan nilai counter sehingga proses terus berlangsung sampai counter = 5 }
\NormalTok{  counter <-}\StringTok{ }\NormalTok{counter }\OperatorTok{+}\StringTok{ }\DecValTok{1}
\NormalTok{\}}
\end{Highlighting}
\end{Shaded}

\begin{verbatim}
## [1] "Contoh"
## [1] "Contoh"
## [1] "Contoh"
## [1] "Contoh"
\end{verbatim}

\emph{Loop} akan dimulai dari blok \emph{statement while} sampai dengan
\emph{counter} \textless{}- 1. \emph{Loop} hanya akan dilakukan
sepanjang nilai \emph{counter} \textless{} 5.

\subsection{Repeat Loop}\label{repeat-loop}

\emph{Repeat loop} akan menjalankan \emph{statement}/kode yang sama
berulang-ulang hingga \emph{stop condition} tercapai. Berikut adalah
pola dari \emph{repeat loop}.

\begin{Shaded}
\begin{Highlighting}[]
\ControlFlowTok{repeat}\NormalTok{ \{}
\NormalTok{  commands}
  \ControlFlowTok{if}\NormalTok{(condition)\{}
    \ControlFlowTok{break}
\NormalTok{  \}}
\NormalTok{\}}
\end{Highlighting}
\end{Shaded}

Berikut adalah contoh penerapan dari \emph{repeat loop}:

\begin{Shaded}
\begin{Highlighting}[]
\NormalTok{coba <-}\StringTok{ }\KeywordTok{c}\NormalTok{(}\StringTok{"contoh"}\NormalTok{)}
\NormalTok{counter <-}\StringTok{ }\DecValTok{1}
\ControlFlowTok{repeat}\NormalTok{ \{}
  \KeywordTok{print}\NormalTok{(coba)}
\NormalTok{  counter <-}\StringTok{ }\NormalTok{counter }\OperatorTok{+}\StringTok{ }\DecValTok{1}
  \ControlFlowTok{if}\NormalTok{(counter }\OperatorTok{<}\StringTok{ }\DecValTok{5}\NormalTok{)\{}
\ControlFlowTok{break}
\NormalTok{  \}}
\NormalTok{\}}
\end{Highlighting}
\end{Shaded}

\begin{verbatim}
## [1] "contoh"
\end{verbatim}

\emph{Loop} akan dimulai dari blok \emph{statement while} sampai dengan
\emph{break}. \emph{Loop} hanya akan dilakukan sepanjang nilai
\emph{counter} \textless{} 5. Hasil yang diperoleh berbeda dengan
\emph{while loop}, dimana kita memperoleh 4 buah kata ``contoh''. Hal
ini disebabkan karena \emph{repeat loop} melakukan pengecekan \emph{stop
condition} tidak di awal loop seperti \emph{while loop} sehingga
berapapun nilainya, selama nilainya sesuai dengan \emph{stop condition}
maka \emph{loop} akan dihentikan. Hal ini berbeda dengan \emph{while
loop} dimana proses dilakukan berulang-ulang sampai jumlahnya mendekati
\emph{stop condition}.

\subsection{Break}\label{break}

\emph{Break} sebenarnya bukan bagian dari \emph{loop}, namun sering
digunakan dalam \emph{loop}. \emph{Break} dapat digunakan pada
\emph{loop} manakala dirasa perlu, yaitu saat kondisi yang disyaratkan
pada \emph{break} tercapai.

Berikut adalah contoh penerapan \emph{break} pada beberapa jenis
\emph{loop}.

\begin{Shaded}
\begin{Highlighting}[]
\CommentTok{# for loop}
\NormalTok{a =}\StringTok{ }\KeywordTok{c}\NormalTok{(}\DecValTok{2}\NormalTok{,}\DecValTok{4}\NormalTok{,}\DecValTok{6}\NormalTok{,}\DecValTok{8}\NormalTok{,}\DecValTok{10}\NormalTok{,}\DecValTok{12}\NormalTok{,}\DecValTok{14}\NormalTok{)}
\ControlFlowTok{for}\NormalTok{(i }\ControlFlowTok{in}\NormalTok{ a)\{}
  \ControlFlowTok{if}\NormalTok{(i}\OperatorTok{>}\DecValTok{8}\NormalTok{)\{}
    \ControlFlowTok{break}
\NormalTok{  \}}
  \KeywordTok{print}\NormalTok{(i)}
\NormalTok{\}}
\end{Highlighting}
\end{Shaded}

\begin{verbatim}
## [1] 2
## [1] 4
## [1] 6
## [1] 8
\end{verbatim}

\begin{Shaded}
\begin{Highlighting}[]
\CommentTok{# while loop}
\NormalTok{a =}\StringTok{ }\DecValTok{2}
\NormalTok{b =}\StringTok{ }\DecValTok{4}
\ControlFlowTok{while}\NormalTok{(a}\OperatorTok{<}\DecValTok{7}\NormalTok{)\{}
  \KeywordTok{print}\NormalTok{(a)}
\NormalTok{  a =}\StringTok{ }\NormalTok{a }\OperatorTok{+}\DecValTok{1}
  \ControlFlowTok{if}\NormalTok{(b}\OperatorTok{+}\NormalTok{a}\OperatorTok{>}\DecValTok{10}\NormalTok{)\{}
    \ControlFlowTok{break}
\NormalTok{  \}}
\NormalTok{\}}
\end{Highlighting}
\end{Shaded}

\begin{verbatim}
## [1] 2
## [1] 3
## [1] 4
## [1] 5
## [1] 6
\end{verbatim}

\begin{Shaded}
\begin{Highlighting}[]
\CommentTok{# repeat loop}
\NormalTok{a =}\StringTok{ }\DecValTok{1}
\ControlFlowTok{repeat}\NormalTok{\{}
  \KeywordTok{print}\NormalTok{(a)}
\NormalTok{  a =}\StringTok{ }\NormalTok{a}\OperatorTok{+}\DecValTok{1}
  \ControlFlowTok{if}\NormalTok{(a}\OperatorTok{>}\DecValTok{6}\NormalTok{)\{}
    \ControlFlowTok{break}
\NormalTok{  \}}
\NormalTok{\}}
\end{Highlighting}
\end{Shaded}

\begin{verbatim}
## [1] 1
## [1] 2
## [1] 3
## [1] 4
## [1] 5
## [1] 6
\end{verbatim}

\section{Decision Making}\label{decision-making}

\emph{Decicion Making} atau sering disebut sebagai \emph{if then else
statement} merupakan bentuk percabagan yang digunakan manakala kita
ingin agar program dapat melakukan pengujian terhadap syarat kondisi
tertentu. Pada Tabel \ref{tab:percabangan} disajikan daftar percabangan
yang digunakan pada \texttt{R}.

\begin{longtable}[]{@{}ll@{}}
\caption{\label{tab:percabangan} Daftar percabangan pada
\texttt{R}.}\tabularnewline
\toprule
\begin{minipage}[b]{0.15\columnwidth}\raggedright\strut
\textbf{Statement}\strut
\end{minipage} & \begin{minipage}[b]{0.79\columnwidth}\raggedright\strut
\textbf{Keterangan}\strut
\end{minipage}\tabularnewline
\midrule
\endfirsthead
\toprule
\begin{minipage}[b]{0.15\columnwidth}\raggedright\strut
\textbf{Statement}\strut
\end{minipage} & \begin{minipage}[b]{0.79\columnwidth}\raggedright\strut
\textbf{Keterangan}\strut
\end{minipage}\tabularnewline
\midrule
\endhead
\begin{minipage}[t]{0.15\columnwidth}\raggedright\strut
\emph{if statement}\strut
\end{minipage} & \begin{minipage}[t]{0.79\columnwidth}\raggedright\strut
\emph{if statement} hanya terdiri atas sebuah ekspresi \emph{Boolean},
dan diikuti satu atau lebih \emph{statement}\strut
\end{minipage}\tabularnewline
\begin{minipage}[t]{0.15\columnwidth}\raggedright\strut
\emph{if\ldots{}else statement}\strut
\end{minipage} & \begin{minipage}[t]{0.79\columnwidth}\raggedright\strut
\emph{if else statement} terdiri atas beberapa buah ekspresi
\emph{Boolean}. Ekspressi \emph{Boolean} berikutnya akan dijalankan jika
ekspresi *Boolan sebelumnya bernilai FALSE\strut
\end{minipage}\tabularnewline
\begin{minipage}[t]{0.15\columnwidth}\raggedright\strut
\emph{switch statement}\strut
\end{minipage} & \begin{minipage}[t]{0.79\columnwidth}\raggedright\strut
\emph{switch statement} digunakan untuk mengevaluasi sebuah variabel
beberapa pilihan\strut
\end{minipage}\tabularnewline
\bottomrule
\end{longtable}

\subsection{if statement}\label{if-statement}

Pola \emph{if statement} disajikan pada Gambar \ref{fig:ifstatement}

\begin{figure}

{\centering \includegraphics[width=0.4\linewidth]{ifstatement} 

}

\caption{Diagram if statement (sumber: Primartha, 2018).}\label{fig:ifstatement}
\end{figure}

Berikut adalah contoh penerapan \emph{if statement}:

\begin{Shaded}
\begin{Highlighting}[]
\NormalTok{x <-}\StringTok{ }\KeywordTok{c}\NormalTok{(}\DecValTok{1}\OperatorTok{:}\DecValTok{5}\NormalTok{)}
\ControlFlowTok{if}\NormalTok{(}\KeywordTok{is.vector}\NormalTok{(x))\{}
  \KeywordTok{print}\NormalTok{(}\StringTok{"x adalah sebuah vector"}\NormalTok{)}
\NormalTok{\}}
\end{Highlighting}
\end{Shaded}

\begin{verbatim}
## [1] "x adalah sebuah vector"
\end{verbatim}

\subsection{if else statement}\label{if-else-statement}

Pola dari \emph{if else statement} disajikan pada Gambar
\ref{fig:ifelse}

\begin{figure}

{\centering \includegraphics[width=0.4\linewidth]{ifelse} 

}

\caption{Diagram if else statement (sumber: Primartha, 2018).}\label{fig:ifelse}
\end{figure}

Berikut adalah contoh penerapan \emph{if else statement}:

\begin{Shaded}
\begin{Highlighting}[]
\NormalTok{x <-}\StringTok{ }\KeywordTok{c}\NormalTok{(}\StringTok{"Andi"}\NormalTok{,}\StringTok{"Iwan"}\NormalTok{, }\StringTok{"Adi"}\NormalTok{)}
\ControlFlowTok{if}\NormalTok{(}\StringTok{"Rina"} \OperatorTok\StringTok{ }\NormalTok{x)\{}
  \KeywordTok{print}\NormalTok{(}\StringTok{"Rina ditemukan"}\NormalTok{)}
\NormalTok{\} }\ControlFlowTok{else} \ControlFlowTok{if}\NormalTok{(}\StringTok{"Adi"} \OperatorTok\StringTok{ }\NormalTok{x)\{}
  \KeywordTok{print}\NormalTok{(}\StringTok{"Adi ditemukan"}\NormalTok{)}
\NormalTok{\} }\ControlFlowTok{else}\NormalTok{\{}
  \KeywordTok{print}\NormalTok{(}\StringTok{"tidak ada yang ditemukan"}\NormalTok{)}
\NormalTok{\}}
\end{Highlighting}
\end{Shaded}

\begin{verbatim}
## [1] "Adi ditemukan"
\end{verbatim}

\subsection{switch statement}\label{switch-statement}

Pola dari \emph{switch statement} disajikan pada Gambar \ref{fig:switch}

\begin{figure}

{\centering \includegraphics[width=0.4\linewidth]{switch} 

}

\caption{Diagram switch statement (sumber: Primartha, 2018).}\label{fig:switch}
\end{figure}

Berikut adalah contoh penerapan \emph{switch statement}:

\begin{Shaded}
\begin{Highlighting}[]
\NormalTok{y =}\StringTok{ }\DecValTok{3}

\NormalTok{x =}\StringTok{ }\ControlFlowTok{switch}\NormalTok{(}
\NormalTok{  y,}
  \StringTok{"Selamat Pagi"}\NormalTok{,}
  \StringTok{"Selamat Siang"}\NormalTok{,}
  \StringTok{"Selamat Sore"}\NormalTok{,}
  \StringTok{"Selamat Malam"}
\NormalTok{)}

\KeywordTok{print}\NormalTok{(x)}
\end{Highlighting}
\end{Shaded}

\begin{verbatim}
## [1] "Selamat Sore"
\end{verbatim}

\section{Fungsi}\label{fungsi}

Fungsi merupakan sekumpulan instruksi atau \emph{statement} yang dapat
melakukan tugas khusus. Sebagai contoh fungsi perkalian untuk
menyelesaikan operasi perkalian, fungsi pemangkatan hanya untuk operasi
pemangkatan, dll.

Pada \texttt{R} terdapat 2 jenis fungsi, yaitu: \emph{build in fuction}
dan \emph{user define function}. \emph{build in fnction} merupakan
fungsi bawaan \texttt{R} saat pertama kita menginstall \texttt{R}.
Contohnya adalah \texttt{mean()}, \texttt{sum()}, \texttt{ls()},
\texttt{rm()}, dll. Sedangkan \emph{user define fuction} merupakan
fungsi-fungsi yang dibuat sendiri oleh pengguna.

Fungsi-fungsi buatan pengguna haruslah dideklarasikan (dibuat) terlebih
dahulu sebelum dapat dijalankan. Pola pembentukan fungsi adalah sebagai
berikut:

\begin{Shaded}
\begin{Highlighting}[]
\NormalTok{function_name <-}\StringTok{ }\ControlFlowTok{function}\NormalTok{(argument_}\DecValTok{1}\NormalTok{, argument_}\DecValTok{2}\NormalTok{, ...)\{}
  \ControlFlowTok{function}\NormalTok{ body}
\NormalTok{\}}
\end{Highlighting}
\end{Shaded}

\begin{quote}
\textbf{Note: }

\begin{itemize}
\tightlist
\item
  \textbf{function\_name} : Nama dari fungsi \texttt{R}. \texttt{R} akan
  menyimpan fungsi tersebut sebagai objek
\item
  \textbf{argument\_1, argument\_2,\ldots{}} : \emph{Argument} bersifat
  opsional (tidak wajib). \emph{Argument} dapat digunakan untuk memberi
  inputan kepada fungsi
\item
  \textbf{function body} : Merupakan inti dari fungsi. Fuction body
  dapat terdiri atas 0 statement (kosong) hingga banyak statement.
\item
  \textbf{return} : Fungsi ada yang memiliki \emph{output} atau
  \emph{return value} ada juga yang tidak. Jika fungsi memiliki
  \emph{return value} maka \emph{return value} dapat diproses lebih
  lanjut
\end{itemize}
\end{quote}

Berikut adalah contoh penerapan \emph{user define function}:

\begin{Shaded}
\begin{Highlighting}[]
\CommentTok{# Fungsi tanpa argument}
\NormalTok{bilang <-}\StringTok{ }\ControlFlowTok{function}\NormalTok{()\{}
  \KeywordTok{print}\NormalTok{(}\StringTok{"Hello World!!"}\NormalTok{)}
\NormalTok{\}}

\CommentTok{# Print}
\KeywordTok{bilang}\NormalTok{()}
\end{Highlighting}
\end{Shaded}

\begin{verbatim}
## [1] "Hello World!!"
\end{verbatim}

\begin{Shaded}
\begin{Highlighting}[]
\CommentTok{# Fungsi dengan argumen}
\NormalTok{tambah <-}\StringTok{ }\ControlFlowTok{function}\NormalTok{(a,b)\{}
  \KeywordTok{print}\NormalTok{(a}\OperatorTok{+}\NormalTok{b)}
\NormalTok{\}}

\CommentTok{# Print}
\KeywordTok{tambah}\NormalTok{(}\DecValTok{5}\NormalTok{,}\DecValTok{3}\NormalTok{)}
\end{Highlighting}
\end{Shaded}

\begin{verbatim}
## [1] 8
\end{verbatim}

\begin{Shaded}
\begin{Highlighting}[]
\CommentTok{# Fungsi dengan return value}
\NormalTok{kali <-}\StringTok{ }\ControlFlowTok{function}\NormalTok{(a,b)\{}
  \KeywordTok{return}\NormalTok{(a}\OperatorTok{*}\NormalTok{b)}
\NormalTok{\}}

\CommentTok{# Print}
\KeywordTok{kali}\NormalTok{(}\DecValTok{4}\NormalTok{,}\DecValTok{3}\NormalTok{)}
\end{Highlighting}
\end{Shaded}

\begin{verbatim}
## [1] 12
\end{verbatim}

\section{Referensi}\label{referensi-1}

\begin{enumerate}
\def\labelenumi{\arabic{enumi}.}
\tightlist
\item
  Primartha, R. 2018. \textbf{Belajar Machine Learning Teori dan
  Praktik}. Penerbit Informatika : Bandung.
\item
  Rosadi,D. 2016. \textbf{Analisis Statistika dengan R}. Gadjah Mada
  University Press: Yogyakarta.
\item
  STHDA. \textbf{Easy R Programming Basics}.
  \url{http://www.sthda.com/english/wiki/easy-r-programming-basics}
\item
  Venables, W.N. Smith D.M. and R Core Team. 2018. \textbf{An
  Introduction to R}. R Manuals.
\item
  The R Core Team. 2018. \textbf{R: A Language and Environment for
  Statistical Computing}. R Manuals.
\end{enumerate}

\chapter{Manajemen Data R}\label{manajemen-data-r}

Data manajemen merupakan bagian penting dalam setiap proses analisa
data. Proses import dan eksport data pada berbagai format penting untuk
dipelajari. Selain itu, proses perapihan data sebelum analisa menjadi
bagian yang harus ada pada awal proses analisa. Proses-proses tersebut
akan kita ulas secara mendalam pada \emph{chapter} ini.\emph{Chapter}
ini juga akan membahas bagaimana kita dapat melakukan sejumlah
manipulasi data untuk memperoleh informasi lebih yang terkandung pada.

\section{Import File}\label{import-file}

Pada sesi bagian ini penulis akan menjelaskan cara mengimport file pada
\texttt{R}. File yang diimport ke dalam \texttt{R} terdiri atas file
yang sering digunakan pada saat akan melakukan analisis data, antara
lain: TXT, CSv, Excel, SPSS, SAS, dan STATA.

Pada bagian ini akan dijelaskan pula bagaimana melakukan import data
menggunakan library \texttt{readr} serta kelebihan dari metode import
data yang digunakan. Berikut adalah cara mengimport data berbagai format
pada \texttt{R}.

\begin{quote}
\textbf{Note: } Pastikan kita telah mengatur lokasi \emph{working
directory} pada tempat dimana lokasi file yang akan kita baca berada
untuk mempermudah dalam melakukan import file.
\end{quote}

\subsection{Import File Menggunakan Fungsi Bawaan
R}\label{import-file-menggunakan-fungsi-bawaan-r}

Fungsi bawaan \texttt{R} secara umum hanya dapat membaca data dengan
format TXT dan CSV. Pada \texttt{RStudio} fungsi ini bertambah dengan
adanya library tambahan yang telah terinstall di \texttt{RStudio} untuk
membaca file dengan format EXCEL, SPSS, SAS dan STATA.

Secara umum fungsi yang digunakan untuk membaca data dengan format tabel
seperti TXT dan CSV adalah fungsi\texttt{read.table()}. Berikut adalah
list fungsi dasar lainnya untuk membaca file dengan format TXT dan CSV
pada \texttt{R}:

\begin{itemize}
\tightlist
\item
  \textbf{read.csv()}: untuk membaca file dengan format \emph{comma
  separated value}(``.csv'').
\item
  \textbf{read.csv2()}: varian yang digunakan jika pada file ``.csv''
  yang akan dibaca mengandung koma (``,'') sebagai desimal dan semicolon
  (``;'') sebagai pemisah antar variabel atau kolom.
\item
  \textbf{read.delim()}: untuk membaca file dengan format
  \emph{tab-separated value}(``.txt'').
\item
  \textbf{read.delim2()}: membaca file dengan format ``.txt'' dengan
  tanda koma (``,'') sebagai penujuk bilangan desimal.
\end{itemize}

Masing-masing fungsi diatas dapat dituliskan kedalam \texttt{R} dengan
format sebagai berikut:

\begin{Shaded}
\begin{Highlighting}[]
\CommentTok{# Membaca tabular data pada  R}
\KeywordTok{read.table}\NormalTok{(file, }\DataTypeTok{header =} \OtherTok{FALSE}\NormalTok{, }\DataTypeTok{sep =} \StringTok{""}\NormalTok{, }\DataTypeTok{dec =} \StringTok{"."}\NormalTok{)}
\CommentTok{# Membaca"comma separated value" files (".csv")}
\KeywordTok{read.csv}\NormalTok{(file, }\DataTypeTok{header =} \OtherTok{TRUE}\NormalTok{, }\DataTypeTok{sep =} \StringTok{","}\NormalTok{, }\DataTypeTok{dec =} \StringTok{"."}\NormalTok{, ...)}
\CommentTok{# atau gunakan read.csv2 jika tanda desimal pada data adalah "," dan pemisah kolom adalah ";"}
\KeywordTok{read.csv2}\NormalTok{(file, }\DataTypeTok{header =} \OtherTok{TRUE}\NormalTok{, }\DataTypeTok{sep =} \StringTok{";"}\NormalTok{, }\DataTypeTok{dec =} \StringTok{","}\NormalTok{, ...)}
\CommentTok{# MembacaTAB delimited files}
\KeywordTok{read.delim}\NormalTok{(file, }\DataTypeTok{header =} \OtherTok{TRUE}\NormalTok{, }\DataTypeTok{sep =} \StringTok{"}\CharTok{\textbackslash{}t}\StringTok{"}\NormalTok{, }\DataTypeTok{dec =} \StringTok{"."}\NormalTok{, ...)}
\KeywordTok{read.delim2}\NormalTok{(file, }\DataTypeTok{header =} \OtherTok{TRUE}\NormalTok{, }\DataTypeTok{sep =} \StringTok{"}\CharTok{\textbackslash{}t}\StringTok{"}\NormalTok{, }\DataTypeTok{dec =} \StringTok{","}\NormalTok{, ...)}
\end{Highlighting}
\end{Shaded}

\begin{quote}
\textbf{Note: }

\begin{itemize}
\tightlist
\item
  \textbf{file}: nama file diakhiri dengan format file (misal:
  ``nama\_file.txt'') yang akan di import ke dalam file. Dapat pula
  diisi lokasi file tersebut berada, misal:(C:/Users/My
  PC/Documents/nama\_file.txt atau .csv)
\item
  \textbf{sep}: pemisah antar kolom. ``\t'' digunakan untuk
  tab-delimited file.
\item
  \textbf{header}: nilai logik. jika TRUE, maka \texttt{read.table()}
  akan menganggap bahwa file yang akan dibaca pada baris pertama file
  merupakan header data.
\item
  \textbf{dec}: karakter yang digunakan sebagai penunjuk desimal pada
  data.
\end{itemize}
\end{quote}

Untuk info lebih lanjut terkait fungsi-fungsi tersebut dan contoh
bagaimana menggunakannya, pembaca dapat mengakses fitur batuan dari
fungsi tersebut menggunakan sintaks berikut:

\begin{Shaded}
\begin{Highlighting}[]
\CommentTok{# mengakses menu bantuan}
\NormalTok{?read.table}
\NormalTok{?read.csv}
\NormalTok{?read.csv2}
\NormalTok{?read.delim}
\NormalTok{?read.delim2}
\end{Highlighting}
\end{Shaded}

Misalkan penulis memiliki data pada file bernama ``mtcars.csv'' dengan
desimal berupa titik pada datanya. Penulsi ingin membaca file tersebut,
maka penulis akan menuliskan sintaks berikut:

\begin{Shaded}
\begin{Highlighting}[]
\NormalTok{data <-}\StringTok{ }\KeywordTok{read.csv}\NormalTok{(}\StringTok{"mtcars.csv"}\NormalTok{)}
\end{Highlighting}
\end{Shaded}

Secara default perintah tersebut akan membaca baris pertama data sebagai
header serta data berupa karakter menjadi factor. Untuk mencegah agar
data berupa karakter menjadi faktor, perintah tersebut dapat ditambahkan
parameter \texttt{stringAsFactor\ =\ FALSE}.

Kita juga dapat memilih file yang akan kita baca secara interakti. Misal
pada \emph{working directory} terdapat beberapa file yang akan kita
baca. Kita ingin melihat file dengan format tertentu yang hendak kita
baca, namun kita malas mengecek file explorer pada windows. Untuk
mengatasi masalah tersebut, kita dapat menggunakan fungsi
\texttt{file.choose()} pada \texttt{R}. Fungsi tersebut akan menampilkan
jendela windows explores sehingga kita dapat memilih file apa yang
hendak dibaca. Berikut adalah contoh penerapannya:

\begin{Shaded}
\begin{Highlighting}[]
\NormalTok{data <-}\StringTok{ }\KeywordTok{read.csv}\NormalTok{(}\KeywordTok{file.choose}\NormalTok{())}
\end{Highlighting}
\end{Shaded}

\begin{quote}
\textbf{Note: } pastikan format file yang dibaca sama dengan fungsi
import yang digunakan.
\end{quote}

Kita juga dapat membaca file dari internet. Untuk melakukannya kit hanya
perlu meng-copy url file tersebut. Berikut adalah contoh file yang
dibaca dari internet:

\begin{Shaded}
\begin{Highlighting}[]
\CommentTok{# Membaca file dari internet}
\NormalTok{data <-}\StringTok{ }\KeywordTok{read.delim}\NormalTok{(}\StringTok{"http://www.sthda.com/upload/boxplot_format.txt"}\NormalTok{)}

\CommentTok{# mengecek 6 observasi awal}
\KeywordTok{head}\NormalTok{(data)}
\end{Highlighting}
\end{Shaded}

\begin{verbatim}
##    Nom variable Group
## 1 IND1       10     A
## 2 IND2        7     A
## 3 IND3       20     A
## 4 IND4       14     A
## 5 IND5       14     A
## 6 IND6       12     A
\end{verbatim}

\subsection{Membaca File CSV dan TXT Menggunakan Library
readr}\label{membaca-file-csv-dan-txt-menggunakan-library-readr}

Pada bagian sebelumnya kita telah belajar bagaimana cara membaca file
dengan format CSV dan TXT menggunakan paket dasar \texttt{R}. Pada
bagian ini penulis akan menjelaskan bagaimana cara membaca file dengan
format TXT dan CSV pada \texttt{R} menggunakan paket \texttt{readr}.

\texttt{readr} dikembangkan oleh Hadley Wickham. paket \texttt{readr}
memberikan solusi cepat dan ramah untuk membaca delimited file ke dalam
\texttt{R}.

Dibandingkan dengan paket dasar \texttt{R}, \texttt{readr} memiliki
kelebihan sebagai berikut:

\begin{itemize}
\tightlist
\item
  Mampu membaca file 10x lebih cepat dibandingkan pada paket bawaan
  \texttt{R}.
\item
  Menampilkan \emph{progress bar} yang bermanfaat jika proses pemuatan
  berlangsung agak lama.
\item
  semua fungsi bekerja dengan cara yang persis sama dengan paket bawaan
  \texttt{R}.
\end{itemize}

Untuk dapat menggunakan \texttt{readr}, kita perlu menginstall paketnya
terlebih dahulu. Untuk melakukannya jalankan sintaks berikut:

\begin{Shaded}
\begin{Highlighting}[]
\CommentTok{# Menginstall paket}
\KeywordTok{install.packages}\NormalTok{(}\StringTok{"readr"}\NormalTok{)}

\CommentTok{# Memuat paket}
\KeywordTok{library}\NormalTok{(readr)}
\end{Highlighting}
\end{Shaded}

Berikut adalah format bebrapa fungsi yang dapat digunakan:

\begin{Shaded}
\begin{Highlighting}[]
\CommentTok{# Fungsi umum (membaca TXT dan CSV) dapat juga membaca flat file dan tsv}
\KeywordTok{read_delim}\NormalTok{(file, delim, }\DataTypeTok{col_names =} \OtherTok{TRUE}\NormalTok{)}
\CommentTok{# Membaca comma (",") separated values}
\KeywordTok{read_csv}\NormalTok{(file, }\DataTypeTok{col_names =} \OtherTok{TRUE}\NormalTok{)}
\CommentTok{# Membaca semicolon (";") separated values}
\KeywordTok{read_csv2}\NormalTok{(file, }\DataTypeTok{col_names =} \OtherTok{TRUE}\NormalTok{)}
\CommentTok{# Membaca tab separated values}
\KeywordTok{read_tsv}\NormalTok{(file, }\DataTypeTok{col_names =} \OtherTok{TRUE}\NormalTok{)}
\end{Highlighting}
\end{Shaded}

\begin{quote}
\textbf{Note: }

\begin{itemize}
\tightlist
\item
  \textbf{file}: path file, koneksi atau raw vector. File yang
  berakhiran .gz, .bz2, .xz, atau .zip akan secara otomatis tidak
  terkompresi. File yang dimulai dengan ``http: //'', ``https: //'',
  ``ftp: //'', atau ``ftps: //'' akan diunduh secara otomatis. File gz
  jarak jauh juga dapat diunduh \& didekompresi secara otomatis.
\item
  \textbf{delim}: karakter yang membatasi tiap nilai pada file.
\item
  \textbf{col\_names}: nilai logik. Jika TRUE, maka baris pertama akan
  menjadi header.
\end{itemize}
\end{quote}

Berikut adalah contoh bagaimana cara membaca file menggunakan fungsi
pada paket \texttt{readr}:

\begin{Shaded}
\begin{Highlighting}[]
\CommentTok{# Membaca file lokal}
\NormalTok{data <-}\StringTok{ }\KeywordTok{read_csv}\NormalTok{(}\StringTok{"mtcars.csv"}\NormalTok{)}

\CommentTok{# atau}
\NormalTok{data <-}\StringTok{ }\KeywordTok{read_csv}\NormalTok{(}\KeywordTok{file.choose}\NormalTok{())}

\CommentTok{# Membaca dari internet}
\NormalTok{data <-}\StringTok{ }\KeywordTok{read_tsv}\NormalTok{(}\StringTok{"http://www.sthda.com/upload/boxplot_format.txt"}\NormalTok{)}
\end{Highlighting}
\end{Shaded}

Kita juga dapat menspesifikasi jenis data pada kolom yang akan dibaca.
Keuntungan dari penentuan jenis kolom (tipe data) akan memastikan data
yang telah dibaca tidak salah berdasarkan jenis data pada masing-masing
kolom.

Beberapa format jenis kolom yang tersedia pada \texttt{readr} adalah
sebagi berikut:

\begin{itemize}
\tightlist
\item
  \textbf{col\_integer()}: untuk menentukan integer (alias = ``i'').
\item
  \textbf{col\_double()}: untuk menentukan kolom sebagai jenis data
  double (alias = ``d'').
\item
  \textbf{col\_logical()}: untuk menentukan variabel logis (alias =
  ``l'').
\item
  \textbf{col\_character()}: meninggalkan string apa adanya.Tidak
  mengonversinya menjadi faktor (alias = ``c'').
\item
  \textbf{col\_factor()}: untuk menentukan variabel faktor (atau
  pengelompokan) (alias = ``f'')
\item
  \textbf{col\_skip()}: untuk mengabaikan kolom (alias = ``-'' atau
  ``\_``)
\item
  \textbf{col\_date()} (alias = ``D''), \textbf{col\_datetime()} (alias
  = ``T'') dan \textbf{col\_time()} (``t'') untuk menentukan tanggal,
  waktu tanggal, dan waktu.
\end{itemize}

Berikut adalah contoh penerapannya:

\begin{Shaded}
\begin{Highlighting}[]
\NormalTok{data <-}\StringTok{ }\KeywordTok{read_csv}\NormalTok{(}\StringTok{"my_file.csv"}\NormalTok{, }\DataTypeTok{col_types =} \KeywordTok{cols}\NormalTok{(}
  \DataTypeTok{x =} \StringTok{"i"}\NormalTok{, }\CommentTok{# kolom integer}
  \DataTypeTok{treatment =} \StringTok{"c"} \CommentTok{# kolom karakter/string}
\NormalTok{))}
\end{Highlighting}
\end{Shaded}

\subsection{Import File Excel Pada R}\label{import-file-excel-pada-r}

Keunggulan penggunaan excel sebagai format penyimpan data adalah kita
dapat menyimpan banyak data dan memisahkannya pada lembar (\emph{sheet})
yang berbeda sebagai suatu data yang independen dibandingkan pembacaan
pada file csv yang hanya berisikan satu tabel data saja tiap file.

Pada \texttt{R} kita dapat melakukan pembacaan file menggunakan berbagai
macam cara seperti menggunakan paket bawaan \texttt{R} maupun
menggunakan library yang perlu kita install. Berikut adalah beberapa
cara membaca file excel pada \texttt{R}.

\begin{enumerate}
\def\labelenumi{\alph{enumi}.}
\item
  Mengkonversi terlebih dahulu satu sheet excel yang akan kita baca
  menjadi format ``.csv'' maupun ``.txt'' sehingga dapat dibaca seperti
  pada sub-bab 3.1.1.
\item
  Menyalin data dari excel dan mengimport data pada \texttt{R}.
\end{enumerate}

Cara ini sedikit mirip dengan cara sebelumnya, dimana kita perlu membuka
file excel dan melakukan \textbf{select} dan \textbf{copy} (ctrl+c)
tabel data yang hendak dibaca. Data tersebut selanjutnya akan tersimpan
pada \textbf{clipboard}.

Data yang telah tersalin selanjutnya diimport ke \texttt{R} dengan
mengetikkan sintaks berikut:

\begin{Shaded}
\begin{Highlighting}[]
\NormalTok{data <-}\StringTok{ }\KeywordTok{read.table}\NormalTok{(}\DataTypeTok{file=} \StringTok{"clipboard"}\NormalTok{,}
                   \DataTypeTok{sep =} \StringTok{"}\CharTok{\textbackslash{}t}\StringTok{"}\NormalTok{, }\DataTypeTok{header =} \OtherTok{TRUE}\NormalTok{)}
\end{Highlighting}
\end{Shaded}

Cara ini merupakan cara yang paling sering penulis gunakan. Kelemahan
penggunaan cara ini adalah ketika kita melakukan proses \textbf{select}
dan \textbf{copy} (ctrl+c) tabel yang jumlahnya sangat banyak dan
terdapat teks-teks penjelasan terkait tabel data pada lembar kerja excel
yang tidak ingin kita sertakan akan memakan waktu yang lebih lama pada
proses \textbf{select}.

\begin{enumerate}
\def\labelenumi{\alph{enumi}.}
\setcounter{enumi}{2}
\tightlist
\item
  Mengimport data menggunakan library readxl.
\end{enumerate}

Paket \texttt{readxl}, yang dikembangkan oleh Hadley Wickham, dapat
digunakan untuk dengan mudah mengimpor file Excel (xls \textbar{} xlsx)
ke \texttt{R} tanpa ada ketergantungan eksternal.

Untuk dapat menggunakan library \texttt{readxl} kita harus
menginstallnya terlebih dahulu menggunakan sintaks berikut:

\begin{Shaded}
\begin{Highlighting}[]
\CommentTok{# Instal paket}
\KeywordTok{install.packages}\NormalTok{(}\StringTok{"readxl"}\NormalTok{)}

\CommentTok{# memuat paket}
\KeywordTok{library}\NormalTok{(readxl)}
\end{Highlighting}
\end{Shaded}

Berikut adalah contoh cara mengimport data dengan format xls atau xlsx
pada \texttt{R}.

\begin{Shaded}
\begin{Highlighting}[]
\CommentTok{# Tentukan sheet dengan nama sheet pada file}
\NormalTok{data <-}\StringTok{ }\KeywordTok{read_excel}\NormalTok{(}\StringTok{"my_file.xlsx"}\NormalTok{, }\DataTypeTok{sheet =} \StringTok{"data"}\NormalTok{)}

\CommentTok{# Tentukan sheet berdasarkan indeks sheet}
\NormalTok{data <-}\StringTok{ }\KeywordTok{read_excel}\NormalTok{(}\StringTok{"my_file.xlsx"}\NormalTok{, }\DataTypeTok{sheet =} \DecValTok{2}\NormalTok{) }\CommentTok{# membaca sheet ke-2}
\end{Highlighting}
\end{Shaded}

\begin{enumerate}
\def\labelenumi{\alph{enumi}.}
\setcounter{enumi}{3}
\tightlist
\item
  Mengimport data menggunakan library xlsx
\end{enumerate}

Paket \texttt{xlsx}, solusi berbasis \texttt{java}, adalah salah satu
paket \texttt{R} yang ampuh untuk membaca, menulis, dan memformat file
Excel. Untuk dapat menggunakannya kita harus menginstall dan memuatnya
terlebih dahulu. Berikut sintaks yang digunakan:

\begin{Shaded}
\begin{Highlighting}[]
\CommentTok{# Menginstall paket}
\KeywordTok{install.packages}\NormalTok{(}\StringTok{"xlsx"}\NormalTok{)}

\CommentTok{# Memuat paket}
\KeywordTok{library}\NormalTok{(xlsx)}
\end{Highlighting}
\end{Shaded}

Terdapat dua buah fungsi yang disediakan pada paket tersebut yaitu
\texttt{read.xlsx()} dan \texttt{read.xlsx2()}. Perbedaan keduanya
adalah \texttt{read.xlsx2()} digunakan pada file data dengan ukuran yang
besar serta proses pembacaan data yang lebih cepat dibandingkan dengan
\texttt{read.xlsx()}. Fromat yang digunakan untuk kedua fungsi tersebut
disajikan sebagai berikut:

\begin{Shaded}
\begin{Highlighting}[]
\KeywordTok{read.xlsx}\NormalTok{(file, sheetIndex, }\DataTypeTok{header=}\OtherTok{TRUE}\NormalTok{)}
\KeywordTok{read.xlsx2}\NormalTok{(file, sheetIndex, }\DataTypeTok{header=}\OtherTok{TRUE}\NormalTok{)}
\end{Highlighting}
\end{Shaded}

\begin{quote}
\textbf{Note: }

\begin{itemize}
\tightlist
\item
  \textbf{file}: nama atau lokasi file berada
\item
  \textbf{sheetIndex}: Indeks dari sheet yang hendak dibaca
\item
  \textbf{header}: nilai logik. Jika bernilai TRUE, maka baris pertama
  dari sheet menjadi header.
\end{itemize}
\end{quote}

Berikut adalah contoh penggunaanya:

\begin{Shaded}
\begin{Highlighting}[]
\NormalTok{data <-}\StringTok{ }\KeywordTok{read.xlsx}\NormalTok{(}\KeywordTok{file.choose}\NormalTok{(), }\DecValTok{1}\NormalTok{) }\CommentTok{# membaca sheet 1}
\end{Highlighting}
\end{Shaded}

\begin{quote}
\textbf{Note: } kita juga dapat membaca file dari internet seperti pada
sub-bab 3.1.1.
\end{quote}

\subsection{Membaca File Dari Format Aplikasi
Statistik}\label{membaca-file-dari-format-aplikasi-statistik}

Untuk membaca file yang berasal dari format aplikasi statistik seperti
SPSS, SAS, dan STATA kita perlu menginstal dan memuat paket-paket yang
dibutuhkan sesuai dengan file yang akan kita install. Berikut adalah
sintaks bagaimana cara mengimport file dari berbagai format aplikasi
statistik.

\begin{Shaded}
\begin{Highlighting}[]
\CommentTok{# membaca file SPSS}
\KeywordTok{install.packages}\NormalTok{(}\StringTok{"Hmisc"}\NormalTok{) }\CommentTok{# menginstall paket}
\KeywordTok{library}\NormalTok{(Hmisc) }\CommentTok{# memuat paket}
\CommentTok{# simpan SPSS dataset pada transport format}
\NormalTok{get file=}\StringTok{'c:\textbackslash{}mydata.sav'}\NormalTok{.}
\NormalTok{export outfile=}\StringTok{'c:\textbackslash{}mydata.por'}\NormalTok{. }
\NormalTok{data <-}\StringTok{ }\KeywordTok{spss.get}\NormalTok{(}\StringTok{"c:\textbackslash{}mydata.por"}\NormalTok{, }\DataTypeTok{use.value.labels=} \OtherTok{TRUE}\NormalTok{) }
\CommentTok{# use.value.labels digunakan untuk mengubah label menjadi factor}


\CommentTok{# membaca file SAS}
\KeywordTok{install.packages}\NormalTok{(}\StringTok{"Hmisc"}\NormalTok{) }\CommentTok{# menginstall paket}
\KeywordTok{library}\NormalTok{(Hmisc) }\CommentTok{# memuat paket}
\CommentTok{# simpan SAS dataset pada transport format}
\NormalTok{libname out xport }\StringTok{'c:/mydata.xpt'}\NormalTok{;}
\NormalTok{data out.mydata;}
\NormalTok{set sasuser.mydata;}
\NormalTok{run;}
\NormalTok{data <-}\StringTok{ }\KeywordTok{sasxport.get}\NormalTok{(}\StringTok{"c:/mydata.xpt"}\NormalTok{) }
\CommentTok{# Variabel yang berupa karakter akan dikonversi menjadi factor}


\CommentTok{# membaca file STATA}
\KeywordTok{install.packages}\NormalTok{(}\StringTok{"foreign"}\NormalTok{) }\CommentTok{# menginstall paket}
\KeywordTok{library}\NormalTok{(foreign) }\CommentTok{# memuat paket}
\NormalTok{data <-}\StringTok{ }\KeywordTok{read.dta}\NormalTok{(}\StringTok{"c:/mydata.dta"}\NormalTok{)}
\end{Highlighting}
\end{Shaded}

\section{Eksport File}\label{eksport-file}

Setelah kita melakukan analisa dan telah memperoleh hasil yang kita
inginkan dan memperoleh data frame berupa hasil prediksi suatu model
atau data yang telah dibersihakan, kita ingin melakukan pelaporan dalam
bentuk file dengan format seperti EXCEL, CSV atau TXT. Untuk
melakukannya kita perlu melakukan eksport data yang telah dihasilkan.

Pada bagian ini penulis akan menjelaskan bagaimana cara mengeksport data
dari \texttt{R} kedalam format TXT, CSV, maupun EXCEL. Sebenarnya
\texttt{R} memungkinkan untuk melakukan eksport dalam format lain
seperti RDA maupun RDS yang tidak dibahas dalam buku ini karena berada
diluar lingkup buku ini.

\subsection{Eksport Data Menjadi Format TXT dan
CSV}\label{eksport-data-menjadi-format-txt-dan-csv}

Terdapat dua cara untuk melakukan ekport data dari \texttt{R} menjadi
format TXT atau CSV, yaitu melalui paket dasar \texttt{R} maupun
menggunakan library \texttt{readr}. Kedua cara tersebut memiliki
sejumlah kemiripan dari segi fungsi, namun berbeda dari segi kecepatan
eksport.

Fungsi dasar yang digunakan pada \texttt{R} untuk melakukan eksport file
kedalam format TXT dan CSv adalah \texttt{write.tabel()}. Format umum
yang digunakan adalah sebagai berikut:

\begin{Shaded}
\begin{Highlighting}[]
\KeywordTok{write.table}\NormalTok{(x, file, }\DataTypeTok{sep=} \StringTok{" "}\NormalTok{, }\DataTypeTok{dec =} \StringTok{","}\NormalTok{,}
            \DataTypeTok{row.names =} \OtherTok{TRUE}\NormalTok{, }\DataTypeTok{col.names =} \OtherTok{TRUE}\NormalTok{)}
\end{Highlighting}
\end{Shaded}

\begin{quote}
\textbf{Note: }

\begin{itemize}
\tightlist
\item
  \textbf{x}: matriks atau data frame yang akan ditulis.
\item
  \textbf{file}: karakter yang menentukan nama file yang dihasilkan.
\item
  \textbf{sep}: string pemisah bidang atau kolom, mis., sep = ``~t''
  (untuk nilai yang dipisahkan tab).
\item
  \textbf{dec}: string yang akan digunakan sebagai pemisah desimal.
  Standarnya adalah ``.''.
\item
  \textbf{row.names}: nilai logik yang menunjukkan apakah nama baris x
  harus ditulis bersama dengan x, atau vektor karakter nama baris yang
  akan ditulis.
\item
  \textbf{col.names}: baik nilai logik yang menunjukkan apakah nama
  kolom x harus ditulis bersama dengan x, atau vektor karakter nama
  kolom yang akan ditulis. Jika \texttt{col.names\ =\ NA} dan
  \texttt{row.names\ =\ TRUE} ditambahkan nama kolom kosong, yang
  merupakan konvensi yang digunakan untuk file CSV untuk dibaca oleh
  spreadsheet.
\end{itemize}
\end{quote}

Selain menggunakan fungsi tersebut, untuk eksport ke dalam format CSV
juga dapa menggunakan fungsi \texttt{write.csv()} atau
\texttt{write.csv2()}. Berikut adalah format yang digunakan:

\begin{Shaded}
\begin{Highlighting}[]
\KeywordTok{write.csv}\NormalTok{(data, }\DataTypeTok{file=}\StringTok{"data.csv"}\NormalTok{)}
\KeywordTok{write.csv2}\NormalTok{(data, }\DataTypeTok{file=}\StringTok{"data.csv"}\NormalTok{)}
\end{Highlighting}
\end{Shaded}

Secara penampakan kedua fungsi tersebut pada dasarnya sama dengan fungsi
\texttt{write.table()}, bedanya adalah kedua fungsi tersebut spesifik
digunakan untuk eksport file kedalam format CSV.

\begin{quote}
\textbf{Note: }

\begin{itemize}
\tightlist
\item
  \textbf{write.csv()} menggunakan ``.'' sebagai titik desimal serta
  ``,'' sebagai pemisah antar kolom data.
\item
  \textbf{write.csv2()} menggunakan ``,'' sebagai titik desimal serta
  ``;'' sebagai pemisah antar kolom data.
\end{itemize}
\end{quote}

Misalkan kita ingin melakukan eksport data objek \texttt{mtcars} kedalam
format CSV. Untuk melakukannya dapat dilakukan dengan sintaks berikut:

\begin{Shaded}
\begin{Highlighting}[]
\KeywordTok{write.csv}\NormalTok{(mtcars, }\DataTypeTok{file=}\StringTok{"mtcars.csv"}\NormalTok{, }\DataTypeTok{row.names =} \OtherTok{FALSE}\NormalTok{)}
\end{Highlighting}
\end{Shaded}

\begin{quote}
\textbf{Note: } Hasil ekspoet ditampilkan pada \emph{working directory}
\end{quote}

Kita juga dapat menggunakan fungsi \texttt{write\_delim()} dari library
\texttt{readr} untuk melakukan eksport data kedalam format CSV atau TXT.
Berdasarkan format file yang hendak dihasilkan kita juga dapat
menggunakan fungsi \texttt{write\_csv()} atau \texttt{write\_tsv()}.
Berikut adalah penjelasan terkait kedua fungsi tersebut:

\begin{itemize}
\tightlist
\item
  \textbf{write\_csv()}: untuk mengeksport kedalam format CSV.
\item
  \textbf{write\_tsv()}: untuk mengeksport kedalam format TXT.
\end{itemize}

Format sederhana ketiga fungsi fungsi tersebut adalah sebagai berikut:

\begin{Shaded}
\begin{Highlighting}[]
\CommentTok{# Fungsi umum}
\KeywordTok{write_delim}\NormalTok{(x, path, }\DataTypeTok{delim =} \StringTok{" "}\NormalTok{)}
\CommentTok{# Write comma (",") separated value files}
\KeywordTok{write_csv}\NormalTok{(file, path)}
\CommentTok{# Write tab ("\textbackslash{}t") separated value files}
\KeywordTok{write_tsv}\NormalTok{(file, path)}
\end{Highlighting}
\end{Shaded}

\begin{quote}
\textbf{Note: }

\begin{itemize}
\tightlist
\item
  \textbf{x}: data frame yang akan ditulis
\item
  \textbf{path}: path ke file hasil (dapat berupa nama file disertai
  ekstensi file yang akan dibuat)
\item
  \textbf{delim}: Delimiter digunakan untuk memisahkan nilai. Harus
  karakter tunggal.
\end{itemize}
\end{quote}

Berikut adalah contoh penerapan dari fungsi tersebut:

\begin{Shaded}
\begin{Highlighting}[]
\CommentTok{# memuat mtcars data}
\KeywordTok{data}\NormalTok{(mtcars)}
\KeywordTok{library}\NormalTok{(readr)}

\CommentTok{# eksport mtcars menjadi tsv atau txt}
\KeywordTok{write_tsv}\NormalTok{(mtcars, }\DataTypeTok{path =} \StringTok{"mtcars.txt"}\NormalTok{)}

\CommentTok{# eksport mycars menjadi csv}
\KeywordTok{write_csv}\NormalTok{(mtcars, }\DataTypeTok{path =} \StringTok{"mtcars.csv"}\NormalTok{)}
\end{Highlighting}
\end{Shaded}

\subsection{Eksport Data Menjadi Format
Excel}\label{eksport-data-menjadi-format-excel}

Untuk mengeksport data menjadi format EXCEL (``.xls'' atau ``.xlsx'')
kita dapat menggunakan fungsi \texttt{write.xlsx()} dan
\texttt{write.xlsx2()} dari library \texttt{xlsx}. Berikut adalah format
sederhana yanga digunakan:

\begin{Shaded}
\begin{Highlighting}[]
\KeywordTok{write.xlsx}\NormalTok{(x, file, }\DataTypeTok{sheetName =} \StringTok{"Sheet1"}\NormalTok{, }
  \DataTypeTok{col.names =} \OtherTok{TRUE}\NormalTok{, }\DataTypeTok{row.names =} \OtherTok{TRUE}\NormalTok{, }\DataTypeTok{append =} \OtherTok{FALSE}\NormalTok{)}
\KeywordTok{write.xlsx2}\NormalTok{(x, file, }\DataTypeTok{sheetName =} \StringTok{"Sheet1"}\NormalTok{,}
  \DataTypeTok{col.names =} \OtherTok{TRUE}\NormalTok{, }\DataTypeTok{row.names =} \OtherTok{TRUE}\NormalTok{, }\DataTypeTok{append =} \OtherTok{FALSE}\NormalTok{)}
\end{Highlighting}
\end{Shaded}

\begin{quote}
\textbf{Note: }

\begin{itemize}
\tightlist
\item
  \textbf{x}: sebuah data frame untuk ditulis ke dalam worksheet.
\item
  \textbf{file}: path ke file output.
\item
  \textbf{sheetName}: string karakter yang digunakan untuk nama sheet.
\item
  \textbf{col.names, row.names}: nilai logik yang menentukan apakah nama
  kolom / nama baris x akan ditulis ke file.
\item
  \textbf{append}: nilai logis yang menunjukkan apakah x harus
  ditambahkan ke file yang ada.
\end{itemize}
\end{quote}

Berikut adalah contoh penerapannya:

\begin{Shaded}
\begin{Highlighting}[]
\KeywordTok{library}\NormalTok{(}\StringTok{"xlsx"}\NormalTok{)}
\CommentTok{# Menuliskan dataset pertama pada workbook}
\KeywordTok{write.xlsx}\NormalTok{(USArrests, }\DataTypeTok{file =} \StringTok{"myworkbook.xlsx"}\NormalTok{,}
      \DataTypeTok{sheetName =} \StringTok{"USA-ARRESTS"}\NormalTok{, }\DataTypeTok{append =} \OtherTok{FALSE}\NormalTok{)}
\CommentTok{# Menambahkan dataset kedua pada workbook}
\KeywordTok{write.xlsx}\NormalTok{(mtcars, }\DataTypeTok{file =} \StringTok{"myworkbook.xlsx"}\NormalTok{, }
           \DataTypeTok{sheetName=}\StringTok{"MTCARS"}\NormalTok{, }\DataTypeTok{append=}\OtherTok{TRUE}\NormalTok{)}
\CommentTok{# Menambahkan dataset kedua pada workbook}
\KeywordTok{write.xlsx}\NormalTok{(iris, }\DataTypeTok{file =} \StringTok{"myworkbook.xlsx"}\NormalTok{,}
           \DataTypeTok{sheetName=}\StringTok{"IRIS"}\NormalTok{, }\DataTypeTok{append=}\OtherTok{TRUE}\NormalTok{)}
\end{Highlighting}
\end{Shaded}

\section{Tibble Data Format}\label{tibble-data-format}

Tibble adalah data frame yang menyediakan metode print yang lebih bagus,
berguna saat bekerja dengan kumpulan data besar. Pada bagian ini penulis
akan menjelaskan penggunaan tibble sebagai alternatif kita dalam
berinteraksi dengan data frame.

Untuk membuat tibble kita perlu menginstall dan memuat library
\texttt{tibble} yang dikembangkan oleh \textbf{Hadley Wichham}. Berikut
adalah sintaks yang digunakan:

\begin{Shaded}
\begin{Highlighting}[]
\CommentTok{# menginstall paket}
\KeywordTok{install.packages}\NormalTok{(}\StringTok{"tibble"}\NormalTok{)}

\CommentTok{# memuat paket}
\KeywordTok{library}\NormalTok{(tibble)}
\end{Highlighting}
\end{Shaded}

\subsection{Membuat Tibble}\label{membuat-tibble}

Untuk dapat membuat tibble kita dapat melakukan konversi data frame yang
sudah ada menjadi tibble menggunakan fungsi \texttt{as\_tibble()}.
Berikut adalah contoh bagaimana membuat tibble mengunakan data
\texttt{iris}:

\begin{Shaded}
\begin{Highlighting}[]
\CommentTok{# memuat data mtcars}
\KeywordTok{data}\NormalTok{(}\StringTok{"iris"}\NormalTok{)}

\CommentTok{# print}
\KeywordTok{head}\NormalTok{(iris, }\DecValTok{10}\NormalTok{)}
\end{Highlighting}
\end{Shaded}

\begin{verbatim}
##    Sepal.Length Sepal.Width Petal.Length Petal.Width
## 1           5.1         3.5          1.4         0.2
## 2           4.9         3.0          1.4         0.2
## 3           4.7         3.2          1.3         0.2
## 4           4.6         3.1          1.5         0.2
## 5           5.0         3.6          1.4         0.2
## 6           5.4         3.9          1.7         0.4
## 7           4.6         3.4          1.4         0.3
## 8           5.0         3.4          1.5         0.2
## 9           4.4         2.9          1.4         0.2
## 10          4.9         3.1          1.5         0.1
##    Species
## 1   setosa
## 2   setosa
## 3   setosa
## 4   setosa
## 5   setosa
## 6   setosa
## 7   setosa
## 8   setosa
## 9   setosa
## 10  setosa
\end{verbatim}

\begin{Shaded}
\begin{Highlighting}[]
\CommentTok{# konversi mtcars menjadi tibble}
\NormalTok{iris_tbl <-}\StringTok{ }\KeywordTok{as_tibble}\NormalTok{(iris)}

\CommentTok{# print}
\NormalTok{iris_tbl}
\end{Highlighting}
\end{Shaded}

\begin{verbatim}
## # A tibble: 150 x 5
##    Sepal.Length Sepal.Width Petal.Length Petal.Width
##           <dbl>       <dbl>        <dbl>       <dbl>
##  1          5.1         3.5          1.4         0.2
##  2          4.9         3            1.4         0.2
##  3          4.7         3.2          1.3         0.2
##  4          4.6         3.1          1.5         0.2
##  5          5           3.6          1.4         0.2
##  6          5.4         3.9          1.7         0.4
##  7          4.6         3.4          1.4         0.3
##  8          5           3.4          1.5         0.2
##  9          4.4         2.9          1.4         0.2
## 10          4.9         3.1          1.5         0.1
## # ... with 140 more rows, and 1 more variable:
## #   Species <fct>
\end{verbatim}

\begin{quote}
\textbf{Note: } Kita dapat mengkonversi tibble menjadi data frame
menggunakan fungsi \texttt{as.data.frame()}
\end{quote}

Secara default saat kita print tibble, maka akan dimunculkan 10
observasi pertama. Pada data frame biasa jika kita print data tersebut
maka seluruh observasi akan ditampilkan.

Penggunaan tibble ini cenderung menguntungkan saat kita bekerja dengan
jumlah data yang besar dan ingin mengecek observasi yang ada. Hal ini
berbeda dengan data frame biasa dimana untuk mengecek observasi awal
kita perlu menggunakan fungsi \texttt{head()} agar seluruh data tidak
ditampilkan. Sehingga penggunaan tibble cenderung membuat proses analisa
menjadi lebih rapi.

Kita juga dapat membuat tibble dari kumpulan sejumlah vektor menggunakan
fungsi \texttt{tibble()}. \texttt{tibble()} akan secara otomatis mendaur
ulang input dengan panjang 1 (variabel \texttt{y}), dan memungkinkan
kita untuk merujuk ke variabel yang baru saja kita buat, seperti yang
ditunjukkan pada sintaks berikut:

\begin{Shaded}
\begin{Highlighting}[]
\KeywordTok{tibble}\NormalTok{(}
  \DataTypeTok{x =} \DecValTok{1}\OperatorTok{:}\DecValTok{20}\NormalTok{,}
  \DataTypeTok{y =} \DecValTok{1}\NormalTok{,}
  \DataTypeTok{z =} \DecValTok{2}\OperatorTok{*}\NormalTok{x}\OperatorTok{+}\DecValTok{5}\OperatorTok{*}\NormalTok{y}
\NormalTok{)}
\end{Highlighting}
\end{Shaded}

\begin{verbatim}
## # A tibble: 20 x 3
##        x     y     z
##    <int> <dbl> <dbl>
##  1     1     1     7
##  2     2     1     9
##  3     3     1    11
##  4     4     1    13
##  5     5     1    15
##  6     6     1    17
##  7     7     1    19
##  8     8     1    21
##  9     9     1    23
## 10    10     1    25
## 11    11     1    27
## 12    12     1    29
## 13    13     1    31
## 14    14     1    33
## 15    15     1    35
## 16    16     1    37
## 17    17     1    39
## 18    18     1    41
## 19    19     1    43
## 20    20     1    45
\end{verbatim}

Jika pembaca telah mulai familiar dengan fungsi \texttt{data.frame()},
perlu diingat bahwa \texttt{tibble()} melakukan lebih sedikit: tidak
pernah mengubah jenis input (mis., tidak pernah mengubah string menjadi
faktor!), tidak pernah mengubah nama variabel, dan tidak pernah membuat
nama baris seperti yang biasa terjadi saat kita menggunakan fungsi
\texttt{data.frame()}.

Cara lain yang dapat digunakan untuk membuat tibble adalah dengan
menggunakan fungsi \texttt{tribble()} yang merupakan singkatan dari
\emph{transposed tibble}. \texttt{tribble()} dikustomisasi untuk entri
data dalam kode: judul kolom didefinisikan oleh rumus (yaitu, mereka
mulai dengan \textasciitilde{}), dan entri dipisahkan oleh koma. Hal ini
memungkinkan untuk menata sejumlah kecil data dalam bentuk yang mudah
dibaca. Berikut adalah contoh penerapannya:

\begin{Shaded}
\begin{Highlighting}[]
\KeywordTok{tribble}\NormalTok{(}
  \OperatorTok{~}\NormalTok{x, }\OperatorTok{~}\NormalTok{y, }\OperatorTok{~}\NormalTok{z,}
  \CommentTok{#--/--/----}
  \StringTok{"a"}\NormalTok{, }\DecValTok{2}\NormalTok{, }\DecValTok{5}\NormalTok{,}
  \StringTok{"b"}\NormalTok{, }\DecValTok{5}\NormalTok{, }\DecValTok{7}
\NormalTok{)}
\end{Highlighting}
\end{Shaded}

\begin{verbatim}
## # A tibble: 2 x 3
##   x         y     z
##   <chr> <dbl> <dbl>
## 1 a         2     5
## 2 b         5     7
\end{verbatim}

Penambahahan komen (\#--/--/----) dilakukan untuk memperjelas posisi
dari header sehingga meminimalisir kesalahan dalam input data.

\subsection{Tibble vs Data Frame}\label{tibble-vs-data-frame}

terdapat dua buah perbedaan utama antara tibble dan data frame , yaitu:
\emph{printing} dan \emph{subsetting}.

\begin{enumerate}
\def\labelenumi{\alph{enumi}.}
\tightlist
\item
  \textbf{Printing}
\end{enumerate}

Tibbles memiliki metode print halus yang hanya menampilkan 10 baris
pertama observasi, dan semua kolom yang sesuai dengan lebar layar. Ini
membuatnya lebih mudah untuk bekerja dengan data besar. Selain namanya,
setiap kolom melaporkan jenis datanya, fitur bagus yang dipinjam dari
fungsi \texttt{str()}. Berikut adalah contohnya:

\begin{Shaded}
\begin{Highlighting}[]
\KeywordTok{tribble}\NormalTok{(}
  \OperatorTok{~}\NormalTok{x, }\OperatorTok{~}\NormalTok{y, }\OperatorTok{~}\NormalTok{z,}
  \CommentTok{#--/---/--------}
  \StringTok{"a"}\NormalTok{, }\FloatTok{2.1}\NormalTok{, }\OtherTok{FALSE}\NormalTok{,}
  \StringTok{"b"}\NormalTok{, }\FloatTok{5.5}\NormalTok{, }\OtherTok{TRUE}
\NormalTok{)}
\end{Highlighting}
\end{Shaded}

\begin{verbatim}
## # A tibble: 2 x 3
##   x         y z    
##   <chr> <dbl> <lgl>
## 1 a       2.1 FALSE
## 2 b       5.5 TRUE
\end{verbatim}

Tibbles dirancang agar kita tidak secara sengaja menampilkan data yang
sangat banyak saat melakukan perintah \texttt{print()}. Tetapi terkadang
kita membutuhkan lebih banyak output daripada tampilan default. Ada
beberapa opsi yang dapat membantu.

Pertama, kita dapat secara eksplisit melakukan print data frame dan
mengontrol jumlah baris (n) dan lebar tampilan. \texttt{width\ =\ Inf}
akan menampilkan semua kolom. Berikut adalah contoh penerapannya

\begin{Shaded}
\begin{Highlighting}[]
\KeywordTok{print}\NormalTok{(iris_tbl, }\DataTypeTok{n=}\DecValTok{15}\NormalTok{, }\DataTypeTok{width=}\OtherTok{Inf}\NormalTok{)}
\end{Highlighting}
\end{Shaded}

\begin{verbatim}
## # A tibble: 150 x 5
##    Sepal.Length Sepal.Width Petal.Length Petal.Width
##           <dbl>       <dbl>        <dbl>       <dbl>
##  1          5.1         3.5          1.4         0.2
##  2          4.9         3            1.4         0.2
##  3          4.7         3.2          1.3         0.2
##  4          4.6         3.1          1.5         0.2
##  5          5           3.6          1.4         0.2
##  6          5.4         3.9          1.7         0.4
##  7          4.6         3.4          1.4         0.3
##  8          5           3.4          1.5         0.2
##  9          4.4         2.9          1.4         0.2
## 10          4.9         3.1          1.5         0.1
## 11          5.4         3.7          1.5         0.2
## 12          4.8         3.4          1.6         0.2
## 13          4.8         3            1.4         0.1
## 14          4.3         3            1.1         0.1
## 15          5.8         4            1.2         0.2
##    Species
##    <fct>  
##  1 setosa 
##  2 setosa 
##  3 setosa 
##  4 setosa 
##  5 setosa 
##  6 setosa 
##  7 setosa 
##  8 setosa 
##  9 setosa 
## 10 setosa 
## 11 setosa 
## 12 setosa 
## 13 setosa 
## 14 setosa 
## 15 setosa 
## # ... with 135 more rows
\end{verbatim}

Kita juga dapat mengontrol print default dengan melakukan pengaturan
menggunakan fungsi \texttt{options()}. Berikut adalah contoh
penerapannya:

\begin{itemize}
\tightlist
\item
  \textbf{options(tibble.print\_max= n, tibble.print\_min= m)}: jika
  terdapat lebih dari ``m'' baris, print hanya sejumlah ``n'' baris.
\item
  \textbf{options(dplyr.print\_min = Inf)}: untuk selalu menampilkan
  seluruh baris. Perlu diingat fungsi ini dapat digunakan saat kita
  telah memuat library \texttt{dplyr}.
\item
  \textbf{options(tibble.width = Inf)}: menampilkan seluruh kolom tanpa
  mempedulikan lebar tampilan layar.
\end{itemize}

Cara terakhir untuk menampilkan seluruh observasi adalh dengan fungsi
\texttt{view()}. Berikut adalah contoh penerapannya pada data
\texttt{iris\_tbl}:

\begin{Shaded}
\begin{Highlighting}[]
\KeywordTok{view}\NormalTok{(iris_tbl)}
\end{Highlighting}
\end{Shaded}

\begin{enumerate}
\def\labelenumi{\alph{enumi}.}
\setcounter{enumi}{1}
\tightlist
\item
  \textbf{Subsetting}
\end{enumerate}

Sejauh ini semua alat yang kita pelajari telah bekerja dengan data frame
yang lengkap. Jika kita ingin mengeluarkan variabel tunggal, kita
memerlukan beberapa alat baru, dollar sign (\texttt{\$}) dan {[}{[}.
{[}{[}dapat mengekstraksi berdasarkan nama atau posisi; \texttt{\$}
hanya mengekstraksi berdasarkan nama. Berikut adalah contoh
penerapannya:

\begin{Shaded}
\begin{Highlighting}[]
\CommentTok{# print tibble}
\NormalTok{iris_tbl}
\end{Highlighting}
\end{Shaded}

\begin{verbatim}
## # A tibble: 150 x 5
##    Sepal.Length Sepal.Width Petal.Length Petal.Width
##           <dbl>       <dbl>        <dbl>       <dbl>
##  1          5.1         3.5          1.4         0.2
##  2          4.9         3            1.4         0.2
##  3          4.7         3.2          1.3         0.2
##  4          4.6         3.1          1.5         0.2
##  5          5           3.6          1.4         0.2
##  6          5.4         3.9          1.7         0.4
##  7          4.6         3.4          1.4         0.3
##  8          5           3.4          1.5         0.2
##  9          4.4         2.9          1.4         0.2
## 10          4.9         3.1          1.5         0.1
## # ... with 140 more rows, and 1 more variable:
## #   Species <fct>
\end{verbatim}

\begin{Shaded}
\begin{Highlighting}[]
\CommentTok{# subset berdasarkan nama kolom}
\NormalTok{iris_tbl}\OperatorTok{$}\NormalTok{Sepal.Length}
\end{Highlighting}
\end{Shaded}

\begin{verbatim}
##   [1] 5.1 4.9 4.7 4.6 5.0 5.4 4.6 5.0 4.4 4.9 5.4 4.8
##  [13] 4.8 4.3 5.8 5.7 5.4 5.1 5.7 5.1 5.4 5.1 4.6 5.1
##  [25] 4.8 5.0 5.0 5.2 5.2 4.7 4.8 5.4 5.2 5.5 4.9 5.0
##  [37] 5.5 4.9 4.4 5.1 5.0 4.5 4.4 5.0 5.1 4.8 5.1 4.6
##  [49] 5.3 5.0 7.0 6.4 6.9 5.5 6.5 5.7 6.3 4.9 6.6 5.2
##  [61] 5.0 5.9 6.0 6.1 5.6 6.7 5.6 5.8 6.2 5.6 5.9 6.1
##  [73] 6.3 6.1 6.4 6.6 6.8 6.7 6.0 5.7 5.5 5.5 5.8 6.0
##  [85] 5.4 6.0 6.7 6.3 5.6 5.5 5.5 6.1 5.8 5.0 5.6 5.7
##  [97] 5.7 6.2 5.1 5.7 6.3 5.8 7.1 6.3 6.5 7.6 4.9 7.3
## [109] 6.7 7.2 6.5 6.4 6.8 5.7 5.8 6.4 6.5 7.7 7.7 6.0
## [121] 6.9 5.6 7.7 6.3 6.7 7.2 6.2 6.1 6.4 7.2 7.4 7.9
## [133] 6.4 6.3 6.1 7.7 6.3 6.4 6.0 6.9 6.7 6.9 5.8 6.8
## [145] 6.7 6.7 6.3 6.5 6.2 5.9
\end{verbatim}

\begin{Shaded}
\begin{Highlighting}[]
\CommentTok{#subset berdasarkan posisi}
\NormalTok{iris_tbl[[}\DecValTok{1}\NormalTok{]]}
\end{Highlighting}
\end{Shaded}

\begin{verbatim}
##   [1] 5.1 4.9 4.7 4.6 5.0 5.4 4.6 5.0 4.4 4.9 5.4 4.8
##  [13] 4.8 4.3 5.8 5.7 5.4 5.1 5.7 5.1 5.4 5.1 4.6 5.1
##  [25] 4.8 5.0 5.0 5.2 5.2 4.7 4.8 5.4 5.2 5.5 4.9 5.0
##  [37] 5.5 4.9 4.4 5.1 5.0 4.5 4.4 5.0 5.1 4.8 5.1 4.6
##  [49] 5.3 5.0 7.0 6.4 6.9 5.5 6.5 5.7 6.3 4.9 6.6 5.2
##  [61] 5.0 5.9 6.0 6.1 5.6 6.7 5.6 5.8 6.2 5.6 5.9 6.1
##  [73] 6.3 6.1 6.4 6.6 6.8 6.7 6.0 5.7 5.5 5.5 5.8 6.0
##  [85] 5.4 6.0 6.7 6.3 5.6 5.5 5.5 6.1 5.8 5.0 5.6 5.7
##  [97] 5.7 6.2 5.1 5.7 6.3 5.8 7.1 6.3 6.5 7.6 4.9 7.3
## [109] 6.7 7.2 6.5 6.4 6.8 5.7 5.8 6.4 6.5 7.7 7.7 6.0
## [121] 6.9 5.6 7.7 6.3 6.7 7.2 6.2 6.1 6.4 7.2 7.4 7.9
## [133] 6.4 6.3 6.1 7.7 6.3 6.4 6.0 6.9 6.7 6.9 5.8 6.8
## [145] 6.7 6.7 6.3 6.5 6.2 5.9
\end{verbatim}

Dibandingkan dengan data frame, tibble lebih ketat: tibble tidak pernah
melakukan \emph{partial matching}, dan mereka akan menghasilkan
peringatan jika kolom yang kita coba akses tidak ada.

\section{Merapikan Data}\label{merapikan-data}

Sebelum memulai analisa terhadap data yang kita miliki, umumnya kita
akan merapikan data yang akan kita gunakan. Tujuannya adalah agar data
yang akan digunakan sudah siap untuk dilakukan analisa dengan software
tertentu seperti \texttt{R}, dimana pada dataset perlu jelas antara
variabel dan nilai (\emph{value}), serta untuk mempermudah dalah
memperoleh informasi pada data. Berikut adalah beberapa contoh dataset
yang dapat pembaca cermati terkait manakah data yang telah rapi
(\emph{tidy data}) dan mana yang belum (\emph{messy data}):

\begin{Shaded}
\begin{Highlighting}[]
\CommentTok{# Install paket dataset EDAWR}
\CommentTok{# install.packages("devtools")}
\CommentTok{# devtools::install_github("rstudio/EDAWR")}

\CommentTok{# hilangkan tanda # jika pembaca belum menginstall}
\end{Highlighting}
\end{Shaded}

\begin{Shaded}
\begin{Highlighting}[]
\KeywordTok{library}\NormalTok{(EDAWR)}
\CommentTok{# memuat dataset}
\NormalTok{storms <-}\StringTok{ }\NormalTok{EDAWR}\OperatorTok{::}\NormalTok{storms}
\NormalTok{cases}
\end{Highlighting}
\end{Shaded}

\begin{verbatim}
##   country  2011  2012  2013
## 1      FR  7000  6900  7000
## 2      DE  5800  6000  6200
## 3      US 15000 14000 13000
\end{verbatim}

\begin{Shaded}
\begin{Highlighting}[]
\NormalTok{pollution}
\end{Highlighting}
\end{Shaded}

\begin{verbatim}
##       city  size amount
## 1 New York large     23
## 2 New York small     14
## 3   London large     22
## 4   London small     16
## 5  Beijing large    121
## 6  Beijing small     56
\end{verbatim}

Sebelum kita melakukan analisa di dataset tersebut, kita harus tahu
terlebih dahulu apa saja syarat suatu dataset dikatakan rapi
(\emph{tidy}). Berikut adalah syaratnya:

\begin{itemize}
\tightlist
\item
  Setiap variabel harus memiliki kolomnya sendiri
\item
  Setiap observasi harus memiliki barisnya sendiri
\item
  Setiap nilai berada pada sel tersendiri
\end{itemize}

Ketiga syarat tersebut saling berhubungan sehingga jika salah satu
syarat tersebut tidak terpenuhi, maka dataset belum bisa dikatakan
\emph{tidy}. Ketiga syarat tersebut dapat divisualisasikan melalui
Gambar \ref{fig:tidy}

\begin{figure}

{\centering \includegraphics[width=8.14in]{tidy} 

}

\caption{Visualisasi 3 rule tidy data}\label{fig:tidy}
\end{figure}

Pada dataset \texttt{storms} terdapat 4 buah kolom dan 6 buah baris.
Masing-masing kolom menyatakan variabel pada masing-masing observasi
seperti nama badai , kecepatan angin, tekanan dan waktu . Ketiga syarat
kerapihan data sudah terpenuhi pada data tersebut sehingga kita bisa
melakukan analisa terhadap data tersebut, misalnya kecepatan angin dan
tekanan pada masing-masing badai. Selain itu kita juga dapat dengan
mudah menginput variabel baru pada dataset tersebut, misal: rasio
(kecepatan angin/tekanan).

Berikut adalah contoh bagaimana kita dapat dengan mudah menarik nilai
variabel pada masing-masing kolom dan membentuk variabel baru pada
dataset tersebut:

\begin{Shaded}
\begin{Highlighting}[]
\CommentTok{# subset variabel}
\NormalTok{storms}\OperatorTok{$}\NormalTok{storm}
\end{Highlighting}
\end{Shaded}

\begin{verbatim}
## [1] "Alberto" "Alex"    "Allison" "Ana"     "Arlene" 
## [6] "Arthur"
\end{verbatim}

\begin{Shaded}
\begin{Highlighting}[]
\NormalTok{storms}\OperatorTok{$}\NormalTok{wind}
\end{Highlighting}
\end{Shaded}

\begin{verbatim}
## [1] 110  45  65  40  50  45
\end{verbatim}

\begin{Shaded}
\begin{Highlighting}[]
\NormalTok{storms}\OperatorTok{$}\NormalTok{pressure}
\end{Highlighting}
\end{Shaded}

\begin{verbatim}
## [1] 1007 1009 1005 1013 1010 1010
\end{verbatim}

\begin{Shaded}
\begin{Highlighting}[]
\NormalTok{storms}\OperatorTok{$}\NormalTok{date}
\end{Highlighting}
\end{Shaded}

\begin{verbatim}
## [1] "2000-08-03" "1998-07-27" "1995-06-03" "1997-06-30"
## [5] "1999-06-11" "1996-06-17"
\end{verbatim}

\begin{Shaded}
\begin{Highlighting}[]
\CommentTok{# membuat variabel baru}
\NormalTok{storms_new <-}\StringTok{ }\NormalTok{storms}
\NormalTok{storms_new}\OperatorTok{$}\NormalTok{ratio <-}\StringTok{ }\NormalTok{storms_new}\OperatorTok{$}\NormalTok{wind}\OperatorTok{/}\NormalTok{storms_new}\OperatorTok{$}\NormalTok{pressure}
\NormalTok{storms_new}
\end{Highlighting}
\end{Shaded}

\begin{verbatim}
##     storm wind pressure       date   ratio
## 1 Alberto  110     1007 2000-08-03 0.10924
## 2    Alex   45     1009 1998-07-27 0.04460
## 3 Allison   65     1005 1995-06-03 0.06468
## 4     Ana   40     1013 1997-06-30 0.03949
## 5  Arlene   50     1010 1999-06-11 0.04950
## 6  Arthur   45     1010 1996-06-17 0.04455
\end{verbatim}

Pada dataset \texttt{cases} terdapat 3 buah kolom dan 3 baris. Pada
kolom pertama berupa kode Negara, sedangkan kolom sisanya merupakan
tahun. Jika kita perhatikan dengan seksama dataset tersebut merupakan
sebuah \emph{contingency table} dimana tabel tersebut menyatakan
frekuensi kejadian pada tahun tertentu dan negara tertentu. Dataset
tersebut belum dapat dikatan \emph{tidy} karena kolom \texttt{2011}
sampai \texttt{2013} merupakan sebuah nilai dari observasi dan bukan
sebuah variabel sehingga dataset tersebut masih tergolong dataset
\emph{messy}. Selain itu sangat sulit untuk dilakukan penarikan terhadap
nilai pada setiap kolom serta pembentukan variabel baru sebagai
pendukung analisa juga sulit dilakukan. Berikut adalah contoh melakukan
penarikan nilai / subset pada masing variabel:

\begin{Shaded}
\begin{Highlighting}[]
\NormalTok{cases}\OperatorTok{$}\NormalTok{country}
\end{Highlighting}
\end{Shaded}

\begin{verbatim}
## [1] "FR" "DE" "US"
\end{verbatim}

\begin{Shaded}
\begin{Highlighting}[]
\KeywordTok{names}\NormalTok{(cases[}\OperatorTok{-}\DecValTok{1}\NormalTok{])}
\end{Highlighting}
\end{Shaded}

\begin{verbatim}
## [1] "2011" "2012" "2013"
\end{verbatim}

\begin{Shaded}
\begin{Highlighting}[]
\KeywordTok{unlist}\NormalTok{(cases[}\DecValTok{1}\OperatorTok{:}\DecValTok{3}\NormalTok{, }\DecValTok{2}\OperatorTok{:}\DecValTok{4}\NormalTok{])}
\end{Highlighting}
\end{Shaded}

\begin{verbatim}
## 20111 20112 20113 20121 20122 20123 20131 20132 20133 
##  7000  5800 15000  6900  6000 14000  7000  6200 13000
\end{verbatim}

Pada dataset \texttt{pollution}terdapat 3 buah kolom dan 6 baris.
Masing-masing kolom menyatakan lokasi berupa nama kota, keterangan
ukuran partikel, serta nilai dari ukuran partikel. Beberapa dari kita
mungkin menganggap dataset ini telah memenuhi syarat kerapihan data.
Namun, coba kita cermati jika mita ingin membuat variabel baru terkait
dengan berapa rentang ukuran partikel (range ukuran partikel) pada
masing-masing kota. Hal tersebut tentu sangat sulit dilakukan pada
dataset tersebut, namun dataset tersebut memungkinkan kita dengan mudah
mengambil nilai dari masing-masing variabelnya seperti contoh berikut:

\begin{Shaded}
\begin{Highlighting}[]
\NormalTok{pollution}\OperatorTok{$}\NormalTok{city}
\end{Highlighting}
\end{Shaded}

\begin{verbatim}
## [1] "New York" "New York" "London"   "London"  
## [5] "Beijing"  "Beijing"
\end{verbatim}

\begin{Shaded}
\begin{Highlighting}[]
\NormalTok{pollution}\OperatorTok{$}\NormalTok{size}
\end{Highlighting}
\end{Shaded}

\begin{verbatim}
## [1] "large" "small" "large" "small" "large" "small"
\end{verbatim}

\begin{Shaded}
\begin{Highlighting}[]
\NormalTok{pollution}\OperatorTok{$}\NormalTok{amount}
\end{Highlighting}
\end{Shaded}

\begin{verbatim}
## [1]  23  14  22  16 121  56
\end{verbatim}

Berdasarkan contoh-contoh tersebut pada pembahasan kali ini penulis akan
menjelaskan bagaiman cara melakukan perapihan data menggunakan library
\texttt{tidyr}. Sebelum kita melakukannya berikut adalah sintaks untuk
menginstall library tersebut:

\begin{Shaded}
\begin{Highlighting}[]
\CommentTok{# memasang paket}
\KeywordTok{install.packages}\NormalTok{(}\StringTok{"tidyr"}\NormalTok{)}
\end{Highlighting}
\end{Shaded}

\begin{Shaded}
\begin{Highlighting}[]
\CommentTok{# memuat paket}
\KeywordTok{library}\NormalTok{(tidyr)}
\end{Highlighting}
\end{Shaded}

\subsection{Gather}\label{gather}

Pada dataset \texttt{cases} kolom \texttt{2011} sampai \texttt{2013}
perlu dijadikan satu variabel yaitu tahun. untuk melakukannya kita dapat
menggunakan fungsi \texttt{gather()}. Secara sederhana fungsi tersebut
dapat dituliskan dengan format sebagai berikut:

\begin{Shaded}
\begin{Highlighting}[]
\KeywordTok{gather}\NormalTok{(data, key, value, ...)}
\end{Highlighting}
\end{Shaded}

\begin{quote}
\textbf{Note: }

\begin{itemize}
\tightlist
\item
  \textbf{data}: data frame
\item
  \textbf{key, value}: nama kunci dan kolom nilai yang akan dibuat di
  output
\item
  \textbf{\ldots{}}: Spesifikasi kolom untuk dikumpulkan. Nilai yang
  diizinkan adalah:

  \begin{itemize}
  \tightlist
  \item
    nama variabel
  \item
    jika kita ingin memilih semua variabel antara a dan e, gunakan a:e
  \item
    jika kita ingin mengecualikan nama kolom y gunakan -y
  \item
    untuk opsi lainnya, lihat: \texttt{dplyr::select()}
  \end{itemize}
\end{itemize}
\end{quote}

Berikut adalah contoh penerapannya pada dataset \texttt{cases}:

\begin{Shaded}
\begin{Highlighting}[]
\CommentTok{# Ubah dataset cases menjadi tibble simpan sebagai objek cases_new}
\KeywordTok{library}\NormalTok{(tibble)}
\NormalTok{cases_tbl <-}\StringTok{ }\KeywordTok{as_tibble}\NormalTok{(cases)}

\CommentTok{# print}
\NormalTok{cases_tbl}
\end{Highlighting}
\end{Shaded}

\begin{verbatim}
## # A tibble: 3 x 4
##   country `2011` `2012` `2013`
##   <chr>    <dbl>  <dbl>  <dbl>
## 1 FR        7000   6900   7000
## 2 DE        5800   6000   6200
## 3 US       15000  14000  13000
\end{verbatim}

\begin{Shaded}
\begin{Highlighting}[]
\CommentTok{# gather}
\NormalTok{cases_new <-}\StringTok{ }\KeywordTok{gather}\NormalTok{(cases_tbl, }
                    \CommentTok{# variabel kunci}
                    \DataTypeTok{key =} \StringTok{"year"}\NormalTok{,}
                    \CommentTok{# nilai variabel}
                    \DataTypeTok{value =} \StringTok{"frequency"}\NormalTok{,}
                    \CommentTok{# kecualikan kolom country}
                    \OperatorTok{-}\NormalTok{country)}

\CommentTok{# print}
\NormalTok{cases_new}
\end{Highlighting}
\end{Shaded}

\begin{verbatim}
## # A tibble: 9 x 3
##   country year  frequency
##   <chr>   <chr>     <dbl>
## 1 FR      2011       7000
## 2 DE      2011       5800
## 3 US      2011      15000
## 4 FR      2012       6900
## 5 DE      2012       6000
## 6 US      2012      14000
## 7 FR      2013       7000
## 8 DE      2013       6200
## 9 US      2013      13000
\end{verbatim}

Berdasarkan hasil yang diperoleh terlihat bahwa variabel tahun memiliki
jenis data karakter. Jenis data ini masih belum sesuai sehingga perlu
dikonversi agar menjadi jenis data numerik (\emph{dbl = double}). Untuk
melakukannya jalankan sintaks berikut:

\begin{Shaded}
\begin{Highlighting}[]
\CommentTok{# Ubah jenis variabel tahun menjadi numerik}
\NormalTok{cases_new}\OperatorTok{$}\NormalTok{year <-}\StringTok{ }\KeywordTok{as.numeric}\NormalTok{(cases_new}\OperatorTok{$}\NormalTok{year)}
\NormalTok{cases_new}
\end{Highlighting}
\end{Shaded}

\begin{verbatim}
## # A tibble: 9 x 3
##   country  year frequency
##   <chr>   <dbl>     <dbl>
## 1 FR       2011      7000
## 2 DE       2011      5800
## 3 US       2011     15000
## 4 FR       2012      6900
## 5 DE       2012      6000
## 6 US       2012     14000
## 7 FR       2013      7000
## 8 DE       2013      6200
## 9 US       2013     13000
\end{verbatim}

Data yang diperoleh sekaran telah rapi (\emph{tidy}), sehingga sudah
siap untuk dilakukan analisa data.

\subsection{Spread}\label{spread}

Fungsi \texttt{spread()} berkebalikan dengan \texttt{gather()}. Fungsi
\texttt{gather()} menggabungkan beberapa kolom menjadi 2 buah kolom
kolom kunci sedangkan \texttt{spread()} merubah dua kolom menjadi
beberapa kolom. Format sederhanya adalah sebagai berikut:

\begin{quote}
\textbf{Note: }

\begin{itemize}
\tightlist
\item
  \textbf{data}: data frame
\item
  \textbf{key}: nama kolom yang akan dijadikan heading pada kolom baru
\item
  \textbf{value}: nama kolom yang nilainya akan mengisi setiap sel
\end{itemize}
\end{quote}

Pada contoh kasus pada data \texttt{pollution}, kita dapat memisahkan
kolom 2 menjadi kolom baru yaitu kolom \texttt{big\ size} dan
\texttt{small\ size}. Untuk melakukannya jalankan sintaks berikut:

\begin{Shaded}
\begin{Highlighting}[]
\CommentTok{# merubah objek pollution menjadi tibble}
\NormalTok{pollution_tbl <-}\StringTok{ }\KeywordTok{as_tibble}\NormalTok{(pollution)}

\CommentTok{# print}
\NormalTok{pollution_tbl}
\end{Highlighting}
\end{Shaded}

\begin{verbatim}
## # A tibble: 6 x 3
##   city     size  amount
##   <chr>    <chr>  <dbl>
## 1 New York large     23
## 2 New York small     14
## 3 London   large     22
## 4 London   small     16
## 5 Beijing  large    121
## 6 Beijing  small     56
\end{verbatim}

\begin{Shaded}
\begin{Highlighting}[]
\CommentTok{# spread}
\NormalTok{pollution_new <-}\StringTok{ }\KeywordTok{spread}\NormalTok{(pollution_tbl,}
                        \DataTypeTok{key =}\NormalTok{ size,}
                        \DataTypeTok{value =}\NormalTok{ amount)}

\CommentTok{#print}
\NormalTok{pollution_new}
\end{Highlighting}
\end{Shaded}

\begin{verbatim}
## # A tibble: 3 x 3
##   city     large small
##   <chr>    <dbl> <dbl>
## 1 Beijing    121    56
## 2 London      22    16
## 3 New York    23    14
\end{verbatim}

Terlihat bahwa data \texttt{pollution} tampak memnuhi syarat kerapihan
data (\emph{tidy}). Kita sekarang dapat menginput variabel baru dan
melakukan analisa terhadap data tersebut. Berikut adalah contoh
penerapannya:

\begin{Shaded}
\begin{Highlighting}[]
\CommentTok{# input variabel range (large-small)}
\NormalTok{pollution_new}\OperatorTok{$}\NormalTok{range <-}\StringTok{ }\NormalTok{pollution_new}\OperatorTok{$}\NormalTok{large }\OperatorTok{-}\StringTok{ }\NormalTok{pollution_new}\OperatorTok{$}\NormalTok{small}

\CommentTok{# print}
\NormalTok{pollution_new}
\end{Highlighting}
\end{Shaded}

\begin{verbatim}
## # A tibble: 3 x 4
##   city     large small range
##   <chr>    <dbl> <dbl> <dbl>
## 1 Beijing    121    56    65
## 2 London      22    16     6
## 3 New York    23    14     9
\end{verbatim}

Berdasarkan hasil yang diperoleh diketahui bahwa nilai range ukuran
partikel terbesar berada di Kota Beijing.

\subsection{Separate}\label{separate}

Fungsi \texttt{separate()} merupakan fungsi yang digunakan untuk
memisahkan sejumlah nilai pada sebuah kolom menjadi beberapa kolom
berdasarkan karakter pemisah yang ada di dalam nilai suatu kolom. Fungsi
ini berbeda dengan fungsi sebelumnya seperti \texttt{gather()} dan
\texttt{spread()} yang menggabung atau memisahkan 2 atau beberapa kolom.
Format sederhana fungsi \texttt{separate()} adalah sebagai berikut:

\begin{Shaded}
\begin{Highlighting}[]
\KeywordTok{separate}\NormalTok{(data, col, into, }\DataTypeTok{sep =} \StringTok{"[^[:alnum:]]+"}\NormalTok{, }\DataTypeTok{convert=} \OtherTok{TRUE}\NormalTok{)}
\end{Highlighting}
\end{Shaded}

\begin{quote}
\textbf{Note: }

\begin{itemize}
\tightlist
\item
  \textbf{data}: data frame.
\item
  \textbf{col}: Nama kolom yang tidak dikutip.
\item
  \textbf{into}: Vektor karakter menentukan nama variabel baru yang akan
  dibuat.
\item
  \textbf{sep}: Pemisah antar kolom:
\item
  Jika karakter, diartikan sebagai ekspresi reguler. Jika numerik,
  diartikan sebagai posisi untuk dibelah. Nilai-nilai positif mulai dari
  1 di ujung kiri string; nilai negatif mulai dari -1 di ujung kanan
  string.
\item
  \textbf{convert}: nilai logik. Jika bernilai TRUE maka kolom baru yang
  akan diperoleh akan dikonversi berdasarkan jenis data yang seharusnya.
\end{itemize}
\end{quote}

Pada dataset \texttt{storms} kita ingin memisahkan kolom \texttt{date}
menjadi beberapa kolom seperti \texttt{year}, \texttt{month}, dan
\texttt{day}, Kita dapat menggunakan fungsi \texttt{separate()} untuk
memisahkan nilai pada kolom tersebut berdasarkan karakter pemisah pada
nilai kolom tersebut dalam hal ini adalah ``-''. Berikut adalah cara
melakukannya:

\begin{Shaded}
\begin{Highlighting}[]
\CommentTok{# merubah storms menjadi tibble}
\NormalTok{storms_tbl <-}\StringTok{ }\KeywordTok{as_tibble}\NormalTok{(storms)}

\CommentTok{# print}
\NormalTok{storms_tbl}
\end{Highlighting}
\end{Shaded}

\begin{verbatim}
## # A tibble: 6 x 4
##   storm    wind pressure date      
##   <chr>   <int>    <int> <date>    
## 1 Alberto   110     1007 2000-08-03
## 2 Alex       45     1009 1998-07-27
## 3 Allison    65     1005 1995-06-03
## 4 Ana        40     1013 1997-06-30
## 5 Arlene     50     1010 1999-06-11
## 6 Arthur     45     1010 1996-06-17
\end{verbatim}

\begin{Shaded}
\begin{Highlighting}[]
\CommentTok{# separate}
\NormalTok{storms_new <-}\StringTok{ }\KeywordTok{separate}\NormalTok{(storms_tbl,}
                       \DataTypeTok{col =}\NormalTok{ date,}
                       \DataTypeTok{into =} \KeywordTok{c}\NormalTok{(}\StringTok{"year"}\NormalTok{,}\StringTok{"month"}\NormalTok{,}\StringTok{"days"}\NormalTok{),}
                       \DataTypeTok{sep =} \StringTok{"-"}\NormalTok{,}
                       \DataTypeTok{convert =} \OtherTok{TRUE}\NormalTok{)}

\CommentTok{# print}
\NormalTok{storms_new}
\end{Highlighting}
\end{Shaded}

\begin{verbatim}
## # A tibble: 6 x 6
##   storm    wind pressure  year month  days
##   <chr>   <int>    <int> <int> <int> <int>
## 1 Alberto   110     1007  2000     8     3
## 2 Alex       45     1009  1998     7    27
## 3 Allison    65     1005  1995     6     3
## 4 Ana        40     1013  1997     6    30
## 5 Arlene     50     1010  1999     6    11
## 6 Arthur     45     1010  1996     6    17
\end{verbatim}

Berdasarkan hasil yang diperoleh terlihat bahwa data telah terpisah
dengan benar yang ditunjukkan dari nilai yang terpisah dan jenis data
yang dihasilkan.

\subsection{Unite}\label{unite}

Fungsi \texttt{unite()} merupakan kebalikan dari fungsi
\texttt{separate()}, dimana fungsi ini menggabungkan sejumlah kolom
menjadi 1 kolom. Format sederhana untuk melakukanya disajikan sebagai
berikut:

\begin{Shaded}
\begin{Highlighting}[]
\KeywordTok{unite}\NormalTok{(data, col, ..., }\DataTypeTok{sep =} \StringTok{"_"}\NormalTok{)}
\end{Highlighting}
\end{Shaded}

\begin{quote}
\textbf{Note: }

\begin{itemize}
\tightlist
\item
  \textbf{data}: data frame.
\item
  \textbf{col}: nama kolom baru (tanpa tanda kutip) untuk ditambahkan.
\item
  \textbf{sep}: pemisah yang akan digunakan pada antar nilai.
\end{itemize}
\end{quote}

Pada dataset \texttt{storms\_new} kita ingin menggabungkan kembali kolom
\texttt{year}, \texttt{month}, dan \texttt{days} dengan karakter pemisah
``/''. Berikut adalah cara melakukannya:

\begin{Shaded}
\begin{Highlighting}[]
\CommentTok{# unite}
\NormalTok{storms_old <-}\StringTok{ }\KeywordTok{unite}\NormalTok{(storms_new,}
                   \DataTypeTok{col =} \StringTok{"date"}\NormalTok{,}
\NormalTok{                   year, month, days,}
                   \DataTypeTok{sep =} \StringTok{"-"}\NormalTok{)}

\CommentTok{# print}
\NormalTok{storms_old}
\end{Highlighting}
\end{Shaded}

\begin{verbatim}
## # A tibble: 6 x 4
##   storm    wind pressure date     
##   <chr>   <int>    <int> <chr>    
## 1 Alberto   110     1007 2000-8-3 
## 2 Alex       45     1009 1998-7-27
## 3 Allison    65     1005 1995-6-3 
## 4 Ana        40     1013 1997-6-30
## 5 Arlene     50     1010 1999-6-11
## 6 Arthur     45     1010 1996-6-17
\end{verbatim}

\begin{Shaded}
\begin{Highlighting}[]
\CommentTok{# ubah jenis kolom menjadi date}
\NormalTok{storms_old}\OperatorTok{$}\NormalTok{date <-}\StringTok{ }\KeywordTok{as.Date}\NormalTok{(storms_old}\OperatorTok{$}\NormalTok{date)}

\CommentTok{# print}
\NormalTok{storms_old}
\end{Highlighting}
\end{Shaded}

\begin{verbatim}
## # A tibble: 6 x 4
##   storm    wind pressure date      
##   <chr>   <int>    <int> <date>    
## 1 Alberto   110     1007 2000-08-03
## 2 Alex       45     1009 1998-07-27
## 3 Allison    65     1005 1995-06-03
## 4 Ana        40     1013 1997-06-30
## 5 Arlene     50     1010 1999-06-11
## 6 Arthur     45     1010 1996-06-17
\end{verbatim}

\section{Transformasi Data}\label{transformasi-data}

Data frame merupakan struktur data utama dalam statistik dan dalam
\texttt{R}. Struktur dasar data frame ialah ada satu observasi tiap
baris dan setiap kolom mewakili variabel, ukuran, fitur, atau
karakteristik pengamatan itu yang telah dijelaskan pada bagian
sebelumya. \texttt{R} memiliki implementasi internal data frame yang
kemungkinan besar akan kita gunakan paling sering. Namun, ada paket di
CRAN yang mengimplementasikan data frame layaknya basis data relasional
yang memungkinkan kita untuk beroperasi pada data frame yang sangat
besar.

Mengingat pentingnya mengelola dat frame, penting bagi kita untuk
memiliki alat yang baik untuk melakukannya. \texttt{R} memiliki beberapa
paket seperti fungsi \texttt{subset()} dan penggunaan operator ``{[}''
dan ``\$'' untuk mengekstrak himpunan bagian dari frame data. Namun,
operasi lain, seperti pemfilteran, pengurutan, dan pengelompokan data,
seringkali dapat menjadi operasi yang membosankan di \texttt{R} yang
sintaksisnya tidak terlalu intuitif. Paket \texttt{dplyr} dirancang
untuk mengurangi banyak masalah ini dan menyediakan serangkaian
rutinitas yang dioptimalkan secara khusus untuk menangani data frame.

\subsection{Paket dplyr}\label{paket-dplyr}

Paket \texttt{dplyr} dikembangkan oleh \textbf{Hadley Wickham} dari
\textbf{RStudio} dan merupakan versi yang dioptimalkan dari paket
\texttt{plyr}-nya. Paket \texttt{dplyr} tidak menyediakan fungsionalitas
baru untuk \texttt{R} sendiri, dalam arti bahwa semua yang dilakukan
\texttt{dplyr} sudah dapat dilakukan dengan fungsi basis \texttt{R},
tetapi sangat menyederhanakan fungsi yang ada di \texttt{R}.

Salah satu kontribusi penting dari paket \texttt{dplyr} adalah ia
menyediakan ``\emph{grammar}'' (khususnya, kata kerja) untuk manipulasi
data dan untuk beroperasi pada data frame. Melalui \emph{grammar} ini,
kita dapat berkomunikasi dengan masuk akal apa yang telah kita lakukan
terhadap data frame dapat pula dipahami orang lain (dengan asumsi mereka
juga tahu \emph{grammar}-nya). Hal ini berguna karena memberikan
abstraksi untuk manipulasi data yang sebelumnya tidak ada. Kontribusi
lain yang bermanfaat adalah bahwa fungsi \texttt{dplyr} sangat cepat,
karena banyak operasi utama dikodekan dalam C++.

Pada bagian ini pembaca akan belajar \textbf{6} fungsi utama yang ada
pada paket \texttt{dplyr}. Fungsi tersebut antara lain:

\begin{enumerate}
\def\labelenumi{\arabic{enumi}.}
\tightlist
\item
  Mengambil sejumlah observasi berdasarkan nilainya (\texttt{filter()}).
\item
  Mengurutkan kembali baris data frame berdasarkan nilai pada sebuah
  atau beberapa variabel (\texttt{arrange()}).
\item
  Mengambil atau subset terhadap sebuah atau beberapa variabel
  berdasarkan nama variabel/kolom (\texttt{select()}).
\item
  Membuat variabel baru atau menambahkan kolom baru (\texttt{mutate()}).
\item
  Membuat ringkasan terhadap data frame (\texttt{summarize()})
\item
  Mengelompokkan operasi berdasarkan grup data (\texttt{group\_by()}).
\end{enumerate}

Keseluruhan fungsi tersebut format fungsi yang seragam, yaitu:

\begin{enumerate}
\def\labelenumi{\arabic{enumi}.}
\tightlist
\item
  Argumen pertama adalah data frame.
\item
  Argumen selanjutnya adalah deskripsi yang akan dilakukan terhadap data
  frame (filter, pengurutan kembali, membuat ringkasan, dll) menggunakan
  nama variabel (tanpa tanda kutip).
\item
  Hasil operasi yang diperoleh adalah data frame baru.
\end{enumerate}

Untuk menginstall dan memuat paket \texttt{dplyr} jalankan sintaks
berikut:

\begin{Shaded}
\begin{Highlighting}[]
\CommentTok{# Memasang paket}
\KeywordTok{install.packages}\NormalTok{(}\StringTok{"dplyr"}\NormalTok{)}
\end{Highlighting}
\end{Shaded}

\begin{Shaded}
\begin{Highlighting}[]
\CommentTok{# memuat paket}
\KeywordTok{library}\NormalTok{(dplyr)}
\end{Highlighting}
\end{Shaded}

\subsection{filter()}\label{filter}

Fungsi \texttt{filter()} digunakan untuk mengekstrak himpunan bagian
(subset) baris dari data frame. Fungsi ini mirip dengan fungsi
\texttt{subset()} yang ada di \texttt{R}. Secara sederhana format fungsi
\texttt{filter()} dapat dituliskan sebagai berikut:

\begin{Shaded}
\begin{Highlighting}[]
\KeywordTok{filter}\NormalTok{(data, ....)}
\end{Highlighting}
\end{Shaded}

\begin{quote}
\textbf{Note: }

\begin{itemize}
\tightlist
\item
  \textbf{data} : data frame
\item
  \textbf{\ldots{}.} : Predikat logis didefinisikan dalam istilah
  variabel dalam \textbf{data}. Beberapa kondisi digabungkan dengan \&
  (lihat Chapter 2 opeator relasi dan operator logika. Hanya baris
  tempat kondisi bernilai TRUE disimpan.
\end{itemize}
\end{quote}

Misalkan kita akan melakukan melakukan filter terhadap data frame
\texttt{pollution\_tbl} terhadap variabel \texttt{size} dengan kriteria
\texttt{large} dan \texttt{amount} \textgreater{} 12. Berikut adalah
sintaks yang digunakan:

\begin{Shaded}
\begin{Highlighting}[]
\KeywordTok{filter}\NormalTok{(pollution_tbl, size}\OperatorTok{==}\StringTok{"large"} \OperatorTok{&}\StringTok{ }\NormalTok{amount }\OperatorTok{>}\StringTok{ }\DecValTok{12}\NormalTok{)}
\end{Highlighting}
\end{Shaded}

\begin{verbatim}
## # A tibble: 3 x 3
##   city     size  amount
##   <chr>    <chr>  <dbl>
## 1 New York large     23
## 2 London   large     22
## 3 Beijing  large    121
\end{verbatim}

Jika menggunakan paket dasar \texttt{R}:

\begin{Shaded}
\begin{Highlighting}[]
\KeywordTok{subset}\NormalTok{(pollution_tbl,size}\OperatorTok{==}\StringTok{"large"} \OperatorTok{&}\StringTok{ }\NormalTok{amount }\OperatorTok{>}\StringTok{ }\DecValTok{12}\NormalTok{)}
\end{Highlighting}
\end{Shaded}

\begin{verbatim}
## # A tibble: 3 x 3
##   city     size  amount
##   <chr>    <chr>  <dbl>
## 1 New York large     23
## 2 London   large     22
## 3 Beijing  large    121
\end{verbatim}

Operator ``\textgreater{}'' merupakan operator relasi (lihat chapter 2:
operator relasi). Operator tersebut banyak digunakan untuk melakukan
filter terhadap variabel/kolom yang mengandung nilai numerik.

Operator ``=='' merupakan operator logika (lihat chapter 2: operator
logika). Operator tersebut digunakan untuk melakukan filter terhadap
sejumlah syarat atau kondisi yang kita tetapkan. Jika nilai yang
dihasilkan TRUE, maka hanya observasi tersebut yang akan ditampilkan.
Untuk lebih memahami penerapan masing-masing operator logika pada proses
filter perhatikan Gambar \ref{fig:filter} berikut:

\begin{figure}

{\centering \includegraphics[width=6.57in]{filter} 

}

\caption{Diagram operasi Boolean}\label{fig:filter}
\end{figure}

\begin{quote}
\textbf{Note: } Bagian yang di arsir adalah observasi yang akan
ditampilkan pada output.
\end{quote}

Salah satu bagian terpenting dan paling sering penulis gunakan pada
fungsi ini memfilter \emph{missing value} (melihat observasi yang
mengandung \emph{missing value} atau tidak melibatkan \emph{missing
value}). Berikut adalah contoh filter terhadap data pada
\texttt{pollution\_tbl} yang tidak mengandung \emph{missing value} dan
nilai \texttt{amount}\textgreater{}0.

\begin{Shaded}
\begin{Highlighting}[]
\KeywordTok{filter}\NormalTok{(pollution_tbl,}\OperatorTok{!}\NormalTok{(}\KeywordTok{is.na}\NormalTok{(amount)}\OperatorTok{|}\NormalTok{amount}\OperatorTok{<=}\DecValTok{0}\NormalTok{))}
\end{Highlighting}
\end{Shaded}

\begin{verbatim}
## # A tibble: 6 x 3
##   city     size  amount
##   <chr>    <chr>  <dbl>
## 1 New York large     23
## 2 New York small     14
## 3 London   large     22
## 4 London   small     16
## 5 Beijing  large    121
## 6 Beijing  small     56
\end{verbatim}

Berdasarkan hasil yang diperoleh seluruh data tidak ada yang di drop
sehingga dapat disimpulkan bahwa data tersebut tidak mengandung
\emph{missing value} dan nol.

\subsection{arrange()}\label{arrange}

Fungsi \texttt{arrange()} bekerja mirip dengan fungsi \texttt{filter()}
kecuali bahwa alih-alih memilih baris, fungsi ini mengubah urutan
observasinya (mengurutkan dari yang terbesar atau sebaliknya).
Dibutuhkan data frame dan sekumpulan nama kolom (atau ekspresi yang
lebih rumit) untuk dipesan. Jika kita memberikan lebih dari satu nama
kolom pada fungsi, setiap kolom tambahan akan digunakan untuk menentukan
urutan nilai yang sama berdasarkan nilai kolom sebelumnya.

Fungsi \texttt{arrange()} mirip dengan fungsi \texttt{order()} pada
paket dasar \texttt{R}. Format sederhana fungsi ini adalah sebagai
berikut:

\begin{Shaded}
\begin{Highlighting}[]
\KeywordTok{arrange}\NormalTok{(data, ....)}
\end{Highlighting}
\end{Shaded}

\begin{quote}
\textbf{Note: }

\begin{itemize}
\tightlist
\item
  \textbf{data} : data frame
\item
  \textbf{\ldots{}.} : daftar nama variabel yang tidak dikutip yang
  dipisahkan tanda koma, atau ekspresi yang melibatkan nama variabel.
  Gunakan \texttt{desc()} untuk mengurutkan variabel dalam urutan
  menurun.
\end{itemize}
\end{quote}

Misalkan kita ingin melihat urutan mobil pada data \texttt{mtcars}
berdasarkan penggunaan bahan bakar (\texttt{mpg}) dan bobot mobil
(\texttt{wt}) tersebut. Berikut adalah sintaks yang digunakan:

\begin{Shaded}
\begin{Highlighting}[]
\KeywordTok{data}\NormalTok{(}\StringTok{"mtcars"}\NormalTok{)}

\CommentTok{# Ubah mtcars menjadi tibble}
\NormalTok{mtcars<-}\StringTok{ }\KeywordTok{as_tibble}\NormalTok{(mtcars)}

\KeywordTok{arrange}\NormalTok{(mtcars, mpg, wt)}
\end{Highlighting}
\end{Shaded}

\begin{verbatim}
## # A tibble: 32 x 11
##      mpg   cyl  disp    hp  drat    wt  qsec    vs
##    <dbl> <dbl> <dbl> <dbl> <dbl> <dbl> <dbl> <dbl>
##  1  10.4     8  472    205  2.93  5.25  18.0     0
##  2  10.4     8  460    215  3     5.42  17.8     0
##  3  13.3     8  350    245  3.73  3.84  15.4     0
##  4  14.3     8  360    245  3.21  3.57  15.8     0
##  5  14.7     8  440    230  3.23  5.34  17.4     0
##  6  15       8  301    335  3.54  3.57  14.6     0
##  7  15.2     8  304    150  3.15  3.44  17.3     0
##  8  15.2     8  276.   180  3.07  3.78  18       0
##  9  15.5     8  318    150  2.76  3.52  16.9     0
## 10  15.8     8  351    264  4.22  3.17  14.5     0
## # ... with 22 more rows, and 3 more variables:
## #   am <dbl>, gear <dbl>, carb <dbl>
\end{verbatim}

Jika ingin urutan yang digunakan adalah dari yang terbesar ke terkecil
untuk kedua variabel tersebut jalankan sintaks berikut:

\begin{Shaded}
\begin{Highlighting}[]
\KeywordTok{arrange}\NormalTok{(mtcars, }\KeywordTok{desc}\NormalTok{(mpg), }\KeywordTok{desc}\NormalTok{(wt))}
\end{Highlighting}
\end{Shaded}

\begin{verbatim}
## # A tibble: 32 x 11
##      mpg   cyl  disp    hp  drat    wt  qsec    vs
##    <dbl> <dbl> <dbl> <dbl> <dbl> <dbl> <dbl> <dbl>
##  1  33.9     4  71.1    65  4.22  1.84  19.9     1
##  2  32.4     4  78.7    66  4.08  2.2   19.5     1
##  3  30.4     4  75.7    52  4.93  1.62  18.5     1
##  4  30.4     4  95.1   113  3.77  1.51  16.9     1
##  5  27.3     4  79      66  4.08  1.94  18.9     1
##  6  26       4 120.     91  4.43  2.14  16.7     0
##  7  24.4     4 147.     62  3.69  3.19  20       1
##  8  22.8     4 141.     95  3.92  3.15  22.9     1
##  9  22.8     4 108      93  3.85  2.32  18.6     1
## 10  21.5     4 120.     97  3.7   2.46  20.0     1
## # ... with 22 more rows, and 3 more variables:
## #   am <dbl>, gear <dbl>, carb <dbl>
\end{verbatim}

Jika menggunakan fungsi \texttt{order()}:

\begin{Shaded}
\begin{Highlighting}[]
\KeywordTok{attach}\NormalTok{(mtcars)}
\CommentTok{# urutan dari kecil ke besar}
\NormalTok{mtcars[}\KeywordTok{order}\NormalTok{(mpg, wt), ]}
\end{Highlighting}
\end{Shaded}

\begin{verbatim}
## # A tibble: 32 x 11
##      mpg   cyl  disp    hp  drat    wt  qsec    vs
##    <dbl> <dbl> <dbl> <dbl> <dbl> <dbl> <dbl> <dbl>
##  1  10.4     8  472    205  2.93  5.25  18.0     0
##  2  10.4     8  460    215  3     5.42  17.8     0
##  3  13.3     8  350    245  3.73  3.84  15.4     0
##  4  14.3     8  360    245  3.21  3.57  15.8     0
##  5  14.7     8  440    230  3.23  5.34  17.4     0
##  6  15       8  301    335  3.54  3.57  14.6     0
##  7  15.2     8  304    150  3.15  3.44  17.3     0
##  8  15.2     8  276.   180  3.07  3.78  18       0
##  9  15.5     8  318    150  2.76  3.52  16.9     0
## 10  15.8     8  351    264  4.22  3.17  14.5     0
## # ... with 22 more rows, and 3 more variables:
## #   am <dbl>, gear <dbl>, carb <dbl>
\end{verbatim}

\begin{Shaded}
\begin{Highlighting}[]
\CommentTok{# urutan dari besar ke kecil}
\NormalTok{mtcars[}\KeywordTok{order}\NormalTok{(}\OperatorTok{-}\NormalTok{mpg, }\OperatorTok{-}\NormalTok{wt), ]}
\end{Highlighting}
\end{Shaded}

\begin{verbatim}
## # A tibble: 32 x 11
##      mpg   cyl  disp    hp  drat    wt  qsec    vs
##    <dbl> <dbl> <dbl> <dbl> <dbl> <dbl> <dbl> <dbl>
##  1  33.9     4  71.1    65  4.22  1.84  19.9     1
##  2  32.4     4  78.7    66  4.08  2.2   19.5     1
##  3  30.4     4  75.7    52  4.93  1.62  18.5     1
##  4  30.4     4  95.1   113  3.77  1.51  16.9     1
##  5  27.3     4  79      66  4.08  1.94  18.9     1
##  6  26       4 120.     91  4.43  2.14  16.7     0
##  7  24.4     4 147.     62  3.69  3.19  20       1
##  8  22.8     4 141.     95  3.92  3.15  22.9     1
##  9  22.8     4 108      93  3.85  2.32  18.6     1
## 10  21.5     4 120.     97  3.7   2.46  20.0     1
## # ... with 22 more rows, and 3 more variables:
## #   am <dbl>, gear <dbl>, carb <dbl>
\end{verbatim}

\begin{quote}
\textbf{Note: } \emph{missing value} akan selalu diurutkan pada
observasi terakhir baik menggunakan urutan dari terbesar ke terkecil
maupun sebaliknya.
\end{quote}

\subsection{select()}\label{select}

Fungsi \texttt{select()} dapat digunakan untuk memilih kolom dari data
frame yang ingin kita fokuskan. Seringkali kita memiliki data frame yang
besar yang berisi semua data, tetapi setiap analisis yang diberikan
hanya menggunakan subset variabel atau pengamatan. Fungsi
\texttt{select()} memungkinkan kita untuk mendapatkan beberapa kolom
yang mungkin kita butuhkan.

Fungsi \texttt{select()} memiliki kesamaan dengan subset menggunakan
tanda ``{[}'' dan ``\$''. Perbedaanya adalah kita dapat melakukan hal
lebih melalui fungsi ini seperti memilih berdasarkan kriteria tertentu
menggunakan fungsi bantuan sebagai berikut:

\begin{enumerate}
\def\labelenumi{\arabic{enumi}.}
\tightlist
\item
  \texttt{starts\_with("abcd")}, pilih kolom yang memiliki awalan
  ``abcd''.
\item
  \texttt{end\_with("abcd")}, pilih kolom yang memiliki akhiran
  ``abcd''.
\item
  \texttt{contains("abcd")}, pilih kolom yang mengandung nama ``abcd''
\item
  \texttt{matches("(.)\textbackslash{}\textbackslash{}1")}, pilih
  variabel yang mengandung \emph{regular expression}. Fungsi ini memilih
  variabel yang mengandung perulangan karakter.
\item
  \texttt{num\_range("x",\ 1:3)}, cocokkan berdasarkan kolom dengan nama
  x1,x2,x3.
\end{enumerate}

Berdasarkan fungsi bantuan tersebut, fungsi \texttt{select()} lebih
powerfull dibandingkan dengan cara subset biasa serta lebih mudah dalam
melakukannya. Berikut adalah format dari fungsi \texttt{select()}:

\begin{Shaded}
\begin{Highlighting}[]
\KeywordTok{select}\NormalTok{(data, ....)}
\end{Highlighting}
\end{Shaded}

\begin{quote}
\textbf{Note: }

\begin{itemize}
\tightlist
\item
  \textbf{data} : data frame
\item
  \textbf{\ldots{}.} : Satu atau lebih ekspresi kutip yang dipisahkan
  oleh koma. kita dapat memperlakukan nama variabel seperti posisi,
  sehingga kita dapat menggunakan ekspresi seperti x: y untuk memilih
  rentang variabel.Nilai positif pilih variabel; nilai negatif drop
  variabel. Jika ekspresi pertama negatif, \texttt{select()} akan secara
  otomatis dimulai dengan semua variabel. Gunakan argumen bernama, mis.
  \texttt{new\_name\ =\ old\_name}, untuk mengganti nama variabel yang
  dipilih.
\end{itemize}
\end{quote}

Berikut adalah contoh penerapan \texttt{selct()} pada data frame
\texttt{flights}.

\begin{Shaded}
\begin{Highlighting}[]
\CommentTok{# memasang paket}
\CommentTok{# install.packages("nycflights13")}

\CommentTok{# memuat data frame}
\KeywordTok{library}\NormalTok{(nycflights13)}

\CommentTok{# data}
\NormalTok{flights}
\end{Highlighting}
\end{Shaded}

\begin{verbatim}
## # A tibble: 336,776 x 19
##     year month   day dep_time sched_dep_time dep_delay
##    <int> <int> <int>    <int>          <int>     <dbl>
##  1  2013     1     1      517            515         2
##  2  2013     1     1      533            529         4
##  3  2013     1     1      542            540         2
##  4  2013     1     1      544            545        -1
##  5  2013     1     1      554            600        -6
##  6  2013     1     1      554            558        -4
##  7  2013     1     1      555            600        -5
##  8  2013     1     1      557            600        -3
##  9  2013     1     1      557            600        -3
## 10  2013     1     1      558            600        -2
## # ... with 336,766 more rows, and 13 more variables:
## #   arr_time <int>, sched_arr_time <int>,
## #   arr_delay <dbl>, carrier <chr>, flight <int>,
## #   tailnum <chr>, origin <chr>, dest <chr>,
## #   air_time <dbl>, distance <dbl>, hour <dbl>,
## #   minute <dbl>, time_hour <dttm>
\end{verbatim}

\begin{Shaded}
\begin{Highlighting}[]
\CommentTok{# pilih kolom berdasarkan nama kolom}
\KeywordTok{select}\NormalTok{(flights, year, month, day)}
\end{Highlighting}
\end{Shaded}

\begin{verbatim}
## # A tibble: 336,776 x 3
##     year month   day
##    <int> <int> <int>
##  1  2013     1     1
##  2  2013     1     1
##  3  2013     1     1
##  4  2013     1     1
##  5  2013     1     1
##  6  2013     1     1
##  7  2013     1     1
##  8  2013     1     1
##  9  2013     1     1
## 10  2013     1     1
## # ... with 336,766 more rows
\end{verbatim}

\begin{Shaded}
\begin{Highlighting}[]
\CommentTok{# pilih seluruh kolom dari year sampai day}
\KeywordTok{select}\NormalTok{(flights, year}\OperatorTok{:}\NormalTok{day)}
\end{Highlighting}
\end{Shaded}

\begin{verbatim}
## # A tibble: 336,776 x 3
##     year month   day
##    <int> <int> <int>
##  1  2013     1     1
##  2  2013     1     1
##  3  2013     1     1
##  4  2013     1     1
##  5  2013     1     1
##  6  2013     1     1
##  7  2013     1     1
##  8  2013     1     1
##  9  2013     1     1
## 10  2013     1     1
## # ... with 336,766 more rows
\end{verbatim}

\begin{Shaded}
\begin{Highlighting}[]
\CommentTok{# drop kolom dari year sampai day}
\KeywordTok{select}\NormalTok{(flights, }\OperatorTok{-}\NormalTok{(year}\OperatorTok{:}\NormalTok{day))}
\end{Highlighting}
\end{Shaded}

\begin{verbatim}
## # A tibble: 336,776 x 16
##    dep_time sched_dep_time dep_delay arr_time
##       <int>          <int>     <dbl>    <int>
##  1      517            515         2      830
##  2      533            529         4      850
##  3      542            540         2      923
##  4      544            545        -1     1004
##  5      554            600        -6      812
##  6      554            558        -4      740
##  7      555            600        -5      913
##  8      557            600        -3      709
##  9      557            600        -3      838
## 10      558            600        -2      753
## # ... with 336,766 more rows, and 12 more variables:
## #   sched_arr_time <int>, arr_delay <dbl>,
## #   carrier <chr>, flight <int>, tailnum <chr>,
## #   origin <chr>, dest <chr>, air_time <dbl>,
## #   distance <dbl>, hour <dbl>, minute <dbl>,
## #   time_hour <dttm>
\end{verbatim}

\begin{Shaded}
\begin{Highlighting}[]
\CommentTok{# pilih kolom dengan akhiran time}
\KeywordTok{select}\NormalTok{(flights, }\KeywordTok{ends_with}\NormalTok{(}\StringTok{"time"}\NormalTok{))}
\end{Highlighting}
\end{Shaded}

\begin{verbatim}
## # A tibble: 336,776 x 5
##    dep_time sched_dep_time arr_time sched_arr_time
##       <int>          <int>    <int>          <int>
##  1      517            515      830            819
##  2      533            529      850            830
##  3      542            540      923            850
##  4      544            545     1004           1022
##  5      554            600      812            837
##  6      554            558      740            728
##  7      555            600      913            854
##  8      557            600      709            723
##  9      557            600      838            846
## 10      558            600      753            745
## # ... with 336,766 more rows, and 1 more variable:
## #   air_time <dbl>
\end{verbatim}

\begin{Shaded}
\begin{Highlighting}[]
\CommentTok{# pilih kolom yang mengandung karakter "arr"}
\KeywordTok{select}\NormalTok{(flights, }\KeywordTok{contains}\NormalTok{(}\StringTok{"arr"}\NormalTok{))}
\end{Highlighting}
\end{Shaded}

\begin{verbatim}
## # A tibble: 336,776 x 4
##    arr_time sched_arr_time arr_delay carrier
##       <int>          <int>     <dbl> <chr>  
##  1      830            819        11 UA     
##  2      850            830        20 UA     
##  3      923            850        33 AA     
##  4     1004           1022       -18 B6     
##  5      812            837       -25 DL     
##  6      740            728        12 UA     
##  7      913            854        19 B6     
##  8      709            723       -14 EV     
##  9      838            846        -8 B6     
## 10      753            745         8 AA     
## # ... with 336,766 more rows
\end{verbatim}

Kita juga dapat menggunakan fungsi tambahan \texttt{everithing()} yang
berguna jika kita ingin memindahkan variabel yang menjadi fokus kita ke
awal data frame tanpa melakukan drop variabel. Berikut adalah contoh
sintaksnya:

\begin{Shaded}
\begin{Highlighting}[]
\CommentTok{# pindahkan kolom yang mengandung time di awal}
\KeywordTok{select}\NormalTok{(flights, }\KeywordTok{contains}\NormalTok{(}\StringTok{"time"}\NormalTok{), }\KeywordTok{everything}\NormalTok{())}
\end{Highlighting}
\end{Shaded}

\begin{verbatim}
## # A tibble: 336,776 x 19
##    dep_time sched_dep_time arr_time sched_arr_time
##       <int>          <int>    <int>          <int>
##  1      517            515      830            819
##  2      533            529      850            830
##  3      542            540      923            850
##  4      544            545     1004           1022
##  5      554            600      812            837
##  6      554            558      740            728
##  7      555            600      913            854
##  8      557            600      709            723
##  9      557            600      838            846
## 10      558            600      753            745
## # ... with 336,766 more rows, and 15 more variables:
## #   air_time <dbl>, time_hour <dttm>, year <int>,
## #   month <int>, day <int>, dep_delay <dbl>,
## #   arr_delay <dbl>, carrier <chr>, flight <int>,
## #   tailnum <chr>, origin <chr>, dest <chr>,
## #   distance <dbl>, hour <dbl>, minute <dbl>
\end{verbatim}

\subsection{mutate()}\label{mutate}

Fungsi \texttt{mutate()} ada untuk menghitung transformasi variabel
dalam data frame. Seringkali, kita ingin membuat variabel baru yang
berasal dari variabel yang ada dan fungsi \texttt{mutate()} menyediakan
antarmuka yang bersih untuk melakukan itu. Format yang digunakan adalah
sebagai berikut:

\begin{Shaded}
\begin{Highlighting}[]
\KeywordTok{mutate}\NormalTok{(data, ....)}
\end{Highlighting}
\end{Shaded}

\begin{quote}
\textbf{Note: }

\begin{itemize}
\tightlist
\item
  \textbf{data} : data frame
\item
  \textbf{\ldots{}.} : Pasangan nama-nilai ekspresi, masing-masing
  dengan panjang 1 atau panjang yang sama dengan jumlah baris dalam grup
  (jika menggunakan group\_by ()) atau di seluruh input (jika tidak
  menggunakan grup). Nama setiap argumen akan menjadi nama variabel
  baru, dan nilainya akan menjadi nilai yang sesuai. Gunakan nilai NULL
  dalam mutasi untuk menjatuhkan drop variabel lama, sehingga variabel
  baru menimpa variabel yang ada dengan nama yang sama.
\end{itemize}
\end{quote}

\begin{Shaded}
\begin{Highlighting}[]
\CommentTok{# subset data frame}
\NormalTok{flights_sml <-}\StringTok{ }\KeywordTok{select}\NormalTok{(flights,}
\NormalTok{  year}\OperatorTok{:}\NormalTok{day,}
  \KeywordTok{ends_with}\NormalTok{(}\StringTok{"delay"}\NormalTok{),}
\NormalTok{  distance,}
\NormalTok{  air_time}
\NormalTok{)}

\CommentTok{# mutate()}
\KeywordTok{mutate}\NormalTok{(flights_sml,}
  \DataTypeTok{gain =}\NormalTok{ arr_delay }\OperatorTok{-}\StringTok{ }\NormalTok{dep_delay,}
  \DataTypeTok{hours =}\NormalTok{ air_time }\OperatorTok{/}\StringTok{ }\DecValTok{60}\NormalTok{,}
  \DataTypeTok{gain_per_hour =}\NormalTok{ gain }\OperatorTok{/}\StringTok{ }\NormalTok{hours}
\NormalTok{)}
\end{Highlighting}
\end{Shaded}

\begin{verbatim}
## # A tibble: 336,776 x 10
##     year month   day dep_delay arr_delay distance
##    <int> <int> <int>     <dbl>     <dbl>    <dbl>
##  1  2013     1     1         2        11     1400
##  2  2013     1     1         4        20     1416
##  3  2013     1     1         2        33     1089
##  4  2013     1     1        -1       -18     1576
##  5  2013     1     1        -6       -25      762
##  6  2013     1     1        -4        12      719
##  7  2013     1     1        -5        19     1065
##  8  2013     1     1        -3       -14      229
##  9  2013     1     1        -3        -8      944
## 10  2013     1     1        -2         8      733
## # ... with 336,766 more rows, and 4 more variables:
## #   air_time <dbl>, gain <dbl>, hours <dbl>,
## #   gain_per_hour <dbl>
\end{verbatim}

Jika hanya ingin menyisakan variabel output fungsi \texttt{mutate()}
pada data frame (variabel lain di drop), kita dapat menggunakan fungsi
\texttt{transmute()}. Berikut adalah contoh sintaks yang digunakan:

\begin{Shaded}
\begin{Highlighting}[]
\KeywordTok{transmute}\NormalTok{(flights,}
  \DataTypeTok{gain =}\NormalTok{ arr_delay }\OperatorTok{-}\StringTok{ }\NormalTok{dep_delay,}
  \DataTypeTok{hours =}\NormalTok{ air_time }\OperatorTok{/}\StringTok{ }\DecValTok{60}\NormalTok{,}
  \DataTypeTok{gain_per_hour =}\NormalTok{ gain }\OperatorTok{/}\StringTok{ }\NormalTok{hours}
\NormalTok{)}
\end{Highlighting}
\end{Shaded}

\begin{verbatim}
## # A tibble: 336,776 x 3
##     gain hours gain_per_hour
##    <dbl> <dbl>         <dbl>
##  1     9 3.78           2.38
##  2    16 3.78           4.23
##  3    31 2.67          11.6 
##  4   -17 3.05          -5.57
##  5   -19 1.93          -9.83
##  6    16 2.5            6.4 
##  7    24 2.63           9.11
##  8   -11 0.883        -12.5 
##  9    -5 2.33          -2.14
## 10    10 2.3            4.35
## # ... with 336,766 more rows
\end{verbatim}

Adapaun fungsi-fungsi dan operator yang dapat digunakan pada
\texttt{mutate()} untuk membuat variabel baru adalah sebagai berikut:

\begin{enumerate}
\def\labelenumi{\arabic{enumi}.}
\tightlist
\item
  \textbf{Operator aritmatik} (+,-,*,/,\^{}, \%/\%, \%\%). operator
  aritmetik seperti \%/\% dan \%\% sangat berguna dalam memecah integer
  menjadi beberapa bagian seperti hasil bagi tanpa sisa (\%/\%) dan sisa
  hasil bagi (\%\%). Berikut adalah contoh penerapannya:
\end{enumerate}

\begin{Shaded}
\begin{Highlighting}[]
\KeywordTok{transmute}\NormalTok{(flights,}
\NormalTok{  dep_time,}
  \DataTypeTok{hour =}\NormalTok{ dep_time }\OperatorTok\StringTok{ }\DecValTok{100}\NormalTok{,}
  \DataTypeTok{minute =}\NormalTok{ dep_time }\OperatorTok\StringTok{ }\DecValTok{100}
\NormalTok{)}
\end{Highlighting}
\end{Shaded}

\begin{verbatim}
## # A tibble: 336,776 x 3
##    dep_time  hour minute
##       <int> <dbl>  <dbl>
##  1      517     5     17
##  2      533     5     33
##  3      542     5     42
##  4      544     5     44
##  5      554     5     54
##  6      554     5     54
##  7      555     5     55
##  8      557     5     57
##  9      557     5     57
## 10      558     5     58
## # ... with 336,766 more rows
\end{verbatim}

\begin{enumerate}
\def\labelenumi{\arabic{enumi}.}
\setcounter{enumi}{1}
\tightlist
\item
  \textbf{Fungsi aritmetik}
  (\texttt{log()},\texttt{sin()},\texttt{cos()},dll)
\item
  \textbf{Fungsi Offsets} (\texttt{lead()}dan \texttt{lag()}).
  memungkinkan kita untuk merujuk pada nilai-nilai memimpin atau
  tertinggal. Berikut adalah contoh penerapannya:
\end{enumerate}

\begin{Shaded}
\begin{Highlighting}[]
\NormalTok{(x <-}\StringTok{ }\DecValTok{1}\OperatorTok{:}\DecValTok{10}\NormalTok{)}
\end{Highlighting}
\end{Shaded}

\begin{verbatim}
##  [1]  1  2  3  4  5  6  7  8  9 10
\end{verbatim}

\begin{Shaded}
\begin{Highlighting}[]
\KeywordTok{lag}\NormalTok{(x)}
\end{Highlighting}
\end{Shaded}

\begin{verbatim}
##  [1] NA  1  2  3  4  5  6  7  8  9
\end{verbatim}

\begin{Shaded}
\begin{Highlighting}[]
\KeywordTok{lead}\NormalTok{(x)}
\end{Highlighting}
\end{Shaded}

\begin{verbatim}
##  [1]  2  3  4  5  6  7  8  9 10 NA
\end{verbatim}

\begin{enumerate}
\def\labelenumi{\arabic{enumi}.}
\setcounter{enumi}{3}
\tightlist
\item
  \textbf{Fungsi kumulatif}
  (\texttt{cumsum()},\texttt{cumprod()},\texttt{cummin()},\texttt{cummax()},
  dan \texttt{cummean()}). Jika kita membutuhkan agregat bergulir (mis.,
  Jumlah yang dihitung di atas jendela bergulir). Berikut adalah contoh
  penerapannya:
\end{enumerate}

\begin{Shaded}
\begin{Highlighting}[]
\NormalTok{x}
\end{Highlighting}
\end{Shaded}

\begin{verbatim}
##  [1]  1  2  3  4  5  6  7  8  9 10
\end{verbatim}

\begin{Shaded}
\begin{Highlighting}[]
\KeywordTok{cumsum}\NormalTok{(x)}
\end{Highlighting}
\end{Shaded}

\begin{verbatim}
##  [1]  1  3  6 10 15 21 28 36 45 55
\end{verbatim}

\begin{Shaded}
\begin{Highlighting}[]
\KeywordTok{cummean}\NormalTok{(x)}
\end{Highlighting}
\end{Shaded}

\begin{verbatim}
##  [1] 1.0 1.5 2.0 2.5 3.0 3.5 4.0 4.5 5.0 5.5
\end{verbatim}

\begin{enumerate}
\def\labelenumi{\arabic{enumi}.}
\setcounter{enumi}{4}
\item
  \textbf{Operator logik} (\textless{}, \textless{}=, \textgreater{},
  \textgreater{}=, !=). Jika kita melakukan urutan operasi logis yang
  kompleks, seringkali ide yang baik untuk menyimpan nilai sementara
  dalam variabel baru sehingga kita dapat memeriksa bahwa setiap langkah
  berfungsi seperti yang diharapkan.
\item
  Rangking (\texttt{min\_rank()}, \texttt{row\_number()},
  \texttt{dense\_rank()}, \texttt{percent\_rank()},
  \texttt{cume\_dist()}dan \texttt{ntile()}).
\end{enumerate}

\subsection{summarize() dan group\_by()}\label{summarize-dan-group_by}

Kita dapat membuat ringkasan data menggunakan fungsi
\texttt{summarize()}. Fungsi tersebut akan merubah data frame menjadi
sebuah baris berisi ringkasan data yang kita inginkan. Berikut adalh
contoh penerapannya:

\begin{Shaded}
\begin{Highlighting}[]
\KeywordTok{summarize}\NormalTok{(flights, }\DataTypeTok{delay =} \KeywordTok{mean}\NormalTok{(dep_delay, }\DataTypeTok{na.rm =} \OtherTok{TRUE}\NormalTok{))}
\end{Highlighting}
\end{Shaded}

\begin{verbatim}
## # A tibble: 1 x 1
##   delay
##   <dbl>
## 1  12.6
\end{verbatim}

FUngsi ini akan lebih berguna saat digunakan dengan fungsi
\texttt{group\_by()} sehingga dapat diperoleh ringkasan data pada setiap
grup. berikut adalah contoh penerapannya:

\begin{Shaded}
\begin{Highlighting}[]
\NormalTok{by_day <-}\StringTok{ }\KeywordTok{group_by}\NormalTok{(flights, year, month, day)}
    \KeywordTok{summarize}\NormalTok{(by_day, }\DataTypeTok{delay =} \KeywordTok{mean}\NormalTok{(dep_delay, }\DataTypeTok{na.rm =} \OtherTok{TRUE}\NormalTok{))}
\end{Highlighting}
\end{Shaded}

\begin{verbatim}
## # A tibble: 365 x 4
## # Groups:   year, month [12]
##     year month   day delay
##    <int> <int> <int> <dbl>
##  1  2013     1     1 11.5 
##  2  2013     1     2 13.9 
##  3  2013     1     3 11.0 
##  4  2013     1     4  8.95
##  5  2013     1     5  5.73
##  6  2013     1     6  7.15
##  7  2013     1     7  5.42
##  8  2013     1     8  2.55
##  9  2013     1     9  2.28
## 10  2013     1    10  2.84
## # ... with 355 more rows
\end{verbatim}

\subsection{Mengkombinasikan Beberapa Operasi Menggunakan Operator Pipe
(\%\textgreater{}\%)}\label{mengkombinasikan-beberapa-operasi-menggunakan-operator-pipe}

Operator pipa (\%\textgreater{}\%) sangat berguna untuk merangkai
bersama beberapa fungsi \texttt{dplyr} dalam suatu urutan operasi.
Perhatikan contoh sebelumnya dimana setiap kali kita ingin menerapkan
lebih dari satu fungsi, urutannya akan dimulai dalam urutan panggilan
fungsi bersarang yang sulit dibaca. Secara ringkas dapat kita tulis
sebagai berikut:

\begin{Shaded}
\begin{Highlighting}[]
\KeywordTok{third}\NormalTok{(}\KeywordTok{second}\NormalTok{(}\KeywordTok{first}\NormalTok{(x)))}
\end{Highlighting}
\end{Shaded}

Jika dituliskan menggunakan operator pipa akan menghasilkan sintak
berikut:

\begin{Shaded}
\begin{Highlighting}[]
\NormalTok{x }\OperatorTok
\StringTok{  }\KeywordTok{first}\NormalTok{() }\OperatorTok
\StringTok{  }\KeywordTok{second}\NormalTok{() }\OperatorTok
\StringTok{  }\KeywordTok{third}\NormalTok{()}
\end{Highlighting}
\end{Shaded}

Dengan menuliskannya melalui cara tersebut kita dapat membacanya lebih
mudah.

Misal kita ingin mengetahui hubungan antara variabel jarak
(\texttt{dist}) terhadap rata-rata delay (\texttt{arr\_delay}).
Langkah-langkah untuk melakukannya dengan menggunakan operator pipa
adalah sebagai berikut:

\begin{enumerate}
\def\labelenumi{\arabic{enumi}.}
\tightlist
\item
  Kelompokkan penerbangan berdasarkan destinasinya
  (\texttt{group\_by()}).
\item
  Hitung ringkasan data berdasarkan jarak, rata-rata delay, dan jumlah
  penerbangan.
\item
  Lakukan filter untuk membuang \emph{noisy point} (jika diperlukan).
  Dalam hal ini jumlah penerbangan \textgreater{} 20 dan tujuan
  penerbangan Honolulu (``HNL'') adalah \emph{outlier} atau \emph{noisy
  point}.
\end{enumerate}

Berikut adalah sintaks untuk melakukannya:

\begin{Shaded}
\begin{Highlighting}[]
\CommentTok{# Tanpa pipe operator}
\NormalTok{by_dest <-}\StringTok{ }\KeywordTok{group_by}\NormalTok{(flights, dest)}
\NormalTok{delay <-}\StringTok{ }\KeywordTok{summarize}\NormalTok{(by_dest,}
  \DataTypeTok{count =} \KeywordTok{n}\NormalTok{(),}
  \DataTypeTok{dist =} \KeywordTok{mean}\NormalTok{(distance, }\DataTypeTok{na.rm =} \OtherTok{TRUE}\NormalTok{),}
  \DataTypeTok{delay =} \KeywordTok{mean}\NormalTok{(arr_delay, }\DataTypeTok{na.rm =} \OtherTok{TRUE}\NormalTok{)}
\NormalTok{)}
\NormalTok{delay <-}\StringTok{ }\KeywordTok{filter}\NormalTok{(delay, count }\OperatorTok{>}\StringTok{ }\DecValTok{20}\NormalTok{, dest }\OperatorTok{!=}\StringTok{ "HNL"}\NormalTok{)}

\CommentTok{# Dengan pipe operator}
\KeywordTok{library}\NormalTok{(magrittr)}
\NormalTok{delays <-}\StringTok{ }\NormalTok{flights }\OperatorTok
\StringTok{  }\KeywordTok{group_by}\NormalTok{(dest) }\OperatorTok
\StringTok{  }\KeywordTok{summarize}\NormalTok{(}
  \DataTypeTok{count =} \KeywordTok{n}\NormalTok{(),}
  \DataTypeTok{dist =} \KeywordTok{mean}\NormalTok{(distance, }\DataTypeTok{na.rm =} \OtherTok{TRUE}\NormalTok{),}
  \DataTypeTok{delay =} \KeywordTok{mean}\NormalTok{(arr_delay, }\DataTypeTok{na.rm =} \OtherTok{TRUE}\NormalTok{)}
\NormalTok{  ) }\OperatorTok
\StringTok{  }\KeywordTok{filter}\NormalTok{(count }\OperatorTok{>}\StringTok{ }\DecValTok{20}\NormalTok{, dest }\OperatorTok{!=}\StringTok{ "HNL"}\NormalTok{)}

\CommentTok{# Print}
\NormalTok{delays}
\end{Highlighting}
\end{Shaded}

\begin{verbatim}
## # A tibble: 96 x 4
##    dest  count  dist delay
##    <chr> <int> <dbl> <dbl>
##  1 ABQ     254 1826   4.38
##  2 ACK     265  199   4.85
##  3 ALB     439  143  14.4 
##  4 ATL   17215  757. 11.3 
##  5 AUS    2439 1514.  6.02
##  6 AVL     275  584.  8.00
##  7 BDL     443  116   7.05
##  8 BGR     375  378   8.03
##  9 BHM     297  866. 16.9 
## 10 BNA    6333  758. 11.8 
## # ... with 86 more rows
\end{verbatim}

\begin{verbatim}
## `geom_smooth()` using method = 'loess' and formula 'y ~ x'
\end{verbatim}

\begin{figure}

{\centering \includegraphics[width=0.7\linewidth]{EnvStat_files/figure-latex/distvsave-1} 

}

\caption{Jarak vs rata-rata delay}\label{fig:distvsave}
\end{figure}

Berdasarkan Gambar \ref{fig:distvsave}, rata-rata delay meningkat
seiring dengan pertambahan jarak penerbangan.

\section{Referensi}\label{referensi-2}

\begin{enumerate}
\def\labelenumi{\arabic{enumi}.}
\tightlist
\item
  Wickham, H. Grolemund G. 2016. \textbf{R For Data Science: Import,
  Tidy, Transform, Visualize, And Model Data}. O'Reilly Media, Inc.
\item
  Peng, R.D. 2015. \textbf{Exploratory Data Analysis with R}. Leanpub
  book.
\item
  Dplyr Documentation. \url{https://dplyr.tidyverse.org/}
\item
  Quick-R. \textbf{Data Input}.
  \url{https://www.statmethods.net/input/index.html}
\item
  Quick-R. \textbf{Data Management}.
  \url{https://www.statmethods.net/management/index.html}
\item
  STHDA. \textbf{Importing Data Into R }.
  \url{http://www.sthda.com/english/wiki/importing-data-into-r}
\item
  STHDA. \textbf{Exporting Data From R}.
  \url{http://www.sthda.com/english/wiki/exporting-data-from-r}
\end{enumerate}

\part*{Visualisasi Data - R}\label{part-visualisasi-data---r}
\addcontentsline{toc}{part}{Visualisasi Data - R}

\chapter{Visualisasi Data Menggunakan Fungsi Dasar
R}\label{visualisasi-data-menggunakan-fungsi-dasar-r}

Visualisasi data merupakan bagian yang sangat penting untuk
mengkomunikasikan hasil analisa yang telah kita lakukan. Selain itu,
komunikasi juga membantu kita untuk memperoleh gambaran terkait data
selama proses analisa data sehingga membantu kita dalam memutuskan
metode analisa apa yang dapat kita terapkan pada data tersebut.

\texttt{R} memiliki library visualisasi yang sangat beragam, baik yang
merupakan fungsi dasar pada \texttt{R} maupun dari sumber lain seperti
ggplot dan lattice. Seluruh library visualisasi tersebut memiliki
kelebihan dan kekurangannya masing-masing.

Pada \emph{chapter} ini kita tidak akan membahas seluruh library
tersebut. Kita akab berfokus pada fungsi visualisasi dasar bawaan dari
\texttt{R}. kita akan mempelajari mengenai jenis visualisasi data sampai
dengan melakukan kustomisasi pada parameter grafik yang kita buat.

\section{Visualisasi Data Menggunakan Fungsi
plot()}\label{visualisasi-data-menggunakan-fungsi-plot}

Fungsi \texttt{plot()} merupakan fungsi umum yang digunakan untuk
membuat plot pada \texttt{R}. Format dasarnya adalah sebagai berikut:

\begin{Shaded}
\begin{Highlighting}[]
\KeywordTok{plot}\NormalTok{(x, y, }\DataTypeTok{type=}\StringTok{"p"}\NormalTok{)}
\end{Highlighting}
\end{Shaded}

\begin{quote}
\textbf{Note: }

\begin{itemize}
\tightlist
\item
  \textbf{x dan y}: titik koordinat plot Berupa variabel dengan panjang
  atau jumlah observasi yang sama.
\item
  \textbf{type}: jenis grafik yang hendak dibuat. Nilai yang dapat
  dimasukkan antara lain:
\item
  type=``p'' : membuat plot titik atau scatterplot. Nilai ini merupakan
  default pada fungsi \texttt{plot()}.
\item
  type=``l'' : membuat plot garis.
\item
  type=``b'' : membuat plot titik yang terhubung dengan garis.
\item
  type=``o'' : membuat plot titik yang ditimpa oleh garis.
\item
  type=``h'' : membuat plot garis vertikal dari titik ke garis y=0.
\item
  type=``s'' : membuat fungsi tangga.
\item
  type=``n'' : tidak membuat grafik plot sama sekali, kecuali plot dari
  axis. Dapat digunakan untuk mengatur tampilan suatu plot utama yang
  diikuti oleh sekelompok plot tambahan.
\end{itemize}
\end{quote}

Untuk lebih memahaminya berikut penulis akan sajikan contoh untuk
masing-masing grafik tersebut. Berikut adalah contoh sintaks dan hasil
plot yang disajikan pada Gambar \ref{fig:plot}:

\begin{Shaded}
\begin{Highlighting}[]
\CommentTok{# membuat vektor data }
\NormalTok{x <-}\StringTok{ }\KeywordTok{c}\NormalTok{(}\DecValTok{1}\OperatorTok{:}\DecValTok{10}\NormalTok{); y <-}\StringTok{ }\NormalTok{x}\OperatorTok{^}\DecValTok{2}
\end{Highlighting}
\end{Shaded}

\begin{Shaded}
\begin{Highlighting}[]
\CommentTok{# membagi jendela grafik menajdi 4 baris dan 2 kolom}
\KeywordTok{par}\NormalTok{(}\DataTypeTok{mfrow=}\KeywordTok{c}\NormalTok{(}\DecValTok{3}\NormalTok{,}\DecValTok{3}\NormalTok{))}

\CommentTok{# loop}
\NormalTok{type <-}\StringTok{ }\KeywordTok{c}\NormalTok{(}\StringTok{"p"}\NormalTok{,}\StringTok{"l"}\NormalTok{,}\StringTok{"b"}\NormalTok{,}\StringTok{"o"}\NormalTok{,}\StringTok{"h"}\NormalTok{,}\StringTok{"s"}\NormalTok{,}\StringTok{"n"}\NormalTok{)}
\ControlFlowTok{for}\NormalTok{ (i }\ControlFlowTok{in}\NormalTok{ type)\{}
  \KeywordTok{plot}\NormalTok{(x,y, }\DataTypeTok{type=}\NormalTok{ i,}
       \DataTypeTok{main=} \KeywordTok{paste}\NormalTok{(}\StringTok{"type="}\NormalTok{, i))}
\NormalTok{\}}
\end{Highlighting}
\end{Shaded}

\begin{figure}

{\centering \includegraphics[width=0.8\linewidth]{EnvStat_files/figure-latex/plot-1} 

}

\caption{Plot berbagai jenis setting type}\label{fig:plot}
\end{figure}

Pada contoh selanjutnya akan dilakukan plot terhadap dataset
\texttt{trees}. Untuk memuatnya jalankan sintaks berikut:

\begin{Shaded}
\begin{Highlighting}[]
\KeywordTok{library}\NormalTok{(tibble)}
\end{Highlighting}
\end{Shaded}

\begin{Shaded}
\begin{Highlighting}[]
\CommentTok{# memuat dataset}
\NormalTok{trees <-}\StringTok{ }\KeywordTok{as_tibble}\NormalTok{(trees)}

\CommentTok{# print }
\NormalTok{trees}
\end{Highlighting}
\end{Shaded}

\begin{verbatim}
## # A tibble: 31 x 3
##    Girth Height Volume
##    <dbl>  <dbl>  <dbl>
##  1   8.3     70   10.3
##  2   8.6     65   10.3
##  3   8.8     63   10.2
##  4  10.5     72   16.4
##  5  10.7     81   18.8
##  6  10.8     83   19.7
##  7  11       66   15.6
##  8  11       75   18.2
##  9  11.1     80   22.6
## 10  11.2     75   19.9
## # ... with 21 more rows
\end{verbatim}

Pada dataset tersebut kita ingin membuat scatterplot untuk melihat
korelasi antara variabel \texttt{Height} dan \texttt{Volume}. Untuk
melakukannya jalankan sintaks berikut:

\begin{Shaded}
\begin{Highlighting}[]
\KeywordTok{plot}\NormalTok{(trees}\OperatorTok{$}\NormalTok{Height, trees}\OperatorTok{$}\NormalTok{Volume)}
\end{Highlighting}
\end{Shaded}

\begin{Shaded}
\begin{Highlighting}[]
\CommentTok{# atau }
\KeywordTok{with}\NormalTok{(trees, }\KeywordTok{plot}\NormalTok{(Height, Volume))}
\end{Highlighting}
\end{Shaded}

\begin{figure}

{\centering \includegraphics[width=0.7\linewidth]{EnvStat_files/figure-latex/scatter-1} 

}

\caption{Scatterplot Height vs Volume}\label{fig:scatter}
\end{figure}

Kita juga dapat menggunakan formula untuk membuat scatterplot pada
Gambar \ref{fig:scatter}. Berikut adalah contoh sintaks yang digunakan:

\begin{Shaded}
\begin{Highlighting}[]
\NormalTok{x <-}\StringTok{ }\NormalTok{trees}\OperatorTok{$}\NormalTok{Height}
\NormalTok{y <-}\StringTok{ }\NormalTok{trees}\OperatorTok{$}\NormalTok{Volume}

\KeywordTok{plot}\NormalTok{(y}\OperatorTok{~}\NormalTok{x)}
\end{Highlighting}
\end{Shaded}

Fungsi \texttt{plot()} juga dapat digunakan untuk membentuk matriks
scatterplot. Untuk membuatnya kita hanya perlu memasukkan seluruh
dataset kedalam fungsi \texttt{plot()}. Berikut adalah sintaks dan
output yang dihasilkan berupa Gambar \ref{fig:scatter2}:

\begin{Shaded}
\begin{Highlighting}[]
\KeywordTok{plot}\NormalTok{(trees)}
\end{Highlighting}
\end{Shaded}

\begin{figure}

{\centering \includegraphics[width=0.8\linewidth]{EnvStat_files/figure-latex/scatter2-1} 

}

\caption{Matriks scatterplot dataset trees}\label{fig:scatter2}
\end{figure}

Selain itu jika kita memasukkan objek \texttt{lm()} yang merupakan
fungsi untuk melakukan operasi regresi linier pada fungsi
\texttt{plot()}, output yang dihasilkan berupa plot diagnostik yang
berguna untuk menguji asumsi model regresi linier. Berikut adalah contoh
sintaks dan output yang dihasilkan pada Gambar \ref{fig:diag}:

\begin{Shaded}
\begin{Highlighting}[]
\CommentTok{# membagi jendela grafik menjadi 2 baris dan 2 kolom}
\KeywordTok{par}\NormalTok{(}\DataTypeTok{mfrow=}\KeywordTok{c}\NormalTok{(}\DecValTok{2}\NormalTok{,}\DecValTok{2}\NormalTok{))}

\CommentTok{# plot}
\KeywordTok{plot}\NormalTok{(}\KeywordTok{lm}\NormalTok{(Volume}\OperatorTok{~}\NormalTok{Height, }\DataTypeTok{data=}\NormalTok{trees))}
\end{Highlighting}
\end{Shaded}

\begin{figure}

{\centering \includegraphics[width=0.8\linewidth]{EnvStat_files/figure-latex/diag-1} 

}

\caption{Plot diagnostik regresi linier}\label{fig:diag}
\end{figure}

Selain objek-objek tersebut, fungsi \texttt{plot()} akan banyak
digunakan dalam analisis statistika kita pada chapter lainnya.

\section{Matriks Scatterplot}\label{matriks-scatterplot}

Pada bagian sebelumnya kita telah belajar bagaimana membuat matriks
scatterplot mengggunakan fungsi \texttt{plot()}. Pada bagian ini kita
akan belajar cara membuat matriks scatterplot menggunakan fungsi
\texttt{pairs()}. Secara umum format fungsi dituliskan sebagai berikut:

\begin{Shaded}
\begin{Highlighting}[]
\KeywordTok{pairs}\NormalTok{(data, }\DataTypeTok{lower.panel=}\OtherTok{NULL}\NormalTok{)}
\end{Highlighting}
\end{Shaded}

\begin{quote}
\textbf{Note: }

\begin{itemize}
\tightlist
\item
  \textbf{data}: data frame
\item
  \textbf{lower.panel}: menampilkan atau tidak menampilkan panel bawah
\end{itemize}
\end{quote}

Untuk lebih memahami penggunaan fungsi tersebut, berikut akan disajikan
contoh penggunaannya pada dataset \texttt{iris}. Sebelum melakukannya
jalankan sintaks berikut untuk memuat dataset:

\begin{Shaded}
\begin{Highlighting}[]
\CommentTok{# memuat dataset irir}
\NormalTok{iris <-}\StringTok{ }\KeywordTok{as_tibble}\NormalTok{(iris)}

\CommentTok{# print}
\NormalTok{iris}
\end{Highlighting}
\end{Shaded}

\begin{verbatim}
## # A tibble: 150 x 5
##    Sepal.Length Sepal.Width Petal.Length Petal.Width
##           <dbl>       <dbl>        <dbl>       <dbl>
##  1          5.1         3.5          1.4         0.2
##  2          4.9         3            1.4         0.2
##  3          4.7         3.2          1.3         0.2
##  4          4.6         3.1          1.5         0.2
##  5          5           3.6          1.4         0.2
##  6          5.4         3.9          1.7         0.4
##  7          4.6         3.4          1.4         0.3
##  8          5           3.4          1.5         0.2
##  9          4.4         2.9          1.4         0.2
## 10          4.9         3.1          1.5         0.1
## # ... with 140 more rows, and 1 more variable:
## #   Species <fct>
\end{verbatim}

Untuk membuat matriks scatterplot kita hanya perlu memasukkan objek
\texttt{iris} kedalam fungsi \texttt{pairs()}. Berikut adalah sintaks
yang digunakan dan output yang dihasilkan pada Gambar \ref{fig:matscat}:

\begin{Shaded}
\begin{Highlighting}[]
\KeywordTok{pairs}\NormalTok{(iris)}
\end{Highlighting}
\end{Shaded}

\begin{figure}

{\centering \includegraphics[width=0.8\linewidth]{EnvStat_files/figure-latex/matscat-1} 

}

\caption{Matriks scatterplot iris}\label{fig:matscat}
\end{figure}

Kita dapat melakukan drop terhadap panel bawah grafik tersebut. Untuk
melakukannya kita perlu memasukkan parameter \texttt{lower.panel=NULL}.
Output yang dihasilkan akan tampak seperti pada Gambar
\ref{fig:matscat2}.

\begin{Shaded}
\begin{Highlighting}[]
\KeywordTok{pairs}\NormalTok{(iris, }\DataTypeTok{lower.panel=}\OtherTok{NULL}\NormalTok{)}
\end{Highlighting}
\end{Shaded}

\begin{figure}

{\centering \includegraphics[width=0.8\linewidth]{EnvStat_files/figure-latex/matscat2-1} 

}

\caption{Matriks scatterplot iris tanpa panel bawah}\label{fig:matscat2}
\end{figure}

Kita dapat merubah warna titik berdasarkan factor \texttt{Species}.
Langkah pertama yang perlu dilakukan adalah melakukan drop variabel
\texttt{Species} pada dataset dan memasukkan objek baru tanpa variabel
tersebut kedalam fungsi \texttt{pairs()}. Warna berdasarkan grup
diberikan dengan menambahkan parameter \texttt{col=} pada fungsi
\texttt{pairs()}. Berikut adalah contoh penerapannya dan output yang
dihasilkan pada Gambar \ref{fig:matscat3}:

\begin{Shaded}
\begin{Highlighting}[]
\CommentTok{# drop variabel Species}
\CommentTok{# simpan dataset baru pada objek iris2}
\NormalTok{iris2 <-}\StringTok{ }\NormalTok{iris[ ,}\DecValTok{1}\OperatorTok{:}\DecValTok{4}\NormalTok{]}

\CommentTok{# print}
\NormalTok{iris2}
\end{Highlighting}
\end{Shaded}

\begin{verbatim}
## # A tibble: 150 x 4
##    Sepal.Length Sepal.Width Petal.Length Petal.Width
##           <dbl>       <dbl>        <dbl>       <dbl>
##  1          5.1         3.5          1.4         0.2
##  2          4.9         3            1.4         0.2
##  3          4.7         3.2          1.3         0.2
##  4          4.6         3.1          1.5         0.2
##  5          5           3.6          1.4         0.2
##  6          5.4         3.9          1.7         0.4
##  7          4.6         3.4          1.4         0.3
##  8          5           3.4          1.5         0.2
##  9          4.4         2.9          1.4         0.2
## 10          4.9         3.1          1.5         0.1
## # ... with 140 more rows
\end{verbatim}

\begin{Shaded}
\begin{Highlighting}[]
\CommentTok{# spesifikasi vaktor warna titik berdasarkan spesies}
\NormalTok{my_col <-}\StringTok{ }\KeywordTok{c}\NormalTok{(}\StringTok{"#00AFBB"}\NormalTok{, }\StringTok{"#E7B800"}\NormalTok{, }\StringTok{"#FC4E07"}\NormalTok{)}

\CommentTok{# plot}
\KeywordTok{pairs}\NormalTok{(iris2, }\DataTypeTok{lower.panel=}\OtherTok{NULL}\NormalTok{,}
      \CommentTok{# spesifikasi warna}
      \DataTypeTok{col=}\NormalTok{ my_col[iris}\OperatorTok{$}\NormalTok{Species])}
\end{Highlighting}
\end{Shaded}

\begin{figure}

{\centering \includegraphics[width=0.8\linewidth]{EnvStat_files/figure-latex/matscat3-1} 

}

\caption{Matriks scatterplot iris tanpa panel bawah}\label{fig:matscat3}
\end{figure}

Kita juga dapat mengganti panel bawah menjadi nilai korelasi antar
variabel. Untuk melakukannya kita perlu mendefinisikan sebuah fungsi
untuk panel bawah dan panel atas (jika ingin warna titik berdasarkan
factor). Setelah fungsi panel bawah dan atas didefinisikan, langkah
selanjutnya adalah melakukan memasukkan nilainya kedalam fungsi
\texttt{pairs()}. Berikut adalah sintaks yang digunakan serta output
yang dihasilkan pada Gambar \ref{fig:matscat4}:

\begin{Shaded}
\begin{Highlighting}[]
\CommentTok{# membuat fungsi untuk menghitung}
\CommentTok{# nilai korelasi yang ditempatkan pada panel bawah}
\NormalTok{panel.cor <-}\StringTok{ }\ControlFlowTok{function}\NormalTok{(x, y)\{}
    \CommentTok{# definisi parameter grafik }
\NormalTok{    usr <-}\StringTok{ }\KeywordTok{par}\NormalTok{(}\StringTok{"usr"}\NormalTok{); }\KeywordTok{on.exit}\NormalTok{(}\KeywordTok{par}\NormalTok{(usr))}
    \KeywordTok{par}\NormalTok{(}\DataTypeTok{usr =} \KeywordTok{c}\NormalTok{(}\DecValTok{0}\NormalTok{, }\DecValTok{1}\NormalTok{, }\DecValTok{0}\NormalTok{, }\DecValTok{1}\NormalTok{))}
    \CommentTok{# menghitung koefisien korelas}
\NormalTok{    r <-}\StringTok{ }\KeywordTok{round}\NormalTok{(}\KeywordTok{cor}\NormalTok{(x, y), }\DataTypeTok{digits=}\DecValTok{2}\NormalTok{)}
    \CommentTok{# menambahkan text berdasarkan koefisien korelasi}
\NormalTok{    txt <-}\StringTok{ }\KeywordTok{paste0}\NormalTok{(}\StringTok{"R = "}\NormalTok{, r)}
    \CommentTok{# mengatur besar text sesuai besarnya nilai korelasi}
\NormalTok{    cex.cor <-}\StringTok{ }\FloatTok{0.8}\OperatorTok{/}\KeywordTok{strwidth}\NormalTok{(txt)}
    \KeywordTok{text}\NormalTok{(}\FloatTok{0.5}\NormalTok{, }\FloatTok{0.5}\NormalTok{, txt, }\DataTypeTok{cex =}\NormalTok{ cex.cor }\OperatorTok{*}\StringTok{ }\KeywordTok{abs}\NormalTok{(r))}
\NormalTok{\}}

\CommentTok{# kustomisasi panel atas agar}
\CommentTok{# warna titik berdasarkan factor}
\NormalTok{my_col <-}\StringTok{ }\KeywordTok{c}\NormalTok{(}\StringTok{"#00AFBB"}\NormalTok{, }\StringTok{"#E7B800"}\NormalTok{, }\StringTok{"#FC4E07"}\NormalTok{)}
\NormalTok{upper.panel<-}\ControlFlowTok{function}\NormalTok{(x, y)\{}
  \KeywordTok{points}\NormalTok{(x,y, }\DataTypeTok{col =}\NormalTok{ my_col[iris}\OperatorTok{$}\NormalTok{Species])}
\NormalTok{\}}

\KeywordTok{pairs}\NormalTok{(iris2,}
      \DataTypeTok{lower.panel=}\NormalTok{ panel.cor,}
      \DataTypeTok{upper.panel=}\NormalTok{ upper.panel)}
\end{Highlighting}
\end{Shaded}

\begin{figure}

{\centering \includegraphics[width=0.8\linewidth]{EnvStat_files/figure-latex/matscat4-1} 

}

\caption{Matriks scatterplot iris dengan koefisien korelasi}\label{fig:matscat4}
\end{figure}

Jika kita tidak ingin nilai korelasi ditampilkan di panel bawah, kita
dapat merubahnya sehingga dapat tampil pada panel atas bersamaan dengan
scatterplot. Untuk melakukannya kita perlu mendefinisikan fungsi pada
panel atas dan memasukkannya pada parameter \texttt{upper.panel=}.
Berikut adalah sintaks yang digunakan beserta output yang dihasilkan
pada Gambar \ref{fig:matscat5}:

\begin{Shaded}
\begin{Highlighting}[]
\CommentTok{# kustomisasi panel atas}
\NormalTok{upper.panel<-}\ControlFlowTok{function}\NormalTok{(x, y)\{}
  \KeywordTok{points}\NormalTok{(x,y, }\DataTypeTok{col=}\KeywordTok{c}\NormalTok{(}\StringTok{"#00AFBB"}\NormalTok{, }\StringTok{"#E7B800"}\NormalTok{, }\StringTok{"#FC4E07"}\NormalTok{)[iris}\OperatorTok{$}\NormalTok{Species])}
\NormalTok{  r <-}\StringTok{ }\KeywordTok{round}\NormalTok{(}\KeywordTok{cor}\NormalTok{(x, y), }\DataTypeTok{digits=}\DecValTok{2}\NormalTok{)}
\NormalTok{  txt <-}\StringTok{ }\KeywordTok{paste0}\NormalTok{(}\StringTok{"R = "}\NormalTok{, r)}
\NormalTok{  usr <-}\StringTok{ }\KeywordTok{par}\NormalTok{(}\StringTok{"usr"}\NormalTok{); }\KeywordTok{on.exit}\NormalTok{(}\KeywordTok{par}\NormalTok{(usr))}
  \KeywordTok{par}\NormalTok{(}\DataTypeTok{usr =} \KeywordTok{c}\NormalTok{(}\DecValTok{0}\NormalTok{, }\DecValTok{1}\NormalTok{, }\DecValTok{0}\NormalTok{, }\DecValTok{1}\NormalTok{))}
  \KeywordTok{text}\NormalTok{(}\FloatTok{0.5}\NormalTok{, }\FloatTok{0.9}\NormalTok{, txt)}
\NormalTok{\}}

\CommentTok{# plot}
\KeywordTok{pairs}\NormalTok{(iris2, }\DataTypeTok{lower.panel =} \OtherTok{NULL}\NormalTok{, }
      \DataTypeTok{upper.panel =}\NormalTok{ upper.panel)}
\end{Highlighting}
\end{Shaded}

\begin{figure}

{\centering \includegraphics[width=0.8\linewidth]{EnvStat_files/figure-latex/matscat5-1} 

}

\caption{Matriks scatterplot iris dengan koefisien korelasi di panel atas}\label{fig:matscat5}
\end{figure}

\section{Box plot}\label{box-plot}

Box plot pada \texttt{R} dapat dibuat menggunakan fungsi
\texttt{boxplot()}. Berikut adalah sintaks untuk membuat boxplot
variabel \texttt{Sepal.Lenght} pada dataset \texttt{iris} dan output
yang dihasilkan pada Gambar \ref{fig:boxplot}:

\begin{Shaded}
\begin{Highlighting}[]
\KeywordTok{boxplot}\NormalTok{(iris}\OperatorTok{$}\NormalTok{Sepal.Length)}
\end{Highlighting}
\end{Shaded}

\begin{figure}

{\centering \includegraphics[width=0.7\linewidth]{EnvStat_files/figure-latex/boxplot-1} 

}

\caption{Boxplot variabel Sepal.Length}\label{fig:boxplot}
\end{figure}

Boxplot juga dapat dibuat berdasarkan variabel factor. Hal ini berguna
untuk melihat perbedaan ditribusi data pada masing-masing grup. Pada
sintaks berikut dibuat boxplot berdasarkan variabel \texttt{Species}.
Output yang dihasilkan disajikan pada Gambar \ref{fig:boxplot2}:

\begin{Shaded}
\begin{Highlighting}[]
\KeywordTok{boxplot}\NormalTok{(iris}\OperatorTok{$}\NormalTok{Sepal.Length}\OperatorTok{~}\NormalTok{iris}\OperatorTok{$}\NormalTok{Species)}
\end{Highlighting}
\end{Shaded}

\begin{figure}

{\centering \includegraphics[width=0.7\linewidth]{EnvStat_files/figure-latex/boxplot2-1} 

}

\caption{Boxplot berdasarkan variabel species}\label{fig:boxplot2}
\end{figure}

Kita juga dapat mengubah warna outline dan box pada boxplot. Berikut
adalah contoh sintaks yang digunakan untuk melakukannya dan output yang
dihasilkan disajikan pada Gambar \ref{fig:boxplot3}:

\begin{Shaded}
\begin{Highlighting}[]
\KeywordTok{boxplot}\NormalTok{(iris}\OperatorTok{$}\NormalTok{Sepal.Length}\OperatorTok{~}\NormalTok{iris}\OperatorTok{$}\NormalTok{Species,}
        \CommentTok{# ubah warna outline menjadi steelblue}
        \DataTypeTok{border =} \StringTok{"steelblue"}\NormalTok{,}
        \CommentTok{# ubah warna box berdasarkan grup}
        \DataTypeTok{col=} \KeywordTok{c}\NormalTok{(}\StringTok{"#999999"}\NormalTok{, }\StringTok{"#E69F00"}\NormalTok{, }\StringTok{"#56B4E9"}\NormalTok{))}
\end{Highlighting}
\end{Shaded}

\begin{figure}

{\centering \includegraphics[width=0.7\linewidth]{EnvStat_files/figure-latex/boxplot3-1} 

}

\caption{Boxplot dengan warna berdasarkan spesies}\label{fig:boxplot3}
\end{figure}

Kita juga dapat membuat boxplot pada \emph{multiple group}. Data yang
digunakan untuk contoh tersebut adalah dataset \texttt{ToothGrowth}.
Berikut adalah sintaks untuk memuat dataset tersebut:

\begin{Shaded}
\begin{Highlighting}[]
\CommentTok{# memuat dataset sebagai tibble}
\NormalTok{ToothGrowth <-}\StringTok{ }\KeywordTok{as_tibble}\NormalTok{(ToothGrowth)}

\CommentTok{# print}
\NormalTok{ToothGrowth}
\end{Highlighting}
\end{Shaded}

\begin{verbatim}
## # A tibble: 60 x 3
##      len supp   dose
##    <dbl> <fct> <dbl>
##  1   4.2 VC      0.5
##  2  11.5 VC      0.5
##  3   7.3 VC      0.5
##  4   5.8 VC      0.5
##  5   6.4 VC      0.5
##  6  10   VC      0.5
##  7  11.2 VC      0.5
##  8  11.2 VC      0.5
##  9   5.2 VC      0.5
## 10   7   VC      0.5
## # ... with 50 more rows
\end{verbatim}

\begin{Shaded}
\begin{Highlighting}[]
\CommentTok{# ubah variable dose menjadi factor}
\NormalTok{ToothGrowth}\OperatorTok{$}\NormalTok{dose <-}\StringTok{ }\KeywordTok{as.factor}\NormalTok{(ToothGrowth}\OperatorTok{$}\NormalTok{dose)}

\CommentTok{# print}
\NormalTok{ToothGrowth}
\end{Highlighting}
\end{Shaded}

\begin{verbatim}
## # A tibble: 60 x 3
##      len supp  dose 
##    <dbl> <fct> <fct>
##  1   4.2 VC    0.5  
##  2  11.5 VC    0.5  
##  3   7.3 VC    0.5  
##  4   5.8 VC    0.5  
##  5   6.4 VC    0.5  
##  6  10   VC    0.5  
##  7  11.2 VC    0.5  
##  8  11.2 VC    0.5  
##  9   5.2 VC    0.5  
## 10   7   VC    0.5  
## # ... with 50 more rows
\end{verbatim}

Contoh sintaks dan output boxplot \emph{multiple group} disajikan pada
Gambar \ref{fig:boxplot4}:

\begin{Shaded}
\begin{Highlighting}[]
\KeywordTok{boxplot}\NormalTok{(len }\OperatorTok{~}\StringTok{ }\NormalTok{supp}\OperatorTok{*}\NormalTok{dose, }\DataTypeTok{data =}\NormalTok{ ToothGrowth,}
        \DataTypeTok{col =} \KeywordTok{c}\NormalTok{(}\StringTok{"white"}\NormalTok{, }\StringTok{"steelblue"}\NormalTok{))}
\end{Highlighting}
\end{Shaded}

\begin{figure}

{\centering \includegraphics[width=0.7\linewidth]{EnvStat_files/figure-latex/boxplot4-1} 

}

\caption{Boxplot multiple group}\label{fig:boxplot4}
\end{figure}

\section{Bar Plot}\label{bar-plot}

Barplot pada \texttt{R} dapat dibuat menggunakan fungsi
\texttt{barplot()}. Untuk lebih memahaminya berikut disajikan contoh
barplot menggunakan dataset \texttt{VADeaths}. Untuk memuatnya jalankan
sintaks berikut:

\begin{Shaded}
\begin{Highlighting}[]
\NormalTok{VADeaths}
\end{Highlighting}
\end{Shaded}

\begin{verbatim}
##       Rural Male Rural Female Urban Male Urban Female
## 50-54       11.7          8.7       15.4          8.4
## 55-59       18.1         11.7       24.3         13.6
## 60-64       26.9         20.3       37.0         19.3
## 65-69       41.0         30.9       54.6         35.1
## 70-74       66.0         54.3       71.1         50.0
\end{verbatim}

Contoh bar plot untuk variabel \texttt{Rural\ Male} disajikan pada
Gambar \ref{fig:barplot}:

\begin{Shaded}
\begin{Highlighting}[]
\KeywordTok{par}\NormalTok{(}\DataTypeTok{mfrow=}\KeywordTok{c}\NormalTok{(}\DecValTok{1}\NormalTok{,}\DecValTok{2}\NormalTok{))}
\KeywordTok{barplot}\NormalTok{(VADeaths[, }\StringTok{"Rural Male"}\NormalTok{], }\DataTypeTok{main=}\StringTok{"a"}\NormalTok{)}
\KeywordTok{barplot}\NormalTok{(VADeaths[, }\StringTok{"Rural Male"}\NormalTok{], }\DataTypeTok{main=}\StringTok{"b"}\NormalTok{, }\DataTypeTok{horiz=}\OtherTok{TRUE}\NormalTok{)}
\end{Highlighting}
\end{Shaded}

\begin{figure}

{\centering \includegraphics[width=0.7\linewidth]{EnvStat_files/figure-latex/barplot-1} 

}

\caption{a. bar plot vertikal; b. bar plot horizontal}\label{fig:barplot}
\end{figure}

\begin{Shaded}
\begin{Highlighting}[]
\KeywordTok{par}\NormalTok{(}\DataTypeTok{mfrow=}\KeywordTok{c}\NormalTok{(}\DecValTok{1}\NormalTok{,}\DecValTok{1}\NormalTok{))}
\end{Highlighting}
\end{Shaded}

Kita dapat mengubah warna pada masing-masing bar, baik outline bar
maupun box pada bar. Selain itu kita juga dapat mengubah nama grup yang
telah dihasilkan sebelumnya. Berikut sintaks untuk melakukannya dan
output yang dihasilkan pada Gambar \ref{fig:barplot2}:

\begin{Shaded}
\begin{Highlighting}[]
\KeywordTok{barplot}\NormalTok{(VADeaths[, }\StringTok{"Rural Male"}\NormalTok{],}
        \CommentTok{# ubah warna ouline menjadi steelblue}
        \DataTypeTok{border=}\StringTok{"steelblue"}\NormalTok{,}
        \CommentTok{# ubah wana box}
        \DataTypeTok{col=} \KeywordTok{c}\NormalTok{(}\StringTok{"grey"}\NormalTok{, }\StringTok{"yellow"}\NormalTok{, }\StringTok{"steelblue"}\NormalTok{, }\StringTok{"green"}\NormalTok{, }\StringTok{"orange"}\NormalTok{),}
        \CommentTok{# ubah nama grup dari A sampai E}
        \DataTypeTok{names.arg =}\NormalTok{ LETTERS[}\DecValTok{1}\OperatorTok{:}\DecValTok{5}\NormalTok{],}
        \CommentTok{# ubah orientasi menajadi horizontal}
        \DataTypeTok{horiz=}\OtherTok{TRUE}\NormalTok{)}
\end{Highlighting}
\end{Shaded}

\begin{figure}

{\centering \includegraphics[width=0.7\linewidth]{EnvStat_files/figure-latex/barplot2-1} 

}

\caption{Kustomisasi bar plot}\label{fig:barplot2}
\end{figure}

Untuk bar plot dengan \emph{multiple group}, tersedia dua pengaturan
posisi yaitu \emph{stacked bar plot}(menunjukkan proporsi penyusun pada
masing-masing grup) dan \emph{grouped bar plot}(melihat perbedaan
individual pada masing-masing grup). Pada Gambar \ref{fig:barplot3} dan
Gambar \ref{fig:barplot4} , disajikan kedua jenis bar plot tersebut.

\begin{Shaded}
\begin{Highlighting}[]
\CommentTok{# staked}
\KeywordTok{barplot}\NormalTok{(VADeaths,}
         \DataTypeTok{col =} \KeywordTok{c}\NormalTok{(}\StringTok{"lightblue"}\NormalTok{, }\StringTok{"mistyrose"}\NormalTok{, }\StringTok{"lightcyan"}\NormalTok{, }
                 \StringTok{"lavender"}\NormalTok{, }\StringTok{"cornsilk"}\NormalTok{),}
        \DataTypeTok{legend =} \KeywordTok{rownames}\NormalTok{(VADeaths))}
\end{Highlighting}
\end{Shaded}

\begin{figure}

{\centering \includegraphics[width=0.7\linewidth]{EnvStat_files/figure-latex/barplot3-1} 

}

\caption{Stacked bar plot}\label{fig:barplot3}
\end{figure}

\begin{Shaded}
\begin{Highlighting}[]
\CommentTok{# grouped}
\KeywordTok{barplot}\NormalTok{(VADeaths,}
         \DataTypeTok{col =} \KeywordTok{c}\NormalTok{(}\StringTok{"lightblue"}\NormalTok{, }\StringTok{"mistyrose"}\NormalTok{, }\StringTok{"lightcyan"}\NormalTok{, }
                 \StringTok{"lavender"}\NormalTok{, }\StringTok{"cornsilk"}\NormalTok{),}
        \DataTypeTok{legend =} \KeywordTok{rownames}\NormalTok{(VADeaths), }\DataTypeTok{beside =} \OtherTok{TRUE}\NormalTok{)}
\end{Highlighting}
\end{Shaded}

\begin{figure}

{\centering \includegraphics[width=0.7\linewidth]{EnvStat_files/figure-latex/barplot4-1} 

}

\caption{Grouped bar plot}\label{fig:barplot4}
\end{figure}

\section{Line Plot}\label{line-plot}

Line plot pada \texttt{R} dapat dibentuk menggunakan fungsi
\texttt{plot()}. Selain itu fungsi \texttt{lines()} dapat pula digunakan
untuk menambahkan line plot pada grafik. Berikut adalah sintaks untuk
membuat line plot dan outputnya pada Gambar \ref{fig:line}:

\begin{Shaded}
\begin{Highlighting}[]
\CommentTok{# Membuat vektor data}
\NormalTok{x <-}\StringTok{ }\KeywordTok{c}\NormalTok{(}\DecValTok{1}\OperatorTok{:}\DecValTok{20}\NormalTok{)}
\NormalTok{y <-}\StringTok{ }\DecValTok{2}\OperatorTok{*}\NormalTok{x}
\NormalTok{z <-}\StringTok{ }\NormalTok{x}\OperatorTok{^}\DecValTok{2}

\CommentTok{# Membuat line plot x vs y}
\KeywordTok{plot}\NormalTok{(y}\OperatorTok{~}\NormalTok{x, }\DataTypeTok{type=}\StringTok{"b"}\NormalTok{,}
     \DataTypeTok{lty=}\DecValTok{1}\NormalTok{,}
     \DataTypeTok{col=}\StringTok{"blue"}\NormalTok{)}

\CommentTok{# Menambahkan line plot x vs z}
\KeywordTok{lines}\NormalTok{(z}\OperatorTok{~}\NormalTok{x, }\DataTypeTok{type=}\StringTok{"o"}\NormalTok{,}
      \DataTypeTok{lty=}\DecValTok{2}\NormalTok{,}
      \DataTypeTok{col=}\StringTok{"red"}\NormalTok{)}

\CommentTok{# Menambahkan legend}
\KeywordTok{legend}\NormalTok{(}\StringTok{"topleft"}\NormalTok{, }\DataTypeTok{legend=}\KeywordTok{c}\NormalTok{(}\StringTok{"Line 1"}\NormalTok{, }\StringTok{"Line 2"}\NormalTok{),}
       \DataTypeTok{col=}\KeywordTok{c}\NormalTok{(}\StringTok{"red"}\NormalTok{, }\StringTok{"blue"}\NormalTok{), }\DataTypeTok{lty =} \DecValTok{1}\OperatorTok{:}\DecValTok{2}\NormalTok{, }\DataTypeTok{cex=}\FloatTok{0.8}\NormalTok{)}
\end{Highlighting}
\end{Shaded}

\begin{figure}

{\centering \includegraphics[width=0.7\linewidth]{EnvStat_files/figure-latex/line-1} 

}

\caption{Line plot}\label{fig:line}
\end{figure}

\section{Pie Chart}\label{pie-chart}

Pie chart digunakan untuk membuat visualisasi proporsi pada sebuah data.
Pie chart pada \texttt{R} dibuat menggunakan fungsi \texttt{pie()}.
Berikut adalah sintaks untuk membuat pie chart dan output yang
dihasilkan pada Gambar \ref{fig:pie}:

\begin{Shaded}
\begin{Highlighting}[]
\KeywordTok{par}\NormalTok{(}\DataTypeTok{mar =} \KeywordTok{c}\NormalTok{(}\DecValTok{0}\NormalTok{, }\DecValTok{1}\NormalTok{, }\DecValTok{0}\NormalTok{, }\DecValTok{1}\NormalTok{))}
\KeywordTok{pie}\NormalTok{(}
  \KeywordTok{c}\NormalTok{(}\DecValTok{280}\NormalTok{, }\DecValTok{60}\NormalTok{, }\DecValTok{20}\NormalTok{),}
  \KeywordTok{c}\NormalTok{(}\StringTok{'Sky'}\NormalTok{, }\StringTok{'Sunny side of pyramid'}\NormalTok{, }\StringTok{'Shady side of pyramid'}\NormalTok{),}
  \DataTypeTok{col =} \KeywordTok{c}\NormalTok{(}\StringTok{'#0292D8'}\NormalTok{, }\StringTok{'#F7EA39'}\NormalTok{, }\StringTok{'#C4B632'}\NormalTok{),}
  \DataTypeTok{init.angle =} \OperatorTok{-}\DecValTok{50}\NormalTok{, }\DataTypeTok{border =} \OtherTok{NA}
\NormalTok{)}
\end{Highlighting}
\end{Shaded}

\begin{figure}

{\centering \includegraphics[width=0.7\linewidth]{EnvStat_files/figure-latex/pie-1} 

}

\caption{Pie chart}\label{fig:pie}
\end{figure}

\section{Histogram dan Density Plot}\label{histogram-dan-density-plot}

Fungsi \texttt{hist()} dapat digunakan untuk membuat histogram pada
\texttt{R}. Secara sederhana fungsi tersebut didefinisikan sebagai
berikut:

\begin{Shaded}
\begin{Highlighting}[]
\KeywordTok{hist}\NormalTok{(x, }\DataTypeTok{breaks=}\StringTok{"Sturges"}\NormalTok{)}
\end{Highlighting}
\end{Shaded}

\begin{quote}
\textbf{Note: }

\begin{itemize}
\tightlist
\item
  \textbf{x}: vektor numerik
\item
  \textbf{breaks}: \emph{breakpoints} antar sel histogram.
\end{itemize}
\end{quote}

Pada dataset \texttt{trees} akan dibuat histogram variabel
\texttt{Height}. Untuk melakukannya jalankan sintaks berikut:

\begin{Shaded}
\begin{Highlighting}[]
\KeywordTok{hist}\NormalTok{(trees}\OperatorTok{$}\NormalTok{Height)}
\end{Highlighting}
\end{Shaded}

Output yang dihasilkan disajikan pada Gambar \ref{fig:hist}:

\begin{figure}

{\centering \includegraphics[width=0.7\linewidth]{EnvStat_files/figure-latex/hist-1} 

}

\caption{Histogram}\label{fig:hist}
\end{figure}

Density plot pada \texttt{R} dapat dibuat menggunakan fungsi
\texttt{density()}. Berbeda dengan fungsi \texttt{hist()}, fungsi ini
tidak langsung menghasilkan grafik densitas. Fungsi \texttt{density()}
hanya menghitung kernel densitas pada data. Densitas yang telah dihitung
selanjutnya diplotkan menggunakan fungsi \texttt{plot()}. Berikut adalah
sintaks dan output yang dihasilkan pada Gambar \ref{fig:dens}:

\begin{Shaded}
\begin{Highlighting}[]
\CommentTok{# menghitung kernel density}
\NormalTok{dens <-}\StringTok{ }\KeywordTok{density}\NormalTok{(trees}\OperatorTok{$}\NormalTok{Height)}

\CommentTok{# plot densitas dengan outline merah}
\KeywordTok{plot}\NormalTok{(dens,}\DataTypeTok{col=}\StringTok{"red"}\NormalTok{)}
\end{Highlighting}
\end{Shaded}

\begin{figure}

{\centering \includegraphics[width=0.7\linewidth]{EnvStat_files/figure-latex/dens-1} 

}

\caption{Density plot}\label{fig:dens}
\end{figure}

Kita juga dapat menambahkan grafik densitas pada histogram sehingga
mempermudah pembacaan pada histogram. Untuk melakukannya kita perlu
mengubah kernel histigram dari frekuensi menjadi density dengan
menambahkan argumen \texttt{freq=FALSE} pada fungsi \texttt{hist()}.
Selanjutnya tambahkan fungsi \texttt{polygon()} untuk memplotkan grafik
densitas. Berikut adalah sintak dan output yang dihasilkan pada Gambar
\ref{fig:denshist}:

\begin{Shaded}
\begin{Highlighting}[]
\CommentTok{# menghitung kernel density}
\NormalTok{dens <-}\StringTok{ }\KeywordTok{density}\NormalTok{(trees}\OperatorTok{$}\NormalTok{Height)}

\CommentTok{# histogram}
\KeywordTok{hist}\NormalTok{(trees}\OperatorTok{$}\NormalTok{Height, }\DataTypeTok{freq=}\OtherTok{FALSE}\NormalTok{, }\DataTypeTok{col=}\StringTok{"steelblue"}\NormalTok{)}

\CommentTok{# tambahkan density plot}
\KeywordTok{polygon}\NormalTok{(dens, }\DataTypeTok{border=}\StringTok{"red"}\NormalTok{)}
\end{Highlighting}
\end{Shaded}

\begin{figure}

{\centering \includegraphics[width=0.7\linewidth]{EnvStat_files/figure-latex/denshist-1} 

}

\caption{Density plot dan histogram}\label{fig:denshist}
\end{figure}

\section{QQ Plot}\label{qq-plot}

QQ plot digunakan untuk mengecek distribusi suatu data apakah
berdistribusi normal atau tidak. Pada \texttt{R} QQ plot dibuat
menggunakan 2 fungsi yaitu: \texttt{qqnorm()} dan \texttt{qqline()}.
Fungsi \texttt{qqnorm()} digunakan untuk memproduksi normal QQ plot
suatu variabel. Sedangkan fungsi \texttt{qqline()} digunakan untuk
membuat garis referensi distiribusi normal. Suatu distribusi dikatan
normal jika titik observasi yang dihasilkan mengikuti garis referensi
tersebut.

Berikut adalah cara membuat QQ plot menggunakan variabel \texttt{Volume}
pada dataset \texttt{trees}. Output yang dihasilkan disajikan pada
Gambar \ref{fig:qq}.

\begin{Shaded}
\begin{Highlighting}[]
\KeywordTok{qqnorm}\NormalTok{(trees}\OperatorTok{$}\NormalTok{Volume)}
\KeywordTok{qqline}\NormalTok{(trees}\OperatorTok{$}\NormalTok{Volume, }\DataTypeTok{col=}\StringTok{"red"}\NormalTok{)}
\end{Highlighting}
\end{Shaded}

\begin{figure}

{\centering \includegraphics[width=0.7\linewidth]{EnvStat_files/figure-latex/qq-1} 

}

\caption{QQ plot}\label{fig:qq}
\end{figure}

\section{Dot Chart}\label{dot-chart}

Fungsi \texttt{dotchart()} pada \texttt{R} digunakan untuk membuat dot
chart. Format yang digunakan adalah sebagai berikut:

\begin{Shaded}
\begin{Highlighting}[]
\KeywordTok{dotchart}\NormalTok{(x, }\DataTypeTok{labels =} \OtherTok{NULL}\NormalTok{, }\DataTypeTok{groups =} \OtherTok{NULL}\NormalTok{, }
         \DataTypeTok{gcolor =} \KeywordTok{par}\NormalTok{(}\StringTok{"fg"}\NormalTok{), }\DataTypeTok{color =} \KeywordTok{par}\NormalTok{(}\StringTok{"fg"}\NormalTok{))}
\end{Highlighting}
\end{Shaded}

\begin{quote}
\textbf{Note: }

\begin{itemize}
\tightlist
\item
  \textbf{x}: vektor atau matriks numerik.
\item
  \textbf{labels}: vektor label untuk tiap titik.
\item
  \textbf{groups}: grouping variabel yang mengindikasikan bagaimana
  \textbf{x} dikelompokkan.
\item
  \textbf{gcolor}: warna yang digunakan pada label grup dan nilai
  observasi.
\item
  \textbf{color}: warna yang digunakan untuk titik dan label.
\end{itemize}
\end{quote}

Pada contoh berikut disajikan cara membuat dot chart pada dataset
\texttt{mtcars} untuk melihat mobil yang paling hemat bahan bakar
berdasarkan variabel \texttt{mpg} dan jumlah silinder (\texttt{cyl}).
Berikut sintaks yang digunakan dan output yang dihasilkan pada Gambar
\ref{fig:dot}:

\begin{Shaded}
\begin{Highlighting}[]
\CommentTok{# mengurutkan dataset mtcars berdasarkan variabel mpg}
\NormalTok{mtcars <-}\StringTok{ }\NormalTok{mtcars[}\KeywordTok{order}\NormalTok{(mtcars}\OperatorTok{$}\NormalTok{mpg), ]}

\CommentTok{# mengubah variabel cyl menjadi factor}
\NormalTok{grps <-}\StringTok{ }\KeywordTok{as.factor}\NormalTok{(mtcars}\OperatorTok{$}\NormalTok{cyl)}

\CommentTok{# membuat vektor warna berdasarkan jumlah grup}
\NormalTok{my_cols <-}\StringTok{ }\KeywordTok{c}\NormalTok{(}\StringTok{"#999999"}\NormalTok{, }\StringTok{"#E69F00"}\NormalTok{, }\StringTok{"#56B4E9"}\NormalTok{)}

\CommentTok{# plot}
\KeywordTok{dotchart}\NormalTok{(mtcars}\OperatorTok{$}\NormalTok{mpg, }\DataTypeTok{labels =} \KeywordTok{row.names}\NormalTok{(mtcars),}
         \DataTypeTok{groups =}\NormalTok{ grps, }\DataTypeTok{gcolor =}\NormalTok{ my_cols,}
         \DataTypeTok{color =}\NormalTok{ my_cols[grps],}
         \DataTypeTok{cex =} \FloatTok{0.6}\NormalTok{,  }\DataTypeTok{pch =} \DecValTok{19}\NormalTok{, }\DataTypeTok{xlab =} \StringTok{"mpg"}\NormalTok{)}
\end{Highlighting}
\end{Shaded}

\begin{figure}

{\centering \includegraphics[width=0.7\linewidth]{EnvStat_files/figure-latex/dot-1} 

}

\caption{Dot chart}\label{fig:dot}
\end{figure}

\section{Kustomisasi Parameter
Grafik}\label{kustomisasi-parameter-grafik}

Pada bagian ini penulis akan menjelaskan cara untuk kustomisasi
parameter grafik seperti:

\begin{enumerate}
\def\labelenumi{\alph{enumi}.}
\tightlist
\item
  menambahkan judul, legend, teks, axis, dan garis.
\item
  mengubah skala axis, simbol plot, jenis garis, dan warna.
\end{enumerate}

\subsection{Menambahkan Judul}\label{menambahkan-judul}

Pada grafik di \texttt{R}, kita dapat menambahkan judul dengan dua cara,
yaitu: pada plot melalui parameter dan melalui fungsi plot(). Kedua cara
tersebut tidak berbeda satu sama lain pada parameter input.

Untuk menambahkan judul pada plot secara langsung, kita dapat
menggunakan argumen tambahan sebagai berikut:

\begin{enumerate}
\def\labelenumi{\alph{enumi}.}
\tightlist
\item
  \textbf{main}: teks untuk judul.
\item
  \textbf{xlab}: teks untuk keterangan axis X.
\item
  \textbf{ylab}: teks untuk keterangan axis y.
\item
  \textbf{sub}: teks untuk sub-judul.
\end{enumerate}

Berikut contoh sintaks penerapan masing-masing argumen tersebut beserta
dengan output yang dihasilkan pada Gambar \ref{fig:title}:

\begin{Shaded}
\begin{Highlighting}[]
\CommentTok{# menambahkan judul}
\KeywordTok{barplot}\NormalTok{(}\KeywordTok{c}\NormalTok{(}\DecValTok{2}\NormalTok{,}\DecValTok{5}\NormalTok{), }\DataTypeTok{main=}\StringTok{"Main title"}\NormalTok{,}
        \DataTypeTok{xlab=}\StringTok{"X axis title"}\NormalTok{,}
        \DataTypeTok{ylab=}\StringTok{"Y axis title"}\NormalTok{,}
        \DataTypeTok{sub=}\StringTok{"Sub-title"}\NormalTok{)}
\end{Highlighting}
\end{Shaded}

\begin{figure}

{\centering \includegraphics[width=0.7\linewidth]{EnvStat_files/figure-latex/title-1} 

}

\caption{Menambahkan Judul}\label{fig:title}
\end{figure}

kita juga dapat melakukan kustomisasi pada warna, \emph{font style}, dan
ukuran font judul. Untuk melakukan kustomisasi pada warna pada judul,
kita dapat menambahkan argumen sebagai berikut:

\begin{enumerate}
\def\labelenumi{\alph{enumi}.}
\tightlist
\item
  \textbf{col.main}: warna untuk judul.
\item
  \textbf{col.lab}: warna untuk keterangan axis.
\item
  \textbf{col.sub}: warna untuk sub-judul
\end{enumerate}

Untuk kustomisasi font judul, kita dapat menambahkan argumen berikut:

\begin{enumerate}
\def\labelenumi{\alph{enumi}.}
\tightlist
\item
  \textbf{font.main}: \emph{font style} untuk judul.
\item
  \textbf{font.lab}: \emph{font style} untuk keterangan axis.
\item
  \textbf{font.sub}: \emph{font style} untuk sub-judul.
\end{enumerate}

\begin{quote}
\textbf{Note: }

Nilai yang dapat dimasukkan antara lain:

\begin{itemize}
\tightlist
\item
  \textbf{1}: untuk teks normal.
\item
  \textbf{2}: untuk teks cetak tebal.
\item
  \textbf{3}: untuk teks cetak miring.
\item
  \textbf{4}: untuk teks cetak tebal dan miring.
\item
  \textbf{5}: untuk font simbol.
\end{itemize}
\end{quote}

Sedangkan untuk ukuran font, kita dapat menambahkan variabel berikut:

\begin{enumerate}
\def\labelenumi{\alph{enumi}.}
\tightlist
\item
  \textbf{cex.main}: ukuran teks judul.
\item
  \textbf{cex.lab}: ukuran teks keterangan axis.
\item
  \textbf{cex.sub}: ukuran teks sub-judul.
\end{enumerate}

Berikut sintaks penerapan seluruh argumen tersebut beserta output yang
dihasilkan pada Gambar \ref{fig:title2}:

\begin{Shaded}
\begin{Highlighting}[]
\CommentTok{# menambahkan judul}
\KeywordTok{barplot}\NormalTok{(}\KeywordTok{c}\NormalTok{(}\DecValTok{2}\NormalTok{,}\DecValTok{5}\NormalTok{), }
        \CommentTok{# menambahkan judul}
        \DataTypeTok{main=}\StringTok{"Main title"}\NormalTok{,}
        \DataTypeTok{xlab=}\StringTok{"X axis title"}\NormalTok{,}
        \DataTypeTok{ylab=}\StringTok{"Y axis title"}\NormalTok{,}
        \DataTypeTok{sub=}\StringTok{"Sub-title"}\NormalTok{,}
        \CommentTok{# kustomisasi warna font}
        \DataTypeTok{col.main=}\StringTok{"red"}\NormalTok{, }
        \DataTypeTok{col.lab=}\StringTok{"blue"}\NormalTok{, }
        \DataTypeTok{col.sub=}\StringTok{"black"}\NormalTok{,}
        \CommentTok{# kustomisasi font style}
        \DataTypeTok{font.main=}\DecValTok{4}\NormalTok{, }
        \DataTypeTok{font.lab=}\DecValTok{4}\NormalTok{, }
        \DataTypeTok{font.sub=}\DecValTok{4}\NormalTok{,}
        \CommentTok{# kustomisasi ukuran font}
        \DataTypeTok{cex.main=}\DecValTok{2}\NormalTok{, }
        \DataTypeTok{cex.lab=}\FloatTok{1.7}\NormalTok{, }
        \DataTypeTok{cex.sub=}\FloatTok{1.2}\NormalTok{)}
\end{Highlighting}
\end{Shaded}

\begin{figure}

{\centering \includegraphics[width=0.7\linewidth]{EnvStat_files/figure-latex/title2-1} 

}

\caption{Menambahkan Judul (2)}\label{fig:title2}
\end{figure}

Kita telah belajar bagaimana menambahkan judul langsung pada fungsi
plot. Selain cara tersebut, telah penulis jelaskan bahwa kita dapat
menambahkan judul melalui fungsi \texttt{title()}. argumen yang
dimasukkan pada dasarnya tidak berbeda dengan ketika kita menambahkan
judul secara langsung pada plot. Berikut adalah contoh sintaks dan
output yang dihasilkan pada Gambar \ref{fig:title3}:

\begin{Shaded}
\begin{Highlighting}[]
\CommentTok{# menambahkan judul}
\KeywordTok{barplot}\NormalTok{(}\KeywordTok{c}\NormalTok{(}\DecValTok{2}\NormalTok{,}\DecValTok{5}\NormalTok{,}\DecValTok{8}\NormalTok{))}

\CommentTok{# menambahkan judul}
\KeywordTok{title}\NormalTok{(}\DataTypeTok{main=}\StringTok{"Main title"}\NormalTok{,}
      \DataTypeTok{xlab=}\StringTok{"X axis title"}\NormalTok{,}
      \DataTypeTok{ylab=}\StringTok{"Y axis title"}\NormalTok{,}
      \DataTypeTok{sub=}\StringTok{"Sub-title"}\NormalTok{,}
      \CommentTok{# kustomisasi warna font}
      \DataTypeTok{col.main=}\StringTok{"red"}\NormalTok{, }
      \DataTypeTok{col.lab=}\StringTok{"blue"}\NormalTok{, }
      \DataTypeTok{col.sub=}\StringTok{"black"}\NormalTok{,}
      \CommentTok{# kustomisasi font style}
      \DataTypeTok{font.main=}\DecValTok{4}\NormalTok{, }
      \DataTypeTok{font.lab=}\DecValTok{4}\NormalTok{, }
      \DataTypeTok{font.sub=}\DecValTok{4}\NormalTok{,}
      \CommentTok{# kustomisasi ukuran font}
      \DataTypeTok{cex.main=}\DecValTok{2}\NormalTok{, }
      \DataTypeTok{cex.lab=}\FloatTok{1.7}\NormalTok{, }
      \DataTypeTok{cex.sub=}\FloatTok{1.2}\NormalTok{)}
\end{Highlighting}
\end{Shaded}

\begin{figure}

{\centering \includegraphics[width=0.7\linewidth]{EnvStat_files/figure-latex/title3-1} 

}

\caption{Menambahkan Judul (3)}\label{fig:title3}
\end{figure}

\subsection{Menambahkan Legend}\label{menambahkan-legend}

Fungsi \texttt{legend()} pada \texttt{R} dapat digunakan untuk
menambahkan legend pada grafik. Format sederhananya adalah sebagai
berikut:

\begin{Shaded}
\begin{Highlighting}[]
\KeywordTok{legend}\NormalTok{(x, }\DataTypeTok{y=}\OtherTok{NULL}\NormalTok{, legend, fill, col, bg)}
\end{Highlighting}
\end{Shaded}

\begin{quote}
\textbf{Note: }

\begin{itemize}
\tightlist
\item
  \textbf{x} dan \textbf{y}: koordinat yang digunakan untuk posisi
  legend.
\item
  \textbf{legend}: teks pada legend
\item
  \textbf{fill}: warna yang digunakan untuk mengisi box disamping teks
  legend.
\item
  \textbf{col}: warna garis dan titik disamping teks legend.
\item
  \textbf{bg}: warna latar belakang legend box.
\end{itemize}
\end{quote}

Berikut adalah contoh sintaks dan ouput penerapan argumen disajikan pada
Gambar \ref{fig:legend}:

\begin{Shaded}
\begin{Highlighting}[]
\CommentTok{# membuat vektor numerik}
\NormalTok{x <-}\StringTok{ }\KeywordTok{c}\NormalTok{(}\DecValTok{1}\OperatorTok{:}\DecValTok{10}\NormalTok{)}
\NormalTok{y <-}\StringTok{ }\NormalTok{x}\OperatorTok{^}\DecValTok{2}
\NormalTok{z <-}\StringTok{ }\NormalTok{x}\OperatorTok{*}\DecValTok{2}

\CommentTok{# membuat line plot}
\KeywordTok{plot}\NormalTok{(x,y, }\DataTypeTok{type=}\StringTok{"o"}\NormalTok{, }\DataTypeTok{col=}\StringTok{"red"}\NormalTok{, }\DataTypeTok{lty=}\DecValTok{1}\NormalTok{)}

\CommentTok{# menambahkan line plot}
\KeywordTok{lines}\NormalTok{(x,z, }\DataTypeTok{type=}\StringTok{"o"}\NormalTok{, }\DataTypeTok{col=}\StringTok{"blue"}\NormalTok{, }\DataTypeTok{lty=}\DecValTok{2}\NormalTok{)}

\CommentTok{# menambahkan legend}
\KeywordTok{legend}\NormalTok{(}\DecValTok{1}\NormalTok{, }\DecValTok{95}\NormalTok{, }\DataTypeTok{legend=}\KeywordTok{c}\NormalTok{(}\StringTok{"Line 1"}\NormalTok{, }\StringTok{"Line 2"}\NormalTok{),}
       \DataTypeTok{col=}\KeywordTok{c}\NormalTok{(}\StringTok{"red"}\NormalTok{, }\StringTok{"blue"}\NormalTok{), }\DataTypeTok{lty=}\DecValTok{1}\OperatorTok{:}\DecValTok{2}\NormalTok{, }\DataTypeTok{cex=}\FloatTok{0.8}\NormalTok{)}
\end{Highlighting}
\end{Shaded}

\begin{figure}

{\centering \includegraphics[width=0.7\linewidth]{EnvStat_files/figure-latex/legend-1} 

}

\caption{Menambahkan legend}\label{fig:legend}
\end{figure}

Kita dapat menambahkan judul, merubah font, dan merubah warna backgroud
pada legend. Argumen yang ditambahkan pada legend adalah sebagai
berikut:

\begin{enumerate}
\def\labelenumi{\alph{enumi}.}
\tightlist
\item
  \textbf{title}: Judul legend
\item
  \textbf{text.font}: integer yang menunjukkan \emph{font style} pada
  teks legend. Nilai yang dapat dimasukkan adalah sebagai berikut:

  \begin{itemize}
  \tightlist
  \item
    \textbf{1}: normal
  \item
    \textbf{2}: cetak tebal
  \item
    \textbf{3}: cetak miring
  \item
    \textbf{4}: cetak tebal dan miring.
  \end{itemize}
\item
  \textbf{bg}: warna background legend box.
\end{enumerate}

Berikut adalah penerapan sintaks dan output yang dihasilkan pada Gambar
\ref{fig:legend2}:

\begin{Shaded}
\begin{Highlighting}[]
\CommentTok{# membuat line plot}
\KeywordTok{plot}\NormalTok{(x,y, }\DataTypeTok{type=}\StringTok{"o"}\NormalTok{, }\DataTypeTok{col=}\StringTok{"red"}\NormalTok{, }\DataTypeTok{lty=}\DecValTok{1}\NormalTok{)}

\CommentTok{# menambahkan line plot}
\KeywordTok{lines}\NormalTok{(x,z, }\DataTypeTok{type=}\StringTok{"o"}\NormalTok{, }\DataTypeTok{col=}\StringTok{"blue"}\NormalTok{, }\DataTypeTok{lty=}\DecValTok{2}\NormalTok{)}

\CommentTok{# menambahkan legend}
\KeywordTok{legend}\NormalTok{(}\DecValTok{1}\NormalTok{, }\DecValTok{95}\NormalTok{, }\DataTypeTok{legend=}\KeywordTok{c}\NormalTok{(}\StringTok{"Line 1"}\NormalTok{, }\StringTok{"Line 2"}\NormalTok{),}
       \DataTypeTok{col=}\KeywordTok{c}\NormalTok{(}\StringTok{"red"}\NormalTok{, }\StringTok{"blue"}\NormalTok{), }\DataTypeTok{lty=}\DecValTok{1}\OperatorTok{:}\DecValTok{2}\NormalTok{, }\DataTypeTok{cex=}\FloatTok{0.8}\NormalTok{,}
       \DataTypeTok{title=}\StringTok{"Line types"}\NormalTok{, }\DataTypeTok{text.font=}\DecValTok{4}\NormalTok{, }\DataTypeTok{bg=}\StringTok{'lightblue'}\NormalTok{)}
\end{Highlighting}
\end{Shaded}

\begin{figure}

{\centering \includegraphics[width=0.7\linewidth]{EnvStat_files/figure-latex/legend2-1} 

}

\caption{Menambahkan legend (2)}\label{fig:legend2}
\end{figure}

Kita dapat melakukan kustomisasi pada border dari legend melalui argumen
\texttt{box.lty=}(jenis garis), \texttt{box.lwd=}(ukuran garis), dan
\texttt{box.col=}(warna box). Berikut adalah penerapan argumen tersebut
beserta output yang dihasilkan pada Gambar \ref{fig:legend3}:

\begin{Shaded}
\begin{Highlighting}[]
\CommentTok{# membuat line plot}
\KeywordTok{plot}\NormalTok{(x,y, }\DataTypeTok{type=}\StringTok{"o"}\NormalTok{, }\DataTypeTok{col=}\StringTok{"red"}\NormalTok{, }\DataTypeTok{lty=}\DecValTok{1}\NormalTok{)}

\CommentTok{# menambahkan line plot}
\KeywordTok{lines}\NormalTok{(x,z, }\DataTypeTok{type=}\StringTok{"o"}\NormalTok{, }\DataTypeTok{col=}\StringTok{"blue"}\NormalTok{, }\DataTypeTok{lty=}\DecValTok{2}\NormalTok{)}

\CommentTok{# menambahkan legend}
\KeywordTok{legend}\NormalTok{(}\DecValTok{1}\NormalTok{, }\DecValTok{95}\NormalTok{, }\DataTypeTok{legend=}\KeywordTok{c}\NormalTok{(}\StringTok{"Line 1"}\NormalTok{, }\StringTok{"Line 2"}\NormalTok{),}
       \DataTypeTok{col=}\KeywordTok{c}\NormalTok{(}\StringTok{"red"}\NormalTok{, }\StringTok{"blue"}\NormalTok{), }\DataTypeTok{lty=}\DecValTok{1}\OperatorTok{:}\DecValTok{2}\NormalTok{, }\DataTypeTok{cex=}\FloatTok{0.8}\NormalTok{,}
       \DataTypeTok{title=}\StringTok{"Line types"}\NormalTok{, }\DataTypeTok{text.font=}\DecValTok{4}\NormalTok{, }\DataTypeTok{bg=}\StringTok{'white'}\NormalTok{,}
       \DataTypeTok{box.lty=}\DecValTok{2}\NormalTok{, }\DataTypeTok{box.lwd=}\DecValTok{2}\NormalTok{, }\DataTypeTok{box.col=}\StringTok{"steelblue"}\NormalTok{)}
\end{Highlighting}
\end{Shaded}

\begin{figure}

{\centering \includegraphics[width=0.7\linewidth]{EnvStat_files/figure-latex/legend3-1} 

}

\caption{Menambahkan legend (3)}\label{fig:legend3}
\end{figure}

Selain menggunakan koordinat, kita juga dapat melakukan kustomisasi
posisi legend menggunakan \emph{keyword} seperti:
bottomright``,''bottom``,''bottomleft``,''left``,''topleft``,''top``,''topright``,''right"
and ``center''. Sejumlah kustomisasi legend berdasarkan \emph{keyword}
disajikan pada Gambar \ref{fig:legend4}:

\begin{Shaded}
\begin{Highlighting}[]
\CommentTok{# plot}
\KeywordTok{plot}\NormalTok{(x,y, }\DataTypeTok{type =} \StringTok{"n"}\NormalTok{)}

\CommentTok{# posisi kiri atas, inset =0.05}
\KeywordTok{legend}\NormalTok{(}\StringTok{"topleft"}\NormalTok{,}
  \DataTypeTok{legend =} \StringTok{"(x,y)"}\NormalTok{,}
  \DataTypeTok{title =} \StringTok{"topleft, inset = .05"}\NormalTok{,}
  \DataTypeTok{inset =} \FloatTok{0.05}\NormalTok{)}
\CommentTok{# posisi atas}
\KeywordTok{legend}\NormalTok{(}\StringTok{"top"}\NormalTok{,}
       \DataTypeTok{legend =} \StringTok{"(x,y)"}\NormalTok{,}
       \DataTypeTok{title =} \StringTok{"top"}\NormalTok{)}
\CommentTok{# posisi kanan atas inset = .02}
\KeywordTok{legend}\NormalTok{(}\StringTok{"topright"}\NormalTok{,}
       \DataTypeTok{legend =} \StringTok{"(x,y)"}\NormalTok{,}
       \DataTypeTok{title =} \StringTok{"topright, inset = .02"}\NormalTok{,}
       \DataTypeTok{inset =} \FloatTok{0.02}\NormalTok{)}
\CommentTok{# posisi kiri}
\KeywordTok{legend}\NormalTok{(}\StringTok{"left"}\NormalTok{,}
       \DataTypeTok{legend =} \StringTok{"(x,y)"}\NormalTok{,}
       \DataTypeTok{title =} \StringTok{"left"}\NormalTok{)}
\CommentTok{# posisi tengah}
\KeywordTok{legend}\NormalTok{(}\StringTok{"center"}\NormalTok{,}
       \DataTypeTok{legend =} \StringTok{"(x,y)"}\NormalTok{,}
       \DataTypeTok{title =} \StringTok{"center"}\NormalTok{)}
\CommentTok{# posisi kanan}
\KeywordTok{legend}\NormalTok{(}\StringTok{"right"}\NormalTok{,}
       \DataTypeTok{legend =} \StringTok{"(x,y)"}\NormalTok{,}
       \DataTypeTok{title =} \StringTok{"right"}\NormalTok{)}
\CommentTok{# posisi kiri bawah}
\KeywordTok{legend}\NormalTok{(}\StringTok{"bottomleft"}\NormalTok{,}
       \DataTypeTok{legend =} \StringTok{"(x,y)"}\NormalTok{,}
       \DataTypeTok{title =} \StringTok{"bottomleft"}\NormalTok{)}
\CommentTok{# posisi bawah}
\KeywordTok{legend}\NormalTok{(}\StringTok{"bottom"}\NormalTok{,}
       \DataTypeTok{legend =} \StringTok{"(x,y)"}\NormalTok{,}
       \DataTypeTok{title =} \StringTok{"bottom"}\NormalTok{)}
\CommentTok{# posisi kanan bawah}
\KeywordTok{legend}\NormalTok{(}\StringTok{"bottomright"}\NormalTok{,}
       \DataTypeTok{legend =} \StringTok{"(x,y)"}\NormalTok{,}
       \DataTypeTok{title =} \StringTok{"bottomright"}\NormalTok{)}
\end{Highlighting}
\end{Shaded}

\begin{figure}

{\centering \includegraphics[width=0.7\linewidth]{EnvStat_files/figure-latex/legend4-1} 

}

\caption{Kustomisasi posisi legend}\label{fig:legend4}
\end{figure}

\subsection{Menambahkan Teks Pada
Grafik}\label{menambahkan-teks-pada-grafik}

Teks pada grafik dapat kita tambahkan baik sebagai keterangan yang
menunjukkan label suatu observasi, keterangan tambahan disekitar bingkai
grafik, maupun sebuah persamaan yang ada pada bidang grafik. Untuk
menambahkannya kita dapat menggunakan dua buah fungsi yaitu:
\texttt{text()} dan \texttt{mtext()}.

FUngsi \texttt{text()} berguna untuk menambahkan teks di dalam bidang
grafik seperti label titik observasi dan persamaan di dalam bidang
grafik. Format yang digunakan adalah sebagai berikut:

\begin{Shaded}
\begin{Highlighting}[]
\KeywordTok{text}\NormalTok{(x, y, labels)}
\end{Highlighting}
\end{Shaded}

\begin{quote}
\textbf{Note: }

\begin{itemize}
\tightlist
\item
  \textbf{x} dan \textbf{y}: vektor numerik yang menunjukkan koordinat
  posisi teks.
\item
  \textbf{labels}: vektor karakter yang menunjukkan teks yang hendak
  ditulis.
\end{itemize}
\end{quote}

Berikut adalah contoh sintaks untuk memberi label pada sejumlah data
yang memiliki kriteria yang kita inginkan dan output yang dihasilkan
pada Gambar \ref{fig:text}:

\begin{Shaded}
\begin{Highlighting}[]
\CommentTok{# tandai observasi yang memiliki nilai}
\CommentTok{# mpg < 15 dan wt > 5}
\NormalTok{d <-}\StringTok{ }\NormalTok{mtcars[mtcars}\OperatorTok{$}\NormalTok{wt }\OperatorTok{>=}\StringTok{ }\DecValTok{5} \OperatorTok{&}\StringTok{ }\NormalTok{mtcars}\OperatorTok{$}\NormalTok{mpg }\OperatorTok{<=}\StringTok{ }\DecValTok{15}\NormalTok{, ]}


\CommentTok{# plot}
\KeywordTok{plot}\NormalTok{(mtcars}\OperatorTok{$}\NormalTok{wt, mtcars}\OperatorTok{$}\NormalTok{mpg, }\DataTypeTok{main=}\StringTok{"Milage vs. Car Weight"}\NormalTok{,}
      \DataTypeTok{xlab=}\StringTok{"Weight"}\NormalTok{, }\DataTypeTok{ylab=}\StringTok{"Miles/(US) gallon"}\NormalTok{)}

\CommentTok{# menambahkan text}
\KeywordTok{text}\NormalTok{(d}\OperatorTok{$}\NormalTok{wt, d}\OperatorTok{$}\NormalTok{mpg,  }\KeywordTok{row.names}\NormalTok{(d),}
     \DataTypeTok{cex=}\FloatTok{0.65}\NormalTok{, }\DataTypeTok{pos=}\DecValTok{3}\NormalTok{,}\DataTypeTok{col=}\StringTok{"red"}\NormalTok{)}
\end{Highlighting}
\end{Shaded}

\begin{figure}

{\centering \includegraphics[width=0.7\linewidth]{EnvStat_files/figure-latex/text-1} 

}

\caption{Menambahkan teks}\label{fig:text}
\end{figure}

Sedangkan sintaks berikut adalah contoh bagaimana menambahkan persamaan
kedalam bidang grafik dan output yang dihasilkan pada Gambar
\ref{fig:text2}:

\begin{Shaded}
\begin{Highlighting}[]
\KeywordTok{plot}\NormalTok{(}\DecValTok{1}\OperatorTok{:}\DecValTok{10}\NormalTok{, }\DecValTok{1}\OperatorTok{:}\DecValTok{10}\NormalTok{, }
     \DataTypeTok{main=}\StringTok{"text(...) examples}\CharTok{\textbackslash{}n}\StringTok{~~~~~~~~~~~"}\NormalTok{)}
\KeywordTok{text}\NormalTok{(}\DecValTok{4}\NormalTok{, }\DecValTok{9}\NormalTok{, }\KeywordTok{expression}\NormalTok{(}\KeywordTok{hat}\NormalTok{(beta) }\OperatorTok{==}\StringTok{ }\NormalTok{(X}\OperatorTok{^}\NormalTok{t }\OperatorTok{*}\StringTok{ }\NormalTok{X)}\OperatorTok{^}\NormalTok{\{}\OperatorTok{-}\DecValTok{1}\NormalTok{\} }\OperatorTok{*}\StringTok{ }\NormalTok{X}\OperatorTok{^}\NormalTok{t }\OperatorTok{*}\StringTok{ }\NormalTok{y))}
\KeywordTok{text}\NormalTok{(}\DecValTok{7}\NormalTok{, }\DecValTok{4}\NormalTok{, }\KeywordTok{expression}\NormalTok{(}\KeywordTok{bar}\NormalTok{(x) }\OperatorTok{==}\StringTok{ }\KeywordTok{sum}\NormalTok{(}\KeywordTok{frac}\NormalTok{(x[i], n), i}\OperatorTok{==}\DecValTok{1}\NormalTok{, n)))}
\end{Highlighting}
\end{Shaded}

\begin{figure}

{\centering \includegraphics[width=0.7\linewidth]{EnvStat_files/figure-latex/text2-1} 

}

\caption{Menambahkan teks (2)}\label{fig:text2}
\end{figure}

Fungsi \texttt{mtext()} berguna untuk menambahkan teks pada frame
sekitar bidang grafik. Format yang digunakan adalah sebagai berikut:

\begin{Shaded}
\begin{Highlighting}[]
\KeywordTok{mtext}\NormalTok{(text, }\DataTypeTok{side=}\DecValTok{3}\NormalTok{)}
\end{Highlighting}
\end{Shaded}

\begin{quote}
\textbf{Note: }

\begin{itemize}
\tightlist
\item
  \textbf{text}: teks yang akan ditulis.
\item
  \textbf{side}: integer yang menunjukkan lokasi teks yang akan ditulis.
  Nilai yang dapat dimasukkan antara lain:
\item
  \textbf{1}: bawah
\item
  \textbf{2}: kiri
\item
  \textbf{3}: atas
\item
  \textbf{4}: kanan.
\end{itemize}
\end{quote}

Berikut adalah contoh penerapan dan output yang dihasilkan pada Gambar
\ref{fig:text3}:

\begin{Shaded}
\begin{Highlighting}[]
\KeywordTok{plot}\NormalTok{(}\DecValTok{1}\OperatorTok{:}\DecValTok{10}\NormalTok{, }\DecValTok{1}\OperatorTok{:}\DecValTok{10}\NormalTok{, }
     \DataTypeTok{main=}\StringTok{"mtext(...) examples}\CharTok{\textbackslash{}n}\StringTok{~~~~~~~~~~~"}\NormalTok{)}
\KeywordTok{mtext}\NormalTok{(}\StringTok{"Magic function"}\NormalTok{, }\DataTypeTok{side=}\DecValTok{3}\NormalTok{)}
\end{Highlighting}
\end{Shaded}

\begin{figure}

{\centering \includegraphics[width=0.7\linewidth]{EnvStat_files/figure-latex/text3-1} 

}

\caption{Menambahkan teks (3)}\label{fig:text3}
\end{figure}

\subsection{Menambahkan Garis Pada
Plot}\label{menambahkan-garis-pada-plot}

Fungsi \texttt{abline()} dapat digunakan untuk menamabahkan garis pada
plot. Garis yang ditambahkan dapat berupa garis vertikal, horizontal,
maupun garis regresi. Format yang digunakan adalah sebagi berikut:

\begin{Shaded}
\begin{Highlighting}[]
\KeywordTok{abline}\NormalTok{(}\DataTypeTok{v=}\NormalTok{y)}
\end{Highlighting}
\end{Shaded}

Berikut adalah contoh sintaks bagaimana menambahkan garis pada sebuah
plot dan output yang dihasilkan disajikan pada Gambar \ref{fig:abline}:

\begin{Shaded}
\begin{Highlighting}[]
\CommentTok{# membuat plot}
\KeywordTok{plot}\NormalTok{(mtcars}\OperatorTok{$}\NormalTok{wt, mtcars}\OperatorTok{$}\NormalTok{mpg, }\DataTypeTok{main=}\StringTok{"Milage vs. Car Weight"}\NormalTok{,}
      \DataTypeTok{xlab=}\StringTok{"Weight"}\NormalTok{, }\DataTypeTok{ylab=}\StringTok{"Miles/(US) gallon"}\NormalTok{)}

\CommentTok{# menambahkan garis vertikal di titik rata-rata weight}
\KeywordTok{abline}\NormalTok{(}\DataTypeTok{v=}\KeywordTok{mean}\NormalTok{(mtcars}\OperatorTok{$}\NormalTok{wt), }\DataTypeTok{col=}\StringTok{"red"}\NormalTok{, }\DataTypeTok{lwd=}\DecValTok{3}\NormalTok{, }\DataTypeTok{lty=}\DecValTok{2}\NormalTok{)}

\CommentTok{# menambahkan garis horizontal di titik rata-rata  mpg}
\KeywordTok{abline}\NormalTok{(}\DataTypeTok{h=}\KeywordTok{mean}\NormalTok{(mtcars}\OperatorTok{$}\NormalTok{mpg), }\DataTypeTok{col=}\StringTok{"blue"}\NormalTok{, }\DataTypeTok{lwd=}\DecValTok{3}\NormalTok{, }\DataTypeTok{lty=}\DecValTok{3}\NormalTok{)}

\CommentTok{# menambahkan garis regresi}
\KeywordTok{abline}\NormalTok{(}\KeywordTok{lm}\NormalTok{(mpg}\OperatorTok{~}\NormalTok{wt, }\DataTypeTok{data=}\NormalTok{mtcars), }\DataTypeTok{lwd=}\DecValTok{4}\NormalTok{, }\DataTypeTok{lty=}\DecValTok{4}\NormalTok{)}
\end{Highlighting}
\end{Shaded}

\begin{figure}

{\centering \includegraphics[width=0.7\linewidth]{EnvStat_files/figure-latex/abline-1} 

}

\caption{Menambahkan garis}\label{fig:abline}
\end{figure}

\subsection{Merubah Simbol plot dan Jenis
Garis}\label{merubah-simbol-plot-dan-jenis-garis}

Simbol plot (jenis titik) dapat diubah dengan menambahkan argumen
\texttt{pch=} pada plot. Nilai yang dimasukkan pada argumen tersebut
adalah integer dengan kemungkinan nilai sebagai berikut:

\begin{itemize}
\tightlist
\item
  pch = 0,square
\item
  pch = 1,circle (default)
\item
  pch = 2,triangle point up
\item
  pch = 3,plus
\item
  pch = 4,cross
\item
  pch = 5,diamond
\item
  pch = 6,triangle point down
\item
  pch = 7,square cross
\item
  pch = 8,star
\item
  pch = 9,diamond plus
\item
  pch = 10,circle plus
\item
  pch = 11,triangles up and down
\item
  pch = 12,square plus
\item
  pch = 13,circle cross
\item
  pch = 14,square and triangle down
\item
  pch = 15, filled square
\item
  pch = 16, filled circle
\item
  pch = 17, filled triangle point-up
\item
  pch = 18, filled diamond
\item
  pch = 19, solid circle
\item
  pch = 20,bullet (smaller circle)
\item
  pch = 21, filled circle blue
\item
  pch = 22, filled square blue
\item
  pch = 23, filled diamond blue
\item
  pch = 24, filled triangle point-up blue
\item
  pch = 25, filled triangle point down blue
\end{itemize}

Untuk lebih memahami bentuk simbol tersebut, penulis akan menyajikan
sintaks yang menampilkan seluruh simbol tersebut pada satu grafik.
Output yang dihasilkan disajikan pada Gambar \ref{fig:symbol}:

\begin{Shaded}
\begin{Highlighting}[]
\NormalTok{generateRPointShapes<-}\ControlFlowTok{function}\NormalTok{()\{}
  \CommentTok{# menentukan parameter plot}
\NormalTok{  oldPar<-}\KeywordTok{par}\NormalTok{()}
  \KeywordTok{par}\NormalTok{(}\DataTypeTok{font=}\DecValTok{2}\NormalTok{, }\DataTypeTok{mar=}\KeywordTok{c}\NormalTok{(}\FloatTok{0.5}\NormalTok{,}\DecValTok{0}\NormalTok{,}\DecValTok{0}\NormalTok{,}\DecValTok{0}\NormalTok{))}
  \CommentTok{# produksi titik axis}
\NormalTok{  y=}\KeywordTok{rev}\NormalTok{(}\KeywordTok{c}\NormalTok{(}\KeywordTok{rep}\NormalTok{(}\DecValTok{1}\NormalTok{,}\DecValTok{6}\NormalTok{),}\KeywordTok{rep}\NormalTok{(}\DecValTok{2}\NormalTok{,}\DecValTok{5}\NormalTok{), }\KeywordTok{rep}\NormalTok{(}\DecValTok{3}\NormalTok{,}\DecValTok{5}\NormalTok{), }\KeywordTok{rep}\NormalTok{(}\DecValTok{4}\NormalTok{,}\DecValTok{5}\NormalTok{), }\KeywordTok{rep}\NormalTok{(}\DecValTok{5}\NormalTok{,}\DecValTok{5}\NormalTok{)))}
\NormalTok{  x=}\KeywordTok{c}\NormalTok{(}\KeywordTok{rep}\NormalTok{(}\DecValTok{1}\OperatorTok{:}\DecValTok{5}\NormalTok{,}\DecValTok{5}\NormalTok{),}\DecValTok{6}\NormalTok{)}
  \CommentTok{# plot seluruh titik dan label}
  \KeywordTok{plot}\NormalTok{(x, y, }\DataTypeTok{pch =} \DecValTok{0}\OperatorTok{:}\DecValTok{25}\NormalTok{, }\DataTypeTok{cex=}\FloatTok{1.5}\NormalTok{, }\DataTypeTok{ylim=}\KeywordTok{c}\NormalTok{(}\DecValTok{1}\NormalTok{,}\FloatTok{5.5}\NormalTok{), }\DataTypeTok{xlim=}\KeywordTok{c}\NormalTok{(}\DecValTok{1}\NormalTok{,}\FloatTok{6.5}\NormalTok{), }
       \DataTypeTok{axes=}\OtherTok{FALSE}\NormalTok{, }\DataTypeTok{xlab=}\StringTok{""}\NormalTok{, }\DataTypeTok{ylab=}\StringTok{""}\NormalTok{, }\DataTypeTok{bg=}\StringTok{"blue"}\NormalTok{)}
  \KeywordTok{text}\NormalTok{(x, y, }\DataTypeTok{labels=}\DecValTok{0}\OperatorTok{:}\DecValTok{25}\NormalTok{, }\DataTypeTok{pos=}\DecValTok{3}\NormalTok{)}
  \KeywordTok{par}\NormalTok{(}\DataTypeTok{mar=}\NormalTok{oldPar}\OperatorTok{$}\NormalTok{mar,}\DataTypeTok{font=}\NormalTok{oldPar}\OperatorTok{$}\NormalTok{font )}
\NormalTok{\}}

\CommentTok{# Print}
\KeywordTok{generateRPointShapes}\NormalTok{()}
\end{Highlighting}
\end{Shaded}

\begin{figure}

{\centering \includegraphics[width=0.7\linewidth]{EnvStat_files/figure-latex/symbol-1} 

}

\caption{Symbol plot}\label{fig:symbol}
\end{figure}

Pada \texttt{R} kita juga dapat mengatur jenis garis yang akan
ditampilkan pada plot dengan menambahkan argumen \texttt{lty=}
(\emph{line type}) pada fungsi plot. Nilai yang dapat dimasukkan adalah
nilai integer. Keterangan masing-masing nilai tersebut adalah sebagai
berikut:

\begin{itemize}
\tightlist
\item
  lty = 0, blank
\item
  lty = 1, solid (default)
\item
  lty = 2, dashed
\item
  lty = 3, dotted
\item
  lty = 4, dotdash
\item
  lty = 5, longdash
\item
  lty = 6, twodash
\end{itemize}

Untuk lebih memahaminya, pada sintaks berikut disajikan plot seluruh
jenis garis tersebut beserta output yang dihasilkannya pada Gambar
\ref{fig:lty}:

\begin{Shaded}
\begin{Highlighting}[]
\NormalTok{generateRLineTypes<-}\ControlFlowTok{function}\NormalTok{()\{}
\NormalTok{  oldPar<-}\KeywordTok{par}\NormalTok{()}
  \KeywordTok{par}\NormalTok{(}\DataTypeTok{font=}\DecValTok{2}\NormalTok{, }\DataTypeTok{mar=}\KeywordTok{c}\NormalTok{(}\DecValTok{0}\NormalTok{,}\DecValTok{0}\NormalTok{,}\DecValTok{0}\NormalTok{,}\DecValTok{0}\NormalTok{))}
  \KeywordTok{plot}\NormalTok{(}\DecValTok{1}\NormalTok{, }\DataTypeTok{pch=}\StringTok{""}\NormalTok{, }\DataTypeTok{ylim=}\KeywordTok{c}\NormalTok{(}\DecValTok{0}\NormalTok{,}\DecValTok{6}\NormalTok{), }\DataTypeTok{xlim=}\KeywordTok{c}\NormalTok{(}\DecValTok{0}\NormalTok{,}\FloatTok{0.7}\NormalTok{), }\DataTypeTok{axes =} \OtherTok{FALSE}\NormalTok{ ,}\DataTypeTok{xlab=}\StringTok{""}\NormalTok{, }\DataTypeTok{ylab=}\StringTok{""}\NormalTok{)}
  \ControlFlowTok{for}\NormalTok{(i }\ControlFlowTok{in} \DecValTok{0}\OperatorTok{:}\DecValTok{6}\NormalTok{) }\KeywordTok{lines}\NormalTok{(}\KeywordTok{c}\NormalTok{(}\FloatTok{0.3}\NormalTok{,}\FloatTok{0.7}\NormalTok{), }\KeywordTok{c}\NormalTok{(i,i), }\DataTypeTok{lty=}\NormalTok{i, }\DataTypeTok{lwd=}\DecValTok{3}\NormalTok{)}
  \KeywordTok{text}\NormalTok{(}\KeywordTok{rep}\NormalTok{(}\FloatTok{0.1}\NormalTok{,}\DecValTok{6}\NormalTok{), }\DecValTok{0}\OperatorTok{:}\DecValTok{6}\NormalTok{, }
       \DataTypeTok{labels=}\KeywordTok{c}\NormalTok{(}\StringTok{"0.'blank'"}\NormalTok{, }\StringTok{"1.'solid'"}\NormalTok{, }\StringTok{"2.'dashed'"}\NormalTok{, }\StringTok{"3.'dotted'"}\NormalTok{, }
                \StringTok{"4.'dotdash'"}\NormalTok{, }\StringTok{"5.'longdash'"}\NormalTok{, }\StringTok{"6.'twodash'"}\NormalTok{))}
  \KeywordTok{par}\NormalTok{(}\DataTypeTok{mar=}\NormalTok{oldPar}\OperatorTok{$}\NormalTok{mar,}\DataTypeTok{font=}\NormalTok{oldPar}\OperatorTok{$}\NormalTok{font )}
\NormalTok{\}}
\KeywordTok{generateRLineTypes}\NormalTok{()}
\end{Highlighting}
\end{Shaded}

\begin{figure}

{\centering \includegraphics[width=0.7\linewidth]{EnvStat_files/figure-latex/lty-1} 

}

\caption{Line type}\label{fig:lty}
\end{figure}

\subsection{Mengatur Axis Plot}\label{mengatur-axis-plot}

Kita dapat melakukan pengaturan lebih jauh terhadap axis, seperti:
menambahkan axis tambahan pada atas dan bawah frame, mengubah rentang
nilai axis, serta kustomisasi \emph{tick mark} pada nilai axis. Hal ini
diperlukan karena fungsi grafik dasar \texttt{R} tidak dapat mengatur
axis secara otomatis saat plot baru ditambahkan pada plot pertama dan
rentang nilai plot baru lebih besar dibanding plot pertama, sehingga
sebagian nilai plot baru tidak ditampilkan pada hasil akhir.

Untuk menambahkan axis pada \texttt{R} kita dapat menambahkan fungsi
\texttt{axis()} setelah plot dilakukan. Format yang digunakan adalah
sebagai berikut:

\begin{Shaded}
\begin{Highlighting}[]
\KeywordTok{axis}\NormalTok{(side, }\DataTypeTok{at=}\OtherTok{NULL}\NormalTok{, }\DataTypeTok{labels=}\OtherTok{TRUE}\NormalTok{)}
\end{Highlighting}
\end{Shaded}

\begin{quote}
\textbf{Note: }

\begin{itemize}
\tightlist
\item
  \textbf{side}: nilai integer yang mengidikasikan posisi axix yang
  hendak ditambahkan. Nilai yang dapat dimasukkan adalah sebagai
  berikut:

  \begin{itemize}
  \tightlist
  \item
    \textbf{1}: bawah
  \item
    \textbf{2}: kiri
  \item
    \textbf{3}: atas
  \item
    \textbf{4}: kanan.
  \end{itemize}
\item
  \textbf{at}: titik dimana \emph{tick-mark} hendak digambarkan. Nilai
  yang dapat dimasukkan sama dengan \textbf{side}.
\item
  \textbf{labels}: Teks label \emph{tick-mark}. Dapat juga secara logis
  menentukan apakah anotasi harus dibuat pada \emph{tick mark}.
\end{itemize}
\end{quote}

Berikut contoh sintaks penerapan fungsi tersebut dan output yang
dihasilkan pada Gambar \ref{fig:axis}:

\begin{Shaded}
\begin{Highlighting}[]
\CommentTok{# membuat vektor numerik}
\NormalTok{x <-}\StringTok{ }\KeywordTok{c}\NormalTok{(}\DecValTok{1}\OperatorTok{:}\DecValTok{4}\NormalTok{)}
\NormalTok{y <-}\StringTok{ }\NormalTok{x}\OperatorTok{^}\DecValTok{2}

\CommentTok{# plot}
\KeywordTok{plot}\NormalTok{(x, y, }\DataTypeTok{pch=}\DecValTok{18}\NormalTok{, }\DataTypeTok{col=}\StringTok{"red"}\NormalTok{, }\DataTypeTok{type=}\StringTok{"b"}\NormalTok{,}
     \DataTypeTok{frame=}\OtherTok{FALSE}\NormalTok{, }\DataTypeTok{xaxt=}\StringTok{"n"}\NormalTok{) }\CommentTok{# Remove x axis}

\CommentTok{# menambahkan axis}
\CommentTok{# bawah}
\KeywordTok{axis}\NormalTok{(}\DecValTok{1}\NormalTok{, }\DecValTok{1}\OperatorTok{:}\DecValTok{4}\NormalTok{, LETTERS[}\DecValTok{1}\OperatorTok{:}\DecValTok{4}\NormalTok{], }\DataTypeTok{col.axis=}\StringTok{"blue"}\NormalTok{)}
\CommentTok{# atas}
\KeywordTok{axis}\NormalTok{(}\DecValTok{3}\NormalTok{, }\DataTypeTok{col =} \StringTok{"darkgreen"}\NormalTok{, }\DataTypeTok{lty =} \DecValTok{2}\NormalTok{, }\DataTypeTok{lwd =} \FloatTok{0.5}\NormalTok{)}
\CommentTok{# kanan}
\KeywordTok{axis}\NormalTok{(}\DecValTok{4}\NormalTok{, }\DataTypeTok{col =} \StringTok{"violet"}\NormalTok{, }\DataTypeTok{col.axis =} \StringTok{"dark violet"}\NormalTok{, }\DataTypeTok{lwd =} \DecValTok{2}\NormalTok{)}
\end{Highlighting}
\end{Shaded}

\begin{figure}

{\centering \includegraphics[width=0.7\linewidth]{EnvStat_files/figure-latex/axis-1} 

}

\caption{Menambahkan axis}\label{fig:axis}
\end{figure}

Kita dapat mengubah rentang nilai pada axis menggunakan fungsi
\texttt{xlim()} dan \texttt{ylim()} yang menyatakan vektor nilai masimum
dan minimum rentang. Selain itu kita dapat juga melakukan tranformasi
baik pada sumbu x dan sumbu y. Berikut adalah argumen yang dapat
ditambahkan pada fungsi grafik:

\begin{itemize}
\tightlist
\item
  \textbf{xlim}: limit nilai sumbu x dengan format:
  \texttt{xlim(min,\ max)}.
\item
  \textbf{ylim}: limit nilai sumbu x dengan format:
  \texttt{ylim(min,\ max)}.
\end{itemize}

Untuk transformasi skala log, kita dapat menambahkan argumen berikut:

\begin{itemize}
\tightlist
\item
  \textbf{log=``x''}: transformasi log sumbu x.
\item
  \textbf{log=``y''}: transformasi log sumbu y.
\item
  \textbf{log=``xy''}: transformasi log sumbu x dan y.
\end{itemize}

Berikut adalah contoh sintaks penerapan argumen tersebut beserta output
yang dihasilkan pada Gambar \ref{fig:axis2}:

\begin{Shaded}
\begin{Highlighting}[]
\CommentTok{# membagi jendela grafik menjadi 1 baris dan 3 kolom}
\KeywordTok{par}\NormalTok{(}\DataTypeTok{mfrow=}\KeywordTok{c}\NormalTok{(}\DecValTok{1}\NormalTok{,}\DecValTok{3}\NormalTok{))}

\CommentTok{# membuat vektor numerik}
\NormalTok{x<-}\KeywordTok{c}\NormalTok{(}\DecValTok{1}\OperatorTok{:}\DecValTok{10}\NormalTok{); y<-x}\OperatorTok{*}\NormalTok{x}

\CommentTok{# simple plot}
\KeywordTok{plot}\NormalTok{(x, y)}

\CommentTok{# plot dengan pengaturan rentang skala}
\KeywordTok{plot}\NormalTok{(x, y, }\DataTypeTok{xlim=}\KeywordTok{c}\NormalTok{(}\DecValTok{1}\NormalTok{,}\DecValTok{15}\NormalTok{), }\DataTypeTok{ylim=}\KeywordTok{c}\NormalTok{(}\DecValTok{1}\NormalTok{,}\DecValTok{150}\NormalTok{))}

\CommentTok{# plot dengan transformasi skala log}
\KeywordTok{plot}\NormalTok{(x, y, }\DataTypeTok{log=}\StringTok{"y"}\NormalTok{)}
\end{Highlighting}
\end{Shaded}

\begin{figure}

{\centering \includegraphics[width=0.8\linewidth]{EnvStat_files/figure-latex/axis2-1} 

}

\caption{Mengubah rentang dan skala axis}\label{fig:axis2}
\end{figure}

Kita dapat melakukan kustomisasi pada \emph{tick mark}. Kustomisasi yang
dapat dilakukan adalah merubah warna, \emph{font style}, ukuran font,
orientasi, serta menyembunyikan \emph{tick mark}.

Argumen yang ditambahkan adalah sebagai berikut:

\begin{itemize}
\item
  \textbf{col.axis}: warna \emph{tick mark}.
\item
  \textbf{font.axis}: integer yang menunjukkan \emph{font style}. Sama
  dengan pengaturan judul.
\item
  \textbf{cex.axis}: pengaturan ukuran \emph{tick mark}.
\item
  \textbf{las}: mengatur orientasi \emph{tick mark}. Nilai yang dapat
  dimasukkan adalah sebagai berikut:
\item
  \textbf{0}: paralel terhadap posisi axis (default)
\item
  \textbf{1}: selalu horizontal
\item
  \textbf{2}: selalu perpendikular dengan posisi axis
\item
  \textbf{3}: selalu vertikal
\item
  \textbf{xaxt} dan \textbf{yaxt}: karakter untuk menunjukkan apakah
  axis akan ditampilkan atau tidak. nilai dapat berupa ``n''(sembunyika)
  dan ``s''(tampilkan).
\end{itemize}

Berikut adalah contoh penerapan argumen tersebut beserta output pada
Gambar \ref{fig:axis3}:

\begin{Shaded}
\begin{Highlighting}[]
\CommentTok{# membuat vektor numerik}
\NormalTok{x<-}\KeywordTok{c}\NormalTok{(}\DecValTok{1}\OperatorTok{:}\DecValTok{10}\NormalTok{); y<-x}\OperatorTok{*}\NormalTok{x}

\CommentTok{# plot}
\KeywordTok{plot}\NormalTok{(x,y,}
     \CommentTok{# warna}
     \DataTypeTok{col.axis=}\StringTok{"red"}\NormalTok{,}
     \CommentTok{# font style}
     \DataTypeTok{font.axis=}\DecValTok{2}\NormalTok{,}
     \CommentTok{# ukuran}
     \DataTypeTok{cex=}\FloatTok{1.5}\NormalTok{,}
     \CommentTok{# orientasi}
     \DataTypeTok{las=}\DecValTok{3}\NormalTok{,}
     \CommentTok{# sembunyikan sumbu x}
     \DataTypeTok{xaxt=}\StringTok{"n"}\NormalTok{)}
\end{Highlighting}
\end{Shaded}

\begin{figure}

{\centering \includegraphics[width=0.7\linewidth]{EnvStat_files/figure-latex/axis3-1} 

}

\caption{Kustomisasi tick mark}\label{fig:axis3}
\end{figure}

\subsection{Mengatur Warna}\label{mengatur-warna}

Pada fungsi dasar \texttt{R}, warna dapat diatur dengan mengetikkan nama
warna maupun kode hexadesimal. Selain itu kita juga dapat menamambahkan
warna lain melalui library lain yang tidak dijelaskan pada chapter ini.

Untuk penggunaan warna hexadesima kita perlu mengetikkan ``\#'' yang
diukuti oleh 6 kode warna. Untuk memperlajari kode-kode dan warna yang
dihasilkan, silahkan pembaca mengunjungi situs
\url{http://www.visibone.com/}.

Pada sintaks berikut disajikan visualisasi nama-nama warna bawaan yang
ada pada \texttt{R}. Output yang dihasilkan disajikan pada Gambar
\ref{fig:color}:

\begin{Shaded}
\begin{Highlighting}[]
\NormalTok{showCols <-}\StringTok{ }\ControlFlowTok{function}\NormalTok{(}\DataTypeTok{cl=}\KeywordTok{colors}\NormalTok{(), }\DataTypeTok{bg =} \StringTok{"grey"}\NormalTok{,}
                     \DataTypeTok{cex =} \FloatTok{0.75}\NormalTok{, }\DataTypeTok{rot =} \DecValTok{30}\NormalTok{) \{}
\NormalTok{    m <-}\StringTok{ }\KeywordTok{ceiling}\NormalTok{(}\KeywordTok{sqrt}\NormalTok{(n <-}\KeywordTok{length}\NormalTok{(cl)))}
    \KeywordTok{length}\NormalTok{(cl) <-}\StringTok{ }\NormalTok{m}\OperatorTok{*}\NormalTok{m; cm <-}\StringTok{ }\KeywordTok{matrix}\NormalTok{(cl, m)}
    \KeywordTok{require}\NormalTok{(}\StringTok{"grid"}\NormalTok{)}
    \KeywordTok{grid.newpage}\NormalTok{(); vp <-}\StringTok{ }\KeywordTok{viewport}\NormalTok{(}\DataTypeTok{w =}\NormalTok{ .}\DecValTok{92}\NormalTok{, }\DataTypeTok{h =}\NormalTok{ .}\DecValTok{92}\NormalTok{)}
    \KeywordTok{grid.rect}\NormalTok{(}\DataTypeTok{gp=}\KeywordTok{gpar}\NormalTok{(}\DataTypeTok{fill=}\NormalTok{bg))}
    \KeywordTok{grid.text}\NormalTok{(cm, }\DataTypeTok{x =} \KeywordTok{col}\NormalTok{(cm)}\OperatorTok{/}\NormalTok{m, }\DataTypeTok{y =} \KeywordTok{rev}\NormalTok{(}\KeywordTok{row}\NormalTok{(cm))}\OperatorTok{/}\NormalTok{m, }\DataTypeTok{rot =}\NormalTok{ rot,}
              \DataTypeTok{vp=}\NormalTok{vp, }\DataTypeTok{gp=}\KeywordTok{gpar}\NormalTok{(}\DataTypeTok{cex =}\NormalTok{ cex, }\DataTypeTok{col =}\NormalTok{ cm))}
\NormalTok{\}}

\CommentTok{# print 60 nama warna pertama}
\KeywordTok{showCols}\NormalTok{(}\DataTypeTok{bg=}\StringTok{"gray20"}\NormalTok{, }\DataTypeTok{cl=}\KeywordTok{colors}\NormalTok{()[}\DecValTok{1}\OperatorTok{:}\DecValTok{60}\NormalTok{], }\DataTypeTok{rot=}\DecValTok{30}\NormalTok{, }\DataTypeTok{cex=}\FloatTok{0.9}\NormalTok{)}
\end{Highlighting}
\end{Shaded}

\begin{figure}

{\centering \includegraphics[width=0.7\linewidth]{EnvStat_files/figure-latex/color-1} 

}

\caption{Nama warna}\label{fig:color}
\end{figure}

\section{Alternatif Library Dasar
Lain}\label{alternatif-library-dasar-lain}

Kita juga dapat melakukan visualisasi menggunakan library lain yang
memiliki tampilan mirip dengan fungsi visualisasi dasar \texttt{R}.
Bedanya adalah library-library ini memberikan fungsi tambahan sehingga
visualisasi yang dihasilkan menjadi lebih praktis.

\subsection{Scatterplot Menggunakan Library
car}\label{scatterplot-menggunakan-library-car}

Library \texttt{car} menyediakan alternatif lain visualisasi menggunakan
scatterplot. Berikut adalah contoh sintaks dan output yang dihasilkan
pada Gambar \ref{fig:carscatter}:

\begin{Shaded}
\begin{Highlighting}[]
\CommentTok{# memasang paket}
\CommentTok{# install.packages("car")}

\CommentTok{# memuat paket}
\KeywordTok{library}\NormalTok{(car)}

\CommentTok{# plot}
\KeywordTok{scatterplot}\NormalTok{(Volume}\OperatorTok{~}\NormalTok{Height, }\DataTypeTok{data=}\NormalTok{trees)}
\end{Highlighting}
\end{Shaded}

\begin{figure}

{\centering \includegraphics[width=0.7\linewidth]{EnvStat_files/figure-latex/carscatter-1} 

}

\caption{Enhanced scatterplot}\label{fig:carscatter}
\end{figure}

Pada grafik tersebut terkandung beberapa elemen penting, yaitu:

\begin{itemize}
\tightlist
\item
  titik observasi
\item
  garis regresi (garis lurus)
\item
  non-parametric regression smooth (\emph{dashed line})
\item
  garis smoothed conditional (\emph{point dashed line})
\item
  box plot masing-masing variabel.
\end{itemize}

\subsection{Matriks Scatterplot Menggunakan Library
psych}\label{matriks-scatterplot-menggunakan-library-psych}

FUngsi \texttt{pairs.panels()} pada library \texttt{psych} dapat
digunakan untuk membuat matriks scatterplot. Grafik yang dihasilkan juga
lebih ringkas dan menampilkan fungsional lain pada bagian diagonal lain
berupa histogram dan density plot yang dapat menunjukkan distribusi dari
variabel yang ada. Selain itu pada fungsionalitas grafik juga dapat
ditingkatkan dengan penambahan nilai korelasi antar variabel yang secara
default ditambahkan pada panel atas. Berikut adalah contoh sintaks dan
output yang dihasilkan pada Gambar \ref{fig:psychscatter}:

\begin{Shaded}
\begin{Highlighting}[]
\CommentTok{# memasang paket}
\CommentTok{# install.packages("psych")}

\CommentTok{# memuat paket}
\KeywordTok{library}\NormalTok{(psych)}

\CommentTok{# plot}
\KeywordTok{pairs.panels}\NormalTok{(trees, }
             \DataTypeTok{method =} \StringTok{"pearson"}\NormalTok{, }\CommentTok{# metode korelasi}
             \DataTypeTok{hist.col =} \StringTok{"grey"}\NormalTok{,}
             \DataTypeTok{density =} \OtherTok{TRUE}\NormalTok{,  }\CommentTok{# menampilkan plot densitas}
             \DataTypeTok{ellipses =} \OtherTok{FALSE}\NormalTok{, }\CommentTok{# menampilkancorrelation ellipses}
             \DataTypeTok{lm =} \OtherTok{TRUE} \CommentTok{# menampilkan garis regresi linier}
\NormalTok{             )}
\end{Highlighting}
\end{Shaded}

\begin{figure}

{\centering \includegraphics[width=0.7\linewidth]{EnvStat_files/figure-latex/psychscatter-1} 

}

\caption{Enhanced scatterplot matrices}\label{fig:psychscatter}
\end{figure}

\subsection{Box Plot Menggunakan Library
gplots}\label{box-plot-menggunakan-library-gplots}

Fungsi \texttt{boxplot2()} pada paket \texttt{gplots} memberikan
fungsionalitas lebih dibandingkan box plot yang dihasilkan dari fungsi
dasar \texttt{R}. Plot yang dihasilkan akan menampilkan jumlah observasi
pada tiap box. Berikut adalah contoh sintask penerapan dan output yang
dihasilkan pada Gambar \ref{fig:gplotsboxplot2}:

\begin{Shaded}
\begin{Highlighting}[]
\CommentTok{# memasang paket}
\CommentTok{# install.packages("gplots")}

\CommentTok{# memuat paket}
\KeywordTok{library}\NormalTok{(gplots)}

\CommentTok{# plot}
\KeywordTok{boxplot2}\NormalTok{(len }\OperatorTok{~}\StringTok{ }\NormalTok{dose, }\DataTypeTok{data =}\NormalTok{ ToothGrowth)}
\end{Highlighting}
\end{Shaded}

\begin{figure}

{\centering \includegraphics[width=0.7\linewidth]{EnvStat_files/figure-latex/gplotsboxplot2-1} 

}

\caption{Enhanced box plot}\label{fig:gplotsboxplot2}
\end{figure}

\subsection{QQ Plot Menggunakan Library
car}\label{qq-plot-menggunakan-library-car}

Fungsi \texttt{qqPlot()} pada library \texttt{car} dapat pula digunakan
untuk membuat qq plot. Kelebihannya adalah qqplot yang dihasilkan akan
dilengkapi dengan garis referensi yang memudahkan dalam membaca apakah
data masih dalam rentang distribusi normal atau tidak. Selain itu, untuk
membuatnya juga hanya diperlukan satu perintah saja. Hal ini tentu
berbeda ketika kita menggunakan fungsi dasar \texttt{R}. Berikut adalah
contoh sintask penerapan dan output yang dihasilkan pada Gambar
\ref{fig:carqqplot}:

\begin{Shaded}
\begin{Highlighting}[]
\CommentTok{# memasang paket}
\CommentTok{# install.packages("car")}

\CommentTok{# memuat paket}
\KeywordTok{library}\NormalTok{(car)}

\CommentTok{# plot}
\KeywordTok{qqPlot}\NormalTok{(trees}\OperatorTok{$}\NormalTok{Height)}
\end{Highlighting}
\end{Shaded}

\begin{figure}

{\centering \includegraphics[width=0.7\linewidth]{EnvStat_files/figure-latex/carqqplot-1} 

}

\caption{Enhanced qq plot}\label{fig:carqqplot}
\end{figure}

\begin{verbatim}
## [1]  3 20
\end{verbatim}

\subsection{Plot Group Means Menggunakan Library
gplots}\label{plot-group-means-menggunakan-library-gplots}

Plot ini akan sering kita gunakan saat melakukan analisis statistik
menggunakan anova baik anova satu arah maupun dua arah. Plot ini berguna
untuk melihat adanya interaksi antar faktor saat melakukan analisis
anova dua arah. Berikut adalah contoh sintask penerapan dan output yang
dihasilkan pada Gambar \ref{fig:plotmeans}:

\begin{Shaded}
\begin{Highlighting}[]
\CommentTok{# memasang paket}
\CommentTok{# install.packages("gplots")}

\CommentTok{# memuat paket}
\KeywordTok{library}\NormalTok{(gplots)}

\CommentTok{# plot}
\KeywordTok{plotmeans}\NormalTok{(len }\OperatorTok{~}\StringTok{ }\NormalTok{dose, }\DataTypeTok{data =}\NormalTok{ ToothGrowth)}
\end{Highlighting}
\end{Shaded}

\begin{figure}

{\centering \includegraphics[width=0.7\linewidth]{EnvStat_files/figure-latex/plotmeans-1} 

}

\caption{Plot group means}\label{fig:plotmeans}
\end{figure}

\section{Referensi}\label{referensi-3}

\begin{enumerate}
\def\labelenumi{\arabic{enumi}.}
\tightlist
\item
  Maindonald, J.H. 2008. \textbf{Using R for Data Analysis and Graphics
  Introduction, Code and Commentary}. Centre for Mathematics and Its
  Applications Australian National University.
\item
  Scherber, C. 2007. \textbf{An introduction to statistical data
  analysis using R}. R\_Manual Goettingen.
\item
  Venables, W.N. Smith D.M. and R Core Team. 2018. \textbf{An
  Introduction to R}. R Manuals.
\item
  STHDA. \textbf{R Base Graphs}.
  \url{http://www.sthda.com/english/wiki/r-base-graphs}
\end{enumerate}

\chapter{Visualisasi Data Menggunakan
GGPLOT}\label{visualisasi-data-menggunakan-ggplot}

Library \texttt{ggplot2} merupakan implementasi dari \emph{The Grammar
of Graphics} yang ditulis oleh \textbf{Leland Wilkinson}.
\texttt{ggplot2} merupakan library yang dikembangkan oleh \textbf{Hadley
Wicham} ketika ia sedang menempuh kuliah di \textbf{Lowa State
Universuty} dan masih dikembangkan hingga sekarang.

\texttt{ggplot2} merupakan paket visualisasi yang powerfull. Kita dapat
menggunakannya bersamaan dengan \emph{piping operator} yang disediakan
oleh paket \texttt{dplyr} sehingga menambah kemudahan kita dalam
melakukan analisis data.

Grafik \texttt{ggplot2} terdiri dari sejumlah komponen kunci. Berikut
adalah sejumlah komponen kunci yang membentuk grafik \texttt{ggplot2}.

\begin{itemize}
\tightlist
\item
  \textbf{data frame}: menyimpan semua data yang akan ditampilkan di
  plot.
\item
  \textbf{aesthetic mapping}: menggambarkan bagaimana data dipetakan ke
  warna, ukuran, bentuk, lokasi. Dalam plot diberikan pada fungsi
  \texttt{aes()}
\item
  \textbf{geoms}: objek geometris seperti titik, garis, bentuk.
\item
  \textbf{facets}: menjelaskan bagaimana plot bersyarat / panel harus
  dibangun.
\item
  \textbf{stats}: transformasi statistik seperti binning, quantiles,
  smoothing.
\item
  \textbf{scales}: skala apa yang digunakan oleh \emph{aesthetic map}
  (contoh: pria = merah, wanita = biru).
\item
  \textbf{coordinate system}: menggambarkan sistem di mana lokasi geom
  akan digambarkan.
\end{itemize}

Sebelum kita mulai memcoba melakukan visualisasi data menggunakan
\texttt{ggplot2}, kita perlu menginstall dan memuat terlebih dahulu
library \texttt{ggplot2}. Berikut adalah sintaks yang digunakan untuk
menginstall dan memuat paket \texttt{ggplot2}:

\begin{Shaded}
\begin{Highlighting}[]
\CommentTok{# memasang paket}
\CommentTok{# install.packages('ggplot2')}

\CommentTok{# memuat paket}
\KeywordTok{library}\NormalTok{(ggplot2)}
\end{Highlighting}
\end{Shaded}

Dataset yang akan kita gunakan adalah dataset \texttt{gapminder}.
Dataset ini berisi data demografi penduduk dari berbagai negara dan
benua. Untuk dapat menggunakannya kita perlu menginstall dan memuatnya
terlebih dahulu. Berikut adalah sintaks untuk menginstall dan memuat
dataset tersebut:

\begin{Shaded}
\begin{Highlighting}[]
\CommentTok{# memasang paket}
\CommentTok{# install.packages("gapminder")}

\CommentTok{# memuat paket}
\KeywordTok{library}\NormalTok{(gapminder)}

\CommentTok{# memuat paket dplyr dan tibble}
\KeywordTok{library}\NormalTok{(dplyr)}
\KeywordTok{library}\NormalTok{(tibble)}
\end{Highlighting}
\end{Shaded}

\begin{Shaded}
\begin{Highlighting}[]
\CommentTok{# melihat struktur dataset}
\KeywordTok{glimpse}\NormalTok{(gapminder)}
\end{Highlighting}
\end{Shaded}

\begin{verbatim}
## Observations: 1,704
## Variables: 6
## $ country   <fct> Afghanistan, Afghanistan, Afghan...
## $ continent <fct> Asia, Asia, Asia, Asia, Asia, As...
## $ year      <int> 1952, 1957, 1962, 1967, 1972, 19...
## $ lifeExp   <dbl> 28.80, 30.33, 32.00, 34.02, 36.0...
## $ pop       <int> 8425333, 9240934, 10267083, 1153...
## $ gdpPercap <dbl> 779.4, 820.9, 853.1, 836.2, 740....
\end{verbatim}

\begin{Shaded}
\begin{Highlighting}[]
\CommentTok{# melihat variabel year}
\KeywordTok{unique}\NormalTok{(gapminder}\OperatorTok{$}\NormalTok{year)}
\end{Highlighting}
\end{Shaded}

\begin{verbatim}
##  [1] 1952 1957 1962 1967 1972 1977 1982 1987 1992 1997
## [11] 2002 2007
\end{verbatim}

Dataset gapminder memiliki 6 variabel dan 1704 observasi. 20 observasi
pertama dataset gapminder dapat dilihat pada Tabel \ref{tab:gapminder}

\begin{table}[t]

\caption{\label{tab:gapminder}20 observasi pertama dataset gapminder}
\centering
\begin{tabular}{l|l|r|r|r|r}
\hline
country & continent & year & lifeExp & pop & gdpPercap\\
\hline
Afghanistan & Asia & 1952 & 28.80 & 8425333 & 779.4\\
\hline
Afghanistan & Asia & 1957 & 30.33 & 9240934 & 820.9\\
\hline
Afghanistan & Asia & 1962 & 32.00 & 10267083 & 853.1\\
\hline
Afghanistan & Asia & 1967 & 34.02 & 11537966 & 836.2\\
\hline
Afghanistan & Asia & 1972 & 36.09 & 13079460 & 740.0\\
\hline
Afghanistan & Asia & 1977 & 38.44 & 14880372 & 786.1\\
\hline
Afghanistan & Asia & 1982 & 39.85 & 12881816 & 978.0\\
\hline
Afghanistan & Asia & 1987 & 40.82 & 13867957 & 852.4\\
\hline
Afghanistan & Asia & 1992 & 41.67 & 16317921 & 649.3\\
\hline
Afghanistan & Asia & 1997 & 41.76 & 22227415 & 635.3\\
\hline
Afghanistan & Asia & 2002 & 42.13 & 25268405 & 726.7\\
\hline
Afghanistan & Asia & 2007 & 43.83 & 31889923 & 974.6\\
\hline
Albania & Europe & 1952 & 55.23 & 1282697 & 1601.1\\
\hline
Albania & Europe & 1957 & 59.28 & 1476505 & 1942.3\\
\hline
Albania & Europe & 1962 & 64.82 & 1728137 & 2312.9\\
\hline
Albania & Europe & 1967 & 66.22 & 1984060 & 2760.2\\
\hline
Albania & Europe & 1972 & 67.69 & 2263554 & 3313.4\\
\hline
Albania & Europe & 1977 & 68.93 & 2509048 & 3533.0\\
\hline
Albania & Europe & 1982 & 70.42 & 2780097 & 3630.9\\
\hline
Albania & Europe & 1987 & 72.00 & 3075321 & 3738.9\\
\hline
\end{tabular}
\end{table}

\section{Scatterplot}\label{scatterplot}

Scatterplot dapat dibuat pada \texttt{ggplot2} menggunakan fungsi
\texttt{geom\_point()}. Format sederhananya dituliskan sebagai berikut:

\begin{Shaded}
\begin{Highlighting}[]
\KeywordTok{ggplot}\NormalTok{(data, }\KeywordTok{aes}\NormalTok{(...))}\OperatorTok{+}
\StringTok{  }\KeywordTok{geom_point}\NormalTok{(size, color, shape)}
\end{Highlighting}
\end{Shaded}

Berikut adalah contoh sederhana scatterplot variabel \texttt{lifeExp}
terhadap variabel \texttt{gdpPercap}. Output yang dihasilkan disajikan
pada Gambar \ref{fig:ggscatter}:

\begin{Shaded}
\begin{Highlighting}[]
\KeywordTok{ggplot}\NormalTok{(gapminder, }\KeywordTok{aes}\NormalTok{(gdpPercap, lifeExp))}\OperatorTok{+}
\StringTok{  }\KeywordTok{geom_point}\NormalTok{()}
\end{Highlighting}
\end{Shaded}

\begin{figure}

{\centering \includegraphics[width=0.7\linewidth]{EnvStat_files/figure-latex/ggscatter-1} 

}

\caption{Scatterplot lifeExp vs gdpPercap}\label{fig:ggscatter}
\end{figure}

Kita dapat mengubah warna, jenis, dan ukuran titik pada scatterplot.
Pengubahan warna dan jenis titik berguna untuk menunjukkan grup data
pada grafik. Sedangkan perubahan ukuran titik sangat berguna untuk
menunjukkan nilai variabel lain khususnya variabel kontinyu pada sebuah
titik. Berikut adalah contoh penerapannya. Output yang dihasilkan
disajikan pada Gambar \ref{fig:ggscatter2} sampai dengan Gambar
\ref{fig:ggscatter4}:

\begin{Shaded}
\begin{Highlighting}[]
\KeywordTok{ggplot}\NormalTok{(gapminder, }\KeywordTok{aes}\NormalTok{(gdpPercap,lifeExp, }\DataTypeTok{color=}\NormalTok{continent))}\OperatorTok{+}
\StringTok{  }\KeywordTok{geom_point}\NormalTok{()}\OperatorTok{+}
\StringTok{  }\CommentTok{# merubah sumbu x kedalam fungsi log}
\StringTok{  }\KeywordTok{scale_x_log10}\NormalTok{()}
\end{Highlighting}
\end{Shaded}

\begin{figure}

{\centering \includegraphics[width=0.7\linewidth]{EnvStat_files/figure-latex/ggscatter2-1} 

}

\caption{Scatterplot lifeExp vs gdpPercap tiap benua (1)}\label{fig:ggscatter2}
\end{figure}

\begin{Shaded}
\begin{Highlighting}[]
\KeywordTok{ggplot}\NormalTok{(gapminder, }\KeywordTok{aes}\NormalTok{(gdpPercap,lifeExp, }\DataTypeTok{shape=}\NormalTok{continent))}\OperatorTok{+}
\StringTok{  }\KeywordTok{geom_point}\NormalTok{()}\OperatorTok{+}
\StringTok{  }\CommentTok{# merubah sumbu x kedalam fungsi log}
\StringTok{  }\KeywordTok{scale_x_log10}\NormalTok{()}
\end{Highlighting}
\end{Shaded}

\begin{figure}

{\centering \includegraphics[width=0.7\linewidth]{EnvStat_files/figure-latex/ggscatter3-1} 

}

\caption{Scatterplot lifeExp vs gdpPercap tiap benua (2)}\label{fig:ggscatter3}
\end{figure}

\begin{Shaded}
\begin{Highlighting}[]
\KeywordTok{ggplot}\NormalTok{(gapminder, }\KeywordTok{aes}\NormalTok{(gdpPercap,lifeExp, }
                      \DataTypeTok{size=}\NormalTok{pop, }\DataTypeTok{color=}\NormalTok{continent))}\OperatorTok{+}
\StringTok{  }\KeywordTok{geom_point}\NormalTok{()}\OperatorTok{+}
\StringTok{  }\CommentTok{# merubah sumbu x kedalam fungsi log}
\StringTok{  }\KeywordTok{scale_x_log10}\NormalTok{()}
\end{Highlighting}
\end{Shaded}

\begin{figure}

{\centering \includegraphics[width=0.7\linewidth]{EnvStat_files/figure-latex/ggscatter4-1} 

}

\caption{Scatterplot lifeExp vs gdpPercap dan populasi tiap negara dan benua}\label{fig:ggscatter4}
\end{figure}

Untuk menujukkan asosiasi antara dua variabel kontinyu kita juga dapat
menambahkan garis regresi dan confidence interval garis regresinya.
Fungsi yang digunakan adalah \texttt{geom\_smooth()}. Secara default
fungsi tersebut akan membuat garis loess regression pada grafik. Agar
dapat membuat garis regresi linier kita perlu menambahkan argumen
\texttt{method="lm"}. Selain itu, jika kita tidak ingin menampilkan
garis confidence interval kita dapat menambahkan argumen
\texttt{se=FALSE}. Format sederhananya disajikan pada sintaks berikut:

\begin{Shaded}
\begin{Highlighting}[]
\KeywordTok{geom_smooth}\NormalTok{(}\DataTypeTok{method=}\StringTok{"auto"}\NormalTok{, }\DataTypeTok{se=}\OtherTok{TRUE}\NormalTok{, }\DataTypeTok{fullrange=}\OtherTok{FALSE}\NormalTok{, }\DataTypeTok{level=}\FloatTok{0.95}\NormalTok{)}
\end{Highlighting}
\end{Shaded}

\begin{quote}
\textbf{Note: }

\begin{itemize}
\item
  \textbf{method}: metode penghalusan yang digunakan. Nilai yang dapat
  dimasukkan adalah lm, glm, gam, loess, rlm.
\item
  method=``loess'': merupakan nilai default pada fungsi dan menghasilkan
  metode penghalusan loess regression.
\item
  method=``lm'': menghasilkan metode penghalusan regresi linier. Kita
  juga dapat melakukan spesifikasi terhadap fungsi persamaan regresi
  yang digunakan dengan menambahkan argumen
  formula=y\textasciitilde{}x\ldots{}.
\item
  \textbf{se}: nilai logis. Jika TRUE garis confidence interval akan
  ditampilkan sepanjang garis penghalusan.
\item
  \textbf{fullrange}: nilai logis. Jika TRUE kecocokan mencakup seluruh
  plot.
\item
  level: level confidence interal yang digunakan. Secara default
  bernilai 0.95.
\end{itemize}
\end{quote}

Berikut adalah contoh sintaks penerapan pada variabel \texttt{gdpPercap}
dan \texttt{lifeExp}. Output yang dihasilkan disajikan pada Gambar
\ref{fig:ggscatter5}:

\begin{Shaded}
\begin{Highlighting}[]
\KeywordTok{ggplot}\NormalTok{(gapminder, }\KeywordTok{aes}\NormalTok{(gdpPercap,lifeExp))}\OperatorTok{+}
\StringTok{  }\KeywordTok{geom_point}\NormalTok{()}\OperatorTok{+}
\StringTok{  }\CommentTok{# merubah sumbu x kedalam fungsi log}
\StringTok{  }\KeywordTok{scale_x_log10}\NormalTok{()}\OperatorTok{+}
\StringTok{  }\CommentTok{# menambahkan smoothing method}
\StringTok{  }\KeywordTok{geom_smooth}\NormalTok{(}\DataTypeTok{method=}\StringTok{"lm"}\NormalTok{, }\DataTypeTok{level=}\FloatTok{0.99}\NormalTok{)}
\end{Highlighting}
\end{Shaded}

\begin{figure}

{\centering \includegraphics[width=0.7\linewidth]{EnvStat_files/figure-latex/ggscatter5-1} 

}

\caption{Scatterplot lifeExp vs gdpPercap dengan garis penghalusan regresi linier}\label{fig:ggscatter5}
\end{figure}

\section{Box Plot dan Violin Plot}\label{box-plot-dan-violin-plot}

Box plot merupakan visualisasi yang powerful dalam menggambarkan
distribusi data, melihat adanya outlier, serta membandingkan distribusi
antar data. Format visualisasi dapat dituliskan sebagai berikut:

\begin{Shaded}
\begin{Highlighting}[]
\KeywordTok{ggplot}\NormalTok{(data, }\KeywordTok{aes}\NormalTok{(...))}\OperatorTok{+}
\StringTok{  }\KeywordTok{geom_boxplot}\NormalTok{(}\KeywordTok{geom_boxplot}\NormalTok{(}\DataTypeTok{outlier.colour=}\StringTok{"black"}\NormalTok{, }
                            \DataTypeTok{outlier.shape=}\DecValTok{16}\NormalTok{,}
                            \DataTypeTok{outlier.size=}\DecValTok{2}\NormalTok{, }
                            \DataTypeTok{notch=}\OtherTok{FALSE}\NormalTok{))}
\end{Highlighting}
\end{Shaded}

\begin{quote}
\textbf{Note: }

\begin{itemize}
\tightlist
\item
  \textbf{outlier.colour, outlier.shape, outlier.size}: Warna, bentuk
  dan ukuran untuk titik-titik outlier.
\item
  \textbf{notch}: nilai logis. Jika TRUE, buat \textbf{notched box
  plot}. \emph{Notch} menunjukkan \emph{confidence interval} di sekitar
  median yang biasanya didasarkan pada median
  \(\pm1,58\cdot\frac{\left(IQR\right)}{\sqrt{\left(n\right)}}\).
  \emph{Notch} digunakan untuk membandingkan kelompok; jika takik dua
  kotak tidak tumpang tindih, ini adalah bukti kuat bahwa median
  berbeda.
\end{itemize}
\end{quote}

Berikut merupakan contoh visualisasi variabel \texttt{lifeExp} pada
dataset \texttt{gapminder}. Output yang dihasilkan disajikan pada Gambar
\ref{fig:ggboxplot}:

\begin{Shaded}
\begin{Highlighting}[]
\KeywordTok{ggplot}\NormalTok{(gapminder, }\KeywordTok{aes}\NormalTok{(}\StringTok{""}\NormalTok{, lifeExp))}\OperatorTok{+}
\StringTok{  }\KeywordTok{geom_boxplot}\NormalTok{()}
\end{Highlighting}
\end{Shaded}

\begin{figure}

{\centering \includegraphics[width=0.7\linewidth]{EnvStat_files/figure-latex/ggboxplot-1} 

}

\caption{Box plot variabel lifeExp}\label{fig:ggboxplot}
\end{figure}

Kita dapat melakukan visualisasi bagi setiap kelompok data. Pada sintaks
berikut visualisasi dilakukan untuk variabel \texttt{lifeExp} pada tiap
\texttt{continent}. Pada contoh berikut akan ditampilkan cara
menmabahkan titik rata-rata dan warna pada masing-masing grup. Output
yang dihasilkan disajikan pada Gambar \ref{fig:ggboxplot2}:

\begin{Shaded}
\begin{Highlighting}[]
\KeywordTok{ggplot}\NormalTok{(gapminder, }\KeywordTok{aes}\NormalTok{(continent, lifeExp, }\DataTypeTok{color=}\NormalTok{continent))}\OperatorTok{+}
\StringTok{  }\KeywordTok{geom_boxplot}\NormalTok{()}\OperatorTok{+}
\StringTok{  }\KeywordTok{stat_summary}\NormalTok{(}\DataTypeTok{fun.y=}\NormalTok{mean, }\DataTypeTok{geom=}\StringTok{"point"}\NormalTok{, }
               \DataTypeTok{shape=}\DecValTok{23}\NormalTok{, }\DataTypeTok{size=}\DecValTok{3}\NormalTok{, }\DataTypeTok{color=}\StringTok{"red"}\NormalTok{)}
\end{Highlighting}
\end{Shaded}

\begin{figure}

{\centering \includegraphics[width=0.7\linewidth]{EnvStat_files/figure-latex/ggboxplot2-1} 

}

\caption{Box plot variabel lifeExp pada tiap continent}\label{fig:ggboxplot2}
\end{figure}

Misalkan kita ingin mengetahui perubahan distribusi dari variabel
\texttt{lifeExp} pada masing-masing \texttt{continet} pada tahun 1952
dan 2007. Untuk melakukannya kita perlu melakukan subset pada dataset
\texttt{gapminder} untuk memfilter data pada tahun 1952 dan 2007. Data
selanjutnya dilakukan input kedalam fungsi \texttt{ggplot()}. Berikut
adalah contoh sintaks yang digunakan. Output yang dihasilkan disajikan
pada Gambar \ref{fig:ggboxplot3}:

\begin{Shaded}
\begin{Highlighting}[]
\NormalTok{gapminder }\OperatorTok
\StringTok{  }\KeywordTok{filter}\NormalTok{(year}\OperatorTok{==}\DecValTok{1952} \OperatorTok{|}\StringTok{ }\NormalTok{year}\OperatorTok{==}\DecValTok{2007}\NormalTok{) }\OperatorTok
\StringTok{  }\KeywordTok{ggplot}\NormalTok{(}\KeywordTok{aes}\NormalTok{(continent, lifeExp, }\DataTypeTok{fill=}\KeywordTok{factor}\NormalTok{(year)))}\OperatorTok{+}
\StringTok{  }\KeywordTok{geom_boxplot}\NormalTok{(}\DataTypeTok{notch=}\OtherTok{TRUE}\NormalTok{)}
\end{Highlighting}
\end{Shaded}

\begin{figure}

{\centering \includegraphics[width=0.7\linewidth]{EnvStat_files/figure-latex/ggboxplot3-1} 

}

\caption{Box plot variabel lifeExp pada tiap continent (1952 dan 2007)}\label{fig:ggboxplot3}
\end{figure}

Berdasarkan Gambar \ref{fig:ggboxplot3} terlihat bahwa usia harapan
hidup pada tiap benua meningkat sejak tahun 1952 sampai 2007. Selain
itu, peningkatan tersebut bersifat signifikan yang ditunjukkan dari
tidak adanya \emph{notch} yang saling overlap pada masing-masing benua.

Untuk lebih detailnya kita akan coba melakukan visualisasi pada benua
Asia untuk melihat perubahan variabel \texttt{lifeExp}. Berikut adalah
sintaks yang digunakan dan output yang dihasilkan disajikan pada Gambar
\ref{fig:ggboxplot4}:

\begin{Shaded}
\begin{Highlighting}[]
\NormalTok{gapminder }\OperatorTok
\StringTok{  }\KeywordTok{filter}\NormalTok{(continent}\OperatorTok{==}\StringTok{"Asia"}\NormalTok{) }\OperatorTok
\StringTok{  }\KeywordTok{ggplot}\NormalTok{(}\KeywordTok{aes}\NormalTok{(}\KeywordTok{factor}\NormalTok{(year), lifeExp))}\OperatorTok{+}
\StringTok{  }\KeywordTok{geom_boxplot}\NormalTok{()}
\end{Highlighting}
\end{Shaded}

\begin{figure}

{\centering \includegraphics[width=0.7\linewidth]{EnvStat_files/figure-latex/ggboxplot4-1} 

}

\caption{Box plot variabel lifeExp Benua Asia}\label{fig:ggboxplot4}
\end{figure}

Violin plot memiliki kesamaan dengan box plot. Perbedaanya terletak pada
violin plot tidak hanya menyajikan data titik-titikkuartil data, namun
violin plot juga menampilkan kernel probabilitas distibusi data. Fungsi
yang digunakan untuk membuatnya adalah \texttt{geom\_violin()}.

Pada dataset \texttt{gapminder} kita ingin meisualisasikan distribusi
\texttt{lifeExp} pada masing-masing \texttt{continent}. Berikut adalah
contoh sintaks untuk membuat visualisasi dasar violin plot. Output yang
dihasilkan disajikan pada Gambar \ref{fig:ggviolin}:

\begin{Shaded}
\begin{Highlighting}[]
\NormalTok{gapminder }\OperatorTok
\StringTok{  }\KeywordTok{ggplot}\NormalTok{(}\KeywordTok{aes}\NormalTok{(continent, lifeExp, }\DataTypeTok{fill=}\NormalTok{continent))}\OperatorTok{+}
\StringTok{  }\CommentTok{# violin plot}
\StringTok{  }\KeywordTok{geom_violin}\NormalTok{()}
\end{Highlighting}
\end{Shaded}

\begin{figure}

{\centering \includegraphics[width=0.7\linewidth]{EnvStat_files/figure-latex/ggviolin-1} 

}

\caption{Violin plot variabel lifeExp pada masing-masing benua}\label{fig:ggviolin}
\end{figure}

Kita juga dapat melakukan modifikasi terhadap violin plot tersebut
seperti penambahan titik kuartil, titik mean dan modifikasi terhadap
warna tampilaknnya. COntoh sintaksnya dan output disajikan pada Gambar
\ref{fig:ggviolin2}:

\begin{Shaded}
\begin{Highlighting}[]
\NormalTok{gapminder }\OperatorTok
\StringTok{  }\KeywordTok{ggplot}\NormalTok{(}\KeywordTok{aes}\NormalTok{(continent, lifeExp, }\DataTypeTok{fill=}\NormalTok{continent))}\OperatorTok{+}
\StringTok{  }\CommentTok{# violin plot}
\StringTok{  }\KeywordTok{geom_violin}\NormalTok{()}\OperatorTok{+}
\StringTok{  }\CommentTok{# menambahkan boxplot dengan lebar 0.1}
\StringTok{  }\KeywordTok{geom_boxplot}\NormalTok{(}\DataTypeTok{width=}\FloatTok{0.1}\NormalTok{, }\DataTypeTok{fill=}\StringTok{"white"}\NormalTok{)}\OperatorTok{+}
\StringTok{  }\CommentTok{# menambahkan titik mean}
\StringTok{  }\KeywordTok{stat_summary}\NormalTok{(}\DataTypeTok{fun.y=}\NormalTok{mean, }\DataTypeTok{geom=}\StringTok{"point"}\NormalTok{,}
               \CommentTok{# ukuran dan jenis titik}
               \DataTypeTok{size=}\DecValTok{1}\NormalTok{, }\DataTypeTok{shape=}\DecValTok{23}\NormalTok{,}
               \CommentTok{# warna titik}
               \DataTypeTok{color=}\StringTok{"red"}\NormalTok{, }\DataTypeTok{fill=}\StringTok{"white"}\NormalTok{)}
\end{Highlighting}
\end{Shaded}

\begin{figure}

{\centering \includegraphics[width=0.7\linewidth]{EnvStat_files/figure-latex/ggviolin2-1} 

}

\caption{Violin plot variabel lifeExp pada masing-masing benua (2)}\label{fig:ggviolin2}
\end{figure}

\section{Bar Plot}\label{bar-plot-1}

Pada \texttt{ggplot2} bar plot dapat dibuat menggunakan fungsi
\texttt{geom\_bar()}. Untuk membuat bar plot, langkah pertama yang perlu
dilakukan adalah membuat tabulasi data variabel terlebih dahulu. Berikut
adalah contoh sintaks untuk membuat bar plot dari rata-rata
\texttt{lifeExp} pada masing-masing \texttt{continent}. Output yang
dihasilkan disajikan pada Gambar \ref{fig:ggbar}:

\begin{Shaded}
\begin{Highlighting}[]
\NormalTok{gapminder }\OperatorTok
\StringTok{  }\CommentTok{# kelompokkan berdasarkan continet}
\StringTok{  }\KeywordTok{group_by}\NormalTok{(continent)}\OperatorTok
\StringTok{  }\CommentTok{# membuat ringkasan data}
\StringTok{  }\KeywordTok{summarize}\NormalTok{(}\DataTypeTok{mean_lifeExp=}\KeywordTok{mean}\NormalTok{(lifeExp))}\OperatorTok
\StringTok{  }\CommentTok{# urutkan dari yang terbesar}
\StringTok{  }\KeywordTok{arrange}\NormalTok{(}\KeywordTok{desc}\NormalTok{(mean_lifeExp))}\OperatorTok
\StringTok{  }\CommentTok{# plot}
\StringTok{  }\KeywordTok{ggplot}\NormalTok{(}\KeywordTok{aes}\NormalTok{(continent, mean_lifeExp))}\OperatorTok{+}
\StringTok{  }\CommentTok{# membuat bar plot berdasarkan nilai observasi}
\StringTok{  }\KeywordTok{geom_bar}\NormalTok{(}\DataTypeTok{stat=}\StringTok{"identity"}\NormalTok{)}
\end{Highlighting}
\end{Shaded}

\begin{figure}

{\centering \includegraphics[width=0.7\linewidth]{EnvStat_files/figure-latex/ggbar-1} 

}

\caption{Bar plot rata-rata lifeExp masing-masing benua}\label{fig:ggbar}
\end{figure}

Kita juga dapat membuat bar plot dengan garis confidence interval. Untuk
melakukannya kita perlu terlebih dahulu menghitung standard error dari
data. Standard error selanjutnya digunakan untuk menghitung nilai atas
dan bawah dari nilai rata-rata. Berikut adalah contoh visualisasi bar
plot dengan confidence interval (Gambar \ref{fig:ggbar2}):

\begin{Shaded}
\begin{Highlighting}[]
\NormalTok{gapminder }\OperatorTok
\StringTok{  }\CommentTok{# kelompokkan berdasarkan continet}
\StringTok{  }\KeywordTok{group_by}\NormalTok{(continent)}\OperatorTok
\StringTok{  }\CommentTok{# membuat ringkasan data}
\StringTok{  }\KeywordTok{summarize}\NormalTok{(}\DataTypeTok{mean_lifeExp=}\KeywordTok{mean}\NormalTok{(lifeExp),}
            \DataTypeTok{n=}\KeywordTok{n}\NormalTok{(), }\DataTypeTok{sd=}\KeywordTok{sd}\NormalTok{(lifeExp), }
            \DataTypeTok{se=}\NormalTok{sd}\OperatorTok{/}\KeywordTok{sqrt}\NormalTok{(n))}\OperatorTok
\StringTok{  }\CommentTok{# plot}
\StringTok{  }\KeywordTok{ggplot}\NormalTok{(}\KeywordTok{aes}\NormalTok{(continent, mean_lifeExp))}\OperatorTok{+}
\StringTok{  }\CommentTok{# membuat bar plot}
\StringTok{  }\KeywordTok{geom_bar}\NormalTok{(}\DataTypeTok{stat=}\StringTok{"identity"}\NormalTok{, }\DataTypeTok{color=}\StringTok{"white"}\NormalTok{)}\OperatorTok{+}
\StringTok{  }\CommentTok{# menambahkan error bar}
\StringTok{  }\KeywordTok{geom_errorbar}\NormalTok{(}\KeywordTok{aes}\NormalTok{(}\DataTypeTok{ymin=}\NormalTok{mean_lifeExp}\OperatorTok{-}\NormalTok{se,}
                    \DataTypeTok{ymax=}\NormalTok{mean_lifeExp}\OperatorTok{+}\NormalTok{se),}
                \DataTypeTok{width=}\FloatTok{0.2}\NormalTok{)}
\end{Highlighting}
\end{Shaded}

\begin{figure}

{\centering \includegraphics[width=0.7\linewidth]{EnvStat_files/figure-latex/ggbar2-1} 

}

\caption{Bar plot rata-rata lifeExp masing-masing benua dengan confidence interval}\label{fig:ggbar2}
\end{figure}

Kita juga dapat melakukannya pada visualisasi data beberapa grup.
Berikut adalah contoh sintaks dan output (Gambar \ref{fig:ggbar3}) bar
plot dengan beberapa grup:

\begin{Shaded}
\begin{Highlighting}[]
\NormalTok{gapminder }\OperatorTok
\StringTok{  }\CommentTok{# filter data tahun 1952 dan 2007}
\StringTok{  }\KeywordTok{filter}\NormalTok{(year}\OperatorTok{==}\DecValTok{1952}\OperatorTok{|}\NormalTok{year}\OperatorTok{==}\DecValTok{2007}\NormalTok{)}\OperatorTok
\StringTok{  }\CommentTok{# Ubah year menjadi factor}
\StringTok{  }\KeywordTok{mutate}\NormalTok{(}\DataTypeTok{year=}\KeywordTok{as.factor}\NormalTok{(year))}\OperatorTok
\StringTok{  }\CommentTok{# kelompokkan berdasarkan continet}
\StringTok{  }\KeywordTok{group_by}\NormalTok{(continent,year)}\OperatorTok
\StringTok{  }\CommentTok{# membuat ringkasan data}
\StringTok{  }\KeywordTok{summarize}\NormalTok{(}\DataTypeTok{mean_lifeExp=}\KeywordTok{mean}\NormalTok{(lifeExp),}
            \DataTypeTok{n=}\KeywordTok{n}\NormalTok{(), }\DataTypeTok{sd=}\KeywordTok{sd}\NormalTok{(lifeExp), }
            \DataTypeTok{se=}\NormalTok{sd}\OperatorTok{/}\KeywordTok{sqrt}\NormalTok{(n))}\OperatorTok
\StringTok{  }\CommentTok{# plot}
\StringTok{  }\KeywordTok{ggplot}\NormalTok{(}\KeywordTok{aes}\NormalTok{(continent, mean_lifeExp, }
             \DataTypeTok{fill=}\NormalTok{year))}\OperatorTok{+}
\StringTok{  }\CommentTok{# membuat bar plot}
\StringTok{  }\KeywordTok{geom_bar}\NormalTok{(}\DataTypeTok{stat=}\StringTok{"identity"}\NormalTok{, }
           \DataTypeTok{position=}\KeywordTok{position_dodge}\NormalTok{())}\OperatorTok{+}
\StringTok{  }\CommentTok{# menambahkan error bar}
\StringTok{  }\KeywordTok{geom_errorbar}\NormalTok{(}\KeywordTok{aes}\NormalTok{(}\DataTypeTok{ymin=}\NormalTok{mean_lifeExp}\OperatorTok{-}\NormalTok{se,}
                    \DataTypeTok{ymax=}\NormalTok{mean_lifeExp}\OperatorTok{+}\NormalTok{se),}
                \DataTypeTok{width=}\FloatTok{0.2}\NormalTok{,}
                \DataTypeTok{position=}\KeywordTok{position_dodge}\NormalTok{(}\FloatTok{0.9}\NormalTok{))}
\end{Highlighting}
\end{Shaded}

\begin{figure}

{\centering \includegraphics[width=0.7\linewidth]{EnvStat_files/figure-latex/ggbar3-1} 

}

\caption{Bar plot rata-rata lifeExp masing-masing benua (1952 dan 2007) dengan confidence interval}\label{fig:ggbar3}
\end{figure}

\section{Line Plot}\label{line-plot-1}

Line plot dapat digunakan untuk menunjukkan adanya perubahan pada selang
waktu tertentu. Pada \texttt{ggplot2}, line plot dapat dibuat
menggunakan fungsi \texttt{geom\_line()}. Berikut adalah contoh sintaks
dan grafik (Gambar \ref{fig:ggline}) untuk membuat line plot:

\begin{Shaded}
\begin{Highlighting}[]
\NormalTok{gapminder}\OperatorTok
\StringTok{  }\CommentTok{# kelompokkan data berdasarkan year dan continent}
\StringTok{  }\KeywordTok{group_by}\NormalTok{(year,continent)}\OperatorTok
\StringTok{  }\CommentTok{# ringkasan data}
\StringTok{  }\KeywordTok{summarize}\NormalTok{(}\DataTypeTok{mean_lifeExp=}\KeywordTok{mean}\NormalTok{(lifeExp))}\OperatorTok
\StringTok{  }\CommentTok{# plot}
\StringTok{  }\KeywordTok{ggplot}\NormalTok{(}\KeywordTok{aes}\NormalTok{(year, mean_lifeExp, }
             \DataTypeTok{linetype=}\NormalTok{continent))}\OperatorTok{+}
\StringTok{  }\CommentTok{# membuat line plot}
\StringTok{  }\KeywordTok{geom_line}\NormalTok{()}\OperatorTok{+}
\StringTok{  }\CommentTok{# menambahkan point}
\StringTok{  }\KeywordTok{geom_point}\NormalTok{()}
\end{Highlighting}
\end{Shaded}

\begin{figure}

{\centering \includegraphics[width=0.7\linewidth]{EnvStat_files/figure-latex/ggline-1} 

}

\caption{Line plot lifeExp masing-masing benua }\label{fig:ggline}
\end{figure}

Kita juga dapat menambahkan error bar pada line plot. Berikut adalah
contoh sintak dan grafik (Gambar \ref{fig:ggline2}) yang dihasilkan:

\begin{Shaded}
\begin{Highlighting}[]
\NormalTok{gapminder}\OperatorTok
\StringTok{  }\CommentTok{# filter benua asia}
\StringTok{  }\KeywordTok{filter}\NormalTok{(continent}\OperatorTok{==}\StringTok{"Asia"}\NormalTok{)}\OperatorTok
\StringTok{  }\CommentTok{# kelompokkan data berdasarkan year dan continent}
\StringTok{  }\KeywordTok{group_by}\NormalTok{(year)}\OperatorTok
\StringTok{  }\CommentTok{# ringkasan data}
\StringTok{  }\KeywordTok{summarize}\NormalTok{(}\DataTypeTok{mean_lifeExp=}\KeywordTok{mean}\NormalTok{(lifeExp), }
            \DataTypeTok{sd=}\KeywordTok{sd}\NormalTok{(lifeExp))}\OperatorTok
\StringTok{  }\CommentTok{# plot}
\StringTok{  }\KeywordTok{ggplot}\NormalTok{(}\KeywordTok{aes}\NormalTok{(year, mean_lifeExp))}\OperatorTok{+}
\StringTok{  }\CommentTok{# membuat line plot}
\StringTok{  }\KeywordTok{geom_line}\NormalTok{()}\OperatorTok{+}
\StringTok{  }\CommentTok{# menambahkan point}
\StringTok{  }\KeywordTok{geom_point}\NormalTok{(}\DataTypeTok{size=}\DecValTok{2}\NormalTok{)}\OperatorTok{+}
\StringTok{  }\CommentTok{# menambahkan error bar}
\StringTok{  }\KeywordTok{geom_errorbar}\NormalTok{(}\KeywordTok{aes}\NormalTok{(}\DataTypeTok{ymin=}\NormalTok{mean_lifeExp}\OperatorTok{-}\NormalTok{sd,}
                    \DataTypeTok{ymax=}\NormalTok{mean_lifeExp}\OperatorTok{+}\NormalTok{sd),}
                \DataTypeTok{width=}\FloatTok{0.2}\NormalTok{, }\DataTypeTok{color=}\StringTok{"red"}\NormalTok{)}
\end{Highlighting}
\end{Shaded}

\begin{figure}

{\centering \includegraphics[width=0.7\linewidth]{EnvStat_files/figure-latex/ggline2-1} 

}

\caption{Histogram lifeExp }\label{fig:ggline2}
\end{figure}

\section{Pie Chart}\label{pie-chart-1}

Pie chart pada \texttt{ggplot2} dapat dibuat menggunakan fungsi
\texttt{geom\_bar()} dan \texttt{coord\_polar()}.Berikut adalah contoh
sintaks yang digunakan dan output (Gambar \ref{fig:ggpie}) yang
dihasilkan:

\begin{Shaded}
\begin{Highlighting}[]
\NormalTok{total <-}\StringTok{ }\KeywordTok{sum}\NormalTok{(gapminder}\OperatorTok{$}\NormalTok{pop)}
\NormalTok{gapminder}\OperatorTok
\StringTok{  }\CommentTok{# kelompokkan berdasarkan continent}
\StringTok{  }\KeywordTok{group_by}\NormalTok{(continent)}\OperatorTok
\StringTok{  }\CommentTok{# ringkasan data }
\StringTok{  }\KeywordTok{summarize}\NormalTok{(}\DataTypeTok{pop=}\KeywordTok{sum}\NormalTok{(}\KeywordTok{as.numeric}\NormalTok{(pop)), }\DataTypeTok{percent=}\NormalTok{(pop}\OperatorTok{/}\NormalTok{total)}\OperatorTok{*}\DecValTok{100}\NormalTok{)}\OperatorTok
\StringTok{  }\KeywordTok{ggplot}\NormalTok{(}\KeywordTok{aes}\NormalTok{(}\DataTypeTok{x=}\StringTok{""}\NormalTok{, percent, }\DataTypeTok{fill=}\NormalTok{continent))}\OperatorTok{+}
\StringTok{  }\KeywordTok{geom_bar}\NormalTok{(}\DataTypeTok{stat=}\StringTok{"identity"}\NormalTok{)}\OperatorTok{+}
\StringTok{  }\KeywordTok{coord_polar}\NormalTok{(}\StringTok{"y"}\NormalTok{, }\DataTypeTok{start=}\DecValTok{0}\NormalTok{)}
\end{Highlighting}
\end{Shaded}

\begin{figure}

{\centering \includegraphics[width=0.7\linewidth]{EnvStat_files/figure-latex/ggpie-1} 

}

\caption{Pie chart pop}\label{fig:ggpie}
\end{figure}

\section{Histogram dan Desity Plot}\label{histogram-dan-desity-plot}

Histogram pada \texttt{ggplot2} dapat dibuat dengan fungsi
\texttt{geom\_histogram()}. Berikut adalah sintaks untuk membuat
hitogram pada variabel \texttt{lifeExp}. Output yang dihasilkan
disajikan pada Gambar \ref{fig:gghist}:

\begin{Shaded}
\begin{Highlighting}[]
\NormalTok{gapminder }\OperatorTok
\StringTok{  }\KeywordTok{ggplot}\NormalTok{(}\KeywordTok{aes}\NormalTok{(lifeExp))}\OperatorTok{+}
\StringTok{  }\KeywordTok{geom_histogram}\NormalTok{()}
\end{Highlighting}
\end{Shaded}

\begin{figure}

{\centering \includegraphics[width=0.7\linewidth]{EnvStat_files/figure-latex/gghist-1} 

}

\caption{Histogram lifeExp }\label{fig:gghist}
\end{figure}

Kita dapat membuat grafik histogram berdasarkan grup data. Pada contoh
sebelumnya dibuat histogram berdasarkan variabel \texttt{continent}.
Berikut adalah sintaks dan output yang dihasilkan pada Gambar
\ref{fig:gghist2}:

\begin{Shaded}
\begin{Highlighting}[]
\NormalTok{gapminder }\OperatorTok
\StringTok{  }\KeywordTok{ggplot}\NormalTok{(}\KeywordTok{aes}\NormalTok{(lifeExp, }\DataTypeTok{fill=}\NormalTok{continent))}\OperatorTok{+}
\StringTok{  }\KeywordTok{geom_histogram}\NormalTok{(}\DataTypeTok{alpha=}\FloatTok{0.5}\NormalTok{, }
                 \CommentTok{# atur posisi agar sesuai grup}
                 \DataTypeTok{position=}\StringTok{"identity"}\NormalTok{,}
                 \DataTypeTok{color=}\StringTok{"black"}\NormalTok{)}
\end{Highlighting}
\end{Shaded}

\begin{figure}

{\centering \includegraphics[width=0.7\linewidth]{EnvStat_files/figure-latex/gghist2-1} 

}

\caption{Histogram lifeExp berdasarkan benua}\label{fig:gghist2}
\end{figure}

Density plot dapat dibuat dengan menggunakan fungsi
\texttt{geom\_density()}. Berikut adalah contoh sintaks untuk membuat
density plot variabel \texttt{lifeExp}. Output yang dihasilkan disajikan
pada Gambar \ref{fig:ggdens}:

\begin{Shaded}
\begin{Highlighting}[]
\NormalTok{gapminder }\OperatorTok
\StringTok{  }\KeywordTok{ggplot}\NormalTok{(}\KeywordTok{aes}\NormalTok{(lifeExp))}\OperatorTok{+}
\StringTok{  }\KeywordTok{geom_density}\NormalTok{()}
\end{Highlighting}
\end{Shaded}

\begin{figure}

{\centering \includegraphics[width=0.7\linewidth]{EnvStat_files/figure-latex/ggdens-1} 

}

\caption{Density plot lifeExp }\label{fig:ggdens}
\end{figure}

Kita juga dapat membuat grafik density berdasarkan grup data. Pada
contoh sebelumnya dibuat density plot berdasarkan variabel
\texttt{continent}. Berikut adalah sintaks dan output yang dihasilkan
pada Gambar \ref{fig:ggdens2}:

\begin{Shaded}
\begin{Highlighting}[]
\NormalTok{gapminder }\OperatorTok
\StringTok{  }\KeywordTok{ggplot}\NormalTok{(}\KeywordTok{aes}\NormalTok{(lifeExp, }\DataTypeTok{fill=}\NormalTok{continent))}\OperatorTok{+}
\StringTok{  }\KeywordTok{geom_density}\NormalTok{(}\DataTypeTok{alpha=}\FloatTok{0.5}\NormalTok{, }
                 \CommentTok{# atur posisi agar sesuai grup}
                 \DataTypeTok{position=}\StringTok{"identity"}\NormalTok{,}
                 \DataTypeTok{color=}\StringTok{"black"}\NormalTok{)}
\end{Highlighting}
\end{Shaded}

\begin{figure}

{\centering \includegraphics[width=0.7\linewidth]{EnvStat_files/figure-latex/ggdens2-1} 

}

\caption{Density plot lifeExp berdasarkan benua}\label{fig:ggdens2}
\end{figure}

Jika dinginkan kita juga dapat menambahkan density plot pada histogram.
Pada Gambar \ref{fig:hist} ditambahkan density plot sehingga dihasilkan
output seperti Gambar \ref{fig:ggdenshist}.

\begin{Shaded}
\begin{Highlighting}[]
\NormalTok{gapminder }\OperatorTok
\StringTok{  }\KeywordTok{ggplot}\NormalTok{(}\KeywordTok{aes}\NormalTok{(lifeExp))}\OperatorTok{+}
\StringTok{  }\KeywordTok{geom_histogram}\NormalTok{(}\KeywordTok{aes}\NormalTok{(}\DataTypeTok{y=}\NormalTok{..density..),}
                 \CommentTok{# spesifikasi warna bar}
                 \DataTypeTok{color=}\StringTok{"black"}\NormalTok{, }\DataTypeTok{fill=}\StringTok{"white"}\NormalTok{)}\OperatorTok{+}
\StringTok{  }\KeywordTok{geom_density}\NormalTok{(}\DataTypeTok{fill=}\StringTok{"red"}\NormalTok{, }\DataTypeTok{alpha=}\FloatTok{0.3}\NormalTok{)}
\end{Highlighting}
\end{Shaded}

\begin{figure}

{\centering \includegraphics[width=0.7\linewidth]{EnvStat_files/figure-latex/ggdenshist-1} 

}

\caption{histogram dan density plot lifeExp }\label{fig:ggdenshist}
\end{figure}

\section{QQ Plot}\label{qq-plot-1}

QQ plot pada paket \texttt{ggplot2} dapat dibuat dengan menggunakan
fungsi \texttt{stat\_qq()}. Berikut adalah contoh sintaks untuk
melakukannya. Output yang dihasilkan disajikna pada Gambar
\ref{fig:ggqq}.

\begin{Shaded}
\begin{Highlighting}[]
\KeywordTok{ggplot}\NormalTok{(gapminder, }\KeywordTok{aes}\NormalTok{(}\DataTypeTok{sample=}\NormalTok{lifeExp))}\OperatorTok{+}
\StringTok{  }\CommentTok{# qq plot}
\StringTok{  }\KeywordTok{stat_qq}\NormalTok{()}\OperatorTok{+}
\StringTok{  }\CommentTok{# garis referensi}
\StringTok{  }\KeywordTok{stat_qq_line}\NormalTok{()}
\end{Highlighting}
\end{Shaded}

\begin{figure}

{\centering \includegraphics[width=0.7\linewidth]{EnvStat_files/figure-latex/ggqq-1} 

}

\caption{QQ plot variabel lifeExp}\label{fig:ggqq}
\end{figure}

\section{Dot Plot}\label{dot-plot}

Dot plot dapat dibuat menggunakan fungsi \texttt{geom\_dotplot} atau
\texttt{geom\_jitter()}. Perbedaan keduanya adalah
\texttt{geom\_jitter()} menambahkan \emph{noise} pada plot sehingga
mencegah terjadinya \emph{overplotting}. Berikut adalah contoh sintaks
untuk membuat dotplot pada multiple group dan output yang dihasilkan
pada Gambar \ref{fig:dotplot}:

\begin{Shaded}
\begin{Highlighting}[]
\NormalTok{gapminder }\OperatorTok
\StringTok{  }\KeywordTok{filter}\NormalTok{(year}\OperatorTok{==}\DecValTok{1952} \OperatorTok{|}\StringTok{ }\NormalTok{year}\OperatorTok{==}\DecValTok{2007}\NormalTok{) }\OperatorTok
\StringTok{  }\KeywordTok{ggplot}\NormalTok{(}\KeywordTok{aes}\NormalTok{(continent, lifeExp, }\DataTypeTok{fill=}\KeywordTok{factor}\NormalTok{(year)))}\OperatorTok{+}
\StringTok{  }\KeywordTok{geom_dotplot}\NormalTok{(}\DataTypeTok{binaxis=}\StringTok{"y"}\NormalTok{, }
               \CommentTok{# spesifikasi posisi plot}
               \DataTypeTok{stackdir=}\StringTok{"center"}\NormalTok{,}
               \DataTypeTok{position=}\KeywordTok{position_dodge}\NormalTok{(}\FloatTok{0.8}\NormalTok{),}
               \DataTypeTok{size=}\FloatTok{0.1}\NormalTok{)}
\end{Highlighting}
\end{Shaded}

\begin{verbatim}
## Warning: Ignoring unknown parameters: size
\end{verbatim}

\begin{figure}

{\centering \includegraphics[width=0.7\linewidth]{EnvStat_files/figure-latex/dotplot-1} 

}

\caption{Dot plot variabel lifeExp masing-masing benua (1952-2007)}\label{fig:dotplot}
\end{figure}

Kita juga dapat menambahkan plot dari dari plot yang sudah ada seperti
box plot atau violin plot. Berikut adalah contoh sintaks dan output yang
dihasilkan pada Gambar \ref{fig:dotplot2}:

\begin{Shaded}
\begin{Highlighting}[]
\NormalTok{gapminder }\OperatorTok
\StringTok{  }\KeywordTok{filter}\NormalTok{(year}\OperatorTok{==}\DecValTok{1952} \OperatorTok{|}\StringTok{ }\NormalTok{year}\OperatorTok{==}\DecValTok{2007}\NormalTok{) }\OperatorTok
\StringTok{  }\KeywordTok{ggplot}\NormalTok{(}\KeywordTok{aes}\NormalTok{(continent, lifeExp, }\DataTypeTok{fill=}\KeywordTok{factor}\NormalTok{(year)))}\OperatorTok{+}
\StringTok{  }\CommentTok{# box plot dibawah}
\StringTok{  }\KeywordTok{geom_boxplot}\NormalTok{(}\DataTypeTok{position=}\KeywordTok{position_dodge}\NormalTok{(}\FloatTok{0.8}\NormalTok{))}\OperatorTok{+}
\StringTok{  }\CommentTok{# dot plot diatas}
\StringTok{  }\KeywordTok{geom_dotplot}\NormalTok{(}\DataTypeTok{binaxis=}\StringTok{"y"}\NormalTok{, }
               \CommentTok{# spesifikasi posisi plot}
               \DataTypeTok{stackdir=}\StringTok{"center"}\NormalTok{,}
               \DataTypeTok{position=}\KeywordTok{position_dodge}\NormalTok{(}\FloatTok{0.8}\NormalTok{))}
\end{Highlighting}
\end{Shaded}

\begin{figure}

{\centering \includegraphics[width=0.7\linewidth]{EnvStat_files/figure-latex/dotplot2-1} 

}

\caption{Dot plot variabel lifeExp masing-masing benua (1952-2007) (2)}\label{fig:dotplot2}
\end{figure}

\section{ECDF Plot}\label{ecdf-plot}

\emph{Empirical Cumulative Density FUnction} (ECDF) plot merupakan
grafik yang digunakan untuk menggambarkan ditribusi suatu data. Dari
grafik ini kita dapat mengetahui faraksi suatu data baik yang terendah
maupun yang tertinggi. ECDF pada \texttt{ggplot2} dapat dibuat dengan
dua cara yaitu dengan \texttt{geom\_line()} dan \texttt{stat\_ecdf()}.
Jika menggunakan fungsi \texttt{geom\_line()} kita perlu membuat fraksi
kumulatif dari variabel yang akan kita plotkan. Sedangkan dengan
menggunakan \texttt{stat\_ecdf()}, kita tidak perlu melakukannya karena
fungsi tersebut akan secara otomatis memproses data kita. Berikut adalah
sintaks dan output (Gambar \ref{fig:ggecdf}) contoh ecdf:

\begin{Shaded}
\begin{Highlighting}[]
\KeywordTok{ggplot}\NormalTok{(gapminder, }\KeywordTok{aes}\NormalTok{(lifeExp))}\OperatorTok{+}
\StringTok{  }\KeywordTok{stat_ecdf}\NormalTok{(}\DataTypeTok{geom=}\StringTok{"line"}\NormalTok{)}
\end{Highlighting}
\end{Shaded}

\begin{figure}

{\centering \includegraphics[width=0.7\linewidth]{EnvStat_files/figure-latex/ggecdf-1} 

}

\caption{ECDF plot variabel lifeExp}\label{fig:ggecdf}
\end{figure}

\section{Parameter Grafik}\label{parameter-grafik}

Pada bagian ini penulis akan menjelaskan bagaimana cara mengatur
parameter grafik seperti judul grafik, legend, warna, tema, dll.
Pengaturan parameter grafik pada \texttt{ggplot2} sebenarnya jauh lebih
sederhana dibandingkan dengan fungsi dasar visualisasi \texttt{R}.
Selain itu, kita dapat membuat tampilan grafik kita jauh lebih menarik
dengan membuat tema kustom pada grafik kita.

\subsection{Merubah Judul Grafik, Keterangan Axis dan
Legend}\label{merubah-judul-grafik-keterangan-axis-dan-legend}

Untuk merubah judul grafik dan keterangan axis kita dapat melakukannya
melalui dua cara. Cara pertama adalah dengan memasukkan mengubahnya satu
persatu menggunakan fungsi \texttt{ggtitle()} (judul grafik),
\texttt{xlab()} (keterangan sumbu x), dan \texttt{ylab()} (keterangan
pada sumbu y). Cara kedua adalah dengan menggunakan fungsi
\texttt{labs()} dimana selain dapat mengubah judul grafik dan keterangan
axis fungsi tersebut dapat juga digunakan untuk mengubah keterangan
legend.

Pada sintaks berikut penulis akan memberikan contoh bagaimana mengubah
judul grafik dan keterangan axis menggunakan dua cara tersebut. Output
yang dihasilkan disajikan pada Gambar \ref{fig:ggtitle}.

\begin{Shaded}
\begin{Highlighting}[]
\CommentTok{# Cara 1}
\KeywordTok{ggplot}\NormalTok{(gapminder, }\KeywordTok{aes}\NormalTok{(continent, gdpPercap, }\DataTypeTok{fill=}\NormalTok{continent))}\OperatorTok{+}
\StringTok{  }\CommentTok{# membuat box plot}
\StringTok{  }\KeywordTok{geom_boxplot}\NormalTok{()}\OperatorTok{+}
\StringTok{  }\CommentTok{# menambahkan judul}
\StringTok{  }\KeywordTok{ggtitle}\NormalTok{(}\StringTok{"GDP Per Capita Tiap Benua"}\NormalTok{)}\OperatorTok{+}
\StringTok{  }\CommentTok{# mengubah keterangan axis}
\StringTok{  }\KeywordTok{xlab}\NormalTok{(}\StringTok{"Benua"}\NormalTok{)}\OperatorTok{+}
\StringTok{  }\KeywordTok{ylab}\NormalTok{(}\StringTok{"GDP Per Kapita"}\NormalTok{)}
\end{Highlighting}
\end{Shaded}

\begin{Shaded}
\begin{Highlighting}[]
\CommentTok{# cara 2}
\KeywordTok{ggplot}\NormalTok{(gapminder, }\KeywordTok{aes}\NormalTok{(continent, gdpPercap, }\DataTypeTok{fill=}\NormalTok{continent))}\OperatorTok{+}
\StringTok{  }\CommentTok{# membuat box plot}
\StringTok{  }\KeywordTok{geom_boxplot}\NormalTok{()}\OperatorTok{+}
\StringTok{  }\CommentTok{# kustomisasi judul dan keterangan axis}
\StringTok{  }\KeywordTok{labs}\NormalTok{(}\DataTypeTok{title=}\StringTok{"GDP Per Capita Tiap Benua"}\NormalTok{,}
       \DataTypeTok{x=}\StringTok{"Benua"}\NormalTok{, }\DataTypeTok{y=}\StringTok{"GDP Per Kapita"}\NormalTok{)}
\end{Highlighting}
\end{Shaded}

\begin{figure}

{\centering \includegraphics[width=0.7\linewidth]{EnvStat_files/figure-latex/ggtitle-1} 

}

\caption{Mengubah judul grafik dan keterangan axis}\label{fig:ggtitle}
\end{figure}

Pada Gambar \ref{fig:ggtitle} kita belum mengubah keterangan legend.
Berikut adalah sintaks untuk mengubah keterangan legend pada grafik
tersebut beserta output yang disajikan pada Gambar \ref{fig:ggtitle2}.

\begin{Shaded}
\begin{Highlighting}[]
\CommentTok{# cara 2}
\KeywordTok{ggplot}\NormalTok{(gapminder, }\KeywordTok{aes}\NormalTok{(continent, gdpPercap, }
                      \CommentTok{# warna box berdasarkan benua}
                      \DataTypeTok{fill=}\NormalTok{continent))}\OperatorTok{+}
\StringTok{  }\CommentTok{# membuat box plot}
\StringTok{  }\KeywordTok{geom_boxplot}\NormalTok{()}\OperatorTok{+}
\StringTok{  }\CommentTok{# kustomisasi judul dan keterangan axis}
\StringTok{  }\KeywordTok{labs}\NormalTok{(}\DataTypeTok{title=}\StringTok{"GDP Per Capita Tiap Benua"}\NormalTok{,}
       \DataTypeTok{x=}\StringTok{"Benua"}\NormalTok{, }\DataTypeTok{y=}\StringTok{"GDP Per Kapita"}\NormalTok{,}
       \CommentTok{# mengubah keterangan legend}
       \DataTypeTok{fill=}\StringTok{"Benua"}\NormalTok{)}
\end{Highlighting}
\end{Shaded}

\begin{figure}

{\centering \includegraphics[width=0.7\linewidth]{EnvStat_files/figure-latex/ggtitle2-1} 

}

\caption{Mengubah keterangan legend pada grafik}\label{fig:ggtitle2}
\end{figure}

Judul, keterangan axis, dan keterangan legend dapat dikustomisasi
menggunakan fungsi \texttt{theme()} dan \texttt{element\_text()}.
Berikut adalah format yang digunakan:

\begin{Shaded}
\begin{Highlighting}[]
\CommentTok{# Judul}
\OperatorTok{<}\NormalTok{ggplot}\OperatorTok{>}\StringTok{ }\OperatorTok{+}\StringTok{ }\KeywordTok{theme}\NormalTok{(}\DataTypeTok{plot.title =} \KeywordTok{element_text}\NormalTok{(family, face, colour, size))}
\CommentTok{# keterangan sumbu x}
\OperatorTok{<}\NormalTok{ggplot}\OperatorTok{>}\StringTok{ }\OperatorTok{+}\StringTok{ }\KeywordTok{theme}\NormalTok{(}\DataTypeTok{axis.title.x =} \KeywordTok{element_text}\NormalTok{(family, face, colour, size))}
\CommentTok{# keterangan sumbu y}
\OperatorTok{<}\NormalTok{ggplot}\OperatorTok{>}\StringTok{ }\OperatorTok{+}\StringTok{ }\KeywordTok{theme}\NormalTok{(}\DataTypeTok{axis.title.y =} \KeywordTok{element_text}\NormalTok{(family, face, colour, size))}
\CommentTok{# keterangan legend}
\OperatorTok{<}\NormalTok{ggplot}\OperatorTok{>}\StringTok{ }\OperatorTok{+}\StringTok{ }\KeywordTok{theme}\NormalTok{(}\DataTypeTok{axis.title.y =} \KeywordTok{element_text}\NormalTok{(family, face, colour, size))}
\end{Highlighting}
\end{Shaded}

\begin{quote}
\textbf{Note: }

\begin{itemize}
\tightlist
\item
  \textbf{family}: font family.
\item
  \textbf{face}: tampilan font. Nilai yang dapat digunakan antara lain:
  ``plain'', ``italic'', ``bold'' dan ``bold.italic''.
\item
  \textbf{colour}: warna teks.
\item
  \textbf{size}: ukuran teks
\end{itemize}
\end{quote}

Berikut adalah contoh penerapan fungsi tersebut pada grafik Gambar
\ref{fig:ggtitle2}. Output yang dihasilkan disajikan pada Gambar
\ref{fig:ggtitle3}.

\begin{Shaded}
\begin{Highlighting}[]
\CommentTok{# cara 2}
\KeywordTok{ggplot}\NormalTok{(gapminder, }\KeywordTok{aes}\NormalTok{(continent, gdpPercap, }
                      \CommentTok{# warna box berdasarkan benua}
                      \DataTypeTok{fill=}\NormalTok{continent))}\OperatorTok{+}
\StringTok{  }\CommentTok{# membuat box plot}
\StringTok{  }\KeywordTok{geom_boxplot}\NormalTok{()}\OperatorTok{+}
\StringTok{  }\CommentTok{# kustomisasi judul dan keterangan axis}
\StringTok{  }\KeywordTok{labs}\NormalTok{(}\DataTypeTok{title=}\StringTok{"GDP Per Capita Tiap Benua"}\NormalTok{,}
       \DataTypeTok{x=}\StringTok{"Benua"}\NormalTok{, }\DataTypeTok{y=}\StringTok{"GDP Per Kapita"}\NormalTok{,}
       \CommentTok{# mengubah keterangan legend}
       \DataTypeTok{fill=}\StringTok{"Benua"}\NormalTok{)}\OperatorTok{+}
\StringTok{  }\KeywordTok{theme}\NormalTok{(}
      \DataTypeTok{plot.title =} \KeywordTok{element_text}\NormalTok{(}\DataTypeTok{color=}\StringTok{"red"}\NormalTok{, }\DataTypeTok{size=}\DecValTok{14}\NormalTok{, }\DataTypeTok{face=}\StringTok{"bold.italic"}\NormalTok{),}
      \DataTypeTok{axis.title.x =} \KeywordTok{element_text}\NormalTok{(}\DataTypeTok{color=}\StringTok{"blue"}\NormalTok{, }\DataTypeTok{size=}\DecValTok{14}\NormalTok{, }\DataTypeTok{face=}\StringTok{"bold"}\NormalTok{),}
      \DataTypeTok{axis.title.y =} \KeywordTok{element_text}\NormalTok{(}\DataTypeTok{color=}\StringTok{"#993333"}\NormalTok{, }\DataTypeTok{size=}\DecValTok{14}\NormalTok{, }\DataTypeTok{face=}\StringTok{"bold"}\NormalTok{),}
      \DataTypeTok{legend.text =} \KeywordTok{element_text}\NormalTok{(}\DataTypeTok{colour=}\StringTok{"blue"}\NormalTok{, }\DataTypeTok{size=}\DecValTok{10}\NormalTok{, }\DataTypeTok{face=}\StringTok{"bold"}\NormalTok{)}
\NormalTok{      )}
\end{Highlighting}
\end{Shaded}

\begin{figure}

{\centering \includegraphics[width=0.7\linewidth]{EnvStat_files/figure-latex/ggtitle3-1} 

}

\caption{Kustomisasi judul grafik dan keterangan axis}\label{fig:ggtitle3}
\end{figure}

\subsection{Merubah Tampilan dan Posisi
Legend}\label{merubah-tampilan-dan-posisi-legend}

Posisi legend dapat diubah dengan menambahkan argumen
\texttt{legend.position} pada fungsi \texttt{theme()}. Posisi legend
dapat diubah dengan memasukkan nilai berupa karakter seperti
``left'',``top'', ``right'', dan ``bottom''. Selain itu, posisi legend
dapat dispesifikasi menggunakan vektor numerik c(x,Y). Nilai x dan y
berkisar antara 0 sampai 1. Nilai c(0,0) menandakan posisi legend pada
bagian kiri bawah dan c(0,1) menyatakan kiri atas.

Penggunaan karakter dan vektor numerik akan menghasilkan output posisi
legend yang berbeda. Jika menggunakan karakter posisi legend akan diubah
diluar bidang plot. Sedangkan vektor numerik akan mengubah posisi legend
menjadi ada pada bidang plot. Untuk lebih memahaminya berikut disajikan
dua buah gambar. Gambar \ref{fig:gglegend} menyajikan pengaturan legend
menggunakan karakter, sedangkan Gambar \ref{fig:gglegend2} menyajikan
pengaturan legend menggunakan vektor numerik.

\begin{Shaded}
\begin{Highlighting}[]
\CommentTok{# cara 2}
\KeywordTok{ggplot}\NormalTok{(gapminder, }\KeywordTok{aes}\NormalTok{(continent, gdpPercap, }
                      \CommentTok{# warna box berdasarkan benua}
                      \DataTypeTok{fill=}\NormalTok{continent))}\OperatorTok{+}
\StringTok{  }\CommentTok{# membuat box plot}
\StringTok{  }\KeywordTok{geom_boxplot}\NormalTok{()}\OperatorTok{+}
\StringTok{  }\CommentTok{# kustomisasi judul dan keterangan axis}
\StringTok{  }\KeywordTok{labs}\NormalTok{(}\DataTypeTok{title=}\StringTok{"GDP Per Capita Tiap Benua"}\NormalTok{,}
       \DataTypeTok{x=}\StringTok{"Benua"}\NormalTok{, }\DataTypeTok{y=}\StringTok{"GDP Per Kapita"}\NormalTok{,}
       \CommentTok{# mengubah keterangan legend}
       \DataTypeTok{fill=}\StringTok{"Benua"}\NormalTok{)}\OperatorTok{+}
\StringTok{  }\KeywordTok{theme}\NormalTok{(}\DataTypeTok{legend.position=}\StringTok{"top"}\NormalTok{)}
\end{Highlighting}
\end{Shaded}

\begin{figure}

{\centering \includegraphics[width=0.7\linewidth]{EnvStat_files/figure-latex/gglegend-1} 

}

\caption{Kustomisasi posisi legend berdasarkan karakter}\label{fig:gglegend}
\end{figure}

\begin{Shaded}
\begin{Highlighting}[]
\CommentTok{# cara 2}
\KeywordTok{ggplot}\NormalTok{(gapminder, }\KeywordTok{aes}\NormalTok{(continent, gdpPercap, }
                      \CommentTok{# warna box berdasarkan benua}
                      \DataTypeTok{fill=}\NormalTok{continent))}\OperatorTok{+}
\StringTok{  }\CommentTok{# membuat box plot}
\StringTok{  }\KeywordTok{geom_boxplot}\NormalTok{()}\OperatorTok{+}
\StringTok{  }\CommentTok{# kustomisasi judul dan keterangan axis}
\StringTok{  }\KeywordTok{labs}\NormalTok{(}\DataTypeTok{title=}\StringTok{"GDP Per Capita Tiap Benua"}\NormalTok{,}
       \DataTypeTok{x=}\StringTok{"Benua"}\NormalTok{, }\DataTypeTok{y=}\StringTok{"GDP Per Kapita"}\NormalTok{,}
       \CommentTok{# mengubah keterangan legend}
       \DataTypeTok{fill=}\StringTok{"Benua"}\NormalTok{)}\OperatorTok{+}
\StringTok{  }\KeywordTok{theme}\NormalTok{(}\DataTypeTok{legend.position=}\KeywordTok{c}\NormalTok{(}\FloatTok{0.9}\NormalTok{,}\FloatTok{0.75}\NormalTok{))}
\end{Highlighting}
\end{Shaded}

\begin{figure}

{\centering \includegraphics[width=0.7\linewidth]{EnvStat_files/figure-latex/gglegend2-1} 

}

\caption{Kustomisasi posisi legend berdasarkan vektor numerik}\label{fig:gglegend2}
\end{figure}

Pada fungsi \texttt{theme()} kita juga dapat merubah backgroud dari
legend box menggunakan argumen \texttt{legend.bacground} dan
\texttt{element\_rect}. Selain itu kita juga dapat mengubah orientasi
dari legend yang semula vertikal menjadi horizontal dengan menambahkan
argumen \texttt{legend.box}. Berikut adalah contoh sintaks penerapannya.
Output yang dihasilkan disajikan pada Gambar \ref{fig:gglegend3}.

\begin{Shaded}
\begin{Highlighting}[]
\CommentTok{# cara 2}
\KeywordTok{ggplot}\NormalTok{(gapminder, }\KeywordTok{aes}\NormalTok{(continent, gdpPercap, }
                      \CommentTok{# warna box berdasarkan benua}
                      \DataTypeTok{fill=}\NormalTok{continent,}
                      \CommentTok{# warna outline berdasarkan benua}
                      \DataTypeTok{color=}\NormalTok{continent))}\OperatorTok{+}
\StringTok{  }\CommentTok{# membuat box plot}
\StringTok{  }\KeywordTok{geom_boxplot}\NormalTok{()}\OperatorTok{+}
\StringTok{  }\CommentTok{# kustomisasi judul dan keterangan axis}
\StringTok{  }\KeywordTok{labs}\NormalTok{(}\DataTypeTok{title=}\StringTok{"GDP Per Capita Tiap Benua"}\NormalTok{,}
       \DataTypeTok{x=}\StringTok{"Benua"}\NormalTok{, }\DataTypeTok{y=}\StringTok{"GDP Per Kapita"}\NormalTok{,}
       \CommentTok{# mengubah keterangan legend}
       \DataTypeTok{fill=}\StringTok{"Benua (fill)"}\NormalTok{,}
       \DataTypeTok{color=}\StringTok{"Benua (outline)"}\NormalTok{)}\OperatorTok{+}
\StringTok{  }\KeywordTok{theme}\NormalTok{(}\DataTypeTok{legend.position=}\StringTok{"bottom"}\NormalTok{,}
        \CommentTok{# mengubah tampilan legend box }
        \DataTypeTok{legend.background =} \KeywordTok{element_rect}\NormalTok{(}\DataTypeTok{fill=}\StringTok{"lightblue"}\NormalTok{,}
                                  \DataTypeTok{size=}\FloatTok{0.5}\NormalTok{, }\DataTypeTok{linetype=}\StringTok{"solid"}\NormalTok{, }
                                  \DataTypeTok{colour =}\StringTok{"darkblue"}\NormalTok{),}
        \CommentTok{# mengubah orientasi legend}
        \DataTypeTok{legend.box=} \StringTok{"horizontal"}\NormalTok{)}
\end{Highlighting}
\end{Shaded}

\begin{figure}

{\centering \includegraphics[width=0.7\linewidth]{EnvStat_files/figure-latex/gglegend3-1} 

}

\caption{Kustomisasi tampilan legend}\label{fig:gglegend3}
\end{figure}

Kita dapat juga menghilangkan legend baik seluruh legend maupun legend
spesifik. Pada Gambar \ref{fig:gglegend4} dan Gambar \ref{fig:gglegend5}
disajikan contoh cara menghilangkan seluruh legend maupun sebagian
legend.

\begin{Shaded}
\begin{Highlighting}[]
\CommentTok{# Menghilangkan seluruh legend}
\KeywordTok{ggplot}\NormalTok{(gapminder, }\KeywordTok{aes}\NormalTok{(continent, gdpPercap, }
                      \CommentTok{# warna box berdasarkan benua}
                      \DataTypeTok{fill=}\NormalTok{continent,}
                      \CommentTok{# warna outline berdasarkan benua}
                      \DataTypeTok{color=}\NormalTok{continent))}\OperatorTok{+}
\StringTok{  }\CommentTok{# membuat box plot}
\StringTok{  }\KeywordTok{geom_boxplot}\NormalTok{()}\OperatorTok{+}
\StringTok{  }\CommentTok{# kustomisasi judul dan keterangan axis}
\StringTok{  }\KeywordTok{labs}\NormalTok{(}\DataTypeTok{title=}\StringTok{"GDP Per Capita Tiap Benua"}\NormalTok{,}
       \DataTypeTok{x=}\StringTok{"Benua"}\NormalTok{, }\DataTypeTok{y=}\StringTok{"GDP Per Kapita"}\NormalTok{,}
       \CommentTok{# mengubah keterangan legend}
       \DataTypeTok{fill=}\StringTok{"Benua"}\NormalTok{)}\OperatorTok{+}
\StringTok{  }\KeywordTok{theme}\NormalTok{(}\DataTypeTok{legend.position=}\StringTok{"none"}\NormalTok{)}
\end{Highlighting}
\end{Shaded}

\begin{figure}

{\centering \includegraphics[width=0.7\linewidth]{EnvStat_files/figure-latex/gglegend4-1} 

}

\caption{Menghilangkan seluruh legend}\label{fig:gglegend4}
\end{figure}

\begin{Shaded}
\begin{Highlighting}[]
\CommentTok{# Menghilangkan seluruh legend}
\KeywordTok{ggplot}\NormalTok{(gapminder, }\KeywordTok{aes}\NormalTok{(continent, gdpPercap, }
                      \CommentTok{# warna box berdasarkan benua}
                      \DataTypeTok{fill=}\NormalTok{continent,}
                      \CommentTok{# warna outline berdasarkan benua}
                      \DataTypeTok{color=}\NormalTok{continent))}\OperatorTok{+}
\StringTok{  }\CommentTok{# membuat box plot}
\StringTok{  }\KeywordTok{geom_boxplot}\NormalTok{()}\OperatorTok{+}
\StringTok{  }\CommentTok{# kustomisasi judul dan keterangan axis}
\StringTok{  }\KeywordTok{labs}\NormalTok{(}\DataTypeTok{title=}\StringTok{"GDP Per Capita Tiap Benua"}\NormalTok{,}
       \DataTypeTok{x=}\StringTok{"Benua"}\NormalTok{, }\DataTypeTok{y=}\StringTok{"GDP Per Kapita"}\NormalTok{,}
       \CommentTok{# mengubah keterangan legend}
       \DataTypeTok{fill=}\StringTok{"Benua (fill)"}\NormalTok{,}
       \DataTypeTok{color=}\StringTok{"Benua (outline)"}\NormalTok{)}\OperatorTok{+}
\StringTok{  }\KeywordTok{theme}\NormalTok{(}\DataTypeTok{legend.position=}\StringTok{"bottom"}\NormalTok{,}
        \CommentTok{# mengubah tampilan legend box }
        \DataTypeTok{legend.background =} \KeywordTok{element_rect}\NormalTok{(}\DataTypeTok{fill=}\StringTok{"lightblue"}\NormalTok{,}
                                  \DataTypeTok{size=}\FloatTok{0.5}\NormalTok{, }\DataTypeTok{linetype=}\StringTok{"solid"}\NormalTok{, }
                                  \DataTypeTok{colour =}\StringTok{"darkblue"}\NormalTok{))}\OperatorTok{+}
\StringTok{  }\CommentTok{# Menghilangkan legend Benua (outline)}
\StringTok{  }\KeywordTok{guides}\NormalTok{(}\DataTypeTok{color=}\OtherTok{FALSE}\NormalTok{)}
\end{Highlighting}
\end{Shaded}

\begin{figure}

{\centering \includegraphics[width=0.7\linewidth]{EnvStat_files/figure-latex/gglegend5-1} 

}

\caption{Menghilangkan sebagian legend legend}\label{fig:gglegend5}
\end{figure}

\subsection{Merubah Warana Pada Grafik Secara Otomatis dan
Manual}\label{merubah-warana-pada-grafik-secara-otomatis-dan-manual}

Kita dapat merubah warna grafik baik secara otomatis dan manual. Secara
otomatis warna dapat diubah dengan memasukkan nama variabel kedalam
argumen \texttt{fill} dan \texttt{color}. Namun, jika kita inginkan kita
dapat memasukkan kode warna untuk memperoleh warna yang seragam pada
seluruh kelompok data.

Pada contoh sintaks berikut diberikan contoh bagaimana merubah warna
pada seluruh grup data dengan satu warna yang seragam. Output yang
dihasilkan disajikan pada Gambar \ref{fig:ggcolor}:

\begin{Shaded}
\begin{Highlighting}[]
\KeywordTok{ggplot}\NormalTok{(gapminder, }\KeywordTok{aes}\NormalTok{(continent, lifeExp))}\OperatorTok{+}
\StringTok{  }\CommentTok{# spesifikasi warna tunggal}
\StringTok{  }\KeywordTok{geom_boxplot}\NormalTok{(}\DataTypeTok{color=}\StringTok{"darkred"}\NormalTok{,}\DataTypeTok{fill=}\StringTok{"#A4A4A4"}\NormalTok{)}
\end{Highlighting}
\end{Shaded}

\begin{figure}

{\centering \includegraphics[width=0.7\linewidth]{EnvStat_files/figure-latex/ggcolor-1} 

}

\caption{Merubah warna grup berdasarkan satu warna}\label{fig:ggcolor}
\end{figure}

Selain itu, kita dapat mengubah warna berdasarkan grup baik secara
otomatis maupun manual. Berikut adalah contoh sintaks warna berdasarkan
grup secara otomatis. Output yang dihasilkan disajikan pada Gambar
\ref{fig:ggcolor2}.

\begin{Shaded}
\begin{Highlighting}[]
\KeywordTok{ggplot}\NormalTok{(gapminder, }\KeywordTok{aes}\NormalTok{(continent, gdpPercap, }
                      \CommentTok{# warna berdasarkan grup}
                      \DataTypeTok{fill=}\NormalTok{continent))}\OperatorTok{+}
\StringTok{  }\KeywordTok{geom_boxplot}\NormalTok{()}
\end{Highlighting}
\end{Shaded}

\begin{figure}

{\centering \includegraphics[width=0.7\linewidth]{EnvStat_files/figure-latex/ggcolor2-1} 

}

\caption{Merubah warna grup secara otomatis}\label{fig:ggcolor2}
\end{figure}

Kita dapat mengatur pecahayaan (l) dan intensitas warna (c) dari warna
yang kita tampilkan menggunakan fungsi \texttt{scale\_fill\_hue()}.
Berikut adalah sintaks yang digunakan beserta output yang dihasilkan
pada Gambar \ref{fig:ggcolor3}.

\begin{Shaded}
\begin{Highlighting}[]
\KeywordTok{ggplot}\NormalTok{(gapminder, }\KeywordTok{aes}\NormalTok{(continent, gdpPercap, }
                      \CommentTok{# warna berdasarkan grup}
                      \DataTypeTok{fill=}\NormalTok{continent))}\OperatorTok{+}
\StringTok{  }\KeywordTok{geom_boxplot}\NormalTok{()}\OperatorTok{+}
\StringTok{  }\CommentTok{# merubah l dan c}
\StringTok{  }\KeywordTok{scale_color_hue}\NormalTok{(}\DataTypeTok{l=}\DecValTok{40}\NormalTok{, }\DataTypeTok{c=}\DecValTok{35}\NormalTok{)}
\end{Highlighting}
\end{Shaded}

\begin{figure}

{\centering \includegraphics[width=0.7\linewidth]{EnvStat_files/figure-latex/ggcolor3-1} 

}

\caption{Merubah pencahayaan dan intensitas warna}\label{fig:ggcolor3}
\end{figure}

Jika kita tidak menginginkan warna yang secara otomatis ditampilkan oleh
\texttt{ggplot2}, kita dapat mengubahnya secara manual menggunakan
fungsi \texttt{scale\_fill\_manual()} (untuk box plot, bar plot, dll)
dan \texttt{scale\_color\_manual()} (untuk line plot, dot plot dan
scatterplot). Berikut adalah sintaks yang digunakan beserta output yang
dihasilkan pada Gambar \ref{fig:ggcolor4}.

\begin{Shaded}
\begin{Highlighting}[]
\KeywordTok{ggplot}\NormalTok{(gapminder, }\KeywordTok{aes}\NormalTok{(continent, gdpPercap, }
                      \CommentTok{# warna berdasarkan grup}
                      \DataTypeTok{fill=}\NormalTok{continent))}\OperatorTok{+}
\StringTok{  }\KeywordTok{geom_boxplot}\NormalTok{()}\OperatorTok{+}
\StringTok{  }\CommentTok{# merubah warna secara manual}
\StringTok{  }\KeywordTok{scale_fill_manual}\NormalTok{(}\DataTypeTok{values=}\KeywordTok{c}\NormalTok{(}\StringTok{"#999999"}\NormalTok{, }\StringTok{"#E69F00"}\NormalTok{, }\StringTok{"#56B4E9"}\NormalTok{,}
                             \StringTok{"#B47846"}\NormalTok{,}\StringTok{"#B4464B"}\NormalTok{))}
\end{Highlighting}
\end{Shaded}

\begin{figure}

{\centering \includegraphics[width=0.7\linewidth]{EnvStat_files/figure-latex/ggcolor4-1} 

}

\caption{Merubah warna secara manual}\label{fig:ggcolor4}
\end{figure}

JIka kita tidak hafal dengan kode hexadesimal warna tersebut kita dapat
juga menggunakan palet warna. Contoh palet warna yang akan digunakan
adalah dari library \texttt{RColorBrewer}. Berikut adalah contoh sintaks
untuk menginstal dan memuat paket tersebut:

\begin{Shaded}
\begin{Highlighting}[]
\CommentTok{# memasang paket}
\CommentTok{# install.packages("RColorBrewer")}

\CommentTok{# memuat paket}
\KeywordTok{library}\NormalTok{(RColorBrewer)}
\end{Highlighting}
\end{Shaded}

Pada sintak berikut penulis akan menampilkan seluruh palet warna pada
pekt tersebut. Output yang dihasilkan disajikan pada Gambar
\ref{fig:ggcolor5}.

\begin{Shaded}
\begin{Highlighting}[]
\KeywordTok{display.brewer.all}\NormalTok{()}
\end{Highlighting}
\end{Shaded}

\begin{figure}

{\centering \includegraphics{EnvStat_files/figure-latex/ggcolor5-1} 

}

\caption{Palet warna RColorBrewer}\label{fig:ggcolor5}
\end{figure}

Pada Gambar \ref{fig:ggcolor5} terdapat 3 jenis warna antara lain:

\begin{enumerate}
\def\labelenumi{\arabic{enumi}.}
\tightlist
\item
  \textbf{Sequential palettes}, digunakan untuk menunjukkan urutan dari
  rendah ke tinggi atau gradien. Nama palet yang ada antara lain: Blues,
  BuGn, BuPu, GnBu, Greens, Greys, Oranges, OrRd, PuBu, PuBuGn, PuRd,
  Purples, RdPu, Reds, YlGn, YlGnBu YlOrBr,dan YlOrRd.
\item
  \textbf{Diverging palettes}, digunakan untuk menunjukkan perubahan
  pada data yang memiliki nilai positif dan negatif. Palet yang tersedia
  antara lain: BrBG, PiYG, PRGn, PuOr, RdBu, RdGy, RdYlBu, RdYlGn, dan
  Spectral.
\item
  \textbf{Qualitative palettes}, digunakan untuk merepresentasikan
  variabel nominal atau kategori karena tidak menunjukkan besaran atau
  perbedaan nilai antar grup. Palete yang tersedia antara lain: Accent,
  Dark2, Paired, Pastel1, Pastel2, Set1, Set2, dan Set3.
\end{enumerate}

Pada contoh sintaks berikut disajikan contoh penerapan dan output yang
dihasilkan pada Gambar \ref{fig:ggcolor6}.

\begin{Shaded}
\begin{Highlighting}[]
\KeywordTok{ggplot}\NormalTok{(gapminder, }\KeywordTok{aes}\NormalTok{(continent, gdpPercap, }
                      \CommentTok{# warna berdasarkan grup}
                      \DataTypeTok{fill=}\NormalTok{continent))}\OperatorTok{+}
\StringTok{  }\KeywordTok{geom_boxplot}\NormalTok{()}\OperatorTok{+}
\StringTok{  }\CommentTok{# merubah warna menggunakan palet}
\StringTok{  }\KeywordTok{scale_color_brewer}\NormalTok{(}\DataTypeTok{palette=}\StringTok{"Dark2"}\NormalTok{)}
\end{Highlighting}
\end{Shaded}

\begin{figure}

{\centering \includegraphics[width=0.7\linewidth]{EnvStat_files/figure-latex/ggcolor6-1} 

}

\caption{Merubah warna menggunakan palet}\label{fig:ggcolor6}
\end{figure}

Jika kita tidak menginginkan warna-warna terang, kita dapat menggunakan
fungsi \texttt{scale\_color\_grey()} (untuk line plot, dot plot, dan
scatterplot) dan \texttt{scale\_fill\_grey()} (untuk bar plot,
histogram, box plot, dll). Funsi tersebut akan memberikan warna palet
gray pada plot. Berikut adalah sintaks yang digunakan beserta output
yang dihasilkan pada Gambar \ref{fig:ggcolor7}.

\begin{Shaded}
\begin{Highlighting}[]
\KeywordTok{ggplot}\NormalTok{(gapminder, }\KeywordTok{aes}\NormalTok{(continent, gdpPercap, }
                      \CommentTok{# warna berdasarkan grup}
                      \DataTypeTok{fill=}\NormalTok{continent))}\OperatorTok{+}
\StringTok{  }\KeywordTok{geom_boxplot}\NormalTok{()}\OperatorTok{+}
\StringTok{  }\CommentTok{# merubah warna menggunakan palet}
\StringTok{  }\KeywordTok{scale_fill_grey}\NormalTok{()}
\end{Highlighting}
\end{Shaded}

\begin{figure}

{\centering \includegraphics[width=0.7\linewidth]{EnvStat_files/figure-latex/ggcolor7-1} 

}

\caption{Merubah warna menggunakan palet gray}\label{fig:ggcolor7}
\end{figure}

\subsection{Kustomisasi Titik}\label{kustomisasi-titik}

Untuk mengubah jenis titik pada scatterplot, outlier pada box plot, dan
dot plot, kita dapat menambahkan argumen \texttt{shape} pada fungsi
geometrinya. Nilai yang mungkin dimasukkan berupa nilai diskrit yang
berkisar antara 0 sampai 25. Selain itu, ukuran dari titik dapat diinput
dengan menambahkan argumen \texttt{size}. Berikut adalah sintaks yang
digunakan beserta output yang dihasilkan pada Gambar \ref{fig:ggpoint}.

\begin{Shaded}
\begin{Highlighting}[]
\KeywordTok{ggplot}\NormalTok{(gapminder, }\KeywordTok{aes}\NormalTok{(gdpPercap, lifeExp))}\OperatorTok{+}
\StringTok{  }\CommentTok{# spesifikasi jenis, ukuran dan warna titik}
\StringTok{  }\KeywordTok{geom_point}\NormalTok{(}\DataTypeTok{shape=}\DecValTok{4}\NormalTok{, }\DataTypeTok{size=}\DecValTok{2}\NormalTok{, }\DataTypeTok{color=}\StringTok{"blue"}\NormalTok{)}
\end{Highlighting}
\end{Shaded}

\begin{figure}

{\centering \includegraphics[width=0.7\linewidth]{EnvStat_files/figure-latex/ggpoint-1} 

}

\caption{Kustomisasi jenis, ukuran dan warna titik}\label{fig:ggpoint}
\end{figure}

Untuk data dengan multiple group, kita dapat mengubah jenis, ukuran dan
warna secara otomatis dengan memasukkan nama variabel kedalam argumen
\texttt{shape}, \texttt{size} dan \texttt{color}. Sedangkan secara
manual kita dapat menambahkan fungsi \texttt{scale\_shape\_manual()}
(jenis titik), \texttt{scale\_color\_manual()} (warna titik), dan
\texttt{scale\_size\_manual()} (ukuran titik). Berikut adalah sintaks
yang digunakan beserta output yang dihasilkan pada Gambar
\ref{fig:ggpoint2} dan Gambar \ref{fig:ggpoint3}.

\begin{Shaded}
\begin{Highlighting}[]
\CommentTok{# cara otomatis}
\KeywordTok{ggplot}\NormalTok{(gapminder, }\KeywordTok{aes}\NormalTok{(gdpPercap, lifeExp,}
                      \CommentTok{# spesifikasi jenis, ukuran dan warna}
                      \DataTypeTok{shape=}\NormalTok{continent, }\DataTypeTok{color=}\NormalTok{continent,}
                      \DataTypeTok{size=}\NormalTok{pop))}\OperatorTok{+}
\StringTok{  }\KeywordTok{geom_point}\NormalTok{()}
\end{Highlighting}
\end{Shaded}

\begin{figure}

{\centering \includegraphics[width=0.7\linewidth]{EnvStat_files/figure-latex/ggpoint2-1} 

}

\caption{Kustomisasi jenis, ukuran dan warna titik untuk multiple group secara otomatis}\label{fig:ggpoint2}
\end{figure}

\begin{Shaded}
\begin{Highlighting}[]
\CommentTok{# cara manual}
\KeywordTok{ggplot}\NormalTok{(gapminder, }\KeywordTok{aes}\NormalTok{(gdpPercap, lifeExp,}
                      \CommentTok{# spesifikasi jenis, ukuran dan warna}
                      \DataTypeTok{shape=}\NormalTok{continent, }\DataTypeTok{color=}\NormalTok{continent,}
                      \DataTypeTok{size=}\NormalTok{pop))}\OperatorTok{+}
\StringTok{  }\KeywordTok{geom_point}\NormalTok{()}\OperatorTok{+}
\StringTok{  }\KeywordTok{scale_shape_manual}\NormalTok{(}\DataTypeTok{values=}\KeywordTok{c}\NormalTok{(}\DecValTok{1}\OperatorTok{:}\DecValTok{5}\NormalTok{))}\OperatorTok{+}
\StringTok{  }\KeywordTok{scale_color_manual}\NormalTok{(}\DataTypeTok{values=}\KeywordTok{c}\NormalTok{(}\StringTok{"#999999"}\NormalTok{, }\StringTok{"#E69F00"}\NormalTok{, }\StringTok{"#56B4E9"}\NormalTok{,}
                             \StringTok{"#B47846"}\NormalTok{,}\StringTok{"#B4464B"}\NormalTok{))}
\end{Highlighting}
\end{Shaded}

\begin{figure}

{\centering \includegraphics[width=0.7\linewidth]{EnvStat_files/figure-latex/ggpoint3-1} 

}

\caption{Kustomisasi jenis, ukuran dan warna titik untuk multiple group secara manual}\label{fig:ggpoint3}
\end{figure}

\subsection{Kustomisasi Jenis Garis}\label{kustomisasi-jenis-garis}

Jenis, warna dan ukuran garis dapat diatur dengan menambahkan argumen
\texttt{linetype}, \texttt{size} dan \texttt{color}. Berikut adalah
sintaks yang digunakan beserta output yang dihasilkan pada Gambar
\ref{fig:gglty}.

\begin{Shaded}
\begin{Highlighting}[]
\NormalTok{gapminder}\OperatorTok
\StringTok{  }\KeywordTok{filter}\NormalTok{(continent}\OperatorTok{==}\StringTok{"Asia"}\NormalTok{)}\OperatorTok
\StringTok{  }\KeywordTok{group_by}\NormalTok{(year)}\OperatorTok
\StringTok{  }\KeywordTok{summarize}\NormalTok{(}\DataTypeTok{mean_pop=}\KeywordTok{mean}\NormalTok{(pop))}\OperatorTok
\StringTok{  }\CommentTok{# plot}
\StringTok{  }\KeywordTok{ggplot}\NormalTok{(}\KeywordTok{aes}\NormalTok{(year, mean_pop))}\OperatorTok{+}
\StringTok{    }\KeywordTok{geom_line}\NormalTok{(}\DataTypeTok{linetype=}\StringTok{"dashed"}\NormalTok{, }\DataTypeTok{color=}\StringTok{"blue"}\NormalTok{,}
              \DataTypeTok{size=}\DecValTok{1}\NormalTok{)}\OperatorTok{+}
\StringTok{    }\KeywordTok{geom_point}\NormalTok{(}\DataTypeTok{shape=}\DecValTok{1}\NormalTok{, }\DataTypeTok{color=}\StringTok{"red"}\NormalTok{)}
\end{Highlighting}
\end{Shaded}

\begin{figure}

{\centering \includegraphics[width=0.7\linewidth]{EnvStat_files/figure-latex/gglty-1} 

}

\caption{Kustomisasi jenis, ukuran dan warna garis}\label{fig:gglty}
\end{figure}

Untuk data dengan multiple group, kita dapat mengubah jenis garis, warna
dan ukuran secara manual maupun secara otomatis. Secara otomatis kita
dapat menginputkan nama variabel kedalam argumen \texttt{linetype},
\texttt{size} dan \texttt{color}. Secara manual, kita dapat mengubah
jenis, warna dan ukuran menggunakan fungsi
\texttt{scale\_linetype\_manual()} (jenis garis),
\texttt{scale\_color\_manual()} (warna garis), dan
\texttt{scale\_size\_manual()} (ukuran garis). Berikut adalah sintaks
yang digunakan beserta output yang dihasilkan pada Gambar
\ref{fig:gglty2} dan Gambar \ref{fig:gglty3}.

\begin{Shaded}
\begin{Highlighting}[]
\CommentTok{# cara otomatis}
\NormalTok{gapminder}\OperatorTok
\StringTok{  }\KeywordTok{filter}\NormalTok{(continent }\OperatorTok\StringTok{ }\KeywordTok{c}\NormalTok{(}\StringTok{"Asia"}\NormalTok{,}\StringTok{"Africa"}\NormalTok{))}\OperatorTok
\StringTok{  }\KeywordTok{group_by}\NormalTok{(year, continent)}\OperatorTok
\StringTok{  }\KeywordTok{summarize}\NormalTok{(}\DataTypeTok{mean_pop=}\KeywordTok{mean}\NormalTok{(pop))}\OperatorTok
\StringTok{  }\CommentTok{# plot}
\StringTok{  }\KeywordTok{ggplot}\NormalTok{(}\KeywordTok{aes}\NormalTok{(year, mean_pop,}
             \DataTypeTok{linetype=}\NormalTok{continent,}
             \DataTypeTok{color=}\NormalTok{continent))}\OperatorTok{+}
\StringTok{    }\KeywordTok{geom_line}\NormalTok{()}\OperatorTok{+}
\StringTok{    }\KeywordTok{geom_point}\NormalTok{(}\DataTypeTok{shape=}\DecValTok{1}\NormalTok{, }\DataTypeTok{color=}\StringTok{"red"}\NormalTok{)}
\end{Highlighting}
\end{Shaded}

\begin{figure}

{\centering \includegraphics[width=0.7\linewidth]{EnvStat_files/figure-latex/gglty2-1} 

}

\caption{Kustomisasi jenis, ukuran dan warna garis untuk multiple group secara otomatis}\label{fig:gglty2}
\end{figure}

\begin{Shaded}
\begin{Highlighting}[]
\CommentTok{# cara manual}
\NormalTok{gapminder}\OperatorTok
\StringTok{  }\KeywordTok{filter}\NormalTok{(continent }\OperatorTok\StringTok{ }\KeywordTok{c}\NormalTok{(}\StringTok{"Asia"}\NormalTok{,}\StringTok{"Africa"}\NormalTok{))}\OperatorTok
\StringTok{  }\KeywordTok{group_by}\NormalTok{(year, continent)}\OperatorTok
\StringTok{  }\KeywordTok{summarize}\NormalTok{(}\DataTypeTok{mean_pop=}\KeywordTok{mean}\NormalTok{(pop))}\OperatorTok
\StringTok{  }\CommentTok{# plot}
\StringTok{  }\KeywordTok{ggplot}\NormalTok{(}\KeywordTok{aes}\NormalTok{(year, mean_pop,}
             \DataTypeTok{linetype=}\NormalTok{continent,}
             \DataTypeTok{color=}\NormalTok{continent))}\OperatorTok{+}
\StringTok{    }\KeywordTok{geom_line}\NormalTok{()}\OperatorTok{+}
\StringTok{    }\KeywordTok{geom_point}\NormalTok{(}\DataTypeTok{shape=}\DecValTok{1}\NormalTok{, }\DataTypeTok{color=}\StringTok{"red"}\NormalTok{)}\OperatorTok{+}
\StringTok{    }\KeywordTok{scale_linetype_manual}\NormalTok{(}\DataTypeTok{values=}\KeywordTok{c}\NormalTok{(}\StringTok{"dotted"}\NormalTok{, }\StringTok{"twodash"}\NormalTok{))}\OperatorTok{+}
\StringTok{    }\KeywordTok{scale_color_manual}\NormalTok{(}\DataTypeTok{values=}\KeywordTok{c}\NormalTok{(}\StringTok{"red"}\NormalTok{,}\StringTok{"blue"}\NormalTok{))}
\end{Highlighting}
\end{Shaded}

\begin{figure}

{\centering \includegraphics[width=0.7\linewidth]{EnvStat_files/figure-latex/gglty3-1} 

}

\caption{Kustomisasi jenis, ukuran dan warna garis untuk multiple group secara manual}\label{fig:gglty3}
\end{figure}

\subsection{Menambahkan Label Pada Titik Observasi dan Bidang
Plot}\label{menambahkan-label-pada-titik-observasi-dan-bidang-plot}

Pada artikel ini penulis akan menjelaskan bagaimana kita dapat
menambahkan teks pada plot. Fungsi-fungsi yang dapat digunakan antara
lain:

\begin{itemize}
\tightlist
\item
  \texttt{geom\_text()}: menambahkan teks secara langsung pada plot.
\item
  \texttt{geom\_label()}: menambahkan teks dengan kotak disekelilingnya.
\item
  \texttt{annotate()}: menambahkan teks tertentu pada bagian tertentu
  bidang plot.
\item
  \texttt{annotation\_custom()}: menambahkan anotasi statik yang sama
  pada setiap panel.
\end{itemize}

Misal kita akan membuat plot antara variabel \texttt{pop} vs
\texttt{gdpPercap} seperti yang ditunjukkan pada Gambar
\ref{fig:gglabel} berikut:

\begin{Shaded}
\begin{Highlighting}[]
\KeywordTok{ggplot}\NormalTok{(gapminder, }\KeywordTok{aes}\NormalTok{(gdpPercap, pop))}\OperatorTok{+}
\StringTok{  }\KeywordTok{geom_point}\NormalTok{()}
\end{Highlighting}
\end{Shaded}

\begin{figure}

{\centering \includegraphics[width=0.7\linewidth]{EnvStat_files/figure-latex/gglabel-1} 

}

\caption{Scatterplot variabel pop vs gdpPercap}\label{fig:gglabel}
\end{figure}

Misalkan kita ingin menandai negara yang memiliki gdpPercap
\textgreater{} 50000. Berikut adalah sintaks yang digunakan beserta
output yang dihasilkan pada Gambar \ref{fig:gglabel2}.

\begin{Shaded}
\begin{Highlighting}[]
\KeywordTok{ggplot}\NormalTok{(gapminder, }\KeywordTok{aes}\NormalTok{(gdpPercap, pop))}\OperatorTok{+}
\StringTok{  }\KeywordTok{geom_point}\NormalTok{(}\DataTypeTok{shape=}\DecValTok{1}\NormalTok{)}\OperatorTok{+}
\StringTok{  }\KeywordTok{geom_label}\NormalTok{(}
    \CommentTok{# subset data sesua kriteria}
    \DataTypeTok{data=}\KeywordTok{subset}\NormalTok{(gapminder,gdpPercap}\OperatorTok{>}\DecValTok{50000}\NormalTok{),}
    \CommentTok{# label berdasarkan kriteria}
    \KeywordTok{aes}\NormalTok{(}\DataTypeTok{label=}\NormalTok{country),}
    \CommentTok{# ukuran teks}
    \DataTypeTok{size =} \DecValTok{3}\NormalTok{)}
\end{Highlighting}
\end{Shaded}

\begin{figure}

{\centering \includegraphics[width=0.7\linewidth]{EnvStat_files/figure-latex/gglabel2-1} 

}

\caption{Scatterplot variabel pop vs gdpPercap dengan label}\label{fig:gglabel2}
\end{figure}

Selain teks yang menunjukkan observasi, kita dapat menambahkan anotasi
pada grafik. Berikut adalah sintaks yang digunakan beserta output yang
dihasilkan pada Gambar \ref{fig:gglabel3}.

\begin{Shaded}
\begin{Highlighting}[]
\KeywordTok{ggplot}\NormalTok{(gapminder, }\KeywordTok{aes}\NormalTok{(gdpPercap, pop))}\OperatorTok{+}
\StringTok{  }\KeywordTok{geom_point}\NormalTok{(}\DataTypeTok{shape=}\DecValTok{1}\NormalTok{)}\OperatorTok{+}
\StringTok{  }\CommentTok{# menambahkan label sesuai kriteria data}
\StringTok{  }\KeywordTok{geom_label}\NormalTok{(}
    \CommentTok{# subset data sesua kriteria}
    \DataTypeTok{data=}\KeywordTok{subset}\NormalTok{(gapminder,gdpPercap}\OperatorTok{>}\DecValTok{50000}\NormalTok{),}
    \CommentTok{# label berdasarkan kriteria}
    \KeywordTok{aes}\NormalTok{(}\DataTypeTok{label=}\NormalTok{country),}
    \CommentTok{# ukuran teks}
    \DataTypeTok{size =} \DecValTok{3}\NormalTok{)}\OperatorTok{+}
\StringTok{  }\KeywordTok{annotate}\NormalTok{(}\DataTypeTok{geom=}\StringTok{"text"}\NormalTok{, }\DataTypeTok{x=}\DecValTok{90000}\NormalTok{,}
          \DataTypeTok{y=}\FloatTok{2e+08}\NormalTok{, }\DataTypeTok{label=}\StringTok{"outlier"}\NormalTok{,}
          \DataTypeTok{color=}\StringTok{"red"}\NormalTok{)}
\end{Highlighting}
\end{Shaded}

\begin{figure}

{\centering \includegraphics[width=0.7\linewidth]{EnvStat_files/figure-latex/gglabel3-1} 

}

\caption{Scatterplot variabel pop vs gdpPercap dengan label dan notasi}\label{fig:gglabel3}
\end{figure}

Kita dapat pula menambahkan teks statik yang sama pada setiap panel.
Berikut adalah sintaks yang digunakan beserta output yang dihasilkan
pada Gambar \ref{fig:gglabel4}.

\begin{Shaded}
\begin{Highlighting}[]
\KeywordTok{library}\NormalTok{(grid)}

\CommentTok{# membuat teks}
\NormalTok{d <-}\StringTok{ }\NormalTok{grob <-}\StringTok{ }\KeywordTok{grobTree}\NormalTok{(}\KeywordTok{textGrob}\NormalTok{(}\StringTok{"Scatter plot"}\NormalTok{, }\DataTypeTok{x=}\FloatTok{0.1}\NormalTok{,  }\DataTypeTok{y=}\FloatTok{0.95}\NormalTok{, }\DataTypeTok{hjust=}\DecValTok{0}\NormalTok{,}
  \DataTypeTok{gp=}\KeywordTok{gpar}\NormalTok{(}\DataTypeTok{col=}\StringTok{"red"}\NormalTok{, }\DataTypeTok{fontsize=}\DecValTok{13}\NormalTok{, }\DataTypeTok{fontface=}\StringTok{"italic"}\NormalTok{)))}

\CommentTok{# plot}
\KeywordTok{ggplot}\NormalTok{(gapminder, }\KeywordTok{aes}\NormalTok{(gdpPercap, pop))}\OperatorTok{+}
\StringTok{  }\KeywordTok{geom_point}\NormalTok{(}\DataTypeTok{shape=}\DecValTok{1}\NormalTok{)}\OperatorTok{+}
\StringTok{  }\CommentTok{# menambahkan anotasi}
\StringTok{  }\KeywordTok{annotation_custom}\NormalTok{(d)}\OperatorTok{+}
\StringTok{  }\CommentTok{# membagi plot menjadi beberapa panel}
\StringTok{  }\KeywordTok{facet_wrap}\NormalTok{(}\OperatorTok{~}\NormalTok{continent, }\DataTypeTok{scales=}\StringTok{"free"}\NormalTok{)}
\end{Highlighting}
\end{Shaded}

\begin{figure}

{\centering \includegraphics[width=0.7\linewidth]{EnvStat_files/figure-latex/gglabel4-1} 

}

\caption{Scatterplot variabel pop vs gdpPercap dengan label dan notasi pada tiap panel}\label{fig:gglabel4}
\end{figure}

\subsection{Kustomisasi Tema Pada
Plot}\label{kustomisasi-tema-pada-plot}

Kita dapat melakukan kustomisasi tema plot untuk membuat tampilan plot
kita lebih menarik. Pada bagian ini penulis akan membahas tema yang
dapat digunakan serta cara untuk melakukan edit terhadap tema yang telah
ada sebelumnya.

Tema-tema yang telah terpasang secara defautl pada paket
\texttt{ggplot2} antara lain:

\begin{itemize}
\tightlist
\item
  \textbf{theme\_gray}: backround dengan warna abu-abu dengan garis grid
  putih.
\item
  \textbf{theme\_bw}: background putih dan garis grid berwarna abu-abu.
\item
  \textbf{theme\_linedraw}: garis hitam di sekeliling bidang plot.
\item
  \textbf{theme\_light}: garis grid dan axis berwarna abu-abu terang.
\item
  \textbf{theme\_minimal}: tidak memiliki frame disekeliling bidang
  plot.
\item
  \textbf{theme\_classic}: tidak ada garis grid dan axis.
\item
  \textbf{theme\_void}: tema kosong
\item
  \textbf{theme\_dark}: background gelap.
\end{itemize}

Pada contoh berikut disajikan sebagian contoh penerapan tema pada plot.
Output yang dihasilkan pada Gambar \ref{fig:ggtema}.

\begin{Shaded}
\begin{Highlighting}[]
\KeywordTok{ggplot}\NormalTok{(gapminder, }\KeywordTok{aes}\NormalTok{(gdpPercap, lifeExp))}\OperatorTok{+}
\StringTok{  }\KeywordTok{geom_point}\NormalTok{()}\OperatorTok{+}
\StringTok{  }\KeywordTok{theme_bw}\NormalTok{()}
\end{Highlighting}
\end{Shaded}

\begin{figure}

{\centering \includegraphics[width=0.7\linewidth]{EnvStat_files/figure-latex/ggtema-1} 

}

\caption{Scatterplot dengan tema black and white}\label{fig:ggtema}
\end{figure}

Kita juga dapat menggunakan tema kustom yang terdapat pada library
\texttt{ggthemes}. Berikut adalah sintaks yang digunakan untuk
menginstall dan memuat paket tersebut:

\begin{Shaded}
\begin{Highlighting}[]
\CommentTok{# Memasang paket}
\KeywordTok{install.packages}\NormalTok{(}\StringTok{"ggthemes"}\NormalTok{)}
\end{Highlighting}
\end{Shaded}

\begin{Shaded}
\begin{Highlighting}[]
\CommentTok{# memuat paket}
\KeywordTok{library}\NormalTok{(ggthemes)}
\end{Highlighting}
\end{Shaded}

tema-tema yang tersedia pada paket tersebut antara lain:

\begin{itemize}
\tightlist
\item
  \textbf{theme\_tufte}: tema minimalis.
\item
  \textbf{theme\_economist}: tema yang digunakan pada majalah Economist.
\item
  \textbf{theme\_stata}: tema yang digunakan pada visualisasi progra
  stata.
\item
  \textbf{theme\_wsj}: tema yang digunakan pada Wall Street Journal.
\item
  \textbf{theme\_cal}: tema yang digunakan pada LibreOffice Calc dan
  Google Docs.
\item
  \textbf{theme\_hc}: tema yang didasarkan pada Highcharts JS.
\end{itemize}

Pada contoh berikut disajikan sebagian contoh penerapan tema pada plot.
Output yang dihasilkan pada Gambar \ref{fig:ggtema2}.

\begin{Shaded}
\begin{Highlighting}[]
\KeywordTok{ggplot}\NormalTok{(gapminder, }\KeywordTok{aes}\NormalTok{(gdpPercap, lifeExp, }
                      \DataTypeTok{color=}\NormalTok{continent))}\OperatorTok{+}
\StringTok{  }\KeywordTok{geom_point}\NormalTok{()}\OperatorTok{+}
\StringTok{  }\KeywordTok{theme_wsj}\NormalTok{()}
\end{Highlighting}
\end{Shaded}

\begin{figure}

{\centering \includegraphics[width=0.7\linewidth]{EnvStat_files/figure-latex/ggtema2-1} 

}

\caption{Scatterplot dengan tema Wall Street Journal}\label{fig:ggtema2}
\end{figure}

Kita dapat juga membuat tema kustom berdasarkan tema yang telah ada.
Untuk melakukannya kita hanya perlu merubah sejumlah argument default
yang ada pada fungsi tema dan menamai tema sesuai dengan yang kita
inginkan menggunakan \emph{user define function}. Berikut adalah contoh
argumen yang dapat diubah pada \texttt{theme\_wsj}.

\begin{Shaded}
\begin{Highlighting}[]
\NormalTok{theme_wsj}
\end{Highlighting}
\end{Shaded}

\begin{verbatim}
## function (base_size = 12, color = "brown", base_family = "sans", 
##     title_family = "mono") 
## {
##     colorhex <- ggthemes::ggthemes_data$wsj$bg[color]
##     theme_foundation(base_size = base_size, base_family = base_family) + 
##         theme(line = element_line(linetype = 1, colour = "black"), 
##             rect = element_rect(fill = colorhex, linetype = 0, 
##                 colour = NA), text = element_text(colour = "black"), 
##             title = element_text(family = title_family, size = rel(2)), 
##             axis.title = element_blank(), axis.text = element_text(face = "bold", 
##                 size = rel(1)), axis.text.x = element_text(colour = NULL), 
##             axis.text.y = element_text(colour = NULL), axis.ticks = element_line(colour = NULL), 
##             axis.ticks.y = element_blank(), axis.ticks.x = element_line(colour = NULL), 
##             axis.line = element_line(), axis.line.y = element_blank(), 
##             legend.background = element_rect(), legend.position = "top", 
##             legend.direction = "horizontal", legend.box = "vertical", 
##             panel.grid = element_line(colour = NULL, linetype = 3), 
##             panel.grid.major = element_line(colour = "black"), 
##             panel.grid.major.x = element_blank(), panel.grid.minor = element_blank(), 
##             plot.title = element_text(hjust = 0, face = "bold"), 
##             plot.margin = unit(c(1, 1, 1, 1), "lines"), strip.background = element_rect())
## }
## <bytecode: 0x000000001c0337e8>
## <environment: namespace:ggthemes>
\end{verbatim}

Berdasarkan output yang disajikan kita dapat merubah sejumlah argumen
seperti base size, color, base\_family, dll.

\subsection{Penskalaan dan Transformasi
Axis}\label{penskalaan-dan-transformasi-axis}

Pada bagian ini penulis akan menjelaskan bagaimana cara melakukan
modifikasi terhadap sumbu x dan y seperti menetapkan limit nilai
maksimum dan minimum axis serta melakukan transformasi pada tiap axis.

Untuk mengatur rentang nilai axis, kita dapat melakukannya dengan fungsi
sebagai berikut:

\begin{itemize}
\tightlist
\item
  \textbf{xlim()} dan \textbf{ylim()}: mengatur limit aksis sumbu x dan
  y.
\item
  \textbf{expand\_limits()}: mengatur limit sumbu x dan y sekaligus
  dapat mengatur intercept kedua sumbu tersebut.
\item
  \textbf{scale\_x\_continous()} dan \textbf{scale\_y\_continous()}:
  megatur limit axis termasuk axis tick dan label.
\end{itemize}

Pada contoh berikut akan disajikan cara mengatur limit axis dengan
menggunakan \texttt{xlim()} dan \texttt{ylim()} serta menggunakan
\texttt{expand\_limits()}. Output yang dihasilkan disajikan pada Gambar
\ref{fig:gglimits}.

\begin{Shaded}
\begin{Highlighting}[]
\NormalTok{gapminder}\OperatorTok
\StringTok{  }\KeywordTok{filter}\NormalTok{(continent}\OperatorTok{==}\StringTok{"Europe"}\NormalTok{)}\OperatorTok
\StringTok{  }\KeywordTok{ggplot}\NormalTok{(}\KeywordTok{aes}\NormalTok{(gdpPercap, lifeExp))}\OperatorTok{+}
\StringTok{  }\KeywordTok{geom_point}\NormalTok{()}\OperatorTok{+}
\StringTok{  }\KeywordTok{theme_wsj}\NormalTok{(}\DataTypeTok{base_size=}\DecValTok{7}\NormalTok{)}\OperatorTok{+}
\StringTok{  }\KeywordTok{labs}\NormalTok{(}\DataTypeTok{title=}\StringTok{"GDP per Capita vs Life Expectancy"}\NormalTok{,}
       \DataTypeTok{y=}\StringTok{"Life Expectancy"}\NormalTok{,}
       \DataTypeTok{x=}\StringTok{"GDP per Capita (US Dollar)"}\NormalTok{)}\OperatorTok{+}
\StringTok{  }\CommentTok{# mengatur limit axis}
\StringTok{  }\KeywordTok{expand_limits}\NormalTok{(}\DataTypeTok{x=}\KeywordTok{c}\NormalTok{(}\DecValTok{0}\NormalTok{, }\DecValTok{55000}\NormalTok{), }\DataTypeTok{y=}\KeywordTok{c}\NormalTok{(}\DecValTok{0}\NormalTok{, }\DecValTok{90}\NormalTok{))}
\end{Highlighting}
\end{Shaded}

\begin{Shaded}
\begin{Highlighting}[]
\CommentTok{# atau}
\NormalTok{gapminder}\OperatorTok
\StringTok{  }\KeywordTok{filter}\NormalTok{(continent}\OperatorTok{==}\StringTok{"Europe"}\NormalTok{)}\OperatorTok
\StringTok{  }\KeywordTok{ggplot}\NormalTok{(}\KeywordTok{aes}\NormalTok{(gdpPercap, lifeExp))}\OperatorTok{+}
\StringTok{  }\KeywordTok{geom_point}\NormalTok{()}\OperatorTok{+}
\StringTok{  }\KeywordTok{theme_wsj}\NormalTok{(}\DataTypeTok{base_size=}\DecValTok{7}\NormalTok{)}\OperatorTok{+}
\StringTok{  }\KeywordTok{labs}\NormalTok{(}\DataTypeTok{title=}\StringTok{"GDP per Capita vs Life Expectancy"}\NormalTok{,}
       \DataTypeTok{y=}\StringTok{"Life Expectancy"}\NormalTok{,}
       \DataTypeTok{x=}\StringTok{"GDP per Capita (US Dollar)"}\NormalTok{)}\OperatorTok{+}
\StringTok{  }\CommentTok{# mengatur limit axis}
\StringTok{  }\KeywordTok{xlim}\NormalTok{(}\DecValTok{0}\NormalTok{,}\DecValTok{55000}\NormalTok{)}\OperatorTok{+}
\StringTok{  }\KeywordTok{ylim}\NormalTok{(}\DecValTok{0}\NormalTok{,}\DecValTok{90}\NormalTok{)}
\end{Highlighting}
\end{Shaded}

\begin{figure}

{\centering \includegraphics[width=0.7\linewidth]{EnvStat_files/figure-latex/gglimits-1} 

}

\caption{Scatterplot dengan axis limits }\label{fig:gglimits}
\end{figure}

Kita juga dapat menggunakan fungsi \texttt{scale\_x\_continuous()} dan
\texttt{scale\_y\_continuous()} untuk mengatur limit axis ,\emph{axis
tick} dan label. Format yang digunakan adalah sebagai berikut:

\begin{Shaded}
\begin{Highlighting}[]
\KeywordTok{scale_x_continuous}\NormalTok{(name, breaks, labels, limits, trans)}
\KeywordTok{scale_y_continuous}\NormalTok{(name, breaks, labels, limits, trans)}
\end{Highlighting}
\end{Shaded}

\begin{quote}
\textbf{Note: }

\begin{itemize}
\item
  \textbf{name}: label axis sumbu x dan y.
\item
  \textbf{breaks}: untuk mengontrol jeda dalam panduan (\emph{axis
  tick}, garis grid, \ldots{}). Di antara nilai-nilai yang mungkin,
  adalah sebagai berikut:
\item
  NULL: menyembunyikan seluruh breaks.
\item
  \textbf{waiver()}: komputasi break default.
\item
  vektor numerik atau karakter untuk menspesifikasikan break yang akan
  ditampilkan.
\item
  \textbf{labels}: label axis. Nilai yang dapat dimasukkan antara lain;
\item
  NULL: tanpa label.
\item
  \textbf{waiver()}: label default.
\item
  vektor karakter yang digunakan untuk spesifikasi label break.
\item
  \textbf{limits}: vektor numerik untuk spesifikasi limit sumbu x dan y.
\item
  \textbf{trans}: transformasi axis. Nilai yang dapat digunakan adalah
  ``log2'', ``log10'', dll.
\end{itemize}
\end{quote}

Pada contoh berikut disajikan contoh mengatur limit axis dan label axis
menggunakan fungsi \texttt{scale\_x\_continous()} dan
\texttt{scale\_y\_continous()}. Grafik yang dihasilkan akan tampak
seperti Gambar \ref{fig:gglimits2}.

\begin{Shaded}
\begin{Highlighting}[]
\CommentTok{# atau}
\NormalTok{gapminder}\OperatorTok
\StringTok{  }\KeywordTok{filter}\NormalTok{(continent}\OperatorTok{==}\StringTok{"Asia"}\NormalTok{)}\OperatorTok
\StringTok{  }\KeywordTok{ggplot}\NormalTok{(}\KeywordTok{aes}\NormalTok{(gdpPercap, lifeExp))}\OperatorTok{+}
\StringTok{  }\KeywordTok{geom_point}\NormalTok{()}\OperatorTok{+}
\StringTok{  }\KeywordTok{theme_wsj}\NormalTok{(}\DataTypeTok{base_size=}\DecValTok{7}\NormalTok{)}\OperatorTok{+}
\StringTok{  }\KeywordTok{ggtitle}\NormalTok{(}\StringTok{"GDP per Capita vs Life Expectancy"}\NormalTok{)}\OperatorTok{+}
\StringTok{  }\CommentTok{# spesifikasi limit dan label axis}
\StringTok{  }\KeywordTok{scale_x_continuous}\NormalTok{(}\DataTypeTok{name=}\StringTok{"GDP per Capita"}\NormalTok{, }
                     \DataTypeTok{limits=}\KeywordTok{c}\NormalTok{(}\DecValTok{0}\NormalTok{, }\DecValTok{125000}\NormalTok{))}\OperatorTok{+}
\StringTok{  }\KeywordTok{scale_y_continuous}\NormalTok{(}\DataTypeTok{name=}\StringTok{"Life Expectancy"}\NormalTok{,}
                     \DataTypeTok{limits=}\KeywordTok{c}\NormalTok{(}\DecValTok{0}\NormalTok{,}\DecValTok{100}\NormalTok{))}
\end{Highlighting}
\end{Shaded}

\begin{figure}

{\centering \includegraphics[width=0.7\linewidth]{EnvStat_files/figure-latex/gglimits2-1} 

}

\caption{Scatterplot dengan axis limits (2) }\label{fig:gglimits2}
\end{figure}

Tranformasi axis dapat dilakukan dengan fungsi bawaan dari
\texttt{ggplot2}. Fungsi transformasi bawaan berupa transformasi log dan
sqrt. Berikut adalah fungsi bawaan untuk transformasi tersebut:

\begin{itemize}
\tightlist
\item
  \textbf{scale\_x\_log10()} dan \textbf{scale\_y\_log10()}:
  transformasi log basis 10.
\item
  \textbf{scale\_x\_sqrt()} dan \textbf{scale\_y\_sqrt()}: transformasi
  akar kuadrat.
\item
  \textbf{scale\_x\_reverse()} dan \textbf{scale\_x\_reverse()}:
  membalikkan koordinat.
\item
  \textbf{coord\_trans(x=``log10'', y=``log10'')}: memungkinkan
  transformasi untuk kedua axis sesuai fungsi yang diinputkan pada sumbu
  x dan sumbu y seperti ``log2'', ``log10'', ``sqrt'', dll.
\item
  \textbf{scale\_x\_continuous(trans=``log2'')} dan
  \textbf{scale\_y\_continuous(trans=``log2'')}: nilai lain yang dapat
  diinputkan adalah ``log10''.
\end{itemize}

Pada contoh berikut disajikan contoh transformasi sumbu x menggunakan
fungsi \texttt{scale\_x\_log10()}. Grafik yang dihasilkan akan tampak
seperti Gambar \ref{fig:gglimits3}.

\begin{Shaded}
\begin{Highlighting}[]
\CommentTok{# atau}
\NormalTok{gapminder}\OperatorTok
\StringTok{  }\KeywordTok{filter}\NormalTok{(continent}\OperatorTok{==}\StringTok{"Europe"}\NormalTok{)}\OperatorTok
\StringTok{  }\KeywordTok{ggplot}\NormalTok{(}\KeywordTok{aes}\NormalTok{(gdpPercap, lifeExp))}\OperatorTok{+}
\StringTok{  }\KeywordTok{geom_point}\NormalTok{()}\OperatorTok{+}
\StringTok{  }\KeywordTok{theme_wsj}\NormalTok{(}\DataTypeTok{base_size=}\DecValTok{7}\NormalTok{)}\OperatorTok{+}
\StringTok{  }\KeywordTok{labs}\NormalTok{(}\DataTypeTok{title=}\StringTok{"log(GDP per Capita) vs Life Expectancy"}\NormalTok{,}
       \DataTypeTok{y=}\StringTok{"Life Expectancy"}\NormalTok{,}
       \DataTypeTok{x=}\StringTok{"GDP per Capita (US Dollar)"}\NormalTok{)}\OperatorTok{+}
\StringTok{  }\CommentTok{# transformasi sumbu x}
\StringTok{  }\KeywordTok{scale_x_log10}\NormalTok{()}
\end{Highlighting}
\end{Shaded}

\begin{figure}

{\centering \includegraphics[width=0.7\linewidth]{EnvStat_files/figure-latex/gglimits3-1} 

}

\caption{Scatterplot dengan transformasi axis }\label{fig:gglimits3}
\end{figure}

\emph{Tick mark} pada axis juga dapat kita atur menggunakan fungsi
\texttt{scale\_x\_continous()} dan \texttt{scale\_y\_continous()}. Untuk
mengubah format dan label \emph{tick mark} kita perlu menginstall dan
memuat library \texttt{scales} yang berfungsi untuk mengakses fungsi
pada argumen break. Berikut adalah sintaks yang digunakan beserta output
yang dihasilkan pada Gambar \ref{fig:gglimits4}.

\begin{Shaded}
\begin{Highlighting}[]
\CommentTok{# memasang paket}
\CommentTok{# install.packages("scales")}

\CommentTok{# memuat paket}
\KeywordTok{library}\NormalTok{(scales)}

\CommentTok{# plot}
\KeywordTok{ggplot}\NormalTok{(gapminder, }\KeywordTok{aes}\NormalTok{(gdpPercap, lifeExp))}\OperatorTok{+}
\StringTok{  }\KeywordTok{geom_point}\NormalTok{()}\OperatorTok{+}
\StringTok{  }\KeywordTok{theme_bw}\NormalTok{()}\OperatorTok{+}
\StringTok{  }\CommentTok{# kustomisasi tick mark sumbu y}
\StringTok{  }\KeywordTok{scale_y_continuous}\NormalTok{(}\DataTypeTok{trans=} \KeywordTok{log2_trans}\NormalTok{(),}
                     \DataTypeTok{breaks=}\KeywordTok{trans_breaks}\NormalTok{(}\StringTok{"log2"}\NormalTok{, }\ControlFlowTok{function}\NormalTok{(x) }\DecValTok{2}\OperatorTok{^}\NormalTok{x),}
                     \DataTypeTok{labels=} \KeywordTok{trans_format}\NormalTok{(}\StringTok{"log2"}\NormalTok{, }\KeywordTok{math_format}\NormalTok{(}\DecValTok{2}\OperatorTok{^}\NormalTok{.x)))}\OperatorTok{+}
\StringTok{  }\CommentTok{# kustomisasi sumbu x}
\StringTok{  }\KeywordTok{scale_x_continuous}\NormalTok{(}\DataTypeTok{labels =}\NormalTok{ dollar)}
\end{Highlighting}
\end{Shaded}

\begin{figure}

{\centering \includegraphics[width=0.7\linewidth]{EnvStat_files/figure-latex/gglimits4-1} 

}

\caption{Scatterplot dengan transformasi tick mark axis }\label{fig:gglimits4}
\end{figure}

\subsection{Kustomisasi Tick Mark
Axis}\label{kustomisasi-tick-mark-axis}

Pada bagian ini pembaca akan mempelajari bagaimana melakukan kustomisasi
tampilan \emph{tick mark}. Selain itu kita juga akan belajar bagaimana
melakukan pengaturan pada garis axis.

Warna, ukuran font, dan tampilan font (\emph{font style}) pada
\emph{tick mark} dapat diubah menggunakan fungsi \texttt{theme()} dan
\texttt{element\_text()}. Format yang digunakan adalah sebagai berikut:

\begin{Shaded}
\begin{Highlighting}[]
\CommentTok{# x axis tick mark labels}
\OperatorTok{<}\NormalTok{plot}\OperatorTok{>}\StringTok{ }\OperatorTok{+}\StringTok{ }\KeywordTok{theme}\NormalTok{(}\DataTypeTok{axis.text.x=} \KeywordTok{element_text}\NormalTok{(family, face, colour, size, angle))}
\CommentTok{# y axis tick mark labels}
\OperatorTok{<}\NormalTok{plot}\OperatorTok{>}\StringTok{ }\OperatorTok{+}\StringTok{ }\KeywordTok{theme}\NormalTok{(}\DataTypeTok{axis.text.y =} \KeywordTok{element_text}\NormalTok{(family, face, colour, size, angle))}
\end{Highlighting}
\end{Shaded}

\begin{quote}
\textbf{Note: }

\begin{itemize}
\tightlist
\item
  \textbf{family}: \emph{font family}, seperti: ``sans'',``times new
  roman'', dll.
\item
  \textbf{face}: \emph{font face}, nilai yang mungkin adalah ``plain'',
  ``italic'', ``bold'' dan ``bold.italic''.
\item
  \textbf{color}: warna teks.
\item
  \textbf{size}: ukuran teks dalam satuan pts.
\item
  \textbf{angle}: sudut kemiringan teks berkisar antara 0 sampai 360.
\end{itemize}
\end{quote}

Berikut adalah sintaks yang digunakan beserta output yang dihasilkan
pada Gambar \ref{fig:ggtick}.

\begin{Shaded}
\begin{Highlighting}[]
\KeywordTok{ggplot}\NormalTok{(gapminder, }\KeywordTok{aes}\NormalTok{(continent, gdpPercap,}
                      \DataTypeTok{fill=}\NormalTok{continent))}\OperatorTok{+}
\StringTok{  }\KeywordTok{geom_boxplot}\NormalTok{()}\OperatorTok{+}
\StringTok{  }\KeywordTok{theme_economist}\NormalTok{()}\OperatorTok{+}
\StringTok{  }\KeywordTok{scale_fill_economist}\NormalTok{()}\OperatorTok{+}
\StringTok{  }\CommentTok{# kustomisasi tick mark}
\StringTok{  }\KeywordTok{theme}\NormalTok{(}\DataTypeTok{axis.text.x =} \KeywordTok{element_text}\NormalTok{(}\DataTypeTok{face=}\StringTok{"bold"}\NormalTok{, }
                                   \DataTypeTok{color=}\StringTok{"#993333"}\NormalTok{,}
                                   \DataTypeTok{size=}\DecValTok{10}\NormalTok{, }
                                   \DataTypeTok{angle=}\DecValTok{30}\NormalTok{),}
          \DataTypeTok{axis.text.y =} \KeywordTok{element_text}\NormalTok{(}\DataTypeTok{face=}\StringTok{"bold"}\NormalTok{, }
                                     \DataTypeTok{color=}\StringTok{"#993333"}\NormalTok{,}
                                     \DataTypeTok{size=}\DecValTok{10}\NormalTok{, }
                                     \DataTypeTok{angle=}\DecValTok{30}\NormalTok{))}
\end{Highlighting}
\end{Shaded}

\begin{figure}

{\centering \includegraphics[width=0.7\linewidth]{EnvStat_files/figure-latex/ggtick-1} 

}

\caption{Mengubah tampilan dari tick mark}\label{fig:ggtick}
\end{figure}

Untuk menonaktifkan \emph{tick mark} pada plot kita dapat menggunakan
fungsi \texttt{element\_blank()}. Berikut adalah sintaks yang digunakan
beserta output yang dihasilkan pada Gambar \ref{fig:ggtick2}.

\begin{Shaded}
\begin{Highlighting}[]
\KeywordTok{ggplot}\NormalTok{(gapminder, }\KeywordTok{aes}\NormalTok{(continent, gdpPercap,}
                      \DataTypeTok{fill=}\NormalTok{continent))}\OperatorTok{+}
\StringTok{  }\KeywordTok{geom_boxplot}\NormalTok{()}\OperatorTok{+}
\StringTok{  }\KeywordTok{theme_stata}\NormalTok{()}\OperatorTok{+}
\StringTok{  }\KeywordTok{scale_fill_stata}\NormalTok{()}\OperatorTok{+}
\StringTok{  }\CommentTok{# menyembunyikan tick mark dan tick mark label}
\StringTok{  }\KeywordTok{theme}\NormalTok{(}\DataTypeTok{axis.text.x=}\KeywordTok{element_blank}\NormalTok{(),}
  \DataTypeTok{axis.text.y=}\KeywordTok{element_blank}\NormalTok{(),}
  \DataTypeTok{axis.ticks=}\KeywordTok{element_blank}\NormalTok{())}
\end{Highlighting}
\end{Shaded}

\begin{figure}

{\centering \includegraphics[width=0.7\linewidth]{EnvStat_files/figure-latex/ggtick2-1} 

}

\caption{Menyembunyikan tampilan dari tick mark}\label{fig:ggtick2}
\end{figure}

Kita dapat melakukan pengaturan terhadap garis axis menggunakan argumen
\texttt{axis.lines} dan fungsi \texttt{element\_line}. Berikut adalah
format yang digunakan:

\begin{Shaded}
\begin{Highlighting}[]
\OperatorTok{<}\NormalTok{plot}\OperatorTok{>}\StringTok{ }\OperatorTok{+}\StringTok{ }\KeywordTok{theme}\NormalTok{(}\DataTypeTok{axis.line =} \KeywordTok{element_line}\NormalTok{(color,size, linetype,}
\NormalTok{                                        lineend, color))}
\end{Highlighting}
\end{Shaded}

\begin{quote}
\textbf{Note: }

\begin{itemize}
\tightlist
\item
  \textbf{color}: warna garis.
\item
  \textbf{size}: ukuran garis.
\item
  \textbf{linetype}: jenis garis.
\item
  \textbf{lineend}: akhir dari garis. Nilai yang dapat dimasukkan antara
  lain: ``round'', ``butt'' atau ``square''.
\end{itemize}
\end{quote}

Berikut adalah sintaks yang digunakan beserta output yang dihasilkan
pada Gambar \ref{fig:ggtick3}.

\begin{Shaded}
\begin{Highlighting}[]
\KeywordTok{ggplot}\NormalTok{(gapminder, }\KeywordTok{aes}\NormalTok{(continent, gdpPercap,}
                      \DataTypeTok{fill=}\NormalTok{continent))}\OperatorTok{+}
\StringTok{  }\KeywordTok{geom_boxplot}\NormalTok{()}\OperatorTok{+}
\StringTok{  }\KeywordTok{theme_wsj}\NormalTok{()}\OperatorTok{+}
\StringTok{  }\KeywordTok{scale_fill_wsj}\NormalTok{()}\OperatorTok{+}
\StringTok{  }\CommentTok{# kustomisasi garis axis}
\StringTok{  }\KeywordTok{theme}\NormalTok{(}\DataTypeTok{axis.line =} \KeywordTok{element_line}\NormalTok{(}\DataTypeTok{colour =} \StringTok{"darkblue"}\NormalTok{, }
                      \DataTypeTok{size =} \DecValTok{1}\NormalTok{, }\DataTypeTok{linetype =} \StringTok{"solid"}\NormalTok{))}
\end{Highlighting}
\end{Shaded}

\begin{figure}

{\centering \includegraphics[width=0.7\linewidth]{EnvStat_files/figure-latex/ggtick3-1} 

}

\caption{Kustomisasi tampilan dari garis axis}\label{fig:ggtick3}
\end{figure}

Kita dapat mengatur \emph{tick} pada axis baik yang memiliki skala
diskrit maupun kontinyu. Fungsi yang digunakan adalah
\texttt{scale\_x\_continous()} dan \texttt{scale\_y\_continous()} untuk
\emph{tick} dengan nilai kontinyu dan \texttt{scale\_x\_discrete()} dan
\texttt{scale\_y\_discrete()}.

Berikut adalah sintaks yang digunakan beserta output yang dihasilkan
pada Gambar \ref{fig:ggtick4}.

\begin{Shaded}
\begin{Highlighting}[]
\KeywordTok{ggplot}\NormalTok{(gapminder, }\KeywordTok{aes}\NormalTok{(continent, lifeExp,}
                      \DataTypeTok{fill=}\NormalTok{continent))}\OperatorTok{+}
\StringTok{  }\KeywordTok{geom_boxplot}\NormalTok{()}\OperatorTok{+}
\StringTok{  }\KeywordTok{theme_gdocs}\NormalTok{()}\OperatorTok{+}
\StringTok{  }\KeywordTok{scale_fill_gdocs}\NormalTok{()}\OperatorTok{+}
\StringTok{  }\CommentTok{# kustomisasi tick mark}
\StringTok{  }\KeywordTok{scale_y_continuous}\NormalTok{(}
    \CommentTok{# nilai dari 0 sampai 100 tiap 10 tick}
    \DataTypeTok{breaks=}\KeywordTok{seq}\NormalTok{(}\DecValTok{0}\NormalTok{,}\DecValTok{100}\NormalTok{,}\DecValTok{10}\NormalTok{))}
\end{Highlighting}
\end{Shaded}

\begin{figure}

{\centering \includegraphics[width=0.7\linewidth]{EnvStat_files/figure-latex/ggtick4-1} 

}

\caption{Kustomisasi tick mark}\label{fig:ggtick4}
\end{figure}

\subsection{Menambahkan Garis Lurus Pada
Plot}\label{menambahkan-garis-lurus-pada-plot}

Fungsi yang dapat digunakan untuk menambahkan garis lurus antara lain:

\begin{itemize}
\tightlist
\item
  \textbf{geom\_hline()}: menambahkan garis horizontal.
\item
  \textbf{geom\_abline()}: menambahkan garis regresi.
\item
  \textbf{geom\_vline()}: menambahkan garis vertikal.
\item
  \textbf{geom\_segment()}: menambahkan garis segmen.
\end{itemize}

Format yang digunakan untuk fungsi \texttt{geom\_hline()} dan
\texttt{geom\_vline()} adalah sebagai berikut:

\begin{Shaded}
\begin{Highlighting}[]
\KeywordTok{geom_hline}\NormalTok{(yintercept, linetype, color, size)}
\KeywordTok{geom_vline}\NormalTok{(xintercept, linetype, color, size)}
\end{Highlighting}
\end{Shaded}

Berikut adalah contoh penerapan kedua fungsi tersebut yang disajikan
pada Gambar \ref{fig:ggvline} dan Gambar \ref{fig:gghline}:

\begin{Shaded}
\begin{Highlighting}[]
\KeywordTok{ggplot}\NormalTok{(gapminder, }\KeywordTok{aes}\NormalTok{(lifeExp, }\DataTypeTok{fill=}\NormalTok{..count..))}\OperatorTok{+}
\StringTok{  }\KeywordTok{geom_histogram}\NormalTok{()}\OperatorTok{+}
\StringTok{  }\KeywordTok{theme_calc}\NormalTok{()}\OperatorTok{+}
\StringTok{  }\CommentTok{# menambahkan garis vertikal}
\StringTok{  }\KeywordTok{geom_vline}\NormalTok{(}\DataTypeTok{xintercept=}\KeywordTok{mean}\NormalTok{(gapminder}\OperatorTok{$}\NormalTok{lifeExp), }
             \DataTypeTok{linetype=}\StringTok{"twodash"}\NormalTok{,}
             \DataTypeTok{color=}\StringTok{"red"}\NormalTok{,}
             \DataTypeTok{size=}\FloatTok{1.5}\NormalTok{)}
\end{Highlighting}
\end{Shaded}

\begin{figure}

{\centering \includegraphics[width=0.7\linewidth]{EnvStat_files/figure-latex/ggvline-1} 

}

\caption{Penerapan vline}\label{fig:ggvline}
\end{figure}

\begin{Shaded}
\begin{Highlighting}[]
\KeywordTok{ggplot}\NormalTok{(gapminder, }\KeywordTok{aes}\NormalTok{(continent, lifeExp, }
                      \DataTypeTok{fill=}\NormalTok{continent))}\OperatorTok{+}
\StringTok{  }\KeywordTok{geom_boxplot}\NormalTok{()}\OperatorTok{+}
\StringTok{  }\KeywordTok{theme_calc}\NormalTok{()}\OperatorTok{+}
\StringTok{  }\KeywordTok{scale_fill_calc}\NormalTok{()}\OperatorTok{+}
\StringTok{  }\CommentTok{# menambahkan garis horizontal}
\StringTok{  }\KeywordTok{geom_hline}\NormalTok{(}\DataTypeTok{yintercept=}\KeywordTok{mean}\NormalTok{(gapminder}\OperatorTok{$}\NormalTok{lifeExp), }
             \DataTypeTok{linetype=}\StringTok{"twodash"}\NormalTok{,}
             \DataTypeTok{color=}\StringTok{"red"}\NormalTok{,}
             \DataTypeTok{size=}\FloatTok{1.5}\NormalTok{)}
\end{Highlighting}
\end{Shaded}

\begin{figure}

{\centering \includegraphics[width=0.7\linewidth]{EnvStat_files/figure-latex/gghline-1} 

}

\caption{Penerapan hline}\label{fig:gghline}
\end{figure}

Selain menggunakan fungsi \texttt{geom\_smooth()}, garis regresi dapat
ditambahkan melalui fungsi `geom\_abline(). Format yang digunakan adalah
sebagai berikut:

\begin{Shaded}
\begin{Highlighting}[]
\KeywordTok{geom_abline}\NormalTok{(intercept, slope, linetype, color, size)}
\end{Highlighting}
\end{Shaded}

Untuk membuat garis regresi kita perlu membuat model regresi terlebih
dahulu menggunakn fungsi \texttt{lm()}. Berikut adalah contoh model yang
dibuat beserta koefisien regresinya.

\begin{Shaded}
\begin{Highlighting}[]
\CommentTok{# membuat model regresi}
\NormalTok{mod <-}\StringTok{ }\KeywordTok{lm}\NormalTok{(lifeExp}\OperatorTok{~}\NormalTok{gdpPercap, }\DataTypeTok{data=}\NormalTok{gapminder)}

\CommentTok{# print model}
\NormalTok{mod}
\end{Highlighting}
\end{Shaded}

\begin{verbatim}
## 
## Call:
## lm(formula = lifeExp ~ gdpPercap, data = gapminder)
## 
## Coefficients:
## (Intercept)    gdpPercap  
##    5.40e+01     7.65e-04
\end{verbatim}

\begin{Shaded}
\begin{Highlighting}[]
\CommentTok{# koefisien regresi model}
\NormalTok{coef <-}\StringTok{ }\KeywordTok{coefficients}\NormalTok{(mod)}

\CommentTok{# print koefisien}
\NormalTok{coef}
\end{Highlighting}
\end{Shaded}

\begin{verbatim}
## (Intercept)   gdpPercap 
##   5.396e+01   7.649e-04
\end{verbatim}

Berikut adalah sintaks yang digunakan beserta output yang dihasilkan
pada Gambar \ref{fig:ggabline} untuk membuat plot regresi linier.

\begin{Shaded}
\begin{Highlighting}[]
\KeywordTok{ggplot}\NormalTok{(gapminder, }\KeywordTok{aes}\NormalTok{(gdpPercap, lifeExp))}\OperatorTok{+}
\StringTok{  }\KeywordTok{geom_point}\NormalTok{(}\DataTypeTok{shape=}\DecValTok{1}\NormalTok{, }\DataTypeTok{color=}\StringTok{"grey"}\NormalTok{)}\OperatorTok{+}
\StringTok{  }\KeywordTok{theme_stata}\NormalTok{()}\OperatorTok{+}
\StringTok{  }\CommentTok{# menambahkan garis regresi}
\StringTok{  }\KeywordTok{geom_abline}\NormalTok{(}\DataTypeTok{intercept=}\FloatTok{5.395556e+01}\NormalTok{,}
         \DataTypeTok{slope=}\FloatTok{7.648826e-04}\NormalTok{,}
         \DataTypeTok{linetype=}\StringTok{"twodash"}\NormalTok{,}
             \DataTypeTok{color=}\StringTok{"red"}\NormalTok{,}
             \DataTypeTok{size=}\DecValTok{1}\NormalTok{)}
\end{Highlighting}
\end{Shaded}

\begin{figure}

{\centering \includegraphics[width=0.7\linewidth]{EnvStat_files/figure-latex/ggabline-1} 

}

\caption{Penerapan abline}\label{fig:ggabline}
\end{figure}

Kita dapat menambahkan garis segment untuk menunjukkan sebuah observasi.
Format yang digunakan adalah sebagai berikut:

\begin{Shaded}
\begin{Highlighting}[]
\KeywordTok{geom_segment}\NormalTok{(}\KeywordTok{aes}\NormalTok{(x, y, xend, yend))}
\end{Highlighting}
\end{Shaded}

Berikut adalah sintaks yang digunakan beserta output yang dihasilkan
pada Gambar \ref{fig:ggsegment} untuk membuat garis segmen.

\begin{Shaded}
\begin{Highlighting}[]
\KeywordTok{library}\NormalTok{(grid)}
\KeywordTok{ggplot}\NormalTok{(gapminder, }\KeywordTok{aes}\NormalTok{(gdpPercap, lifeExp))}\OperatorTok{+}
\StringTok{  }\KeywordTok{geom_point}\NormalTok{(}\DataTypeTok{shape=}\DecValTok{1}\NormalTok{, }\DataTypeTok{color=}\StringTok{"grey"}\NormalTok{)}\OperatorTok{+}
\StringTok{  }\KeywordTok{theme_stata}\NormalTok{()}\OperatorTok{+}
\StringTok{  }\CommentTok{# menambahkan tanda panah}
\StringTok{  }\KeywordTok{geom_segment}\NormalTok{(}\DataTypeTok{x=}\DecValTok{70000}\NormalTok{, }\DataTypeTok{y=}\DecValTok{80}\NormalTok{,}
                   \DataTypeTok{xend=}\DecValTok{60000}\NormalTok{, }\DataTypeTok{yend=}\DecValTok{70}\NormalTok{,}
                   \DataTypeTok{arrow=}\KeywordTok{arrow}\NormalTok{(}\DataTypeTok{length=}\KeywordTok{unit}\NormalTok{(}\FloatTok{0.1}\NormalTok{, }\StringTok{"inches"}\NormalTok{)),}
               \DataTypeTok{linetype=}\StringTok{"twodash"}\NormalTok{,}
               \DataTypeTok{color=}\StringTok{"red"}\NormalTok{,}
               \DataTypeTok{size=}\DecValTok{1}\NormalTok{)}
\end{Highlighting}
\end{Shaded}

\begin{figure}

{\centering \includegraphics[width=0.7\linewidth]{EnvStat_files/figure-latex/ggsegment-1} 

}

\caption{Penerapan garis segmen}\label{fig:ggsegment}
\end{figure}

\subsection{Melakukan Rotasi Pada
Grafik}\label{melakukan-rotasi-pada-grafik}

Rotasi grafik atau pembalikan axis dapat dilakukan menggunakan fungsi
berikut:

\begin{itemize}
\tightlist
\item
  \textbf{coord\_flip()}: untuk membuat plot horizontal.Rotasi axis
  sehingga sumbu x dapat menjadi sumbu y dan sebaliknya.
\item
  \textbf{scale\_x\_reverse()} dan \textbf{scale\_x\_reverse()}:
  pembalikan skala pada axis.
\end{itemize}

Misalkan kita ingin membuat plot horizontal pada box plot sehingga
mempermudah kita dalam melakukan perbandingan terhadap masing-masing
grup. Berikut adalah sintaks yang digunakan beserta output yang
dihasilkan pada Gambar \ref{fig:ggcoord}.

\begin{Shaded}
\begin{Highlighting}[]
\KeywordTok{ggplot}\NormalTok{(gapminder, }\KeywordTok{aes}\NormalTok{(continent, lifeExp, }
                      \DataTypeTok{fill=}\NormalTok{continent))}\OperatorTok{+}
\StringTok{  }\KeywordTok{geom_boxplot}\NormalTok{()}\OperatorTok{+}
\StringTok{  }\KeywordTok{theme_economist}\NormalTok{()}\OperatorTok{+}
\StringTok{  }\KeywordTok{scale_fill_economist}\NormalTok{()}\OperatorTok{+}
\StringTok{  }\CommentTok{# rotasi axis}
\StringTok{  }\KeywordTok{coord_flip}\NormalTok{()}
\end{Highlighting}
\end{Shaded}

\begin{figure}

{\centering \includegraphics[width=0.7\linewidth]{EnvStat_files/figure-latex/ggcoord-1} 

}

\caption{Rotasi axis}\label{fig:ggcoord}
\end{figure}

Kita dapat juga melakukan pembalikan skala pada axis sehingga skala yang
semula berawal dari min ke max menjadi sebaliknya. Berikut adalah
sintaks yang digunakan beserta output yang dihasilkan pada Gambar
\ref{fig:ggyreverse}.

\begin{Shaded}
\begin{Highlighting}[]
\KeywordTok{ggplot}\NormalTok{(gapminder, }\KeywordTok{aes}\NormalTok{(lifeExp, }\DataTypeTok{fill=}\NormalTok{..count..))}\OperatorTok{+}
\StringTok{  }\KeywordTok{geom_histogram}\NormalTok{()}\OperatorTok{+}
\StringTok{  }\KeywordTok{theme_wsj}\NormalTok{()}\OperatorTok{+}
\StringTok{  }\CommentTok{# pembalikan sumbu y}
\StringTok{  }\KeywordTok{scale_y_reverse}\NormalTok{()}
\end{Highlighting}
\end{Shaded}

\begin{figure}

{\centering \includegraphics[width=0.7\linewidth]{EnvStat_files/figure-latex/ggyreverse-1} 

}

\caption{Pembalikan sumbu y}\label{fig:ggyreverse}
\end{figure}

\subsection{Facet}\label{facet}

Facet digunakan untuk membagi plot menjadi panel matriks. Setiap panel
menunjukkan setiap kelompok data. Fungsi facet yang dapat digunakan
antara lain:

\begin{itemize}
\tightlist
\item
  \textbf{facet\_grid()}
\item
  \textbf{facet\_wrap()}
\end{itemize}

Berikut adalah sintaks yang digunakan beserta output yang dihasilkan
pada Gambar \ref{fig:ggfacetgrid} dan Gambar \ref{fig:ggfacetgrid2}
untuk membuat facet pada satu variabel.

\begin{Shaded}
\begin{Highlighting}[]
\KeywordTok{ggplot}\NormalTok{(gapminder, }\KeywordTok{aes}\NormalTok{(lifeExp, }\DataTypeTok{fill=}\NormalTok{..count..))}\OperatorTok{+}
\StringTok{  }\KeywordTok{geom_histogram}\NormalTok{()}\OperatorTok{+}
\StringTok{  }\KeywordTok{theme_gdocs}\NormalTok{()}\OperatorTok{+}
\StringTok{  }\KeywordTok{facet_grid}\NormalTok{(.}\OperatorTok{~}\NormalTok{continent)}
\end{Highlighting}
\end{Shaded}

\begin{figure}

{\centering \includegraphics[width=0.9\linewidth]{EnvStat_files/figure-latex/ggfacetgrid-1} 

}

\caption{Facet horizontal satu variabel}\label{fig:ggfacetgrid}
\end{figure}

\begin{Shaded}
\begin{Highlighting}[]
\KeywordTok{ggplot}\NormalTok{(gapminder, }\KeywordTok{aes}\NormalTok{(lifeExp, }\DataTypeTok{fill=}\NormalTok{..count..))}\OperatorTok{+}
\StringTok{  }\KeywordTok{geom_histogram}\NormalTok{()}\OperatorTok{+}
\StringTok{  }\KeywordTok{theme_gdocs}\NormalTok{()}\OperatorTok{+}
\StringTok{  }\KeywordTok{facet_grid}\NormalTok{(continent}\OperatorTok{~}\NormalTok{.)}
\end{Highlighting}
\end{Shaded}

\begin{figure}

{\centering \includegraphics[width=0.8\linewidth]{EnvStat_files/figure-latex/ggfacetgrid2-1} 

}

\caption{Facet vertikal satu variabel}\label{fig:ggfacetgrid2}
\end{figure}

Kita dapat pula melakukan facet terhadap dua buah variabel.Berikut
adalah sintaks yang digunakan beserta output yang dihasilkan pada Gambar
\ref{fig:ggfacetgrid3} untuk membuat facet pada dua variabel.

\begin{Shaded}
\begin{Highlighting}[]
\NormalTok{gapminder}\OperatorTok
\StringTok{  }\KeywordTok{filter}\NormalTok{(year}\OperatorTok{==}\DecValTok{1952}\OperatorTok{|}\NormalTok{year}\OperatorTok{==}\DecValTok{2007}\NormalTok{,}
\NormalTok{         continent }\OperatorTok\StringTok{ }\KeywordTok{c}\NormalTok{(}\StringTok{"Asia"}\NormalTok{,}\StringTok{"Americas"}\NormalTok{))}\OperatorTok
\StringTok{  }\KeywordTok{ggplot}\NormalTok{(}\KeywordTok{aes}\NormalTok{(continent, lifeExp, }
             \DataTypeTok{fill=}\KeywordTok{factor}\NormalTok{(year)))}\OperatorTok{+}
\StringTok{  }\KeywordTok{geom_boxplot}\NormalTok{()}\OperatorTok{+}
\StringTok{  }\KeywordTok{theme_stata}\NormalTok{()}\OperatorTok{+}
\StringTok{  }\KeywordTok{scale_fill_stata}\NormalTok{()}\OperatorTok{+}
\StringTok{  }\KeywordTok{facet_grid}\NormalTok{(continent}\OperatorTok{~}\KeywordTok{factor}\NormalTok{(year))}
\end{Highlighting}
\end{Shaded}

\begin{figure}

{\centering \includegraphics[width=0.8\linewidth]{EnvStat_files/figure-latex/ggfacetgrid3-1} 

}

\caption{Facet dua variabel}\label{fig:ggfacetgrid3}
\end{figure}

Kita dapat mengatur skala dari axis menggunakan argument sebagai
berikut:

\begin{itemize}
\tightlist
\item
  \textbf{free}: skala akan disesuaikan berdasarkan pada setiap axis.
\item
  \textbf{free\_x}: skala pada sumbu x akan dibiarkan menyesuaikan
  secara bebas.
\item
  \textbf{free\_y}: skala pada sumbu y akan dibiarkan menyesuaikan
  secara bebas.
\item
  \textbf{fixed} (default): skala axis diseragamkan pada seluruh panel.
\end{itemize}

Berikut adalah sintaks yang digunakan beserta output yang dihasilkan
pada Gambar \ref{fig:ggfacetgrid4} untuk membuat facet pada dua variabel
dengan skala bebas pada sumbu y.

\begin{Shaded}
\begin{Highlighting}[]
\NormalTok{gapminder}\OperatorTok
\StringTok{  }\KeywordTok{filter}\NormalTok{(year}\OperatorTok{==}\DecValTok{1952}\OperatorTok{|}\NormalTok{year}\OperatorTok{==}\DecValTok{2007}\NormalTok{,}
\NormalTok{         continent }\OperatorTok\StringTok{ }\KeywordTok{c}\NormalTok{(}\StringTok{"Asia"}\NormalTok{,}\StringTok{"Americas"}\NormalTok{))}\OperatorTok
\StringTok{  }\KeywordTok{ggplot}\NormalTok{(}\KeywordTok{aes}\NormalTok{(continent, lifeExp, }
             \DataTypeTok{fill=}\KeywordTok{factor}\NormalTok{(year)))}\OperatorTok{+}
\StringTok{  }\KeywordTok{geom_boxplot}\NormalTok{()}\OperatorTok{+}
\StringTok{  }\KeywordTok{theme_stata}\NormalTok{()}\OperatorTok{+}
\StringTok{  }\KeywordTok{scale_fill_stata}\NormalTok{()}\OperatorTok{+}
\StringTok{  }\KeywordTok{facet_grid}\NormalTok{(continent}\OperatorTok{~}\KeywordTok{factor}\NormalTok{(year), }\DataTypeTok{scales=}\StringTok{"free_y"}\NormalTok{)}
\end{Highlighting}
\end{Shaded}

\begin{figure}

{\centering \includegraphics[width=0.8\linewidth]{EnvStat_files/figure-latex/ggfacetgrid4-1} 

}

\caption{Facet dua variabel dengan skala bebas pada sumbu y}\label{fig:ggfacetgrid4}
\end{figure}

\section{Referensi}\label{referensi-4}

\begin{enumerate}
\def\labelenumi{\arabic{enumi}.}
\tightlist
\item
  Wickham, H. Grolemund G. 2016. \textbf{R For Data Science: Import,
  Tidy, Transform, Visualize, And Model Data}. O'Reilly Media, Inc.
\item
  Peng, R.D. 2015. \textbf{Exploratory Data Analysis with R}. Leanpub
  book.
\item
  GGPLOT2 Documentation. \url{https://ggplot2.tidyverse.org/}
\item
  STHDA. ggplot2 - Essentials.
  \url{https://www.sthda.com/english/wiki/ggplot2-essentials}
\end{enumerate}

\part*{Statistika Deskriptif -
R}\label{part-statistika-deskriptif---r}
\addcontentsline{toc}{part}{Statistika Deskriptif - R}

\chapter{Ringkasan Numerik}\label{ringkasan-numerik}

Pada bidang lingkungan kita sering kali menemui sebuah pernyataan
``konsentrasi rata-rata TSS pada sungai tersebut adalah 30 mg/l'' atau
``kedalaman penampang saluran tersebut berkisar antara 1 sampai 2
meter''. Kedua pernyataan tersebut merupakan sebuah penyapaian informasi
terkait karakteristik data yang ada. Pernyataan yang pertama menyatakan
karakteristik nilai pemusatan data, sedangkan yang kedua menyatakan
karakteristik sebaran suatu data.

Karakteristik lain yang sering digunakan untuk menjelaskan suatu data
adalah bentuk distibusi suatu data dan estimasi nilai ekstrim seperti
nilai masimum dan minimum suatu data. Seluruh karakteristik data
tersebut perlu dihitung untuk memperoleh informasi numerik pada data.

Pada chapter ini kita akan membahas terkait metode untuk membuat
ringkasan dan deksripsi data. Pembahasan akan terdiri dari ukuran nilai
pemusatan data, ukuran sebaran atau variabilitas data dan bentuk
distribusi data. Selain itu kita akan membahas nilai ekstrim yang ada
pada sebuah data dan transformasi data.

\section{Ukuran Pemusatan Data}\label{ukuran-pemusatan-data}

Nilai rata-rata (mean) dan nilai tengah (median) merupakan dua nilai
yang paling umum digunakan untuk menyatakan lokasi pemusatan data
meskipun kedua nilai bukanlah satu atau dua ukuran yang tersedia. Apa
sajakah properti dari kedua ukuran tersebut dan kapan salah satu atau
keduanya dapat digunakan bersamaan?.

\subsection{Pengukuran Klasik-Mean}\label{pengukuran-klasik-mean}

Nilai mean (\(\overline{X}\)) diperoleh dengan menjumlahkan seluruh data
dan membaginya dengan jumlah observasinya yang dapat dituliskan seperti
Persamaan \eqref{eq:mean}:

\begin{equation}
  \overline{X}=\text{}\sum_{_{i=1}}^n\frac{X_i}{n}
  \label{eq:mean}
\end{equation}

Nilai mean yang disimbolkan dengan ``X bar'' merupakan nilai mean untuk
sampel. Nilai mean untuk populasi disimbolkan oleh huruf Yunani ``mu
atau \(\mu\)''.

Pada Persamaan \eqref{eq:mean}, jika data terdiri dari banyak grup maka
nilai rata-rata dihitung berdasarkan jumlah nilai observasi dikali
dengan bobotnya. Nilai mean tersebut disebut sebagai \emph{weighted
mean} yang dapat ditulis berdasarkan Persamaan Persamaan
\eqref{eq:weightedmean}.

\begin{equation}
  \overline{X}\ =\text{}\sum_{_{i=1}}^n\overline{X_i}\cdot\frac{n_i}{n}
  \label{eq:weightedmean}
\end{equation}

dimana \(\overline{X_i}\) merupakan nilai rata-rata grup ke-i dan
\(\frac{n_i}{n}\) merupakan bobot pengali yang berupa rasio antara
observasi grup ke-i dengan keseluruhan observasi.

Kita biasanya akan berhadapan dengan nilai observasi yang baru sehingga
nilai mean yang telah ada akan ikut berubah. Perubahan nilai mean
tersebut disebabkan karena setiap observasi yang disertakan dalam
perhitungan mean memiliki pengaruhnya masing-masing. Jika observasi
tersebut cenderung ekstrim besar maka nilai mean akan bergeser menuju
kearahnya begitu juga sebaliknya.

Pengaruh dari sebuah nilai observasi ke-j atau \(X_j\) dapat dilihat
dengan menghitung seluruh observasi secara bersamaan kecuali observasi
ke-j pada sebuah grup. Dapat dituliskan pada Persamaan
\eqref{eq:influencemean2}

\begin{equation}
  \overline{X} =\text{}\overline{X_{\left(j\right)}}\ \cdot\frac{\left(n-1\right)}{n}+X_j\cdot\frac{1}{n}
  \label{eq:influencemean1}
\end{equation}

\begin{equation}
  \overline{X} =\overline{X}_{\left(j\right)}+\left(X_j-\overline{X}_{\left(j\right)}\right)\cdot\frac{1}{n}
  \label{eq:influencemean2}
\end{equation}

dimana \(\overline{X}_{\left(j\right)}\) adalah nilai mean seluruh
observasi kecuali \(X_j\). Setiap observasi yang mempengaruhi nilai mean
keseluruhan (\(\overline{X}\)) didefinisikan oleh
\(\left(X_j-\overline{X}_{\left(j\right)}\right)\) sebagai jarak antara
observasi tersebut dengan nilai rata-rata yang tidak termasuk observasi
tersebut di dalamnya. Sehingga seluruh nilai observasi tidak memiliki
pengaruh yang sama terhadap nilai rata-rata seluruh observasi.

\emph{Outlier} merupakan observasi yang memiliki nilai yang ekstrim
tinggi atau rendah dibanding seluruh observasi yang ada sehingga
memiliki pengaruh yang besar terhadap nilai mean keseluruhan
(\(\overline{X}\)). Pengaruhnya yang sangat besar terhadap nilai
rata-rata keseluruhan akan menyebabkan nilai rata-rata akan bergeser ke
arah \emph{outlier} tersebut. Selain itu penampilan dari distribusi
frekuensi yang terbentuk akan terlihat memiliki ekor yang panjang.

Untuk lebih memahami pengaruh observasi terhadap nilai rata-rata,
disajikan dua buah gambar yaitu: Gambar \ref{fig:mean1} dan Gambar
\ref{fig:mean2}

\begin{figure}

{\centering \includegraphics[width=0.7\linewidth]{mean1} 

}

\caption{Nilai mean (segitiga) sebagai titik kesetimbangan pada data.}\label{fig:mean1}
\end{figure}

\begin{figure}

{\centering \includegraphics[width=0.7\linewidth]{mean2} 

}

\caption{Pergeseran nilai mean (segitiga) ke kiri setelah penghilangan outlier.}\label{fig:mean2}
\end{figure}

Pada Gambar \ref{fig:mean1} disajikan 7 buah data konsentrasi TSS di
suatu sungai. Nilai rata-rata TSS pada sungai tersebut adalah 11 mg/l.
Jika kita amati sebagian besar data (6 observasi) berada pada interval
nilai konsentrasi TSS 2 sampai 12 mg/l. Observasi yang lain terletak
jauh dari mayoritas observasi lainnya yaitu sebesar 37 mg/l. Observasi
yang berbeda secara ekstrim dari nilai secara umum pada suatu data
disebut sebagai \emph{outlier}. Nilai \emph{outlier} tersebut
menyebabkan nilai rata-rata yang terbentuk tidak representatif terhadap
keseluruhan data yang ada dan cenderung menggeser nilai rata-rata
mendekati nilai \emph{outlier} tersebut. Nilai observasi yang ekstrim
biasanya muncul dari adanya kesalahan perlakuan terhadap sampel seperti
botol sampel yang digunakan tidak bersih atau prosedur analisa yang
dilakukan tidak standar sehingga memungkinkan adanya partikulat udara
yang terukur pada proses penimbangan.

Salah satu cara untuk menangani adanya \emph{outlier} tersebut adalah
dengan menghapus observasi yang merupakan \emph{outlier}. Pada Gambar
\ref{fig:mean2} terlihat bahwa penghapusan \emph{outlier} telah
menggeser nilai rata-rata ke kiri. Nilai rata-rata yang baru tersebut
jika diperhatikan dari Gambar \ref{fig:mean2} lebih menggambarkan
keseluruhan data yang ada. Tidak terlihat adanya nilai yang berada jauh
jaraknya dari nilai rata-rata yang baru.

Pada contoh tersebut dapat kita simpulkan bahwa nilai mean sangat
sensitif terhadap adanya \emph{outlier}. Pada prakteknya nilai mean
tidaklah berdiri sendiri selama proses analisa. Nilai mean memerlukan
nilai lain seperti median untuk menganalisa apakah data yang diperoleh
tidak simetris yang dapat mengindikasikan adanya outlier.

Pada \texttt{R} untuk menghitung nilai rata-rata, kita dapat menggunakan
fungsi \texttt{mean()}. Format fungsi yang digunakan dituliskan pada
persamaan berikut:

\begin{Shaded}
\begin{Highlighting}[]
\KeywordTok{mean}\NormalTok{(x, }\DataTypeTok{trim =} \DecValTok{0}\NormalTok{, }\DataTypeTok{na.rm =} \OtherTok{FALSE}\NormalTok{)}
\end{Highlighting}
\end{Shaded}

\begin{quote}
\textbf{Note:}

\begin{itemize}
\tightlist
\item
  \textbf{x}: objek atau vektor numerik.
\item
  \textbf{trim}: menyatakan fraksi data (berkisar antara 0 sampai 0,5)
  yang perlu dilakukan pemotongan (\emph{trim}) pada observasi awal dan
  akhir \textbf{x} (yang telah diurutkan) sebelum nilai mean dihitung.
  \textbf{na.rm}: nilai logis yang menyatakan apakah \emph{missing
  value} perlu disertakan dalam perhitungan atau tidak. Jika disertakan
  maka output yang akan dihasilkan adalah NA.
\end{itemize}
\end{quote}

\textbf{Analisa Nilai Mean Grup Data Tunggal (\emph{Single Group})}

Untuk lebih memahami penerapannya pada \texttt{R}, pada Tabel
\ref{tab:debitsungai} berikut disajikan data terkait debit air suatu
sungai.

\begin{table}[t]

\caption{\label{tab:debitsungai}Data Debit Sampel (m3/detik)}
\centering
\begin{tabular}{r|r}
\hline
observasi & debit\\
\hline
1 & 457\\
\hline
2 & 185\\
\hline
3 & 133\\
\hline
4 & 160\\
\hline
5 & 119\\
\hline
6 & 115\\
\hline
7 & 101\\
\hline
8 & 58\\
\hline
9 & 68\\
\hline
10 & 50\\
\hline
11 & 65\\
\hline
12 & 128\\
\hline
\end{tabular}
\end{table}

Data pada Tabel \ref{tab:debitsungai} dapat divisualisasikan seperti
pada Gambar \ref{fig:debitvis}:

\begin{figure}

{\centering \includegraphics[width=0.7\linewidth]{EnvStat_files/figure-latex/debitvis-1} 

}

\caption{Visualisasi debit sungai pada sampel}\label{fig:debitvis}
\end{figure}

Berdasarkan Gambar \ref{fig:debitvis}, terdapat \emph{outlier} yang
ditunjukkan pada debit sungai yang lebih besar dari 400 m3/detik. Hasil
tersebut dapat terjadi salah satunya karena adanya kondisi ekstrim
seperti banjir yang menyebabkan sungai meluap atau terjadi kesalahan
pengukuran dari alat ukur yang ada di lapangan.

Untuk menghitung nilai rata-rata debit pada data tersebut, masukkan
variabel \texttt{debit} yang telah penulis simpan sebagai objek
\texttt{sungai} kedalam fungsi \texttt{mean()} seperti berikut:

\begin{Shaded}
\begin{Highlighting}[]
\KeywordTok{mean}\NormalTok{(sungai}\OperatorTok{$}\NormalTok{debit)}
\end{Highlighting}
\end{Shaded}

\begin{verbatim}
## [1] 136.6
\end{verbatim}

Berdasarkan hasil yang diperoleh, dapat dilihat bahwa nilai rata-rata
debit pada sungai tersebut adalah 136.5833 \(m^3/detik\).

Kita dapat menghitung nilai mean dengan terlebih dahulu menghilangkan
\emph{outlier} pada data. Untuk melakukannya kita perlu melakukan subset
terhadap data tanpa \emph{outlier} di dalamnya sebelum data tersebut
dimasukkan kedalam fungsi mean(). Berikut sintaks yang digunakan untuk
melakukan hal tersebut:

\begin{Shaded}
\begin{Highlighting}[]
\CommentTok{# memuat paket}
\KeywordTok{library}\NormalTok{(dplyr)}

\CommentTok{# melakukan filter terhadap data}
\NormalTok{sungai_subset<-sungai}\OperatorTok
\StringTok{  }\KeywordTok{filter}\NormalTok{(debit}\OperatorTok{<=}\DecValTok{400}\NormalTok{)}

\CommentTok{# menghitung mean}
\KeywordTok{mean}\NormalTok{(sungai_subset}\OperatorTok{$}\NormalTok{debit)}
\end{Highlighting}
\end{Shaded}

\begin{verbatim}
## [1] 107.5
\end{verbatim}

Berdasarkan hasil yang diperoleh terlihat bahwa nilai rata-rata yang
baru lebih kecil dari yang sebelumnya (bergeser ke kiri) dengan nilai
mean debit sungai yang baru sebesar 107.4545 \(m^3/detik\). Hal ini
terjadi karena pengaruh dari data \emph{outlier} yang telah dihilangkan.

\textbf{Analisa Nilai Rata-Rata Berdarsarkan Grup Data}

Pada contoh sebelumnya kita telah melakukan perhitungan nilai mean untuk
studi kasus grup tunggal. Pada contoh ini akan disajikan contoh kasus
perhitungan nilai mean untuk data berkelompok.

Dataset pada contoh kasus ini diambil dari buku \textbf{Statistical
Methods in Water Resources}. Data yang digunakan adalah data konsentrasi
TDS dan Uranium di airtanah dengan perbedaan konsentrasi bikarbonate
dalam air tanah yaitu \(\leq 50\)\% (0) dan \(>50\)\% (1). Dataset yang
digunakan disajikan pada Tabel \ref{tab:gwtdsur}.

\begin{quote}
\textbf{Note: } data yang digunakan dapat diunduh pada link berikut
\href{https://drive.google.com/open?id=1-k_1Fkl2hWmI9ZohG9hWGrqKe4o8GArW}{google.drive}.
Simpan dataset tersebut pada \emph{working directory} pembaca agar mudah
dalam proses membaca data.
\end{quote}

\begin{Shaded}
\begin{Highlighting}[]
\CommentTok{# memuat library}
\KeywordTok{library}\NormalTok{(readxl)}

\CommentTok{# memuat data excel}
\NormalTok{data_gw <-}\StringTok{ }\KeywordTok{read_excel}\NormalTok{(}\StringTok{"hhappc.xls"}\NormalTok{, }\DataTypeTok{sheet=}\StringTok{"appc16"}\NormalTok{)}

\CommentTok{# membuang kolom ke-4}
\NormalTok{data_gw<-data_gw }\OperatorTok
\StringTok{  }\KeywordTok{select}\NormalTok{(TDS, Uranium, Bicarbonate) }\OperatorTok
\StringTok{  }\KeywordTok{mutate}\NormalTok{(}\DataTypeTok{Bicarbonate=}\KeywordTok{as.factor}\NormalTok{(Bicarbonate))}
\end{Highlighting}
\end{Shaded}

\begin{table}[t]

\caption{\label{tab:gwtdsur}Kosentrasi TDS dan Uranium dalam berbagai kondisi kesadahan}
\centering
\begin{tabular}{r|r|l}
\hline
TDS & Uranium & Bicarbonate\\
\hline
682.6 & 0.9315 & 0\\
\hline
819.1 & 1.9380 & 0\\
\hline
303.8 & 0.2919 & 0\\
\hline
1151.4 & 11.9042 & 0\\
\hline
582.4 & 1.5674 & 0\\
\hline
1043.4 & 2.0623 & 0\\
\hline
634.8 & 3.8858 & 0\\
\hline
1087.2 & 0.9772 & 0\\
\hline
1123.5 & 1.9354 & 0\\
\hline
688.1 & 0.4367 & 0\\
\hline
1174.5 & 10.1142 & 0\\
\hline
599.5 & 0.7551 & 0\\
\hline
1240.8 & 6.8559 & 0\\
\hline
538.4 & 0.4806 & 0\\
\hline
607.8 & 1.1452 & 0\\
\hline
705.9 & 6.0876 & 0\\
\hline
1290.6 & 10.8823 & 0\\
\hline
526.1 & 0.1473 & 0\\
\hline
784.7 & 2.6741 & 0\\
\hline
953.1 & 3.0918 & 0\\
\hline
1149.3 & 0.7592 & 0\\
\hline
1074.2 & 3.7101 & 0\\
\hline
1116.6 & 7.2446 & 0\\
\hline
301.2 & 5.7129 & 1\\
\hline
265.4 & 4.7366 & 1\\
\hline
295.9 & 2.8057 & 1\\
\hline
442.4 & 5.6290 & 1\\
\hline
342.7 & 3.0950 & 1\\
\hline
361.3 & 3.5774 & 1\\
\hline
262.1 & 1.7711 & 1\\
\hline
546.2 & 11.2724 & 1\\
\hline
273.9 & 4.9807 & 1\\
\hline
281.4 & 4.0833 & 1\\
\hline
588.9 & 14.6342 & 1\\
\hline
574.1 & 12.3835 & 1\\
\hline
307.1 & 1.5291 & 1\\
\hline
409.4 & 4.4647 & 1\\
\hline
327.1 & 2.4574 & 1\\
\hline
425.7 & 6.3042 & 1\\
\hline
310.1 & 4.5441 & 1\\
\hline
289.8 & 0.9672 & 1\\
\hline
408.2 & 2.1568 & 1\\
\hline
383.0 & 8.3810 & 1\\
\hline
255.2 & 2.7957 & 1\\
\hline
\end{tabular}
\end{table}

Visualisasi data Tabel \ref{tab:gwtdsur}, disajikan pada Gambar
\ref{fig:gwvis1} dan Gambar \ref{fig:gwvis2}:

\begin{figure}

{\centering \includegraphics[width=0.7\linewidth]{EnvStat_files/figure-latex/gwvis1-1} 

}

\caption{Visualisasi konsentrasi TDS pada air tanah}\label{fig:gwvis1}
\end{figure}

\begin{figure}

{\centering \includegraphics[width=0.7\linewidth]{EnvStat_files/figure-latex/gwvis2-1} 

}

\caption{Visualisasi konsentrasi Uranium pada air tanah}\label{fig:gwvis2}
\end{figure}

Pada dataset tersebut kita ingin melihat apakah terdapat perbedaan
antara konsentrasi TDS dan uranium pada kondisi kesadahan bikarbonat
\(\leq 50\)\% dan \(> 50\)\%. Untuk melakukannya pada \texttt{R} kita
perlu mengelompokkan data tersebut terlebih dahulu berdasarkan variabel
bikarbonat. Setelah itu nilai rata-rata dapat dihitung. Berikut sintaks
yang digunakan:

\begin{Shaded}
\begin{Highlighting}[]
\NormalTok{data_gw }\OperatorTok
\StringTok{  }\KeywordTok{group_by}\NormalTok{(Bicarbonate) }\OperatorTok
\StringTok{  }\KeywordTok{summarize}\NormalTok{(}\DataTypeTok{TDS =} \KeywordTok{mean}\NormalTok{(TDS), }\DataTypeTok{Uranium =} \KeywordTok{mean}\NormalTok{(Uranium))}
\end{Highlighting}
\end{Shaded}

\begin{verbatim}
## # A tibble: 2 x 3
##   Bicarbonate   TDS Uranium
##   <fct>       <dbl>   <dbl>
## 1 0            864.    3.47
## 2 1            364.    5.16
\end{verbatim}

Berdasarkan hasil yang diperoleh konsentrasi TDS dan Uranium dipengaruhi
oleh kesadahan airtanah. Pada konsentrasi Bikarbonate \textgreater{}
50\% konsentrasi TDS akan lebih rendah sedangkan konsentrasi Uranium
sebaliknya. Untuk menguji apakah nilai tersebut berbeda signifikan, kita
perlu melakukan uji hipotesis yang akan dibahas pada Chapter
selanjutnya.

\subsection{Median Sebagai Ukuran Pemusatan Data yang
Resistan}\label{median-sebagai-ukuran-pemusatan-data-yang-resistan}

Median atau persentil 50 (\(P_{50}\)) merupakan nilai pusat dari
distribusi suatu data yang telah dirangkin berdasarkan besar nilai
orbservasinya. Untuk data dengan jumlah observasi ganjil median adalah
titik tengah yang memiliki jumlah observasi yang sama baik di atas nilai
media maupun di bawahnya. Untuk data dengan jumlah observasi genap,
media merupakan rata-rata dari dua titik observasi pusat. Untuk
memperoleh median dari suatu distribusi data, langkah pertama yang perlu
dilakukan adalah mengurutkan data dari observasi dengan nilai terkecil
sampai dengan yang besar sehingga \(x_1\) merupakan observasi terkecil
hingga \(x_n\) merupakan observasi terbesar. Persamaan \eqref{eq:med1}
(untuk data ganjil) dan Persamaan \eqref{eq:med2} (untuk data genap)
merupakan persamaan untuk menghitung median berdasarkan jumlah observasi
yang ada.

\begin{equation}
  Median (P_{0.5}) =\frac{X_{\left(n+1\right)}}{2}
  \label{eq:med1}
\end{equation}

\begin{equation}
  Median (P_{0.5}) =\frac{1}{2}\cdot\left(X_{\left(\frac{n}{2}\right)}+X_{\left(\frac{n}{2}\right)+1}\right)
  \label{eq:med2}
\end{equation}

Median hanya dipengaruhi minimal oleh besarnya nilai observasi tunggal,
yang ditentukan semata-mata oleh urutan relatif observasi. Resitensi
terhadap efek dari perubahan nilai atau kehadiran pengamatan terpencil
(\emph{outlier}) sering merupakan sifat yang diinginkan. Meski demikian
median memiliki kelemahan utama yaitu kurang representatif dalam
mendeskripsikan rata-rata dari data dibandingkan mean. Hal ini
disebabkan karena median tidak menggunakan seluruh nilai yang ada pada
data.

\textbf{Analisa Nilai Median Grup Data Tunggal (\emph{Single Group})}

Kita akan menggunakan kembali data pada Tabel \ref{tab:debitsungai}
untuk menghitung median data tersebut. Pada \texttt{R} median dihitung
menggunakan fungsi \texttt{median()}. Fotmat yang digunakan adalah
sebagai berikut:

\begin{Shaded}
\begin{Highlighting}[]
\KeywordTok{median}\NormalTok{(x, }\DataTypeTok{na.rm =} \OtherTok{FALSE}\NormalTok{)}
\end{Highlighting}
\end{Shaded}

\begin{quote}
\textbf{Note:}

\begin{itemize}
\tightlist
\item
  \textbf{x}: objek atau vektor numerik.
\item
  \textbf{na.rm}: nilai logis yang menyatakan apakah \emph{missing
  value} perlu disertakan dalam komputasi atau tidak.
\end{itemize}
\end{quote}

Untuk data pada Tabel \ref{tab:debitsungai}, median dapat dihitung
menggunakan sintaks berikut:

\begin{Shaded}
\begin{Highlighting}[]
\KeywordTok{median}\NormalTok{(sungai}\OperatorTok{$}\NormalTok{debit)}
\end{Highlighting}
\end{Shaded}

\begin{verbatim}
## [1] 117
\end{verbatim}

Berdasarkan hasil komputasi diperoleh median debit sungai sebesar 117
\(m^3/detik\). Nilai tersebut tidak berbeda juah dengan nilai mean tanpa
\emph{outlier} data sungai sebesar 107.4545 \(m^3/detik\).

Jika kita melakukan perhitungan menggunakan menggunakan data
\texttt{sungai\_subset} (tanpa \emph{outlier}), maka diperoleh 115
\(m^3/detik\) yang nilainya juga tidak bergeser jauh dengan median
sebelumnya yang membuktikan bahwa median resisten terhadap
\emph{outlier}.

\textbf{Analisa Nilai Median Berdarsarkan Grup Data}

Paca contoh ini kita akan menggunakan kembali data pada Tabel
\ref{tab:gwtdsur}. Sintaks berikut adalah cara menghitung median untuk
data berkelompok:

\begin{Shaded}
\begin{Highlighting}[]
\NormalTok{data_gw }\OperatorTok
\StringTok{  }\KeywordTok{group_by}\NormalTok{(Bicarbonate) }\OperatorTok
\StringTok{  }\KeywordTok{summarize}\NormalTok{(}\DataTypeTok{TDS=}\KeywordTok{median}\NormalTok{(TDS), }\DataTypeTok{Uranium=}\KeywordTok{median}\NormalTok{(Uranium))}
\end{Highlighting}
\end{Shaded}

\begin{verbatim}
## # A tibble: 2 x 3
##   Bicarbonate   TDS Uranium
##   <fct>       <dbl>   <dbl>
## 1 0            819.    1.94
## 2 1            327.    4.46
\end{verbatim}

Pada median TDS kita tidak menemui perbedaan dengan nilai rata-ratanya.
Hal ini disebabkan karena bentuk distribusinya yang relatif simetris.
Sedangkan pada Uranium distribusi yang terbentuk memiliki kemencengan
(\emph{skewness}) positif. Hal ini menyebabkan nilai mean yang terbentuk
akan sangat dipengaruhi oleh observasi dengan nilai ekstrim yang
dimiliki.

\subsection{Ukuran Pemusatan Data
Lainnya}\label{ukuran-pemusatan-data-lainnya}

Ukuran pemusatan data lainnya yang kurang sering digunakan adalah modus,
rata-rata geometrik (\emph{geometric mean}), dan \emph{trimmed mean}.
Modus merupakan nilai observasi yang sering muncul. Jika kita
visualisasikan menggunakan histogram maka modus merupakan bar tertinggi
pada histogram. Modus lebih dapat diaplikasikan pada data berkelompok
yang nilai observasinya merupakan integer (\emph{finite number})
dibanding data dengan nilai kontinyu. Modus sangat mudah diperoleh,
namun sangat buruk sebagai ukuran pemusatan data untuk jenis data
kontinyu karena sering bergantung pengelompokan data yang
sewenang-wenang atu semaunya.

\emph{Geometric mean} sering digunakan untuk distribusi data memiliki
bentuk kemencengan positif. \emph{Geometric mean} merupakan rata-rata
logaritmik yang diubah kembali ke unit asalnya. Untuk menghitungnya
digunakan Persamaan \eqref{eq:geomean}.

\begin{equation}
  GM = \exp\left(\overline{Y}\right)
  \label{eq:geomean}
\end{equation}

dimana

\begin{equation}
  Y_i = \ln\left(X_i\right)
  \label{eq:geomean2}
\end{equation}

Untuk data yang memiliki kemencengan positif, \emph{geometric mean}
biasanya cukup dekat dengan median. Bahkan, ketika logaritma data
simetris, \emph{geometric mean} adalah estimasi median. Ini karena
median dan \emph{geometric mean} sama. Ketika ditransformasikan kembali
ke satuan asli, rerata geometris terus menjadi estimasi untuk median,
tetapi bukan merupakan estimasi untuk rerata.

Pada \texttt{R} \emph{geometric mean} dapat kita hitung menggunakan
sintaks fungsi yang kita buat sendiri:

\begin{Shaded}
\begin{Highlighting}[]
\NormalTok{geomean <-}\StringTok{ }\ControlFlowTok{function}\NormalTok{(x)\{}
\NormalTok{  y =}\StringTok{ }\KeywordTok{log}\NormalTok{(x)}
\NormalTok{  GM =}\StringTok{ }\KeywordTok{exp}\NormalTok{(}\KeywordTok{mean}\NormalTok{(y))}
  \KeywordTok{return}\NormalTok{(GM)}
\NormalTok{\}}
\end{Highlighting}
\end{Shaded}

Data pada Tabel \ref{tab:debitsungai} merupakan data dengan kemencengan
positif. Nilai \emph{geometric mean} data tersebut dihitung menggunakan
sintaks berikut:

\begin{Shaded}
\begin{Highlighting}[]
\KeywordTok{geomean}\NormalTok{(sungai}\OperatorTok{$}\NormalTok{debit)}
\end{Highlighting}
\end{Shaded}

\begin{verbatim}
## [1] 112.4
\end{verbatim}

Berdasarkan hasil komputasi diperoleh nilai \emph{geometric mean} debit
sungai sebesar 112.4315 \(m^3/detik\). Nilai yang diperoleh tidak
berbeda dengan nilai median sebesar 117 \(m^3/detik\).

Kompromi antara median dan mean tersedia dengan memotong beberapa
observasi terendah dan tertinggi, dan menghitung mean dari apa yang
tersisa. Perkiraan pemusatan data seperti itu tidak dipengaruhi oleh
observasi yang paling ekstrem (dan mungkin anomali), seperti mean. Namun
mereka memungkinkan besarnya sebagian besar nilai untuk mempengaruhi
estimasi, tidak seperti median. Estimator ini disebut ``\emph{trimmed
mean}'', dan persentase data yang diinginkan dapat dipangkas.
Pemangkasan yang paling umum adalah menghapus 25 persen dari data di
setiap ujung - rata-rata yang dihasilkan dari 50 persen pusat data
biasanya disebut ``\emph{trimmed mean}'', tetapi lebih tepatnya 25
persen \emph{trimmed mean}. ``\emph{trimmed mean} 0\%'' adalah mean
sampel itu sendiri, sementara memangkas semua kecuali 1 atau 2 nilai
pusat menghasilkan median. Persentase pemangkasan harus secara eksplisit
dinyatakan saat digunakan. \emph{Trimmed mean} adalah estimator yang
resistan, karena tidak sangat dipengaruhi oleh \emph{outlier}, dan
bekerja dengan baik untuk berbagai macam bentuk distribusi (normal,
lognormal, dll). Ini dapat dianggap sebagai rata-rata tertimbang
(\emph{weighted mean}), di mana data di luar `jendela' cutoff diberi
bobot 0, dan mereka yang berada di dalam jendela bobot 1,0 (lihat Gambar
\ref{fig:tm}).

\begin{figure}

{\centering \includegraphics[width=0.7\linewidth]{tm} 

}

\caption{Jendela diagram trimmed mean.}\label{fig:tm}
\end{figure}

Pada \texttt{R} \emph{trimmed mean} dapat dihitung dengan spesifikasi
argumen \texttt{trim} pada fungsi \texttt{mean()}. Pada data debit
sungai (Tabel \ref{tab:debitsungai}) dihitung \emph{trimmed mean} dengan
data yang dipangkas adalah 5\% di kedua ujung observasi atau
\texttt{trim=0.1}.

\begin{Shaded}
\begin{Highlighting}[]
\KeywordTok{mean}\NormalTok{(sungai}\OperatorTok{$}\NormalTok{debit, }\DataTypeTok{trim=}\FloatTok{0.1}\NormalTok{)}
\end{Highlighting}
\end{Shaded}

\begin{verbatim}
## [1] 113.2
\end{verbatim}

Nilai yang diperoleh sekarang mendekati nilai median dan \emph{geometric
mean} yaitu sebesar 113.2 \(m^3/detik\).

\section{Ukuran Sebaran Data}\label{ukuran-sebaran-data}

Saat kita mengetahui kedalaman rata-rata sungai, kita pasti ingin
mengetahui berapa interval atau variasi dari kedalamannya. Kita tidak
cukup hanya dengan mengetahui nilai pemusatan datanya saja, kita juga
perlu mengetahui seberapa besar variasi atau variabilitas datanya.

Variabilitas suatu data diukur dengan melihat sebaran data dari nilai
rata-ratanya (mean). Semakin besar sebaran suatu data, semakin tidak
berarti nilai rata-ratanya karena nilai rata-ratanya bisa sangat berbeda
dari sejumlah nilai pada datanya.

\subsection{Pengukuran Klasik (Varian dan Simpangan
Baku)}\label{pengukuran-klasik-varian-dan-simpangan-baku}

Varian sampel dan nilai akar dari varian sampel (Simpangan Baku)
merupakan ukuran penyebaran data klasik. Sama dengan mean varian dan
simpangan baku dipengaruhi oleh \emph{outlier}. Semakin besar nilai
keduanya, semakin besar variabilitas datanya. Kedua ukuran tersebut
dinyatakan pada Persamaan \eqref{eq:var} dan Persamaan \eqref{eq:sd}.

\textbf{Varian Sampel}

\begin{equation}
  s^2=\sum_{i=1}^n\frac{\left(X_i-\overline{X}\right)^2}{\left(n-1\right)}
  \label{eq:var}
\end{equation}

\textbf{simpangan baku}

\begin{equation}
  s=\sqrt{s^2}
  \label{eq:sd}
\end{equation}

Kedua nilai tersebut di hitung berdasarkan kuadrat deviasi nilai
observasi dari rata-ratanya, sehingga jika pada data terdapat
\emph{outlier} maka nilai outlier akan memperbesar deviasi data dari
nilai mean. Ketika \emph{outlier} hadir, pengukuran menjadi tidak
stabil. Hal ini akan memberi kesan sebaran data menjadi jauh lebih besar
daripada yang ditunjukkan oleh mayoritas nilai pada data.

Varian dan simpangan baku pada \texttt{R} dihitung menggunakan fungsi
\texttt{var()} (varian) dan \texttt{sd()}. Format yang digunakan adalah
sebagai berikut:

\begin{Shaded}
\begin{Highlighting}[]
\KeywordTok{var}\NormalTok{(x, }\DataTypeTok{na.rm =} \OtherTok{FALSE}\NormalTok{)}
\KeywordTok{sd}\NormalTok{(x, }\DataTypeTok{na.rm =} \OtherTok{FALSE}\NormalTok{)}
\end{Highlighting}
\end{Shaded}

\begin{quote}
\textbf{Note:}

\begin{itemize}
\tightlist
\item
  \textbf{x}: objek atau vektor numerik.
\item
  \textbf{na.rm}: nilai logis yang menyatakan apakah \emph{missing
  value} perlu disertakan dalam komputasi atau tidak.
\end{itemize}
\end{quote}

\textbf{Analisa Varian dan simpangan baku Grup Tunggal}

Kita akan menggunakan kembali data pada Tabel \ref{tab:debitsungai}
untuk menghitung varian dan simpangan baku data tersebut. Berikut adalah
sintaks untuk melakukannya:

\begin{Shaded}
\begin{Highlighting}[]
\CommentTok{# varian data sungai}
\KeywordTok{var}\NormalTok{(sungai}\OperatorTok{$}\NormalTok{debit)}
\end{Highlighting}
\end{Shaded}

\begin{verbatim}
## [1] 11926
\end{verbatim}

\begin{Shaded}
\begin{Highlighting}[]
\CommentTok{# simpangan baku data sungai}
\KeywordTok{sd}\NormalTok{(sungai}\OperatorTok{$}\NormalTok{debit)}
\end{Highlighting}
\end{Shaded}

\begin{verbatim}
## [1] 109.2
\end{verbatim}

Sekarang mari kita bandingkan dengan data yang tidak menyertakan
outlier.

\begin{Shaded}
\begin{Highlighting}[]
\CommentTok{# varian data sungai}
\KeywordTok{var}\NormalTok{(sungai_subset}\OperatorTok{$}\NormalTok{debit)}
\end{Highlighting}
\end{Shaded}

\begin{verbatim}
## [1] 1919
\end{verbatim}

\begin{Shaded}
\begin{Highlighting}[]
\CommentTok{# simpangan baku data sungai}
\KeywordTok{sd}\NormalTok{(sungai_subset}\OperatorTok{$}\NormalTok{debit)}
\end{Highlighting}
\end{Shaded}

\begin{verbatim}
## [1] 43.8
\end{verbatim}

Berdasarkan hasil yang diperoleh terlihat bahwa nilai varian dan
simpangan baku data dengan \emph{outlier} jauh lebih besar dibanding
data tanpa \emph{outlier}.

\textbf{Analisa Varian dan simpangan baku Multi Grup}

Paca contoh ini kita akan menggunakan kembali data pada Tabel
\ref{tab:gwtdsur}. Sintaks berikut adalah cara menghitung varian dan
simpangan baku untuk data berkelompok:

\begin{Shaded}
\begin{Highlighting}[]
\NormalTok{data_gw }\OperatorTok
\StringTok{  }\KeywordTok{group_by}\NormalTok{(Bicarbonate) }\OperatorTok
\StringTok{  }\KeywordTok{summarize}\NormalTok{(}\DataTypeTok{var_TDS=}\KeywordTok{var}\NormalTok{(TDS), }\DataTypeTok{var_Uranium=}\KeywordTok{var}\NormalTok{(Uranium),}
            \DataTypeTok{sd_TDS=}\KeywordTok{sd}\NormalTok{(TDS), }\DataTypeTok{sd_Uranium=}\KeywordTok{sd}\NormalTok{(Uranium))}
\end{Highlighting}
\end{Shaded}

\begin{verbatim}
## # A tibble: 2 x 5
##   Bicarbonate var_TDS var_Uranium sd_TDS sd_Uranium
##   <fct>         <dbl>       <dbl>  <dbl>      <dbl>
## 1 0            79471.        13.0   282.       3.61
## 2 1            10559.        13.5   103.       3.68
\end{verbatim}

Jika kita perhatikan nilai varian dan simpangan baku Uranium pada dua
kondisi kesadahan memiliki nilai yang nyaris sama. Hal sebaliknya
terjadi pada variabel TDS yang menunjukkan perbedaan pada dua ukuran
sebaran datanya. TDS pada kesadahan \textgreater{}50\% memiliki varian
dan simpangan baku yang lebih kecil dibanding kondisi kesadahan satunya,
yang menunjukkan data pada kondisi kesadahan \textgreater{}50\% lebih
tidak tersebar dibanding kesadahan satunya.

\subsection{Ukuran Sebaran Data yang Resisten Terhadap
Outlier}\label{ukuran-sebaran-data-yang-resisten-terhadap-outlier}

Simpangan kuartil atau \emph{interquartile range} (IQR) merupakan ukuran
sebaran data yang resisten dan paling sering digunakan. IQR mengukur
kisaran 50\% pusat data sehingga pengukuran tidak dipengaruhi oleh
adanya outlier pada 25\% pada data pada setiap ujungnya. Untuk
visualisasinya kita dapat melihat kembali pada ambar \ref{fig:tm}.

IQR didefinisikan sebagai persentil ke-75 dikurangi dengan persentil
ke-25. Persentil ke-75, ke-50 (median) dan ke-25 membagi data menjadi
empat tempat berukuran sama. Persentil ke-75 (\(P_{.75}\)), juga disebut
kuartil atas, adalah nilai yang melebihi tidak lebih dari 75\% data dan
dilampaui oleh tidak lebih dari 25 persen data. Persentil ke-25
(\(P_{.25}\)) atau kuartil lebih rendah adalah nilai yang melebihi tidak
lebih dari 25\% dari data dan dilampaui oleh tidak lebih dari 75\%.
Dengan mempertimbangkan data yang telah diurutkan dari yang terkecil ke
yang terbesar: \(X_{i}\), \(i=1,...n\). Persentil (\(P_j\)) dihitung
berdasarkan Persamaan \eqref{eq:iqr}.

\begin{equation}
  P_j=X_{\left(n+1\right)\cdot j}
  \label{eq:iqr}
\end{equation}

dimana \(n\) merupakan ukuran sampel \(X_j\), dan \(j\) merupakan fraksi
data yang kurang dari atau sama dengan nilai persentil (untuk persentil
ke-25, 50, dan 75, \(j=.25, .50., dan .75\)).

Pada \texttt{R}, IQR dapat dihitung secara langsung menggunakan fungsi
\texttt{IQR()} atau secara tidak langsung menggunakan fungsi
\texttt{quantile()}. Penggunaan fungsi \texttt{quantile()} digunakan
untuk mencari persentil dari data. Telah dijelaskan sebelumnya bahwa IQR
merupakan selisih dari persentil 75 dan persentil 25. Format yang
digunakan untuk menghitung IQR adalah sebagai berikut:

\begin{Shaded}
\begin{Highlighting}[]
\CommentTok{# secara langsung}
\KeywordTok{IQR}\NormalTok{(x, }\DataTypeTok{na.rm=}\OtherTok{FALSE}\NormalTok{)}

\CommentTok{# secara tidak langsung}
\KeywordTok{quantile}\NormalTok{(x, }\DecValTok{3}\OperatorTok{/}\DecValTok{4}\NormalTok{)}\OperatorTok{-}\KeywordTok{quantile}\NormalTok{(x, }\DecValTok{1}\OperatorTok{/}\DecValTok{4}\NormalTok{)}

\CommentTok{# atau}
\KeywordTok{quantile}\NormalTok{(x, .}\DecValTok{75}\NormalTok{)}\OperatorTok{-}\KeywordTok{quantile}\NormalTok{(x, .}\DecValTok{25}\NormalTok{)}
\end{Highlighting}
\end{Shaded}

\begin{quote}
\textbf{Note:}

\begin{itemize}
\tightlist
\item
  \textbf{x}: objek atau vektor numerik.
\item
  \textbf{na.rm}: nilai logis yang menyatakan apakah \emph{missing
  value} perlu disertakan dalam komputasi atau tidak.
\end{itemize}
\end{quote}

Pada Tabel \ref{tab:debitsungai}, kita dapat menghitung IQR dari data.
Berikut adalah contoh sintaks yang digunakan:

\begin{Shaded}
\begin{Highlighting}[]
\KeywordTok{IQR}\NormalTok{(sungai}\OperatorTok{$}\NormalTok{debit)}
\end{Highlighting}
\end{Shaded}

\begin{verbatim}
## [1] 72.5
\end{verbatim}

Salah satu penaksir penyebaran yang resisten selain IQR adalah
\emph{Median Absolute Deviation}, atau MAD. MAD dihitung dengan
pertama-tama mendaftar nilai absolut dari semua selisih \(|d|\) antara
masing-masing pengamatan dan median. Median dari nilai absolut ini
adalah MAD yang ditulis berdasarkan Persamaan \eqref{eq:mad}.

\begin{equation}
  MAD\ \left(X_i\right)=median\left|d\right|
  \label{eq:mad}
\end{equation}

dimana

\begin{equation}
  d_i=X_i-median\left(X_i\right)
  \label{eq:mad2}
\end{equation}

Pada \texttt{R}, MAD tidak dapat dihitung secara langsung. Kita perlu
membuat \emph{user defined function} untuk dapat digunakan
sewaktu-waktu. Berikut adalah fungsi yang dibuat:

\begin{Shaded}
\begin{Highlighting}[]
\NormalTok{MAD <-}\StringTok{ }\ControlFlowTok{function}\NormalTok{(x)\{}
  \CommentTok{# median data}
\NormalTok{  m =}\StringTok{ }\KeywordTok{median}\NormalTok{(x)}
  \CommentTok{# MAD}
\NormalTok{  d =}\StringTok{ }\KeywordTok{abs}\NormalTok{(x}\OperatorTok{-}\NormalTok{m)}
\NormalTok{  mad =}\StringTok{ }\KeywordTok{mean}\NormalTok{(d)}
  \CommentTok{# print}
  \KeywordTok{return}\NormalTok{ (mad)}
\NormalTok{\}}
\end{Highlighting}
\end{Shaded}

Pada Tabel \ref{tab:debitsungai}, kita dapat menghitung MAD dari data
menggunakan fungsi yang telah dibuat. Berikut adalah contoh sintaks yang
digunakan:

\begin{Shaded}
\begin{Highlighting}[]
\KeywordTok{MAD}\NormalTok{(sungai}\OperatorTok{$}\NormalTok{debit)}
\end{Highlighting}
\end{Shaded}

\begin{verbatim}
## [1] 60.42
\end{verbatim}

\section{Ringkasan Data Menggunakan Fungsi summary() dan
stat.desc()}\label{ringkasan-data-menggunakan-fungsi-summary-dan-stat.desc}

Ringkasan data menggunakan fungsi \texttt{summary()} akan memberikan
ringkasan data seperti nilai mean, kuartil, nilai minimum dan maksimum,
serta \emph{missing value}. Jika data berupa variabel tunggal maka
output yang dihasilkan berupa nilai-nilai yang telah penulis sebutkan
sebelumnya. Berikut adalah contoh sintaks yang digunakan:

\begin{Shaded}
\begin{Highlighting}[]
\KeywordTok{summary}\NormalTok{(sungai}\OperatorTok{$}\NormalTok{debit)}
\end{Highlighting}
\end{Shaded}

\begin{verbatim}
##    Min. 1st Qu.  Median    Mean 3rd Qu.    Max. 
##    50.0    67.2   117.0   136.6   139.8   457.0
\end{verbatim}

Jika objek yang diinputkan kedalam fungsi tersebut adalah data frame,
maka ringkasan data akan diberikan pada setiap kolom dengan ketentuan
berikut:

\begin{itemize}
\tightlist
\item
  jika kolom berupa variabel numerik maka output yang diperoleh berupa
  mean, median, min, max dan kuartil.
\item
  jika kolom berupa factor maka output yang dihasilkan berupa rekapan
  jumlah observasi pada masing-masing grup.
\end{itemize}

Berikut adalah contoh sintaks penerapannya:

\begin{Shaded}
\begin{Highlighting}[]
\KeywordTok{summary}\NormalTok{(data_gw)}
\end{Highlighting}
\end{Shaded}

\begin{verbatim}
##       TDS          Uranium       Bicarbonate
##  Min.   : 255   Min.   : 0.147   0:23       
##  1st Qu.: 323   1st Qu.: 1.558   1:21       
##  Median : 560   Median : 3.093              
##  Mean   : 626   Mean   : 4.276              
##  3rd Qu.: 853   3rd Qu.: 5.807              
##  Max.   :1291   Max.   :14.634
\end{verbatim}

Ringkasan data lain dapat dilakukan dengan menggunakan fungsi
\texttt{stat.desc()} dari library \texttt{pastecs}. Kelebihan dari
ringkasan data menggunakan fungsi ini adalah kita tidak hanya memperoleh
ringkasan data dengan ouput seperti diatas, namun kita juga memperoleh
output berupa nilai \emph{standadr error} (SE), \emph{confidence
interval} (CI), dan koefisien variasi (coef.var) yang merupakan hasil
bagi dari simpangan baku dibagi dengan nilai rata-rata.

Berikut adalah sintak yang digunakan untuk menghasilkan ringkasan data
menggunakan fungsi \texttt{stat.desc()}:

\begin{Shaded}
\begin{Highlighting}[]
\CommentTok{# memasang paket}
\KeywordTok{install.packages}\NormalTok{(}\StringTok{"pastecs"}\NormalTok{)}
\end{Highlighting}
\end{Shaded}

\begin{Shaded}
\begin{Highlighting}[]
\CommentTok{# memuat paket}
\KeywordTok{library}\NormalTok{(pastecs)}

\CommentTok{# ringkasan data}
\KeywordTok{stat.desc}\NormalTok{(data_gw)}
\end{Highlighting}
\end{Shaded}

\begin{verbatim}
##                    TDS  Uranium Bicarbonate
## nbr.val      4.400e+01  44.0000          NA
## nbr.null     0.000e+00   0.0000          NA
## nbr.na       0.000e+00   0.0000          NA
## min          2.552e+02   0.1473          NA
## max          1.291e+03  14.6342          NA
## range        1.035e+03  14.4869          NA
## sum          2.753e+04 188.1604          NA
## median       5.602e+02   3.0934          NA
## mean         6.257e+02   4.2764          NA
## SE.mean      4.986e+01   0.5572          NA
## CI.mean.0.95 1.005e+02   1.1238          NA
## var          1.094e+05  13.6623          NA
## std.dev      3.307e+02   3.6963          NA
## coef.var     5.286e-01   0.8643          NA
\end{verbatim}

\section{Ukuran Kemencengan Data}\label{ukuran-kemencengan-data}

Ketika data memiliki kemencengan, nilai mean tidak sama dengan median,
tetapi bergeser ke arah ekor distribusi. Jadi untuk kemencengan positif,
nilai mean melebihi lebih dari 50\% dari data, seperti pada Gambar
\ref{fig:skew} dan Gambar \ref{fig:skew2}. Simpangan baku juga meningkat
dengan data di bagian ekor. Data yang menceng juga mempertanyakan
penerapan tes hipotesis yang didasarkan pada asumsi bahwa data memiliki
distribusi normal. Tes-tes ini, yang disebut tes parametrik, mungkin
bernilai dipertanyakan ketika diterapkan pada data seperti data sumber
daya air, karena data seringkali tidak normal atau bahkan simetris.

\begin{figure}

{\centering \includegraphics[width=0.7\linewidth]{skewness} 

}

\caption{a) Kemencengan negatif, b) Kemencengan positif.}\label{fig:skew}
\end{figure}

\begin{figure}

{\centering \includegraphics[width=0.7\linewidth]{skewnessbox} 

}

\caption{Box plot untuk data dengan a) Kemencengan negatif, b) Kemencengan positif.}\label{fig:skew2}
\end{figure}

\subsection{Ukuran Kemencengan Klasik}\label{ukuran-kemencengan-klasik}

Koefisien kemencengan (\(g\)) merupakan ukuran kemencengan yang sering
digunakan. Koefisien kemencengan dituliskan pada Persamaan \eqref{eq:g}.

\begin{equation}
  g=\frac{n}{\left(n-1\right)\left(n-2\right)}\sum_{i=1}^n\frac{\left(x_i-\overline{X}\right)^3}{s^3}
  \label{eq:g}
\end{equation}

Kemencengan positif (ekor panjang kekanan) memiliki nilai \(g\) positif
sedangkan kemencengan negatif (ekor panjang kekiri) memiliki nilai \(g\)
negatif. Sekali lagi, Pengaruh beberapa \emph{outlier} adalah penting -
suatu distribusi simetris yang memiliki satu \emph{outlier} akan
menghasilkan ukuran kemencengan (\(g\)) yang besar (dan mungkin
menyesatkan).

Pada \texttt{R} Kita dapat menghitung sendiri koefisien kemencengan
(\(g\)) menggunakan \emph{user define function}. Berikut adalah contoh
sintaks fungsi yang dibuat:

\begin{Shaded}
\begin{Highlighting}[]
\NormalTok{skew <-}\StringTok{ }\ControlFlowTok{function}\NormalTok{(x)\{}
\NormalTok{  ave =}\StringTok{ }\KeywordTok{mean}\NormalTok{(x)}
\NormalTok{  n =}\StringTok{ }\KeywordTok{length}\NormalTok{(x)}
\NormalTok{  sd =}\StringTok{ }\KeywordTok{sd}\NormalTok{(x)}
\NormalTok{  g=(n}\OperatorTok{/}\NormalTok{((n}\OperatorTok{-}\DecValTok{1}\NormalTok{)}\OperatorTok{*}\NormalTok{(n}\OperatorTok{-}\DecValTok{2}\NormalTok{)))}\OperatorTok{*}\KeywordTok{sum}\NormalTok{(((x}\OperatorTok{-}\NormalTok{ave)}\OperatorTok{^}\DecValTok{3}\NormalTok{)}\OperatorTok{/}\NormalTok{(sd}\OperatorTok{^}\DecValTok{3}\NormalTok{))}
  \KeywordTok{return}\NormalTok{(g)}
\NormalTok{\}}
\end{Highlighting}
\end{Shaded}

Pada contoh sebelumnya dengan menggunakan fungsi yang telah dibuat
diperoleh koefisien kemencengan sebagai berikut:

\begin{Shaded}
\begin{Highlighting}[]
\KeywordTok{skew}\NormalTok{(data_gw}\OperatorTok{$}\NormalTok{Uranium)}
\end{Highlighting}
\end{Shaded}

\begin{verbatim}
## [1] 1.184
\end{verbatim}

\subsection{Ukuran Kemencengan yang
Resisten}\label{ukuran-kemencengan-yang-resisten}

Ukuran kemencengan yang lebih resisten adalah *quartile skew
coefficient8 (\(qs\)). Merupakan ukuran kemencengan didasarkan pada
ketiga nilai kuartil data seperti yang ditunjukkan pada Persamaan
\eqref{eq:qs} yang menyatakan perbedaan pada jarak kuartil atas dan bawah
terhadap median dibagi dengan IQR.

\begin{equation}
  qs=\frac{\left(P_{.75}-P_{.50}\right)-\left(P_{.75}-P_{.25}\right)}{P_{.75}-P_{.25}}
  \label{eq:qs}
\end{equation}

Kemencengan positif akan memiliki nilai \(qs\) positif dan begitupun
sebaliknya. Pada \texttt{R} kita dapat menghitung nilai \(qs\)
menggunakan \emph{user define function}. Berikut adalah contoh sintaks
fungsi yang dibuat:

\begin{Shaded}
\begin{Highlighting}[]
\NormalTok{qs <-}\StringTok{ }\ControlFlowTok{function}\NormalTok{(x)\{}
\NormalTok{  p75 =}\StringTok{ }\KeywordTok{quantile}\NormalTok{(x, }\DecValTok{3}\OperatorTok{/}\DecValTok{4}\NormalTok{)}
\NormalTok{  p50 =}\StringTok{ }\KeywordTok{median}\NormalTok{(x)}
\NormalTok{  p25 =}\StringTok{ }\KeywordTok{quantile}\NormalTok{(x, }\DecValTok{1}\OperatorTok{/}\DecValTok{4}\NormalTok{)}
\NormalTok{  skew =}\StringTok{ }\NormalTok{((p75}\OperatorTok{-}\NormalTok{p50)}\OperatorTok{-}\NormalTok{(p50}\OperatorTok{-}\NormalTok{p25))}\OperatorTok{/}\NormalTok{(p75}\OperatorTok{-}\NormalTok{p25)}
  \KeywordTok{return}\NormalTok{(skew)}
\NormalTok{\}}
\end{Highlighting}
\end{Shaded}

Pada contoh sebelumnya dengan menggunakan fungsi yang telah dibuat
diperoleh koefisien kemencengan sebagai berikut:

\begin{Shaded}
\begin{Highlighting}[]
\KeywordTok{qs}\NormalTok{(data_gw}\OperatorTok{$}\NormalTok{Uranium)}
\end{Highlighting}
\end{Shaded}

\begin{verbatim}
##    75% 
## 0.2772
\end{verbatim}

\section{Outlier}\label{outlier}

\emph{Outlier} merupakan pengamatan yang nilainya sangat berbeda dari
yang lain dalam kumpulan data, sering menimbulkan kekhawatiran atau
alarm. Meskipun sebenarnya kita tidak perlu khawatir dengan adanya
\emph{outlier} . \emph{Outlier} sering ditangani dengan membuangnya
sebelum mendeskripsikan data, atau sebelum beberapa prosedur uji
hipotesis chapter-chapter selanjutnya. Sekali lagi, mereka seharusnya
tidak perlu dikhawatirkan. \emph{Outlier} mungkin merupakan poin paling
penting dalam kumpulan data dan harus diselidiki lebih lanjut.

Untuk lebih memahami kenapa \emph{outlier} begitu penting pada data kita
berikut merupakan contoh kasus dari asal kata \emph{outlier}. Misalkan
bahwa data pada ``lubang'' ozon Antartika, suatu daerah dengan
konsentrasi ozon yang sangat rendah, telah dikumpulkan selama kurang
lebih 10 tahun sebelum penemuan aktualnya. Namun, rutinitas pengecekan
data otomatis selama pemrosesan data menyertakan instruksi untuk
menghapus ``\emph{outlier}''. Definisi \emph{outlier} didasarkan pada
konsentrasi ozon yang ditemukan pada pertengahan garis lintang. Dengan
demikian semua data yang tidak biasa ini tidak pernah dilihat atau
dipelajari selama beberapa waktu. Jika \emph{outlier} dihapus, risiko
diambil hanya dengan melihat apa yang diharapkan dilihat. Jika hal
tersebut dilakukan maka anomali yang terjadi pada atmosfer dapat luput
kita pelajari.

Berdasarkan kasus tersebut kita perlu dengan baik mempertimbangkan
apakah \emph{outlier} pada data perlu dihapus atau tidak. Jika berkaitan
dengan pembuatan model, penghapusan \emph{outlier} merupakan sesuatu
yang dapat memperbaiki akurasi dari model. Namun, pada sebuah penelitian
terkadang diperlukan informasi lebih lanjut mengapa terdapat
\emph{outlier} pada data sehingga kita dapat memperoleh pengetahuan baru
dari proses pencarian tersebut.

\emph{Outlier} dapat terjadi karena tiga hal, yaitu:

\begin{enumerate}
\def\labelenumi{\arabic{enumi}.}
\tightlist
\item
  Kesalahan pengukuran atau perekaman data.
\item
  Observasi dari populasi tidak sama dengan sebagian besar data seperti
  misalnya data debit banji akibat jebolnya sebuah bendungan akan
  berbeda dengan debit banjir akibat presipitasi.
\item
  Kejadian langka pada sebuah populasi yang sedikit memiliki kemencengan
  pada distribusinya.
\end{enumerate}

Metode grafis seperti box plot sangat membantu dalam mengidentifikasi
\emph{outlier}. Setiap kali \emph{outlier} terjadi, pertama-tama
verifikasi bahwa tidak ada penyalinan, titik desimal, atau kesalahan
nyata lainnya yang telah dibuat. Jika tidak, tidak mungkin untuk
menentukan apakah titik itu valid. Upaya yang dilakukan untuk
verifikasi, seperti menjalankan kembali sampel di laboratorium, akan
tergantung pada manfaat yang diperoleh versus biaya verifikasi. Kejadian
masa lalu mungkin tidak dapat diduplikasi. Jika tidak ada kesalahan yang
dapat dideteksi dan diperbaiki, ** \emph{outlier} tidak boleh dibuang
hanya berdasarkan fakta bahwa mereka tampak tidak biasa**.
\emph{Outlier} sering dibuang untuk membuat data cocok dengan distribusi
teoretis yang sudah terbentuk sebelumnya seperti distribusi normal.
Tidak ada alasan untuk menganggap bahwa mereka seharusnya dibuang!
Seluruh rangkaian data dapat muncul dari distribusi yang memiliki
kemencengan, dan mengambil logaritma atau transformasi lain dapat
menghasilkan data yang cukup simetris. Bahkan jika tidak ada
transformasi yang mencapai simetri, outlier tidak perlu dibuang.
Daripada menghilangkan data aktual (dan mungkin sangat penting) untuk
menggunakan prosedur analisis yang membutuhkan simetri atau normalitas,
prosedur yang tahan terhadap \emph{outlier} harus digunakan. Jika
menghitung rata-rata tampak bernilai kecil karena \emph{outlier}, median
telah terbukti menjadi ukuran lokasi yang lebih tepat untuk data yang
memiliki kemencengan. Jika melakukan uji-t (dijelaskan pada chapter
selanjutnya) tampaknya tidak valid karena set data yang tidak normal,
gunakan \emph{rank-sum test} sebagai gantinya.

Singkatnya, biarkan panduan data prosedur analisis yang digunakan,
daripada mengubah data untuk menggunakan beberapa prosedur yang memiliki
persyaratan terlalu ketat untuk situasi yang dihadapi.

\section{Transformasi Data}\label{transformasi-data-1}

Transformasi data dilakukan untuk memenuhi tiga tujuan, antara lain:

\begin{enumerate}
\def\labelenumi{\arabic{enumi}.}
\tightlist
\item
  membuat data lebih simetris,
\item
  membuat data lebih linier, dan
\item
  membuat data memiliki varian yang konsisten.
\end{enumerate}

Beberapa ilmuwan lingkungan takut bahwa dengan mentransformasikan data,
hasilnya diperoleh yang sesuai dengan gagasan yang telah terbentuk
sebelumnya. Oleh karena itu, transformasi adalah metode untuk
\textbf{melihat apa yang ingin kita lihat} dari data. Namun dalam
kenyataannya, masalah serius dapat terjadi ketika prosedur dengan asumsi
simetri, linieritas, atau homoseksualitas (varians konstan) digunakan
pada data yang tidak memiliki karakteristik yang diperlukan ini.
Transformasi dapat menghasilkan karakteristik ini, dan dengan demikian
penggunaan variabel yang diubah memenuhi tujuan.

Satu unit pengukuran tidak lebih valid secara apriori daripada yang
lainnya. Sebagai contoh, logaritma negatif konsentrasi ion hidrogen
(pH), sama validnya dengan sistem pengukuran dengan konsentrasi ion
hidrogen itu sendiri. Transformasi seperti akar kuadrat kedalaman air
pada sumur sumur, atau akar kubik volume curah hujan, seharusnya tidak
mengandung stigma lebih daripada pH. Skala pengukuran ini mungkin lebih
sesuai untuk analisis data daripada unit aslinya. Hoaglin (1988) telah
menulis artikel yang bagus tentang transformasi tersembunyi, secara
konsisten diterima begitu saja, yang umum digunakan oleh semua orang.
Oktaf dalam musik adalah transformasi frekuensi logaritmik. Setiap kali
piano dimainkan, transformasi logaritmik digunakan! Begitu pula dengan
skala Richter untukgempa bumi, mil per galon untuk konsumsi bensin,
f-stop untuk eksposur kamera, dll. semua menggunakan transformasi. Dalam
ilmu analisis data, keputusan yang menggunakan skala pengukuran harus
ditentukan oleh data, bukan dengan kriteria yang ditentukan sebelumnya.
Tujuan penggunaan transformasi adalah untuk kesimetrian, linieritas, dan
homoskedastisitas. Selain itu, penggunaan banyak teknik tahan seperti
persentil dan prosedur uji nonparametrik (akan dibahas kemudian) tidak
berbeda dengan skala pengukuran. Hasil \emph{rank-sum test}, setara
nonparametrik dari uji-t, akan persis sama apakah unit asli atau
logaritma dari unit tersebut digunakan.

Untuk membuat distribusi asimetris menjadi lebih simetris, data dapat
diubah atau diekspresikan kembali menjadi unit baru. Unit-unit baru ini
mengubah jarak antara pengamatan pada plot garis. Efeknya adalah
memperluas atau mengecilkan jarak ke pengamatan ekstrem di satu sisi
median, membuatnya lebih pada setiap sisinya. Transformasi yang paling
umum digunakan dalam bidang lingkungan adalah logaritma, seperti Log
debit air, konduktivitas hidrolik, atau konsentrasi sering diambil
sebelum analisis statistik dilakukan.

Transformasi data biasanya melibatkan fungsi power seperti pada fungsi
\(y=x^\theta\), dimana x merupakan data yang belum ditransformasi, y
adalah data yang telah ditransformasi, dan \(\theta\) merupakan power
eksponensial. Pada Gambar \ref{fig:power} nilai \(\theta\) di-list
kedalam ``\emph{ladder of powers}'' (Velleman dan Hoaglin, 1981 dalam
helsel dan Hirsch, 2002), sebuah struktur yang berguna untuk menentukan
nilai \(\theta\) yang tepat.

\begin{figure}

{\centering \includegraphics[width=0.7\linewidth]{ladder} 

}

\caption{Ladder of power}\label{fig:power}
\end{figure}

Seperti yang dapat dilihat dari \emph{ladder of powers}, setiap
transformasi dengan \(\theta\) kurang dari 1 dapat digunakan untuk
membuat data dengan kemencengan positif lebih simetris. Dengan membuat
box plot atau plot Q-Q dari data yang diubah kita dapat mengetahui
apakah transformasi yang telah dilakukan sesuai. Jika transformasi
logaritmik memberikan kompensasi yang berlebihan untuk kemiringan yang
tepat dan menghasilkan distribusi yang sedikit kiri (kemencengan
negatif), transformasi `lebih ringan' dengan \(\theta\) lebih dekat ke
1, seperti transformasi kuadrat atau akar kubik, harus digunakan.
Transformasi dengan \(\theta\)\textgreater{} 1 akan membantu membuat
data yang condong ke kiri lebih simetris.

Namun, kecenderungan untuk mencari transformasi `terbaik' harus
dihindari. Misalnya, ketika berhadapan dengan beberapa set data yang
serupa, mungkin lebih baik untuk menemukan satu transformasi yang
bekerja cukup baik untuk semua, daripada menggunakan yang sedikit
berbeda untuk masing-masingnya. Harus diingat bahwa setiap set data
adalah sampel dari populasi yang lebih besar, dan sampel lain dari
populasi yang sama kemungkinan akan menunjukkan transformasi `terbaik'
yang sedikit berbeda. Penentuan `terbaik' dalam ketelitian tinggi adalah
pendekatan yang jarang sepadan dengan usaha.

Pada Gambar \ref{fig:gwvis2} kosentrasi distribusi \texttt{Uranium} pada
tiap grup memiliki kemencengan positif. Untuk membuatnya simetris kita
perlu melakukan transformasi yang sesuai jenis transformasi yang
dilakukan dapat dimulai dari akar kuadrat sampai invers akar kuadrat
(berdasarkan Gambar \ref{fig:power}). Pada contoh ini kita akan mencoba
melakukan trasnformasi logaritmik. Berikut adalah contoh visualisasi
hasil transformasinya (lihat Gambar \ref{fig:urantrans}:

\begin{figure}

{\centering \includegraphics[width=0.7\linewidth]{EnvStat_files/figure-latex/urantrans-1} 

}

\caption{Visualisasi konsentrasi Uranium  hasil tansformasi pada air tanah}\label{fig:urantrans}
\end{figure}

Berdasarkan hasil transformasi, kita telah memperoleh ditribusi yang
cukup simetris untuk kedua grup data tersebut. Pembaca dapat mencobanya
menggunakan transformasi lainnya sendiri.

\section{Referensi}\label{referensi-5}

\begin{enumerate}
\def\labelenumi{\arabic{enumi}.}
\tightlist
\item
  Damanhuri, E. 2011. \textbf{Statitika Lingkunga}. Penerbit ITB.
\item
  Helsel, D.R., Hirsch, R.M. 2002. \textbf{statistical Methods in Water
  Resources}. USGS.
\item
  Ofungwu, J. 2014. \textbf{Statistical Applications For Environmental
  Analysis and Risk Assessment}. John Wiley \& Sons, Inc.
\item
  Rosadi, D. 2015. \textbf{Analisis Statistika dengan R}. Gadjah Mada
  University Press.
\item
  STHDA. \textbf{Descriptive Statistics and Graphics}.
  \url{http://www.sthda.com/english/wiki/descriptive-statistics-and-graphics}.
\end{enumerate}

\chapter{Ekplorasi Data Menggunakan
Grafik}\label{ekplorasi-data-menggunakan-grafik}

Pada Chapter 4 dan 5 kita telah belajar bagaimana cara membuat grafik
menggunakan \texttt{R}. Sejauh ini kita belum belajar kegunaan dari
masing-masing grafik yang telah kita pelajarai. Pada Chapter ini kita
tidak lagi akan membahas bagaimana membuat grafik menggunakan
\texttt{R}. Kita akan fokus terhadap fungsi grafik tersebut dalam
analisa kita. Secara umum grafik dibuat untuk memvisualisasikan
distribusi, perbedaan antar sampel, korelasi dan asosiasi antar sampel,
serta ukuran sampel.

Penulis dan pembaca pasti sepakat bahwa visualisasi data merupakan
tahapan awal yang perlu kita lakukan sebelum memutuskan untuk melakukan
analisa data seperti uji hipotesis dan modeling. Angka yang ditampilkan
dalam ringkasan data tidaklah cukup untuk melihat data terutama
kaitannya dengan pengecekan terhadap asumsi model.

Pada Gambar \ref{fig:visscat} disajikan delapan buah scatterplot dengan
koefisien korelasi yang sama persis. Komputasi statistik tanpa melihat
pada visualisasi data akan menyebabkan misinterpretasi pada data. Grafik
memberikan ringkasan visual data dengan cepat dan lengkap dibandingkan
penyajian data dalam tabel angka.

\begin{figure}

{\centering \includegraphics[width=0.9\linewidth]{visscat} 

}

\caption{Scatterplot dengan koefisien korelasi r=0,7.}\label{fig:visscat}
\end{figure}

Grafik sangat penting untuk dua tujuan:

\begin{enumerate}
\def\labelenumi{\arabic{enumi}.}
\tightlist
\item
  untuk memberikan wawasan bagi analis ke dalam data di bawah
  pengawasan, dan
\item
  untuk mengilustrasikan konsep-konsep penting ketika mempresentasikan
  hasil kepada orang lain.
\end{enumerate}

Tugas pertama disebut \textbf{Analisis Data Eksplorasi (EDA)}, dan
merupakan subjek Chapter ini. Prosedur EDA seringkali merupakan (atau
seharusnya) menjadi `pandangan pertama' pada data. Pola dan teori
tentang bagaimana sistem berperilaku dikembangkan dengan mengamati data
melalui grafik. Ini adalah prosedur induktif - data dirangkum dibanding
dilakukan pengujian. Hasil mereka memberikan panduan untuk pemilihan
prosedur pengujian hipotesis deduktif yang tepat.

Setelah analisis selesai, temuan harus dilaporkan kepada orang lain.
Apakah laporan tertulis atau presentasi lisan, analis harus meyakinkan
audiens bahwa kesimpulan yang dicapai didukung oleh data. Tidak ada cara
yang lebih baik untuk melakukan ini selain melalui grafik. Banyak metode
grafis yang sama yang merangkum informasi dengan ringkas untuk analis
juga akan memberikan wawasan tentang data untuk pembaca atau audiens.

\section{Grafik Untuk Melihat Ditribusi
Data}\label{grafik-untuk-melihat-ditribusi-data}

Analisis yang umumnya dilihat pada distribusi data adalah apakah data
berdistribusi normal atau tidak. Hal ini akan mempengaruhi jenis
analisis statistika yang digunakan pada data. Terdapat beberapa grafik
yang dapat digunakan untuk melihat bentuk ditribusi data. Grafik-grafik
tersebut antara lain: \emph{stem and leaf}, histogram, density plot,
QQ-plot, serta box plot atau violin plot. Pada analisis distribusi
\emph{stem and leaf} kurang populer untuk digunakan. Hal ini disebabkan
karena visualisasinya kurang cocok diterapkan pada data dengan jumlah
observasi besar. Selain itu, kita juga tidak bisa melakukan perbandingan
antar grup menggunakan jenis visualisasi tersebut.

\subsection{Histogram}\label{histogram}

Histogram adalah grafik yang sudah dikenal, dan konstruksinya dirinci
dalam berbagai teks pengantar tentang statistik. Batang digambar dengan
tinggi \(n_i\), atau fraksi \(n_i/n\), dari data yang termasuk dalam
salah satu dari beberapa kategori atau interval (Gambar
\ref{fig:histeda}). Iman dan Conover (1983) mengemukakan bahwa untuk
ukuran sampel \(n\), jumlah interval \(k\) harus bilangan bulat terkecil
sehingga \(2^k≥n\).

Histogram sangat berguna untuk menggambarkan perbedaan besar dalam
bentuk data seperti apakah data simetris seperti distribusi normal atau
memiliki kemencengan. Histogram tidak dapat digunakan untuk penilaian
yang lebih tepat karena tampilan dipengaruhi oleh jumlah batang yang
digunakan. Untuk lebih memahaminya perhatikan Gambar \ref{fig:histeda}
dan Gambar \ref{fig:histeda2}. Kedua histogram tersebut tampak berbeda
meskipun data input yang diberikan sama. Pada Gambar \ref{fig:histeda}
kita akan melihat bahwa debit dengan kejadian terbanyak terjadi pada
rentang 800-900 cfs, sedangkan pada Gambar \ref{fig:histeda2} kita
melihat bahwa debit dengan kejadian terbanyak terjadi pada 800-1200 cfs.

\begin{Shaded}
\begin{Highlighting}[]
\CommentTok{# memuat library}
\KeywordTok{library}\NormalTok{(readxl)}
\KeywordTok{library}\NormalTok{(ggplot2)}
\KeywordTok{library}\NormalTok{(ggthemes)}

\CommentTok{# memuat data excel}
\NormalTok{sungai <-}\StringTok{ }\KeywordTok{read_excel}\NormalTok{(}\StringTok{"hhappc.xls"}\NormalTok{, }\DataTypeTok{sheet=}\StringTok{"appc1"}\NormalTok{)}
\end{Highlighting}
\end{Shaded}

\begin{figure}

{\centering \includegraphics[width=0.7\linewidth]{EnvStat_files/figure-latex/histeda-1} 

}

\caption{Histogram dengan bin.width=default debit sungai Saddle}\label{fig:histeda}
\end{figure}

\begin{figure}

{\centering \includegraphics[width=0.7\linewidth]{EnvStat_files/figure-latex/histeda2-1} 

}

\caption{Histogram dengan bin.width=500 debit sungai Saddle}\label{fig:histeda2}
\end{figure}

\subsection{Density Plot}\label{density-plot}

Density plot memecahkan masalah yang dimiliki histogram dalam melihat
grafik dengan menyajikan data bukan dari jumlah kejadian atau observasi,
namun data disajikan berdasarkan frekuensi relatif data (density) yang
digambarkan dalam bentuk \emph{smooth curve}. Contoh density plot dapat
dilijat pada Gambar \ref{fig:denseda}. Dari grafik yang dihasilkan
sekaran tampak jelas bahwa distibusi data memiliki kemencengan positif
dengan frekuensi relatif debit terbanyak berada pada debit 1000 cfs.

\begin{figure}

{\centering \includegraphics[width=0.7\linewidth]{EnvStat_files/figure-latex/denseda-1} 

}

\caption{Density plot debit sungai Saddle}\label{fig:denseda}
\end{figure}

\subsection{QQ-plot}\label{qq-plot-2}

Kita telah mempelajari sebelumnya pada Chapter 5 bahwa QQ-plot data
digunakan untuk mengecek apakah data yang kita miliki berdistribusi
normal atau tidak. Contoh QQ-plot dapat dilijat pada Gambar
\ref{fig:qqeda}. Pada grafik yang dihasilkan terlihat bahwa data tidak
berdistribusi normal. Hal ini terlihat dari sebagian observasi pada
debit \textless{}1000 cfs yang tidak mengikuti garis referensi.

\begin{Shaded}
\begin{Highlighting}[]
\KeywordTok{ggplot}\NormalTok{(sungai, }\KeywordTok{aes}\NormalTok{(}\DataTypeTok{sample=}\NormalTok{Flow))}\OperatorTok{+}
\StringTok{  }\CommentTok{# qq plot}
\StringTok{  }\KeywordTok{stat_qq}\NormalTok{()}\OperatorTok{+}
\StringTok{  }\CommentTok{# garis referensi}
\StringTok{  }\KeywordTok{stat_qq_line}\NormalTok{()}\OperatorTok{+}
\StringTok{  }\KeywordTok{theme_economist}\NormalTok{()}
\end{Highlighting}
\end{Shaded}

\begin{figure}

{\centering \includegraphics[width=0.7\linewidth]{EnvStat_files/figure-latex/qqeda-1} 

}

\caption{QQ plot debit sungai Saddle}\label{fig:qqeda}
\end{figure}

\subsection{Box Plot dan Violin Plot}\label{box-plot-dan-violin-plot-1}

Grafik lain yang dapat digunakan untuk menggambarkan distribusi data
adalah box plot dan violin plot. Box plot memberikan cara yang simpel
untuk melihat ditribusi data seperti melihat posisi sejumlah kuartil,
nilai minimum dan maksimum. Selain itu kita juga dapat melihat adanya
outlier pada data.

Kita dapat menambah fungsionalitas dari box plot ini dengan menambahkan
violin plot. Pada Chapter 5 kita telah belajar bahwa kita dapat
menambahkan box plot pada violin plot atau sebaliknya sehingga
memudahkan dalam mendeskripsikan bentuk distribusi data. Jika dengan box
plot kita tidak dapat melihat secara baik bentuk dari data yang
sesungguhnya karena hanya menampilkan lokasi sejumlah kuartil. Pada
violin plot kita dapat melihat bentuk data yang ada melalui tampilan dua
denisty plot (tampak seperti biola) yang digambarkan. Kekurangannya
adalah kita tidak dapat melihat observasi mana yang menjadi outlier,
sehingga kedua grafik ini biasa digambarkan secara bersamaan. Berikut
adalah contoh box plot dan violin plot dari data debit sungai Saddle
(Gambar \ref{fig:bpeda}).

\begin{Shaded}
\begin{Highlighting}[]
\KeywordTok{ggplot}\NormalTok{(sungai, }\KeywordTok{aes}\NormalTok{(}\DataTypeTok{x=}\StringTok{""}\NormalTok{, }\DataTypeTok{y=}\NormalTok{Flow))}\OperatorTok{+}
\StringTok{  }\KeywordTok{geom_violin}\NormalTok{(}\DataTypeTok{fill=}\StringTok{"blue"}\NormalTok{, }\DataTypeTok{alpha=}\FloatTok{0.5}\NormalTok{, }\DataTypeTok{color=}\StringTok{"white"}\NormalTok{)}\OperatorTok{+}
\StringTok{  }\KeywordTok{geom_boxplot}\NormalTok{(}\DataTypeTok{width=}\FloatTok{0.1}\NormalTok{)}\OperatorTok{+}
\StringTok{  }\KeywordTok{theme_economist}\NormalTok{()}
\end{Highlighting}
\end{Shaded}

\begin{figure}

{\centering \includegraphics[width=0.7\linewidth]{EnvStat_files/figure-latex/bpeda-1} 

}

\caption{Box plot dan violin plot debit sungai Saddle}\label{fig:bpeda}
\end{figure}

Berdasarkan grafik yang dihasilkan pada Gambar \ref{fig:bpeda} kita
dapat melihat bahwa ditribusi data debit sungai memiliki kemencengan
positif. Hal ini terjadi karena terdapat satu \emph{outlier} pada data
yang disebabkan karena nilai observasinya diluar dari nilai maksimum
data yang ditetapkan sebagai \(max=Q3 + 1,5*IQR\).

\section{Grafik Untuk Melihat Beda Distribusi Data Antar
Grup}\label{grafik-untuk-melihat-beda-distribusi-data-antar-grup}

Grafik yang telah dijelaskan sebelumnya seperti box plot, violin plot,
histogram, dan density plot merupakan grafik yang bagus untuk
memvisualisasikan beda distribusi data antar grup untuk data numerik.
Untuk data berupa kategori kita dapat menggunakan bar plot. Pada
penerapannya bar plot juga dapat memvisualisasikan ringkasan data
seperti nilai mean dan sebarannya pada data.

Pada contoh ini penulis hanya akan memberikan contoh penerapan
menggunakan box plot dan bar plot menggunakan data konsentrasi Antrazine
yang diukur pada bulan Juni dan September. Untuk melakukannya kita perlu
memuat data dan melakukan transformasi terhadap datanya terlebih dahulu.

\begin{Shaded}
\begin{Highlighting}[]
\CommentTok{# memuat data excel}
\NormalTok{atrazine <-}\StringTok{ }\KeywordTok{read_excel}\NormalTok{(}\StringTok{"hhappc.xls"}\NormalTok{, }\DataTypeTok{sheet=}\StringTok{"appc4"}\NormalTok{)}

\CommentTok{# print}
\KeywordTok{head}\NormalTok{(atrazine)}
\end{Highlighting}
\end{Shaded}

\begin{verbatim}
## # A tibble: 6 x 2
##   June_atrazine Sept_atrazine
##           <dbl>         <dbl>
## 1          0.38         2.66 
## 2          0.04         0.63 
## 3         -0.01         0.59 
## 4          0.03         0.05 
## 5          0.03         0.84 
## 6          0.05         0.580
\end{verbatim}

\begin{Shaded}
\begin{Highlighting}[]
\CommentTok{# transformasi data}
\KeywordTok{library}\NormalTok{(tidyr)}
\NormalTok{atrazine <-}\StringTok{ }\KeywordTok{gather}\NormalTok{(atrazine,}
                 \DataTypeTok{key=}\StringTok{"month"}\NormalTok{,}
                 \DataTypeTok{value=}\StringTok{"concentration"}\NormalTok{)}

\CommentTok{# print}
\KeywordTok{head}\NormalTok{(atrazine)}
\end{Highlighting}
\end{Shaded}

\begin{verbatim}
## # A tibble: 6 x 2
##   month         concentration
##   <chr>                 <dbl>
## 1 June_atrazine          0.38
## 2 June_atrazine          0.04
## 3 June_atrazine         -0.01
## 4 June_atrazine          0.03
## 5 June_atrazine          0.03
## 6 June_atrazine          0.05
\end{verbatim}

Pada data konsentrasi Atrazine tersebut terdapat nilai negatif yang
dalam hal ini merupakan kesalahan dalam pengukuran dari alat. Untuk
membersihkannya kita dapat membuat nilai observasi tersebut menjadi NA.

\begin{Shaded}
\begin{Highlighting}[]
\NormalTok{atrazine}\OperatorTok{$}\NormalTok{concentration[atrazine}\OperatorTok{$}\NormalTok{concentration}\OperatorTok{<}\DecValTok{0}\NormalTok{]<-}\OtherTok{NA}

\KeywordTok{head}\NormalTok{(atrazine)}
\end{Highlighting}
\end{Shaded}

\begin{verbatim}
## # A tibble: 6 x 2
##   month         concentration
##   <chr>                 <dbl>
## 1 June_atrazine          0.38
## 2 June_atrazine          0.04
## 3 June_atrazine         NA   
## 4 June_atrazine          0.03
## 5 June_atrazine          0.03
## 6 June_atrazine          0.05
\end{verbatim}

Selanjutnya kita akan memvisualisasikan beda antara distribusi data pada
kedua bulan menggunakan box plot (Gambar \ref{fig:bpgeda}). Konsentrasi
rata-rata Atrazine akan divisualisasikan menggunakan bar plot (Gambar
\ref{fig:bargeda}).

\begin{Shaded}
\begin{Highlighting}[]
\KeywordTok{ggplot}\NormalTok{(atrazine, }\KeywordTok{aes}\NormalTok{(month, concentration, }\DataTypeTok{fill=}\NormalTok{month))}\OperatorTok{+}
\StringTok{  }\KeywordTok{geom_boxplot}\NormalTok{()}\OperatorTok{+}
\StringTok{  }\KeywordTok{theme_economist}\NormalTok{()}\OperatorTok{+}
\StringTok{  }\KeywordTok{scale_fill_economist}\NormalTok{()}
\end{Highlighting}
\end{Shaded}

\begin{figure}

{\centering \includegraphics[width=0.7\linewidth]{EnvStat_files/figure-latex/bpgeda-1} 

}

\caption{Box plot konsentrasi Atrazine pada bulan Juni dan September}\label{fig:bpgeda}
\end{figure}

\begin{Shaded}
\begin{Highlighting}[]
\KeywordTok{library}\NormalTok{(dplyr)}
\NormalTok{atrazine }\OperatorTok
\StringTok{  }\KeywordTok{group_by}\NormalTok{(month) }\OperatorTok
\StringTok{  }\KeywordTok{summarize}\NormalTok{(}\DataTypeTok{mean_atrazine=}\KeywordTok{mean}\NormalTok{(concentration, }\DataTypeTok{na.rm=}\OtherTok{TRUE}\NormalTok{)) }\OperatorTok
\StringTok{  }\KeywordTok{ggplot}\NormalTok{(}\KeywordTok{aes}\NormalTok{(month, mean_atrazine, }\DataTypeTok{fill=}\NormalTok{month))}\OperatorTok{+}
\StringTok{    }\KeywordTok{geom_bar}\NormalTok{(}\DataTypeTok{stat=}\StringTok{"identity"}\NormalTok{)}\OperatorTok{+}
\StringTok{    }\KeywordTok{theme_economist}\NormalTok{()}\OperatorTok{+}
\StringTok{    }\KeywordTok{scale_fill_economist}\NormalTok{()}
\end{Highlighting}
\end{Shaded}

\begin{figure}

{\centering \includegraphics[width=0.7\linewidth]{EnvStat_files/figure-latex/bargeda-1} 

}

\caption{Bar plot konsentrasi Atrazine pada bulan Juni dan September}\label{fig:bargeda}
\end{figure}

Pada visualisasi yang dihasilkan terdapat perbedaan signifikan antara
distribusi dan nilai rata-rata konsentrasi Atrazine pada dua periode
tersebut. Hal ini disebabkan karena terdapat sebuah outlier pada periode
Sepetember yang menyebabkan nilai rata-rata yang dihasilkan bergeser
jauh kearah outlier. Pembaca dapat membuat visualisasi data pada data
tersebut tanpa \emph{outlier} dengan terlebih dahulu melakukan filter
terhadap \emph{outlier}.

\section{Grafik Untuk Memvisualisasikan Korelasi Antar
Variabel}\label{grafik-untuk-memvisualisasikan-korelasi-antar-variabel}

Scatterplot dapat digunakan untuk memvisualisasikan korelasi antar dua
variabel. Pada bagian ini akan diberikan contoh visualisasi antara
variabel konsentrasi TDS dan Uranium pada air tanah.

Untuk melakukannya kita perlu memuat terlebih dahulu dataset yang
digunakan. Visualisasi data disajikan pada Gambar \ref{fig:scateda}.

\begin{Shaded}
\begin{Highlighting}[]
\CommentTok{# memuat data excel}
\NormalTok{gw <-}\StringTok{ }\KeywordTok{read_excel}\NormalTok{(}\StringTok{"hhappc.xls"}\NormalTok{, }\DataTypeTok{sheet=}\StringTok{"appc16"}\NormalTok{)}
\end{Highlighting}
\end{Shaded}

\begin{Shaded}
\begin{Highlighting}[]
\KeywordTok{ggplot}\NormalTok{(gw, }\KeywordTok{aes}\NormalTok{(TDS, Uranium))}\OperatorTok{+}
\StringTok{  }\KeywordTok{geom_point}\NormalTok{()}\OperatorTok{+}
\StringTok{  }\KeywordTok{geom_smooth}\NormalTok{(}\DataTypeTok{method=}\StringTok{"lm"}\NormalTok{)}\OperatorTok{+}
\StringTok{  }\KeywordTok{theme_economist}\NormalTok{()}
\end{Highlighting}
\end{Shaded}

\begin{figure}

{\centering \includegraphics[width=0.7\linewidth]{EnvStat_files/figure-latex/scateda-1} 

}

\caption{Scatterplot hubungan antara konsentrasi TDS dan Uranium pada airtanah}\label{fig:scateda}
\end{figure}

Berdasarkan grafik yang dihasilkan terdapat hubungan linier antara
konsentrasi TDS dan Uranium pada airtanah. Meningkatnya konsentrasi TDS
pada air tanah juga menyebabkan peningkatan konsentrasi Uranium pada
airtanah.

\section{Grafik Yang Digunakan Untuk Memvisualisasikan Asosiasi Antar
Variabel}\label{grafik-yang-digunakan-untuk-memvisualisasikan-asosiasi-antar-variabel}

Asosiasi antar variabel kategori dapat dilakukan baik dengan pie chart
maupun dengan bar plot. Pie chart kurang sering digunakan untuk
visualisasi \emph{multiple group} sehingga bar plot lebih sering
digunakan.

Pada contoh kali ini penulis akan melihat terdapat asoiasi antara musim
dan strata terhadap jumlah Corbicula di sungai Tennessee. Untuk
melakukannya kita perlu memuat terlebih dahulu dataset yang digunakan.
Visualisasi data disajikan pada Gambar \ref{fig:barteneda}.

\begin{Shaded}
\begin{Highlighting}[]
\CommentTok{# memuat data excel}
\NormalTok{corbicula<-}\StringTok{ }\KeywordTok{read_excel}\NormalTok{(}\StringTok{"hhappc.xls"}\NormalTok{, }\DataTypeTok{sheet=}\StringTok{"appc8"}\NormalTok{)}

\CommentTok{# print}
\KeywordTok{head}\NormalTok{(corbicula)}
\end{Highlighting}
\end{Shaded}

\begin{verbatim}
## # A tibble: 6 x 4
##    Year Season Strata Corbicula
##   <dbl> <chr>   <dbl>     <dbl>
## 1  1969 Winter      1        25
## 2  1969 Winter      1        20
## 3  1969 Winter      1        30
## 4  1969 Spring      1         9
## 5  1969 Spring      1         8
## 6  1969 Spring      1         9
\end{verbatim}

\begin{Shaded}
\begin{Highlighting}[]
\NormalTok{corbicula }\OperatorTok
\StringTok{  }\KeywordTok{mutate}\NormalTok{(}\DataTypeTok{Season=}\KeywordTok{as.factor}\NormalTok{(Season),}
         \DataTypeTok{Strata=}\KeywordTok{as.factor}\NormalTok{(Strata)) }\OperatorTok
\StringTok{  }\KeywordTok{group_by}\NormalTok{(Season,Strata) }\OperatorTok
\StringTok{  }\KeywordTok{summarize}\NormalTok{(}\DataTypeTok{Corbicula=}\KeywordTok{mean}\NormalTok{(Corbicula)) }\OperatorTok
\StringTok{  }\KeywordTok{ggplot}\NormalTok{(}\KeywordTok{aes}\NormalTok{(Season, Corbicula, }\DataTypeTok{fill=}\NormalTok{Strata))}\OperatorTok{+}
\StringTok{    }\KeywordTok{geom_bar}\NormalTok{(}\DataTypeTok{stat=}\StringTok{"identity"}\NormalTok{,}\DataTypeTok{position=}\KeywordTok{position_dodge2}\NormalTok{())}\OperatorTok{+}
\StringTok{    }\KeywordTok{theme_economist}\NormalTok{()}\OperatorTok{+}
\StringTok{    }\KeywordTok{scale_fill_economist}\NormalTok{()}
\end{Highlighting}
\end{Shaded}

\begin{figure}

{\centering \includegraphics[width=0.7\linewidth]{EnvStat_files/figure-latex/barteneda-1} 

}

\caption{Bar plot Jumlah rata-rata corbicula pada sungai Tennessee}\label{fig:barteneda}
\end{figure}

Berdasarkan grafik yang dihasilkan terdapat pengaruh musim dan strata
terhadap jumlah corbicula di sungai Tennessee. Jumlah tertinggi berada
saat musim semi pada strata 3, sedangkan terendah berada pada musim
dingin juga pada strata 3.

\section{Grafik Yang Digunakan Untuk Memisualisasikan Ukuran Sampel dan
Perubahan Sepanjang
Waktu}\label{grafik-yang-digunakan-untuk-memisualisasikan-ukuran-sampel-dan-perubahan-sepanjang-waktu}

Untuk memvisualisasikan perubahan sepanjang waktu, kita dapat
menggunakan line plot. Pada data \texttt{corbicula} kita ingin
memvisualisasikan perubahan jumlah corbicula rata-rata pada setiap
tahun. Visualisasi dari data disajikan pada Gambar \ref{fig:lineeda}.

\begin{Shaded}
\begin{Highlighting}[]
\NormalTok{corbicula }\OperatorTok
\StringTok{  }\KeywordTok{group_by}\NormalTok{(Year) }\OperatorTok
\StringTok{  }\KeywordTok{summarize}\NormalTok{(}\DataTypeTok{Corbicula=}\KeywordTok{mean}\NormalTok{(Corbicula)) }\OperatorTok
\StringTok{  }\KeywordTok{ggplot}\NormalTok{(}\KeywordTok{aes}\NormalTok{(Year, Corbicula))}\OperatorTok{+}
\StringTok{    }\KeywordTok{geom_line}\NormalTok{()}\OperatorTok{+}
\StringTok{    }\KeywordTok{geom_point}\NormalTok{(}\DataTypeTok{shape=}\DecValTok{1}\NormalTok{)}\OperatorTok{+}
\StringTok{    }\KeywordTok{theme_economist}\NormalTok{()}
\end{Highlighting}
\end{Shaded}

\begin{figure}

{\centering \includegraphics[width=0.7\linewidth]{EnvStat_files/figure-latex/lineeda-1} 

}

\caption{Line plot perubahan jumlah rata-rata corbicula di sungai Tennessee}\label{fig:lineeda}
\end{figure}

Berdasarkan garfik yang dihasilkan dapat disimpulkan bahwa jumlah
rata-rata corbicula menurun setiap tahunnya.

\section{Referensi}\label{referensi-6}

\begin{enumerate}
\def\labelenumi{\arabic{enumi}.}
\tightlist
\item
  Gardener, M. 2012. \textbf{Statistics for Ecologists Using R and
  Excel-Data collection, exploration, analysis and presentation}.
  Pelagic Publishing.
\item
  Helsel, D.R., Hirsch, R.M. 2002. \textbf{statistical Methods in Water
  Resources}. USGS.
\item
  Ofungwu, J. 2014. \textbf{Statistical Applications For Environmental
  Analysis and Risk Assessment}. John Wiley \& Sons, Inc.
\item
  Peck, R.Devore, J.L. 2012. \textbf{Statistics The Exploration \&
  Analysis of Data- Seventh Edition}. Brooks/Cole.
\end{enumerate}

\part*{Statistika Inferensi -
R}\label{part-statistika-inferensi---r}
\addcontentsline{toc}{part}{Statistika Inferensi - R}

\chapter{Penaksiran Secara
Statistika}\label{penaksiran-secara-statistika}

Pada Chapter 6-Ringkasan Numerik kita telah belajar beberapa atribut
kunci dari data seperti \(\overline{X}\) dan \(s\). Kedua nilai tersebut
disebut sebagai nilai estimasi sampel dari populasi (untuk mean \(\mu\)
dan simpangan baku \(\sigma\)). Pada Chapter ini kita akan melakukan
eksplorasi lebih jauh lagi mengenai interval estimasi (\emph{interval
estimate}) yang akan menyinggung kedua nilai tersebut lebih jauh.

\section{Definisi Interval Estimasi}\label{definisi-interval-estimasi}

Median sampel dan mean sampel menyatakan titik pemusatan data. Estimasi
menggunakan kedua nilai tersebut disebut sebagai estimasi titik
(\emph{point estimation}). Estimasi titik sendiri tidak menggambarkan
reliabilitas atau kurangnya reliabilitas (variabilitas) dari estimasi
ini. Sebagai contoh, anggaplah terdapat dua data X dan Y dengan mean 5
dengan jumlah observasi yang sama. Data Y memiliki nilai mean 5 dengan
sangat sedikit variasi didalamnya, sedangkan data X jauh lebih
bervariasi. Perkiraan titik 5 untuk X jauh lebih tidak dapat diandalkan
dibandingkan dengan untuk Y karena variabilitas yang lebih besar dalam
data X. Dengan kata lain, lebih banyak kehati-hatian diperlukan ketika
menyatakan bahwa 5 memperkirakan mean populasi sebenarnya X daripada
ketika menyatakan ini untuk Y. Melaporkan hanya perkiraan mean sampel
(poin) 5 gagal memberikan petunjuk tentang perbedaan ini.

Sebagai alternatif untuk estimasi titik, estimasi interval adalah
interval yang memiliki probabilitas yang dinyatakan mengandung nilai
populasi sebenarnya. interval lebih lebar untuk set data yang memiliki
variabilitas lebih besar. Jadi dalam contoh di atas, interval antara 4,7
dan 5,3 mungkin memiliki kemungkinan 95\% untuk mengandung mean populasi
Y yang sebenarnya (tidak diketahui). Butuh interval yang jauh lebih
luas, katakanlah antara 2,0 dan 8,0, untuk memiliki probabilitas yang
sama untuk mengandung rerata sebenarnya dari X. Karena itu, perbedaan
keandalan dari dua estimasi dengan jelas dinyatakan menggunakan estimasi
interval. Estimasi interval dapat memberikan dua informasi yang estimasi
poin tidak dapat berikan, antara lain:

\begin{enumerate}
\def\labelenumi{\arabic{enumi}.}
\tightlist
\item
  Pernyataan probabilitas atau kemungkinan bahwa interval berisi nilai
  populasi sebenarnya (keandalannya).
\item
  Pernyataan kemungkinan bahwa satu titik data dengan besaran tertentu
  berasal dari populasi yang diteliti.
\end{enumerate}

Estimasi interval untuk poin pertama disebut sebagai interval
kepercayaan (\emph{confidence interval}), sedangkan yang kedua disebut
sebagai interval prediksi (\emph{prediction interal}). Meskipun salin
terkait, pembaca perlu berhati-hati sebab kedua definisi tersebut sering
kali tertukar satu sama lain.

\section{Interpretasi Interval
Estimasi}\label{interpretasi-interval-estimasi}

Untuk lebih memahami mengenai definisi interval estimasi pada
sub-chapter ini akan diberikan contoh yang diambil dari buku
\textbf{statistical Methods in Water Resources} karya Helsel dan Hirsch
(2012). Misalkan mean populasi sebenarnya \(\mu\) konsentrasi dalam
akuifer adalah 10. Sleain itu, anggaplah bahwa varians populasi
sebenarnya \(\sigma^2\) sama dengan 1. Karena nilai-nilai ini dalam
praktiknya tidak pernah diketahui, sampel diambil untuk memperkirakannya
dengan mean sampel \(x\) dan sampel varianss \(s^2\) . Dana yang cukup
tersedia untuk mengambil 12 sampel air (kira-kira satu per bulan) selama
satu tahun, dan hari-hari di mana pengambilan sampel terjadi dipilih
secara acak. Dari 12 sampel ini \(x\)x dan \(s^2\) dihitung. Meskipun
pada kenyataannya hanya satu set 12 sampel akan diambil setiap tahun,
menggunakan komputer 12 hari dapat dipilih beberapa kali untuk
menggambarkan konsep perkiraan interval. Untuk masing-masing dari 10 set
independen dari 12 sampel, interval kepercayaan pada mean dihitung
dengan menggunakan persamaan yang diberikan pada Tabel \ref{tab:iie} dan
Gambar \ref{fig:iievis}.

\begin{longtable}[]{@{}lllll@{}}
\caption{\label{tab:iie} Sepuluh interval kepercayaan 90\% sekitar nilai
mean sebenarnya sebesar 10 (Data berdistribusi normal dan Tanda plus
menyatakan data tidak disertakan dalam nilai mean
sebenarnya)}\tabularnewline
\toprule
\textbf{No.} & \textbf{N} & \textbf{Mean} & \textbf{St.Dev} &
\textbf{90\% Interval kepercayaan}\tabularnewline
\midrule
\endfirsthead
\toprule
\textbf{No.} & \textbf{N} & \textbf{Mean} & \textbf{St.Dev} &
\textbf{90\% Interval kepercayaan}\tabularnewline
\midrule
\endhead
1 & 12 & 10,06 & 1,11 & 9,46 sampai 10,64\tabularnewline
2 & 12 & 10,60 & 0,81 & \(^+\) 10,18 sampai 11,02\tabularnewline
3 & 12 & 9,95 & 1,26 & 9,29 sampai 10,60\tabularnewline
4 & 12 & 10,18 & 1,26 & 9,52 sampai 10,83\tabularnewline
5 & 12 & 10,17 & 1,33 & 9,48 sampai 10,85\tabularnewline
6 & 12 & 10,22 & 1,19 & 9,60 sampai 10,84\tabularnewline
7 & 12 & 9,71 & 1,51 & 8,92 sampai 10,49\tabularnewline
8 & 12 & 9,90 & 1,01 & 9,38 sampai 10,43\tabularnewline
9 & 12 & 9,95 & 0.10 & 9,43 sampai 10,46\tabularnewline
10 & 12 & 9,88 & 1,37 & 9,17 sampai 10,59\tabularnewline
\bottomrule
\end{longtable}

Kesepuluh interval pada contoh di atas disebut dengan dengan
\textbf{interval kepercayaan 90\%} dari nilai mean sesungguhnya. Nilai
mean sebenarnya akan terdapat pada interval tersebut dengan probabilitas
90\%. Sehingga berdasarkan Tabel \ref{tab:iie} terdapat 9 interval yang
memiliki nilai mean sesungguhnya didalamnya (probabilitas 90\%). Jika
kita sekali lagi melakukan sampling dengan jumlah sampling yang sama
pada interval nilai baru yang dihasilkan akan mengandung nilai mean
sesungguhnya dan dapat juga tidak. Probabilitas interval tersebut
mengandung nilai mean sesungguhnya disebut sebagai \textbf{tingkat
kepercayaan} (\emph{confidence level}). Probabilitas nilai interval
tidak mengandung mean sesungguhnya disebut sebagai \textbf{alpha level}
(\(\alpha\)) yang ditulis berdasarkan Persamaan \eqref{eq:alpha}.

\begin{equation}
  \alpha=1-confidence\ level\ 
  \label{eq:alpha}
\end{equation}

Lebar interval kepercayaan adalah fungsi dari bentuk distribusi data
(variabilitas dan kemencengannya), ukuran sampel, dan tingkat
kepercayaan yang diinginkan. Ketika tingkat kepercayaan meningkat, lebar
interval juga meningkat, karena interval yang lebih besar lebih mungkin
mengandung nilai sebenarnya daripada interval yang lebih pendek. Dengan
demikian interval kepercayaan 95\% akan lebih luas daripada interval
90\% untuk data yang sama.

\begin{figure}

{\centering \includegraphics[width=0.65\linewidth]{iievis} 

}

\caption{Sepuluh interval kepercayaan 90 persen data dengan nilai mean sebenarnya 10}\label{fig:iievis}
\end{figure}

Interval kepercayaan simetris pada rata-rata biasanya dihitung dengan
asumsi data mengikuti distribusi normal. Jika tidak, distribusi rerata
itu sendiri akan mendekati normal sepanjang ukuran sampel besar
(katakanlah 50 pengamatan atau lebih besar). Interval kepercayaan dengan
asumsi normalitas kemudian akan memasukkan mean sebenarnya
(\(1-\alpha\))\% dari waktu. Dalam contoh di atas, data dihasilkan dari
distribusi normal, sehingga ukuran sampel kecil 12 tidak menjadi
masalah. Namun ketika data memiliki kemencengan dan ukuran sampel di
bawah 50 atau lebih, interval kepercayaan simetris tidak akan mengandung
rata-rata (\(1-\alpha\))\% sepanjang waktu. Dalam contoh di bawah ini,
interval kepercayaan simetris secara salah dihitung untuk data yang
miring (Gambar \ref{fig:iiedata}). Hasil (Gambar \ref{fig:iievis2} dan
Tabel \ref{tab:iie2}) menunjukkan bahwa interval kepercayaan kehilangan
nilai sebenarnya dari 1 lebih sering daripada yang seharusnya. Semakin
besar skewness, semakin besar ukuran sampel harus sebelum interval
kepercayaan simetris dapat diandalkan. Sebagai alternatif, interval
kepercayaan asimetris dapat dihitung untuk situasi umum data yang
memiliki kemencengan.

\begin{longtable}[]{@{}lllll@{}}
\caption{\label{tab:iie2} Sepuluh interval kepercayaan 90\% sekitar nilai
mean sebenarnya sebesar 1 (Data tidak berdistribusi normal dan Tanda
plus menyatakan data tidak disertakan dalam nilai mean
sebenarnya)}\tabularnewline
\toprule
\textbf{No.} & \textbf{N} & \textbf{Mean} & \textbf{St.Dev} &
\textbf{90\% Interval kepercayaan}\tabularnewline
\midrule
\endfirsthead
\toprule
\textbf{No.} & \textbf{N} & \textbf{Mean} & \textbf{St.Dev} &
\textbf{90\% Interval kepercayaan}\tabularnewline
\midrule
\endhead
1 & 12 & 0,784 & 0,320 & \(^+\) 0,618 sampai 0,950\tabularnewline
2 & 12 & 0,811 & 0,299 & \(^+\) 0,656 sampai 0,966\tabularnewline
3 & 12 & 1,178 & 0,700 & 0,815 sampai 1,541\tabularnewline
4 & 12 & 1,030 & 0,459 & 0,792 sampai 1,267\tabularnewline
5 & 12 & 1,079 & 0,573 & 0,782 sampai 1,376\tabularnewline
6 & 12 & 0,833 & 0,363 & 0,644 sampai 1,021\tabularnewline
7 & 12 & 0,789 & 0,240 & \(^+\) 0,644 sampai 0,913\tabularnewline
8 & 12 & 1,159 & 0,815 & 0,736 sampai 1,581\tabularnewline
9 & 12 & 0,822 & 0,365 & \(^+\) 0,633 sampai 0,992\tabularnewline
10 & 12 & 0,837 & 0,478 & 0,589 sampai 1,085\tabularnewline
\bottomrule
\end{longtable}

\begin{figure}

{\centering \includegraphics[width=0.7\linewidth]{iiedata} 

}

\caption{Histogram data dengan nilai mean populasi 1 dan simpangan baku populasi 0.75}\label{fig:iiedata}
\end{figure}

\begin{figure}

{\centering \includegraphics[width=0.65\linewidth]{iievis2} 

}

\caption{Sepuluh interval kepercayaan 90 persen data dengan nilai mean sebenarnya 1}\label{fig:iievis2}
\end{figure}

\section{Interval Kepercayaan Median}\label{interval-kepercayaan-median}

Interval kepercayaan median populasi dapat dihitung tanpa perlu
mengikuti asumsi distribusi tertentu. Sehingga nilai median dapat
digunakan untuk memperkirakan nilai pusat data untuk distribusi data
yang tidak berdistribusi normal.

\subsection{Interval Estimasi Median Metode
Nonparametrik}\label{interval-estimasi-median-metode-nonparametrik}

Interval estimasi nonparametrik untuk median populasi sebenarnya
dihitung menggunakan distribusi binomial. Pertama, tingkat signifikansi
yang diinginkan \(\alpha\) dinyatakan, error yang dapat diterima tidak
termasuk median yang sebenarnya. Satu-setengah (\(\alpha/2\)) dari error
ini digunakan untuk setiap akhir interval (Gambar \ref{fig:iemednp}).
Tabel distribusi binomial memberikan nilai kritis bawah dan atas \(x'\)
dan \(x\) pada setengah tingkat alfa yang diinginkan (\(\alpha/2\)).
Nilai-nilai kritis ini ditransformasikan ke dalam rangking \(R_l\) dan
\(R_u\) yang sesuai dengan titik data \(C_l\) dan \(C_u\) di ujung
interval kepercayaan.

\begin{figure}

{\centering \includegraphics[width=0.65\linewidth]{iemednp} 

}

\caption{Probabilitas median populasi P50 pada dua sisi interval estimasi.}\label{fig:iemednp}
\end{figure}

Untuk ukuran sampel kecil, tabel binomial dimasukkan pada kolom p = 0,5
(median) untuk menghitung interval kepercayaan pada median. Nilai kritis
\(x'\) diperoleh dari tabel distribusi binomial yang sesuai dengan
\(\alpha/2\), atau sedekat mungkin dengan \(\alpha/2\). Nilai kritis ini
kemudian digunakan untuk menghitung peringkat \(R_u\) dan \(R_l\) yang
sesuai dengan nilai data pada batas kepercayaan atas dan bawah untuk
median. Batas-batas ini adalah titik data peringkat \(R_l\)th yang masuk
dari setiap ujung daftar n observasi. Interval kepercayaan yang
dihasilkan akan mencerminkan bentuk (menceng atau simetris) dari data
asli.

\begin{equation}
  R_l=x'+1 
  \label{eq:rl}
\end{equation}

\begin{equation}
  R_u=n-x'=x\ untuk\ x'\ dan\ x\ dari\ tabel\ dist.\ binomial 
  \label{eq:ru}
\end{equation}

Interval nonparametrik tidak selalu dapat secara tepat menghasilkan
tingkat kepercayaan yang diinginkan ketika ukuran sampel kecil. Ini
karena mereka terpisah, melompat dari satu nilai data ke yang berikutnya
di ujung interval. Namun, tingkat kepercayaan yang dekat dengan yang
diinginkan tersedia untuk semua kecuali ukuran sampel terkecil.

Untuk lebih memahaminya diberikan data 25 pengukuran konsentrasi arsenin
di air tanah dalam ppb yang disajikan pada Tabel \ref{tab:gwardat}.

\begin{table}[t]

\caption{\label{tab:gwardat}Konsentrasi Arsenik dalam air tanah (ppb)}
\centering
\begin{tabular}{r|r}
\hline
observasi & konsentrasi\\
\hline
1 & 1.3\\
\hline
2 & 1.5\\
\hline
3 & 1.8\\
\hline
4 & 2.6\\
\hline
5 & 2.8\\
\hline
6 & 3.5\\
\hline
7 & 4.0\\
\hline
8 & 4.8\\
\hline
9 & 8.0\\
\hline
10 & 9.5\\
\hline
11 & 12.0\\
\hline
12 & 14.0\\
\hline
13 & 19.0\\
\hline
14 & 23.0\\
\hline
15 & 41.0\\
\hline
16 & 80.0\\
\hline
17 & 100.0\\
\hline
18 & 110.0\\
\hline
19 & 120.0\\
\hline
20 & 190.0\\
\hline
21 & 240.0\\
\hline
22 & 250.0\\
\hline
23 & 300.0\\
\hline
24 & 340.0\\
\hline
25 & 580.0\\
\hline
\end{tabular}
\end{table}

Visualisasi Tabel \ref{tab:gwardat} ditunjukkan pada Gambar
\ref{fig:gwardatvis}. Berdasarkan gambar tersebut terlihat bahwa data
memiliki kemencengan yang positif sehingga penaksiran rata-rata populasi
menggunakan nilai mean tidak dapat dilakukan.

\begin{figure}

{\centering \includegraphics[width=0.7\linewidth]{EnvStat_files/figure-latex/gwardatvis-1} 

}

\caption{Distribusi konsentrasi arsenik dalam air tanah}\label{fig:gwardatvis}
\end{figure}

Berdasarkan data pada Tabel \ref{tab:gwardat}, median konsentrasi
arsenik \(\hat{C}_{0.5}\)=19 yang berada pada urutan data ke-13 dari
data yang telah diurutkan dari yang terkecil ke yang terbesar. Untuk
menentukan interval kepercayaan 95\% median kosentrasi arsenik
\(C_{0.5}\), nilai kritis berdasarkan nilai error mendekati
\(\alpha/2\)=0,025 adalah \(x'\)=7. Untuk lebih memahaminya pembaca
dapat mengunduh tabel distribusi binomial pada laman
\href{https://onlinepubs.trb.org/onlinepubs/nchrp/cd-22/manual/v2appendixc.pdf}{berikut}.
Nilai \(x'\)=7 diperoleh menggunakan Tabel distribusi binomial dengan
\(n\)=25 dan \(p\)=0,5 yang ditampilkan pada Gambar \ref{fig:tabbinom}
dengan nilai probabilitas sebesar 0,022 (mendekati 0,025) yang setara
dengan area yang diarsir pada Gambar \ref{fig:iemednp}.

\begin{figure}

{\centering \includegraphics[width=0.65\linewidth]{tabbinom} 

}

\caption{Lokasi probabilitas x berdasarkan tabel distribusi binomial}\label{fig:tabbinom}
\end{figure}

Berdasarkan Persamaan \eqref{eq:rl} dan Persamaan \eqref{eq:ru}, rangking
\(R_l\) pada observasi yang menyatakan batas kepercayaan bawah
(\emph{lower confidence limit}) adalah 8 (\(R_i\)=7+1) dan \(R_u\) yang
menyatakan batas kepercayaan atas (\emph{upper confidence level}) adalah
25-7=18. Berdasarkan nilai probabilitas \(x'\)=0,022, maka nilai alpha
yang sesunggunya sebesar \(\alpha=2*0,022=0,044\). Nilai tersebut setara
dengan tingkat kepercayaan \(1-0,044\) atau 95,6\%. Nilai interval
kepercayaan median antara observasi ke-8 dan 18 adalah
\(C_l=4,8\le C_{0.5}\le110=C_u\ \ pada\ \alpha=0,044\). Nilai asimetrik
disekitar \(\hat{C}_{0.5}\)=19 mencerminkan kemencengan pada data.

Jika pembaca ingin melakukan perhitungan pada \texttt{R}, pembaca harus
membuat fungsi sebagai berikut:

\begin{Shaded}
\begin{Highlighting}[]
\NormalTok{med_npCI <-}\StringTok{ }\ControlFlowTok{function}\NormalTok{(x,alpha)\{}
  \CommentTok{# mengurutkan data}
\NormalTok{  x_sort=}\KeywordTok{sort}\NormalTok{(x)}
  \CommentTok{# menghitung jumlah observasi}
\NormalTok{  n=}\KeywordTok{length}\NormalTok{(x)}
  \CommentTok{# menghitung median data}
\NormalTok{  med =}\StringTok{ }\KeywordTok{median}\NormalTok{(x)}
  \CommentTok{# loop untuk mencari nilai probabilitas terdekat}
  \CommentTok{# dengan alpha}
  \ControlFlowTok{for}\NormalTok{(i }\ControlFlowTok{in} \DecValTok{1}\OperatorTok{:}\NormalTok{n)\{}
\NormalTok{    b =}\StringTok{ }\KeywordTok{pbinom}\NormalTok{(i,n,}\FloatTok{0.5}\NormalTok{)}
    \ControlFlowTok{if}\NormalTok{(b}\OperatorTok{>}\NormalTok{alpha}\OperatorTok{/}\DecValTok{2}\NormalTok{)\{}
      \ControlFlowTok{break}
\NormalTok{    \}}
\NormalTok{  \}}
  \CommentTok{# mengambil x'}
\NormalTok{  x_i=i}\OperatorTok{-}\DecValTok{1}
  \CommentTok{# menghitung Rl dan Ru}
\NormalTok{  rl=x_i}\OperatorTok{+}\DecValTok{1}
\NormalTok{  ru=n}\OperatorTok{-}\NormalTok{x_i}
  \CommentTok{# menghitung true confidence level}
\NormalTok{  CL=}\DecValTok{1}\OperatorTok{-}\DecValTok{2}\OperatorTok{*}\NormalTok{(}\KeywordTok{pbinom}\NormalTok{(x_i,n,}\FloatTok{0.5}\NormalTok{))}
  \CommentTok{# menghitung Lower dan Upper CL}
\NormalTok{  LCL=x_sort[rl]}
\NormalTok{  UCL=x_sort[ru]}
  \CommentTok{# menggabungkannya dalam satu data}
\NormalTok{  data=}\KeywordTok{data.frame}\NormalTok{(}\StringTok{"median"}\NormalTok{=med,}
                  \StringTok{"True CL %"}\NormalTok{=CL}\OperatorTok{*}\DecValTok{100}\NormalTok{,}
                  \StringTok{"Lower CL"}\NormalTok{=LCL,}
                  \StringTok{"Upper CL"}\NormalTok{=UCL)}
  \KeywordTok{return}\NormalTok{(data)}
\NormalTok{\}}
\end{Highlighting}
\end{Shaded}

Fungsi yang telah dibuat tersebut selanjutnya dapat pembaca gunakan saat
akan menghitung interval kepercayaan median populasi menggunakan metode
Nonparametrik. Berikut penerapannya menggunakan data pada Tabel
\ref{tab:gwardat} yang telah disimpan kedalam objek \texttt{gwardat}.

\begin{Shaded}
\begin{Highlighting}[]
\KeywordTok{med_npCI}\NormalTok{(}\DataTypeTok{x=}\NormalTok{gwardat}\OperatorTok{$}\NormalTok{konsentrasi, }\DataTypeTok{alpha=}\FloatTok{0.05}\NormalTok{)}
\end{Highlighting}
\end{Shaded}

\begin{verbatim}
##   median True.CL.. Lower.CL Upper.CL
## 1     19     95.67      4.8      110
\end{verbatim}

Alternatif lain yang dapat digunakan untuk menghitung interval
kepercayaan jika sampel cukup besar n\textgreater{}20 menggunakan metode
Nonparametrik adalah dengan menggunakan pendekatan tabel distribusi
normal untuk memperkirakan distribusi binomial. Dengan menggunakan
pendekatan ini, hanya sebagian kecil tabel distribusi binomial (n=20)
yang diperlukan untuk melakukannya. Nilai kritis \(z_{\alpha/2}\) dari
tabel distribusi normal menentukan rangking atas dan bawah observasi
yang menyatakan awal dan akhir nilai interval kepercayaan yang
dinyatakan pada Persamaan \eqref{eq:rlz} dan Persamaan \eqref{eq:ruz}.
Pembulatan diperlukan dalam proses ini sebab nilai ranking harus berupa
integer.

\begin{equation}
  R_l=\frac{n-z_{\frac{\alpha}{2}}\cdot\sqrt{n}}{2} 
  \label{eq:rlz}
\end{equation}

\begin{equation}
   R_l=\frac{n-z_{\frac{\alpha}{2}}\cdot\sqrt{n}}{2}+1
  \label{eq:ruz}
\end{equation}

Menggunakan contoh data pada Tabel \ref{tab:gwardat}, dengan 95\%
interval kepercayaan (\(z_{\alpha/2}\)=1,96) median dapat dihitung
seperti berikut:

\[
  R_l=\frac{25-1,96\cdot\sqrt{25}}{2}=7,6 
\]

\[
  R_l=\frac{25+1,96\cdot\sqrt{25}}{2}+1=18,4 
\]

Berdasarkan hasil perhitungan diperoleh rangking bawah adalah data ke-8
dan rangking atas adalah 18. Kedua data dibulatkan berdasarkan integer
terdekat. Nilai interval kepercayaan median yang diperoleh sama dengan
metode sebelumnya sebab rangking batas bawah dan atasnya yang seragam.

Jika pembaca ingin menggunakan \texttt{R}, maka fungsi yang sebelumya
telah kita buat dapat dimodifikasi seperti berikut:

\begin{Shaded}
\begin{Highlighting}[]
\NormalTok{med_norCI <-}\StringTok{ }\ControlFlowTok{function}\NormalTok{(x, alpha)\{}
  \CommentTok{# mengurutkan data dari yang terkecil}
\NormalTok{  x_sort=}\KeywordTok{sort}\NormalTok{(x)}
  \CommentTok{# menghitung jumlah observasi}
\NormalTok{  n =}\StringTok{ }\KeywordTok{length}\NormalTok{(x)}
  \CommentTok{# menghitung median data}
\NormalTok{  med =}\StringTok{ }\KeywordTok{median}\NormalTok{(x)}
  \CommentTok{# menghitung Rl dan Ru}
\NormalTok{  rl=}\KeywordTok{round}\NormalTok{((n}\OperatorTok{-}\KeywordTok{abs}\NormalTok{(}\KeywordTok{qnorm}\NormalTok{(alpha}\OperatorTok{/}\DecValTok{2}\NormalTok{))}\OperatorTok{*}\KeywordTok{sqrt}\NormalTok{(n))}\OperatorTok{/}\DecValTok{2}\NormalTok{,}\DecValTok{0}\NormalTok{)}
\NormalTok{  ru=}\KeywordTok{round}\NormalTok{(((n}\OperatorTok{+}\KeywordTok{abs}\NormalTok{(}\KeywordTok{qnorm}\NormalTok{(alpha}\OperatorTok{/}\DecValTok{2}\NormalTok{))}\OperatorTok{*}\KeywordTok{sqrt}\NormalTok{(n))}\OperatorTok{/}\DecValTok{2}\NormalTok{)}\OperatorTok{+}\DecValTok{1}\NormalTok{,}\DecValTok{0}\NormalTok{)}
  \CommentTok{# menghitung Lower dan Upper CL}
\NormalTok{  LCL=x_sort[rl]}
\NormalTok{  UCL=x_sort[ru]}
  \CommentTok{# menggabungkannya dalam satu data}
\NormalTok{  data=}\KeywordTok{data.frame}\NormalTok{(}\StringTok{"median"}\NormalTok{=med,}
                  \StringTok{"Lower CL"}\NormalTok{=LCL,}
                  \StringTok{"Upper CL"}\NormalTok{=UCL)}
  \KeywordTok{return}\NormalTok{(data)}
\NormalTok{\}}
\end{Highlighting}
\end{Shaded}

\begin{Shaded}
\begin{Highlighting}[]
\KeywordTok{med_norCI}\NormalTok{(}\DataTypeTok{x=}\NormalTok{gwardat}\OperatorTok{$}\NormalTok{konsentrasi, }\DataTypeTok{alpha=}\FloatTok{0.05}\NormalTok{)}
\end{Highlighting}
\end{Shaded}

\begin{verbatim}
##   median Lower.CL Upper.CL
## 1     19      4.8      110
\end{verbatim}

\subsection{Metode Parametrik Interval Estimasi
Median}\label{metode-parametrik-interval-estimasi-median}

Telah dijelaskan pada chapter 6 bahwa rata-rata geometrik merupakan
merupakan nilai rata-rata yang digunakan untuk mengestimasi median
sampel untuk data yang memiliki kemencengan dengan transformasi yang
digunakan agar data simetris adalah transformasi logaritmik
\(y=\ln(x)\). Pada metode ini data diasumsikan memiliki distribusi
lognormal (kemencengan positif). Rerata dan interval geometris akan
lebih efisien (interval lebih pendek) dari median dan interval
kepercayaannya ketika data benar-benar lognormal. Median sampel dan
intervalnya lebih tepat dan lebih efisien jika logaritma data masih
menunjukkan kemencengan dan/atau \emph{outlier}. Estimasi media
menggunakan metode parametrik dituliskan kedalam Persamaan \eqref{eq:gmci}
dan Persamaan \eqref{eq:gmci2}.

\begin{equation}
  GM_x=\exp\left(\overline{y}\right) 
  \label{eq:gmci}
\end{equation}

dimana \(y=\ln(x)\) dan \(\overline{y}\)=mean sampel \(y\).

\begin{equation}
  \exp\left(\overline{y}-t_{\left(\frac{\alpha}{2},n-1\right)}\sqrt{\frac{s_y^2}{n}}\right)\le GM_x\le\exp\left(\overline{y}+t_{\left(\frac{\alpha}{2},n-1\right)}\sqrt{\frac{s_y^2}{n}}\right) 
  \label{eq:gmci2}
\end{equation}

dimana \(s_{y}^2\)= varians sampel y pada unit log natural.

Pada Tabel \ref{tab:gwardat}, untuk menghitung interval keyakinan median
menggunakan pendekatan mean geometrik \(GM_x\) kita perlu
mentransformasi datanya terlebih dahulu sehingga menjadi bentuk natural
log. hasil transformasi disajikan pada Tabel \ref{tab:gwardat2}.

\begin{table}[t]

\caption{\label{tab:gwardat2}Tranformasi logaritmik konsentrasi Arsenik dalam air tanah (ppb)}
\centering
\begin{tabular}{r|r}
\hline
observasi & konsentrasi\\
\hline
1 & 0.2624\\
\hline
2 & 0.4055\\
\hline
3 & 0.5878\\
\hline
4 & 0.9555\\
\hline
5 & 1.0296\\
\hline
6 & 1.2528\\
\hline
7 & 1.3863\\
\hline
8 & 1.5686\\
\hline
9 & 2.0794\\
\hline
10 & 2.2513\\
\hline
11 & 2.4849\\
\hline
12 & 2.6391\\
\hline
13 & 2.9444\\
\hline
14 & 3.1355\\
\hline
15 & 3.7136\\
\hline
16 & 4.3820\\
\hline
17 & 4.6052\\
\hline
18 & 4.7005\\
\hline
19 & 4.7875\\
\hline
20 & 5.2470\\
\hline
21 & 5.4806\\
\hline
22 & 5.5215\\
\hline
23 & 5.7038\\
\hline
24 & 5.8289\\
\hline
25 & 6.3630\\
\hline
\end{tabular}
\end{table}

visualisasi distribusi yang baru disajikan pada Gambar
\ref{fig:gwardatvis2}.

\begin{figure}

{\centering \includegraphics[width=0.7\linewidth]{EnvStat_files/figure-latex/gwardatvis2-1} 

}

\caption{Distribusi logaritmik konsentrasi arsenik dalam air tanah}\label{fig:gwardatvis2}
\end{figure}

Nilai mean dari data tersebut adalah 3,17 dengan simpangan baku sebesar
1,96. Berdasarkan Gambar \ref{fig:gwardatvis2}, kita telah memperoleh
distribusi yang simetris.

Dengan menggunakan Persamaan \eqref{eq:gmci} dan Persamaan \eqref{eq:gmci2}
selanjutnya dapat dihitung interval kepercayaannya dengan derajat
kepercayaan 95\%.

\[
  GM_x=\exp\left(3,17\right)=23,8
\]

\[
  \exp\left(3.17-2,064\cdot\sqrt{\frac{1.96^2}{25}}\right)\le GM_x\le\exp\left(3.17-2,064\cdot\sqrt{\frac{1.96^2}{25}}\right)
\]

\[
  \exp\left(2,36\right)\le GM_x\le\exp\left(3,98\right)
\]

\[
  10,6\le GM_x\le53,5
\]

Dengan menggunakan \texttt{R} dapat dikerjakan menggunakan fungsi
sebagai berikut:

\begin{Shaded}
\begin{Highlighting}[]
\CommentTok{# persamaan hanya dapat digunakan untuk data yang telah ditransformasi lognormal}

\NormalTok{med_gm <-}\StringTok{ }\ControlFlowTok{function}\NormalTok{(x, alpha)\{}
  \CommentTok{# rata-rata geometrik}
\NormalTok{  gm =}\StringTok{ }\KeywordTok{exp}\NormalTok{(}\KeywordTok{mean}\NormalTok{(x))}
  \CommentTok{# menghitung derajat kebebasan}
\NormalTok{  df =}\StringTok{ }\KeywordTok{length}\NormalTok{(x)}\OperatorTok{-}\DecValTok{1}
  \CommentTok{# menghitung batas bawah dan atas}
\NormalTok{  LCL =}\StringTok{ }\KeywordTok{exp}\NormalTok{(}\KeywordTok{mean}\NormalTok{(x)}\OperatorTok{-}\KeywordTok{qt}\NormalTok{(}\DecValTok{1}\OperatorTok{-}\NormalTok{alpha}\OperatorTok{/}\DecValTok{2}\NormalTok{,df)}\OperatorTok{*}\KeywordTok{sqrt}\NormalTok{(}\KeywordTok{var}\NormalTok{(x)}\OperatorTok{/}\KeywordTok{length}\NormalTok{(x)))}
\NormalTok{  UCL =}\StringTok{ }\KeywordTok{exp}\NormalTok{(}\KeywordTok{mean}\NormalTok{(x)}\OperatorTok{+}\KeywordTok{qt}\NormalTok{(}\DecValTok{1}\OperatorTok{-}\NormalTok{alpha}\OperatorTok{/}\DecValTok{2}\NormalTok{,df)}\OperatorTok{*}\KeywordTok{sqrt}\NormalTok{(}\KeywordTok{var}\NormalTok{(x)}\OperatorTok{/}\KeywordTok{length}\NormalTok{(x)))}
  \CommentTok{# menggabungkan hasil}
\NormalTok{  data=}\KeywordTok{data.frame}\NormalTok{(}\StringTok{"GM"}\NormalTok{=gm,}
                  \StringTok{"Lower CL"}\NormalTok{=LCL,}
                  \StringTok{"Upper CL"}\NormalTok{=UCL)}
  \KeywordTok{return}\NormalTok{(data)}
\NormalTok{\}}
\end{Highlighting}
\end{Shaded}

\begin{Shaded}
\begin{Highlighting}[]
\KeywordTok{med_gm}\NormalTok{(}\DataTypeTok{x=}\NormalTok{gwardat2}\OperatorTok{$}\NormalTok{konsentrasi, }\DataTypeTok{alpha=}\FloatTok{0.05}\NormalTok{)}
\end{Highlighting}
\end{Shaded}

\begin{verbatim}
##      GM Lower.CL Upper.CL
## 1 23.87    10.63     53.6
\end{verbatim}

Pembaca perlu berhati-hati dalam menentukan apakah akan menggunakan
metode Nonparametrik atau parametrik. Jika data berditribusi lognormal
kita dapat menggunakan metode parametrik.

\section{Interval Kepercayaan Mean}\label{interval-kepercayaan-mean}

Estimasi interval juga dapat dihitung untuk mean populasi sebenarnya
\(\mu\). Hal ini sangat sesuai jika pusat data menjadi fokus dalam
analisa statistik. Interval simetris di sekitar sampel rata-rata X
paling sering dihitung. Untuk ukuran sampel besar, interval simetris
secara memadai menggambarkan variasi rata-rata, terlepas dari bentuk
distribusi data. Ini karena distribusi rata-rata sampel akan mendekati
dengan distribusi normal ketika ukuran sampel semakin besar, meskipun
data mungkin tidak terdistribusi secara normal. Untuk ukuran sampel yang
lebih kecil, rata-rata tidak akan didistribusikan secara normal kecuali
jika data itu sendiri terdistribusi secara normal. Ketika data meningkat
kemencengannya, lebih banyak data diperlukan sebelum distribusi
rata-rata dapat didekati secara memadai oleh distribusi normal. Untuk
distribusi yang sangat miring atau data yang mengandung \emph{outlier},
mungkin diperlukan lebih dari 100 pengamatan sebelum nilai rata-rata
tidak akan terpengaruh oleh nilai terbesar untuk mengasumsikan bahwa
distribusinya akan simetris.

\subsection{Interval Kepercayaan Mean Untuk Distribusi Yang
Simetris}\label{interval-kepercayaan-mean-untuk-distribusi-yang-simetris}

Interval kepercayaan mean untuk distribusi simetris dihitung menggunakan
tabel distribusi \emph{student's t} yang tersedia dalam buku teks
statistik dan perangkat lunak. Tabel ini dimasukkan untuk menemukan
nilai kritis untuk t pada setengah tingkat alfa yang diinginkan. Pada
buku lain sering dijelaskan bahwa distribusi t hanya digunakan untuk
sampel kecil (beberapa menyebutkan n\textless{}30), sedangkan untuk
distribusi besar digunakan distribusi normal. Penggunaan distribusi
normal jarang digunakan dalam prakiraan. Hal ini disebabkan karena pada
proses perhitungan diperlukan nilai simpangan baku \(\sigma\). Pada
kenyataannya pada pengukuran dilapangan kita sering sekalin melakukan
estimasi terhadap simpangan baku melalui sampel \(s\) karena kita tidak
mengetahui nilai simpangan baku populasinya sehingga pada buku ini akan
digali lebih jauh metode estimasi interval menggunakan persamaan
distribusi t.

Lebar interval kepercayaan adalah fungsi dari nilai-nilai kritis dari
tabel distribusi t, standar deviasi data, dan ukuran sampel. Ketika data
memiliki kemecengan atau mengandung \emph{outlier}, asumsi di balik
interval t dan distribusi normal tidak berlaku. Interval simetris yang
dihasilkan akan sangat luas sehingga sebagian besar pengamatan akan
dimasukkan di dalamnya. Ini juga dapat mencapai di bawah nol di ujung
bawah. Titik akhir negatif dari interval kepercayaan untuk data yang
tidak dapat menjadi negatif adalah sinyal yang jelas bahwa asumsi
interval kepercayaan simetris tidak diperlukan. Untuk data tersebut,
dengan asumsi distribusi lognormal seperti yang dijelaskan dalam sub
chapter sebelumnya (interval kepercayaan median) akan lebih tepat.
Interval kepercayaan dihitung menggunakan Persamaan \eqref{eq:pmeancisim}.

\begin{equation}
  \overline{x}-t_{\left(\frac{\alpha}{2},n-1\right)}\cdot\sqrt{\frac{s^2}{n}}\le\mu\le\overline{x}+t_{\left(\frac{\alpha}{2},n-1\right)}\cdot\sqrt{\frac{s^2}{n}}
  \label{eq:pmeancisim}
\end{equation}

Untuk lebih memahami cara penerapannya, kita akan menggunakan kembali
data pada Tabel \ref{tab:gwardat2}. Langkah pertama yang perlu dilakukan
adalah menghitung mean sampel \(\overline{x}\) dan simpangan baku sampel
\(s\). Berdasarkan hasil perhitungan diperoleh nilai \(\overline{x}\)
=98.352 dan \(s\) = 144.685. Dengan meggunakan Persamaan
\eqref{eq:pmeancisim}, interval estimasi mean dengan tingkat kepercayaan
95\% dapat dihitung sebagai berikut:

\[
  3,17\ -\ t_{\left(0.25,24\right)}\cdot\sqrt{\frac{1,96^2}{25}}\le\mu\le3,17\ -\ t_{\left(0.25,24\right)}\cdot\sqrt{\frac{1,96^2}{25}}
\]

\[
  2,36\le\mu\le3,98
\]

Berdasarkan hasil yang diperoleh terdapat 95\% peluang nilai mean
populasi \(\mu\) terletak pada interval 2,36 sampai 3,98. Perlu diingat
bahwa metode parametrik sangat sensitif dengan adanya \emph{outlier}
sehingga jika pembaca ingin menggunakannya pastikan terlebih dahulu
tidak ada \emph{outlier} pada data dengan cara melakukan visualisasi
data.

Pada \texttt{R} kita dapat menggunakan fungsi \texttt{stat.desc()} untuk
menhitung statistika deskriptif serta interval kepercayaan mean-nya.
Berikut adalah sintaks yang digunakan:

\begin{Shaded}
\begin{Highlighting}[]
\CommentTok{# memuat paket}
\KeywordTok{library}\NormalTok{(pastecs)}

\CommentTok{# ringkasan data}
\NormalTok{r=}\KeywordTok{stat.desc}\NormalTok{(gwardat2}\OperatorTok{$}\NormalTok{konsentrasi)}
\NormalTok{r}
\end{Highlighting}
\end{Shaded}

\begin{verbatim}
##      nbr.val     nbr.null       nbr.na          min 
##      25.0000       0.0000       0.0000       0.2624 
##          max        range          sum       median 
##       6.3630       6.1007      79.3167       2.9444 
##         mean      SE.mean CI.mean.0.95          var 
##       3.1727       0.3919       0.8089       3.8400 
##      std.dev     coef.var 
##       1.9596       0.6176
\end{verbatim}

\begin{Shaded}
\begin{Highlighting}[]
\CommentTok{# batas bawah (LCL)}
\KeywordTok{mean}\NormalTok{(gwardat2}\OperatorTok{$}\NormalTok{konsentrasi)}\OperatorTok{-}\NormalTok{r[[}\DecValTok{11}\NormalTok{]]}
\end{Highlighting}
\end{Shaded}

\begin{verbatim}
## [1] 2.364
\end{verbatim}

\begin{Shaded}
\begin{Highlighting}[]
\CommentTok{# batas atas (UCL)}
\KeywordTok{mean}\NormalTok{(gwardat2}\OperatorTok{$}\NormalTok{konsentrasi)}\OperatorTok{+}\NormalTok{r[[}\DecValTok{11}\NormalTok{]]}
\end{Highlighting}
\end{Shaded}

\begin{verbatim}
## [1] 3.982
\end{verbatim}

\subsection{Interval Kepercayaan Mean Untuk Distribusi Yang
Asimetris}\label{interval-kepercayaan-mean-untuk-distribusi-yang-asimetris}

Mean dan interval kepercayaan dapat dihitung dengan mengasumsikan
distribusi data mengikuti distribusi logaritmik \(y=\ln{(x)}\). Metode
ini berguna untuk jenis data yang memiliki bentuk distribusi data yang
memiliki kemencengan positif (perlu transformasi logaritmik agar
simetris).Metode ini memberikan perkiraan rata-rata yang lebih dapat
diandalkan (varians lebih rendah) daripada perhitungan rata-rata sampel
biasa tanpa transformasi log.

Untuk memperkirakan rata-rata populasi \(\mu_x\) dalam unit aslinya,
anggap datanya berdistribusi normal. Satu-setengah varians logaritma
ditambahkan ke \(\overline{y}\) (rata-rata log) sebelum eksponensial.
Karena varianss sampel \(s^2_y\) hanya perkiraan varianss sebenarnya
dari logaritma, estimasi sampel rata-rata akan menjadi bias. Namun,
untuk sampel dengan \(s^2_y\) kecil dan ukuran sampel besar bias dapat
diabaikan. Interval kepercayaan dapat dituliskan berdasarkan Persamaan
\eqref{eq:pmeanciasim}.

\begin{equation}
  \mu_x=\exp\left(\overline{y}+0,5\cdot s_y^2\right)
  \label{eq:pmeanciasim}
\end{equation}

dimana \(y=\ln{(x)}\), \(\overline{y}\)= mean sampel dan \(s^2_y\)=
varians sampel y dalam unit log natural.

Interval kepercayaan sekitar \(\mu_x\) bukan estimasi interval yang
dihitung untuk rata-rata geometri dalam Persamaan \eqref{eq:gmci2}.
Interval kepercayaan tidak dapat dihitung hanya dengan mengekspansi
interval sekitar \(\overline{y}\). Interval kepercayaan yang tepat dalam
satuan asli untuk rata-rata data lognormal dapat dihitung. Untuk lebih
jelasnya pembaca dapat melihatnya pada situs
\url{http://jse.amstat.org/v13n1/olsson.html}.

Metode Cox dapat digunakan untuk menghitung interval keyakinan dengan
nilai estimasi rata-rata menggunakan Persamaan \eqref{eq:pmeanciasim}.
Persamaan yang digunakan dapat dituliskan sebagai berikut (Persamaan
\eqref{eq:pmeanciasimci}).

\begin{equation}
  \ln\left(\mu_x\right)=\overline{Y}+\frac{s_y^2}{2}\pm z_{\left(\frac{\alpha}{2}\right)}\sqrt{\frac{s_y^2}{n}+\frac{s_y^4}{2\left(n-1\right)}}
  \label{eq:pmeanciasimci}
\end{equation}

Persamaan \eqref{eq:pmeanciasimci} dapat dimodifikasi dengan menggunakan
distribusi t dibanding menggunakan distribusi normal. Penggunaan
distribusi t akan memperbaiki kelemahan penggunaan distribusi normal
pada sampel yang berukuran kecil.

Data Tabel \ref{tab:gwardat} dapat kita gunakan untuk menghitung
rata-rata menggunakan Persamaan \eqref{eq:pmeanciasimci}. Hal ini
disebabkan karena data yang ada memiliki kemencengan positif sehingga
dapat dianggap bahwa transformasi logaritmik dapat membentuk distribusi
ini menjadi lebih simetris.

Berdasarkan hasil perhitungan diperoleh nilai \(\overline{Y}\) =3.173
dan \(s^2_y\) = 1.96. Sehingga nilai interval selanjutnya dapat dihitung
menggunakan Persamaan \eqref{eq:pmeanciasimci} dengan interval keyakinan
95\%.

\[
  \ln\left(\mu_x\right)=3,17+\frac{1,96^2}{2}\pm1,96\sqrt{\frac{1,96^2}{25}+\frac{1,96^4}{2\left(25-1\right)}}
\]

\[
  \ln\left(\mu_x\right)=5,10\pm1,33
\]

Sehingga

\[
  \exp\left(5,10-1,33\right)\le\mu_x\le\exp\left(5,10+1,33\right)
\]

\[
  43,38\le\mu_x\le620,17
\]

Nilai interval yang dihasilkan sangat panjang sehingga nilai rata-rata
yang dihasilkan tidak dapat diandalkan untuk memperkirankan lokasi nilai
mean populasi.

Pada contoh berikut akan disajikan sintaks untuk menghitung interval
kepercayaan mean data pada Tabel \ref{tab:gwardat} berdasarkan Persamaan
\eqref{eq:pmeanciasimci} dan sitribusi yang digunakan adalah distribusi t.
Pembaca dapat memodifikasi sintaks berikut jika ingin menggunakan
distribusi normal.

\begin{Shaded}
\begin{Highlighting}[]
\NormalTok{mean_asci<-}\ControlFlowTok{function}\NormalTok{(x,alpha)\{}
\NormalTok{  m=}\KeywordTok{mean}\NormalTok{(x)}
  \CommentTok{# mean data hasil transformasi logaritmik}
\NormalTok{  ave =}\StringTok{ }\KeywordTok{mean}\NormalTok{(}\KeywordTok{log}\NormalTok{(x))}
  \CommentTok{# simpangan baku data hasil transformasi}
\NormalTok{  sd =}\StringTok{ }\KeywordTok{sd}\NormalTok{(}\KeywordTok{log}\NormalTok{(x))}
  \CommentTok{# jumlah observasi}
\NormalTok{  n =}\StringTok{ }\KeywordTok{length}\NormalTok{(x)}
  \CommentTok{# derajat kebebasa}
\NormalTok{  df =}\StringTok{ }\NormalTok{n}\OperatorTok{-}\DecValTok{1}
  \CommentTok{# interval keyakinan satu sisi}
\NormalTok{  re =}\StringTok{ }\DecValTok{1}\OperatorTok{-}\NormalTok{(alpha}\OperatorTok{/}\DecValTok{2}\NormalTok{)}
  \CommentTok{# CI menggunakan distribusi t}
\NormalTok{  LCL =}\StringTok{ }\KeywordTok{exp}\NormalTok{(ave}\OperatorTok{+}\NormalTok{(}\FloatTok{0.5}\OperatorTok{*}\NormalTok{sd}\OperatorTok{^}\DecValTok{2}\NormalTok{)}\OperatorTok{-}\KeywordTok{qt}\NormalTok{(re,df)}\OperatorTok{*}\KeywordTok{sqrt}\NormalTok{(((sd}\OperatorTok{^}\DecValTok{2}\NormalTok{)}\OperatorTok{/}\NormalTok{n)}\OperatorTok{+}\NormalTok{((sd}\OperatorTok{^}\DecValTok{4}\NormalTok{)}\OperatorTok{/}\NormalTok{(}\DecValTok{2}\OperatorTok{*}\NormalTok{df))))}
\NormalTok{  UCL =}\StringTok{ }\KeywordTok{exp}\NormalTok{(ave}\OperatorTok{+}\NormalTok{(}\FloatTok{0.5}\OperatorTok{*}\NormalTok{sd}\OperatorTok{^}\DecValTok{2}\NormalTok{)}\OperatorTok{+}\KeywordTok{qt}\NormalTok{(re,df)}\OperatorTok{*}\KeywordTok{sqrt}\NormalTok{(((sd}\OperatorTok{^}\DecValTok{2}\NormalTok{)}\OperatorTok{/}\NormalTok{n)}\OperatorTok{+}\NormalTok{((sd}\OperatorTok{^}\DecValTok{4}\NormalTok{)}\OperatorTok{/}\NormalTok{(}\DecValTok{2}\OperatorTok{*}\NormalTok{df))))}
  \CommentTok{# menggabungkan hasil}
\NormalTok{  data =}\StringTok{ }\KeywordTok{data.frame}\NormalTok{(}\StringTok{"Mean"}\NormalTok{=m,}
                  \StringTok{"Lower CL"}\NormalTok{=LCL,}
                  \StringTok{"Upper CL"}\NormalTok{=UCL)}
  \KeywordTok{return}\NormalTok{(data)}
\NormalTok{\}}
\end{Highlighting}
\end{Shaded}

\begin{Shaded}
\begin{Highlighting}[]
\KeywordTok{mean_asci}\NormalTok{(}\DataTypeTok{x=}\NormalTok{gwardat}\OperatorTok{$}\NormalTok{konsentrasi, }\DataTypeTok{alpha=}\FloatTok{0.05}\NormalTok{)}
\end{Highlighting}
\end{Shaded}

\begin{verbatim}
##    Mean Lower.CL Upper.CL
## 1 98.35    40.11    660.9
\end{verbatim}

\section{Interval Prediksi
Nonparametrik}\label{interval-prediksi-nonparametrik}

Pertanyaan yang sering diajukan adalah apakah satu pengamatan baru
kemungkinan berasal dari distribusi yang sama dengan data yang
dikumpulkan sebelumnya, atau sebagai alternatif dari distribusi yang
berbeda. Pertanyaan dapat dievaluasi dengan menentukan apakah pengamatan
baru di luar interval prediksi yang dihitung dari data yang ada.
Interval prediksi mengandung \(100\cdot\left(1-\alpha\right)\) persen
dari distribusi data, sementara \(100\cdot\alpha\) persen berada di luar
interval. Jika pengamatan baru datang dari distribusi yang sama dengan
data yang diukur sebelumnya, ada kemungkinan \(100\cdot\alpha\) persen
bahwa pengamatan baru tersebut akan berada di luar interval prediksi.
Karena pengamatan baru tersebut berada di luar interval tidak
``membuktikan'' pengamatan baru itu berbeda, hanya saja sepertinya
begitu. Seberapa besar kemungkinan ini tergantung pada pilihan
\(\alpha\) yang ditentukan oleh peneliti.

Interval prediksi dihitung dengan tujuan yang berbeda dari interval
kepercayaan. Interval prediksi terkait dengan nilai data individu yang
berlawanan dengan ringkasan statistik seperti nilai mean. Interval
prediksi lebih luas daripada interval kepercayaan, karena pengamatan
individu lebih bervariasi daripada ringkasan statistik yang dihitung
dari beberapa pengamatan. Tidak seperti interval kepercayaan, interval
prediksi memperhitungkan variabilitas titik data tunggal di sekitar
median atau rata-rata, di samping kesalahan dalam memperkirakan pusat
distribusi. Ketika mean \(\pm\) 2 standar deviasi secara keliru
digunakan untuk memperkirakan lebar interval prediksi, data baru
dinyatakan berasal dari populasi yang berbeda lebih sering daripada yang
seharusnya.

\subsection{Interval Prediksi Nonparametrik Dua
Sisi}\label{interval-prediksi-nonparametrik-dua-sisi}

Interval prediksi tingkat kepercayaan nonparametrik \(\alpha\) secara
sederhana dinyatakan sebagai interval antara persentil distribusi
\(\alpha/2\) dan \(1-\left(\frac{\alpha}{2}\right)\) (Gambar
\ref{fig:ipnds}). Interval ini mengandung
\(100\cdot\left(1-\alpha\right)\) data, sedangkan \(100\cdot\alpha\)
persen berada di luar interval. Oleh karena itu jika titik data tambahan
baru berasal dari distribusi yang sama dengan data yang diukur
sebelumnya, ada kemungkinan \(100\cdot\alpha\) persen bahwa itu akan
berada di luar interval prediksi. Interval akan mencerminkan bentuk data
yang dikembangkannya, dan tidak ada asumsi tentang bentuk bentuk yang
perlu dibuat. Interval prediksi nonparametrik dua sisi dinyatakan
berdasarkan Persamaan \eqref{eq:eqipnds}.

\begin{equation}
  PI_{np}=X_{\frac{\alpha}{2}\cdot\left(n+1\right)}\ sampai\ dengan\ X_{\left[1-\left(\frac{\alpha}{2}\right)\right]\cdot\left(n+1\right)}
  \label{eq:eqipnds}
\end{equation}

\begin{figure}

{\centering \includegraphics[width=0.65\linewidth]{ipnds} 

}

\caption{Prediksi interval dua sisi}\label{fig:ipnds}
\end{figure}

Kita akan kembali menggunakan data pada Tabel \ref{tab:gwardat}. Dengan
menggunakan tingkat kepercayaan 90\% kita diminta untuk menentukan
interval prediksi dari konsentrasi arsenik pada data tersebut tanpa
mengasumsikan distribusi dari data.

Untuk melakukannya kita perlu menentukan observasi ke-2,5 dan 97,5
(berdasarkan nilai \(\alpha/2\)) dengan rangking observasi berdasarkan
Persamaan \eqref{eq:eqipnds} adalah \((0,05*26)\) atau rangking observasi
antara observasi 1 (\(R_1\)) dan 2 (\(R_2\)) dan \((0,95*26)\) rangking
observasi antara observasi 24 (\(R_{24}\)) dan 25 (\(R_{25}\)). Dengan
menggunakan interpolasi linier pada observasi ke-1, 2 , 24 dan 25,
interval prediksi yang diperoleh adalah sebagai berikut:

\[
  X_1+\left(\frac{R_{\left(0.05\cdot26\right)}-R_1}{R_2-R_1}\right)\cdot\left(X_2-X_1\right)\ sampai\ dengan\ X_{24}+\left(\frac{R_{\left(0.95\cdot26\right)}-R_{24}}{R_{25}-R_{24}}\right)\cdot\left(X_{25}-X_4\right)
\]

\[
  1,3+\left(\frac{1,3-1}{2-1}\right)\cdot\left(1,5-1,3\right)\ sampai\ dengan\ 340+\left(\frac{24,5-24}{25-24}\right)\cdot\left(580-340\right)
\]

\[
  1,4\ sampai\ dengan\ 508\ ppb
\]

Observasi baru diluar rentang tersebut akan dianggap berasal dari
distribusi yang berbeda dengan tingkat error sebesar 10\%
(\(\alpha\)=10\%).

Dengan menggunakan \texttt{R} pembaca depat menghitung interval prediksi
menggunakan fungsi berikut:

\begin{Shaded}
\begin{Highlighting}[]
\NormalTok{PInp <-}\StringTok{ }\ControlFlowTok{function}\NormalTok{(x, alpha)\{}
  \CommentTok{# mengurutkan data}
\NormalTok{  x_sort =}\StringTok{ }\KeywordTok{sort}\NormalTok{(x)}
  \CommentTok{# jumlah observasi}
\NormalTok{  n =}\StringTok{ }\KeywordTok{length}\NormalTok{(x)}
  \CommentTok{# menghitung alpha masing-masing sisi}
\NormalTok{  err <-}\StringTok{ }\NormalTok{alpha}\OperatorTok{/}\DecValTok{2}
  \CommentTok{# menentukan rangkin observasi sesuai alpha}
\NormalTok{  rl =}\StringTok{ }\NormalTok{err}\OperatorTok{*}\NormalTok{(n}\OperatorTok{+}\DecValTok{1}\NormalTok{)}
\NormalTok{  ru =}\StringTok{ }\NormalTok{(}\DecValTok{1}\OperatorTok{-}\NormalTok{err)}\OperatorTok{*}\NormalTok{(n}\OperatorTok{+}\DecValTok{1}\NormalTok{)}
  \CommentTok{# menentukan observasi untuk interpolasi linier}
\NormalTok{  rl_}\DecValTok{1}\NormalTok{=}\StringTok{ }\KeywordTok{ceiling}\NormalTok{(rl) }\CommentTok{# bulatkan ke bawah}
\NormalTok{  rl_}\DecValTok{2}\NormalTok{=}\StringTok{ }\KeywordTok{floor}\NormalTok{(rl) }\CommentTok{# bulatkan ke atas}
\NormalTok{  ru_}\DecValTok{1}\NormalTok{=}\StringTok{ }\KeywordTok{ceiling}\NormalTok{(ru) }
\NormalTok{  ru_}\DecValTok{2}\NormalTok{=}\StringTok{ }\KeywordTok{floor}\NormalTok{(ru)}
  \CommentTok{# menentukan interval prediksi}
\NormalTok{  LPI =}\StringTok{ }\KeywordTok{round}\NormalTok{(x_sort[rl_}\DecValTok{1}\NormalTok{]}\OperatorTok{+}\NormalTok{((rl}\OperatorTok{-}\NormalTok{rl_}\DecValTok{1}\NormalTok{)}\OperatorTok{/}\NormalTok{(rl_}\DecValTok{2}\OperatorTok{-}\NormalTok{rl_}\DecValTok{1}\NormalTok{))}\OperatorTok{*}\NormalTok{(x_sort[rl_}\DecValTok{2}\NormalTok{]}\OperatorTok{-}\NormalTok{x_sort[rl_}\DecValTok{1}\NormalTok{]),}\DecValTok{1}\NormalTok{)}
\NormalTok{  UPI =}\StringTok{ }\KeywordTok{round}\NormalTok{(x_sort[ru_}\DecValTok{1}\NormalTok{]}\OperatorTok{+}\NormalTok{((ru}\OperatorTok{-}\NormalTok{ru_}\DecValTok{1}\NormalTok{)}\OperatorTok{/}\NormalTok{(ru_}\DecValTok{2}\OperatorTok{-}\NormalTok{ru_}\DecValTok{1}\NormalTok{))}\OperatorTok{*}\NormalTok{(x_sort[ru_}\DecValTok{2}\NormalTok{]}\OperatorTok{-}\NormalTok{x[ru_}\DecValTok{1}\NormalTok{]),}\DecValTok{1}\NormalTok{)}
  \CommentTok{# menggabungkan hasil}
\NormalTok{  data =}\StringTok{ }\KeywordTok{data.frame}\NormalTok{(}\StringTok{"Lower PI"}\NormalTok{=LPI,}
                    \StringTok{"Upper PI"}\NormalTok{=UPI)}
  \KeywordTok{return}\NormalTok{(data)}
\NormalTok{\}}
\end{Highlighting}
\end{Shaded}

\begin{Shaded}
\begin{Highlighting}[]
\KeywordTok{PInp}\NormalTok{(}\DataTypeTok{x=}\NormalTok{gwardat}\OperatorTok{$}\NormalTok{konsentrasi, }\DataTypeTok{alpha=}\FloatTok{0.1}\NormalTok{)}
\end{Highlighting}
\end{Shaded}

\begin{verbatim}
##   Lower.PI Upper.PI
## 1      1.4      508
\end{verbatim}

\subsection{Interval Prediksi Nonparametrik Satu
Sisi}\label{interval-prediksi-nonparametrik-satu-sisi}

Interval prediksi satu sisi digunakan jika kita ingin mengecek apakah
pengamatan baru lebih besar dari data yang ada, atau lebih kecil dari
data yang ada, tetapi tidak keduanya. Keputusan untuk menggunakan
interval satu sisi harus didasarkan sepenuhnya pada pertanyaan yang
menarik. Seharusnya tidak ditentukan setelah melihat data dan memutuskan
bahwa pengamatan baru cenderung hanya lebih besar, atau hanya lebih
kecil, daripada informasi yang ada. Interval satu sisi menggunakan
\(\alpha\) dibanding \(\alpha/2\) sebagai nilai error, menempatkan semua
error di satu sisi interval (Gambar \ref{fig:ipss}). Interval prediksi
dituliskan berdasarkan Persamaan \eqref{eq:eqipss}.

\begin{figure}

{\centering \includegraphics[width=0.65\linewidth]{ipss} 

}

\caption{Prediksi interval satu sisi}\label{fig:ipss}
\end{figure}

\begin{equation}
  PI_{np}:\ x_{baru}<X_{\alpha\cdot\left(n+1\right)}\ atau\ x_{baru}>X_{\left[1-\alpha\right]\cdot\left(n+1\right)}
  \label{eq:eqipss}
\end{equation}

Untuk memahami penerapannya, misalkan kita memiliki nilai arsenik baru
dengan konsentrasi 355 ppb. Kita perlu menentukan apakah nilai tersebut
lebih besar dari sebagian besar data yang ada.

Dengan menggunakan Persamaan \eqref{eq:eqipss} dan \(\alpha\)=0,1 atau
tingkat kepercayaan 90\%, interval prediksi satu sisi atau data teratas
dari persentil ke-90 dari data yang ada adalah \(X_{0,9}*26=X_{23,4}\).
Dengan menggunakan interpolasi linier pada observasi data dengan
rangking ke-23 (\(R_{23}\)) dan 24 (\(R_{23}\)) diperoleh:

\[
  X_{23}+0,4\cdot\left(X_{24}-X_{23}\right)=300+0,4\cdot40=316\ ppb
\]

Berdasarkan data yang diperoleh diketahui bahwa batas atas dari interval
prediksi adalah 316\textless{}355 pbb, sehingga disimpulkan bahwa
konsentrasi 355 pbb lebih besar dari sebagian besar data yang ada.

Dengan menggunakan \texttt{R} interval prediksi menggunakan satu sisi
dapat dihitung menggunakan fungsi berikut:

\begin{Shaded}
\begin{Highlighting}[]
\NormalTok{PInp_os <-}\StringTok{ }\ControlFlowTok{function}\NormalTok{(x, obs, alpha, side)\{}
  \CommentTok{# mengurutkan data dari yang terkecil}
\NormalTok{  x_sort =}\StringTok{ }\KeywordTok{sort}\NormalTok{(x)}
  \CommentTok{# jumlah observasi}
\NormalTok{  n =}\StringTok{ }\KeywordTok{length}\NormalTok{(x)}
  \CommentTok{# rangking observasi}
\NormalTok{  ru =}\StringTok{ }\NormalTok{(}\DecValTok{1}\OperatorTok{-}\NormalTok{alpha)}\OperatorTok{*}\NormalTok{(n}\OperatorTok{+}\DecValTok{1}\NormalTok{)}
\NormalTok{  ru_}\DecValTok{1}\NormalTok{ =}\StringTok{ }\KeywordTok{ceiling}\NormalTok{(ru)}
\NormalTok{  ru_}\DecValTok{2}\NormalTok{ =}\StringTok{ }\KeywordTok{floor}\NormalTok{(ru)}
\NormalTok{  rl =}\StringTok{ }\NormalTok{alpha}\OperatorTok{*}\NormalTok{(n}\OperatorTok{+}\DecValTok{1}\NormalTok{)}
\NormalTok{  rl_}\DecValTok{1}\NormalTok{ =}\StringTok{ }\KeywordTok{ceiling}\NormalTok{(rl)}
\NormalTok{  rl_}\DecValTok{2}\NormalTok{ =}\StringTok{ }\KeywordTok{floor}\NormalTok{(rl)}
  \CommentTok{# perhitungan interval atas dan bawah}
\NormalTok{  PIup =}\StringTok{ }\NormalTok{x_sort[ru_}\DecValTok{1}\NormalTok{]}\OperatorTok{+}\NormalTok{((ru}\OperatorTok{-}\NormalTok{ru_}\DecValTok{1}\NormalTok{)}\OperatorTok{/}\NormalTok{(ru_}\DecValTok{2}\OperatorTok{-}\NormalTok{ru_}\DecValTok{1}\NormalTok{))}\OperatorTok{*}\NormalTok{(x_sort[ru_}\DecValTok{2}\NormalTok{]}\OperatorTok{-}\NormalTok{x_sort[ru_}\DecValTok{1}\NormalTok{])}
\NormalTok{  PIdown =}\StringTok{ }\NormalTok{x_sort[rl_}\DecValTok{1}\NormalTok{]}\OperatorTok{+}\NormalTok{((rl}\OperatorTok{-}\NormalTok{rl_}\DecValTok{1}\NormalTok{)}\OperatorTok{/}\NormalTok{(rl_}\DecValTok{2}\OperatorTok{-}\NormalTok{rl_}\DecValTok{1}\NormalTok{))}\OperatorTok{*}\NormalTok{(x_sort[rl_}\DecValTok{2}\NormalTok{]}\OperatorTok{-}\NormalTok{x_sort[rl_}\DecValTok{1}\NormalTok{])}
  \CommentTok{# decision making}
  \ControlFlowTok{if}\NormalTok{((side}\OperatorTok{==}\StringTok{"upper"}\NormalTok{) }\OperatorTok{&}\StringTok{ }\NormalTok{(PIup}\OperatorTok{<}\NormalTok{obs))\{}
    \KeywordTok{cat}\NormalTok{(}\StringTok{"PI ="}\NormalTok{,PIup,}\StringTok{",observasi baru="}\NormalTok{,obs)}
    \KeywordTok{cat}\NormalTok{(}\StringTok{"}\CharTok{\textbackslash{}n}\StringTok{----------------------------------------------------------------------"}\NormalTok{)}
    \KeywordTok{cat}\NormalTok{(}\StringTok{"}\CharTok{\textbackslash{}n}\StringTok{Kesimpulan:"}\NormalTok{)}
    \KeywordTok{cat}\NormalTok{(}\StringTok{"}\CharTok{\textbackslash{}n}\StringTok{nilai observasi lebih besar dibandingkan sebagian besar nilai yang ada"}\NormalTok{)}
\NormalTok{  \} }\ControlFlowTok{else} \ControlFlowTok{if}\NormalTok{((side}\OperatorTok{==}\StringTok{"lower"}\NormalTok{) }\OperatorTok{&}\StringTok{ }\NormalTok{(PIdown}\OperatorTok{>}\NormalTok{obs))\{}
    \KeywordTok{cat}\NormalTok{(}\StringTok{"PI ="}\NormalTok{,PIdown, }\StringTok{",observasi baru="}\NormalTok{,obs)}
    \KeywordTok{cat}\NormalTok{(}\StringTok{"}\CharTok{\textbackslash{}n}\StringTok{----------------------------------------------------------------------"}\NormalTok{)}
    \KeywordTok{cat}\NormalTok{(}\StringTok{"}\CharTok{\textbackslash{}n}\StringTok{Kesimpulan:"}\NormalTok{)}
    \KeywordTok{cat}\NormalTok{(}\StringTok{"}\CharTok{\textbackslash{}n}\StringTok{nilai observasi lebih kecil dibandingkan sebagian besar nilai yang ada"}\NormalTok{)}
\NormalTok{  \} }\ControlFlowTok{else} \ControlFlowTok{if}\NormalTok{(side}\OperatorTok{==}\StringTok{""}\NormalTok{)\{}
    \KeywordTok{print}\NormalTok{(}\StringTok{"side belum ditentukan tentukan apakah lower atau upper"}\NormalTok{)}
\NormalTok{  \} }\ControlFlowTok{else}\NormalTok{\{}
    \KeywordTok{cat}\NormalTok{(}\StringTok{"batas bawah ="}\NormalTok{,PIdown,}\StringTok{", batas atas ="}\NormalTok{,PIup)}
    \KeywordTok{cat}\NormalTok{(}\StringTok{"}\CharTok{\textbackslash{}n}\StringTok{---------------------------------------------------------"}\NormalTok{)}
    \KeywordTok{cat}\NormalTok{(}\StringTok{"}\CharTok{\textbackslash{}n}\StringTok{Kesimpulan:"}\NormalTok{)}
    \KeywordTok{cat}\NormalTok{(}\StringTok{"}\CharTok{\textbackslash{}n}\StringTok{nilai observasi sama dengan sebagian besar nilai yang ada"}\NormalTok{)}
\NormalTok{  \}}
\NormalTok{\}}
\end{Highlighting}
\end{Shaded}

\begin{Shaded}
\begin{Highlighting}[]
\KeywordTok{PInp_os}\NormalTok{(}\DataTypeTok{x=}\NormalTok{gwardat}\OperatorTok{$}\NormalTok{konsentrasi, }\DataTypeTok{obs=}\DecValTok{355}\NormalTok{, }\DataTypeTok{alpha=}\FloatTok{0.1}\NormalTok{, }\DataTypeTok{side=}\StringTok{"upper"}\NormalTok{)}
\end{Highlighting}
\end{Shaded}

\begin{verbatim}
## PI = 316 ,observasi baru= 355
## ----------------------------------------------------------------------
## Kesimpulan:
## nilai observasi lebih besar dibandingkan sebagian besar nilai yang ada
\end{verbatim}

\section{Interval Prediksi
Parametrik}\label{interval-prediksi-parametrik}

Interval prediksi parametrik juga digunakan untuk menentukan apakah
pengamatan baru kemungkinan berasal dari distribusi yang berbeda dari
data yang dikumpulkan sebelumnya. Namun, pada metode parametrik asumsi
bentuk dari distribusi data akan diperhitungkan. Asumsi ini memberikan
lebih banyak informasi untuk membangun interval, asalkan asumsi tersebut
valid. Jika data tidak mengikuti distribusi yang diasumsikan, interval
prediksi mungkin tidak akurat.

\subsection{Interval Prediksi Distribusi
Simetris}\label{interval-prediksi-distribusi-simetris}

Asumsi yang digunakan untuk melakukan perhitungan interval prediksi
untuk distribusi data yang simetris adalah data haruslah berdistribusi
normal. Interval prediksi selanjutnya dibentuk secara simetris pada
kedua sisi nilai mean. Interval ini lebih lebar rentangnya dibandingkan
dengan interval kepercayaan nilai mean. Persamaan matematis yang
digunakan untuk menghitungnya dituliskan pada Persamaan \eqref{eq:ipds}.

\begin{equation}
  PI=\overline{X}-t_{\left(\frac{\alpha}{2},n-1\right)}\cdot\sqrt{s^2+\left(\frac{s^2}{n}\right)}sampai\ \overline{X}+t_{\left(\frac{\alpha}{2},n-1\right)}\cdot\sqrt{s^2+\left(\frac{s^2}{n}\right)}
  \label{eq:ipds}
\end{equation}

Untuk interval satu sisi Persamaan \eqref{eq:ipds}, menjadi Persamaan
\eqref{eq:ipds2}.

\begin{equation}
  PI=\overline{X}-t_{\left(\alpha,n-1\right)}\cdot\sqrt{s^2+\left(\frac{s^2}{n}\right)}sampai\ \overline{X}+t_{\left(\alpha,n-1\right)}\cdot\sqrt{s^2+\left(\frac{s^2}{n}\right)}
  \label{eq:ipds2}
\end{equation}

Untuk lebih memahaminya misalkan terdapat hasil pengukuran baru
konsentrasi arsenik sebesar 350 ppb dengan menggunakan data pada Tabel
\ref{tab:gwardat} sebagai pembanding. Buktikan bahwa observasi baru
tersebut berasal dari distrubusi yang sama dengan \(\alpha\)=5\%.

Dengan menggunakan Persamaan \eqref{eq:ipds}, interval prediksi dapat
dihitung sebagai berikut:

\[
PI\ =\ 98,4-t_{\left(0.025,24\right)}\cdot\sqrt{144,7^2+\frac{144,7^2}{25}\ }sampai\ \ 98,4+t_{\left(0.025,24\right)}\cdot\sqrt{144,7^2+\frac{144,7^2}{25}}
\]

\[
PI\ =\ 98,4-2,064\cdot147,6\ \ sampai\ \ 98,4+2,064\cdot147,6
\]

\[
PI\ =\ -206,25\ \ sampai\ \ 403,05
\]

Berdasarkan hasil perhitungan yang dilakukan terlihat bahwa limit
interval prediksi yang dihasilkanterdapat nilai negatif. Kosentrasi
negatif mengindikasikan bahwa data yang digunakan tidaklah simetris
sehingga penggunaan interval prediksi untuk data yang simetris tidak
dapat digunakan pada data tersebut. Metode perhitungan interval prediksi
untuk data asimetris lebih cocok untuk digunakan.

Pada \texttt{R} interval prediksi disekitar nilai mean dapat dihitung
menggunakan fungsi berikut:

\begin{Shaded}
\begin{Highlighting}[]
\NormalTok{PI_sim <-}\StringTok{ }\ControlFlowTok{function}\NormalTok{(x, obs, alpha, side)\{}
  \CommentTok{# menghitung nilai mean}
\NormalTok{  ave =}\StringTok{ }\KeywordTok{mean}\NormalTok{(x)}
  \CommentTok{# menghitung nilai varians dara}
\NormalTok{  var =}\StringTok{ }\KeywordTok{var}\NormalTok{(x)}
  \CommentTok{# menghitung df}
\NormalTok{  n =}\StringTok{ }\KeywordTok{length}\NormalTok{(x)}
\NormalTok{  df =}\StringTok{ }\NormalTok{n}\OperatorTok{-}\DecValTok{1}
  \CommentTok{# perhitungan rentang satu sisi}
\NormalTok{  pi_l1 =}\StringTok{ }\NormalTok{ave}\OperatorTok{-}\KeywordTok{qt}\NormalTok{((}\DecValTok{1}\OperatorTok{-}\NormalTok{alpha), df)}\OperatorTok{*}\KeywordTok{sqrt}\NormalTok{(var}\OperatorTok{+}\NormalTok{(var}\OperatorTok{/}\NormalTok{n))}
\NormalTok{  pi_u1 =}\StringTok{ }\NormalTok{ave}\OperatorTok{+}\KeywordTok{qt}\NormalTok{((}\DecValTok{1}\OperatorTok{-}\NormalTok{alpha), df)}\OperatorTok{*}\KeywordTok{sqrt}\NormalTok{(var}\OperatorTok{+}\NormalTok{(var}\OperatorTok{/}\NormalTok{n))}
  \CommentTok{# perhitungan rentang dua sisi}
\NormalTok{  pi_l2 =}\StringTok{ }\NormalTok{ave}\OperatorTok{-}\KeywordTok{qt}\NormalTok{((}\DecValTok{1}\OperatorTok{-}\NormalTok{alpha}\OperatorTok{/}\DecValTok{2}\NormalTok{), df)}\OperatorTok{*}\KeywordTok{sqrt}\NormalTok{(var}\OperatorTok{+}\NormalTok{(var}\OperatorTok{/}\NormalTok{n))}
\NormalTok{  pi_u2 =}\StringTok{ }\NormalTok{ave}\OperatorTok{+}\KeywordTok{qt}\NormalTok{((}\DecValTok{1}\OperatorTok{-}\NormalTok{alpha}\OperatorTok{/}\DecValTok{2}\NormalTok{), df)}\OperatorTok{*}\KeywordTok{sqrt}\NormalTok{(var}\OperatorTok{+}\NormalTok{(var}\OperatorTok{/}\NormalTok{n))}
  \CommentTok{# decision making}
  \ControlFlowTok{if}\NormalTok{(side}\OperatorTok{==}\StringTok{"Upper"} \OperatorTok{&}\StringTok{ }\NormalTok{obs}\OperatorTok{>}\NormalTok{pi_u1)\{}
    \KeywordTok{cat}\NormalTok{(}\StringTok{"PI Atas ="}\NormalTok{,pi_u1,}\StringTok{",observasi baru="}\NormalTok{,obs)}
    \KeywordTok{cat}\NormalTok{(}\StringTok{"}\CharTok{\textbackslash{}n}\StringTok{----------------------------------------------------------------------"}\NormalTok{)}
    \KeywordTok{cat}\NormalTok{(}\StringTok{"}\CharTok{\textbackslash{}n}\StringTok{Kesimpulan:"}\NormalTok{)}
    \KeywordTok{cat}\NormalTok{(}\StringTok{"}\CharTok{\textbackslash{}n}\StringTok{nilai observasi lebih besar dibandingkan sebagian besar nilai yang ada"}\NormalTok{)}
\NormalTok{  \}}\ControlFlowTok{else} \ControlFlowTok{if}\NormalTok{(side}\OperatorTok{==}\StringTok{"Lower"} \OperatorTok{&}\StringTok{ }\NormalTok{obs}\OperatorTok{<}\NormalTok{pi_l1)\{}
    \KeywordTok{cat}\NormalTok{(}\StringTok{"PI Bawah ="}\NormalTok{,pi_l1,}\StringTok{",observasi baru="}\NormalTok{,obs)}
    \KeywordTok{cat}\NormalTok{(}\StringTok{"}\CharTok{\textbackslash{}n}\StringTok{----------------------------------------------------------------------"}\NormalTok{)}
    \KeywordTok{cat}\NormalTok{(}\StringTok{"}\CharTok{\textbackslash{}n}\StringTok{Kesimpulan:"}\NormalTok{)}
    \KeywordTok{cat}\NormalTok{(}\StringTok{"}\CharTok{\textbackslash{}n}\StringTok{nilai observasi lebih kecil dibandingkan sebagian besar nilai yang ada"}\NormalTok{)}
\NormalTok{  \}}\ControlFlowTok{else} \ControlFlowTok{if}\NormalTok{(side}\OperatorTok{==}\DecValTok{2} \OperatorTok{&}\StringTok{ }\NormalTok{obs}\OperatorTok{>}\NormalTok{pi_u2)\{}
    \KeywordTok{cat}\NormalTok{(}\StringTok{"PI Bawah ="}\NormalTok{,pi_l2,}\StringTok{",observasi baru="}\NormalTok{,obs, }\StringTok{",PI Atas ="}\NormalTok{,pi_u2)}
    \KeywordTok{cat}\NormalTok{(}\StringTok{"}\CharTok{\textbackslash{}n}\StringTok{----------------------------------------------------------------------"}\NormalTok{)}
    \KeywordTok{cat}\NormalTok{(}\StringTok{"}\CharTok{\textbackslash{}n}\StringTok{Kesimpulan:"}\NormalTok{)}
    \KeywordTok{cat}\NormalTok{(}\StringTok{"}\CharTok{\textbackslash{}n}\StringTok{nilai observasi lebih besar dibandingkan sebagian besar nilai yang ada"}\NormalTok{)}
\NormalTok{  \}}\ControlFlowTok{else} \ControlFlowTok{if}\NormalTok{(side}\OperatorTok{==}\DecValTok{2} \OperatorTok{&}\StringTok{ }\NormalTok{obs}\OperatorTok{<}\NormalTok{pi_l2)\{}
    \KeywordTok{cat}\NormalTok{(}\StringTok{"PI Bawah ="}\NormalTok{,pi_l2,}\StringTok{",observasi baru="}\NormalTok{,obs, }\StringTok{",PI Atas ="}\NormalTok{,pi_u2)}
    \KeywordTok{cat}\NormalTok{(}\StringTok{"}\CharTok{\textbackslash{}n}\StringTok{----------------------------------------------------------------------"}\NormalTok{)}
    \KeywordTok{cat}\NormalTok{(}\StringTok{"}\CharTok{\textbackslash{}n}\StringTok{Kesimpulan:"}\NormalTok{)}
    \KeywordTok{cat}\NormalTok{(}\StringTok{"}\CharTok{\textbackslash{}n}\StringTok{nilai observasi lebih kecil dibandingkan sebagian besar nilai yang ada"}\NormalTok{)}
\NormalTok{  \}}\ControlFlowTok{else} \ControlFlowTok{if}\NormalTok{(side}\OperatorTok{==}\StringTok{"Upper"} \OperatorTok{&}\StringTok{ }\NormalTok{obs}\OperatorTok{<}\NormalTok{pi_u1)\{}
    \KeywordTok{cat}\NormalTok{(}\StringTok{"PI Atas ="}\NormalTok{,pi_u1,}\StringTok{",observasi baru="}\NormalTok{,obs)}
    \KeywordTok{cat}\NormalTok{(}\StringTok{"}\CharTok{\textbackslash{}n}\StringTok{----------------------------------------------------------------------"}\NormalTok{)}
    \KeywordTok{cat}\NormalTok{(}\StringTok{"}\CharTok{\textbackslash{}n}\StringTok{Kesimpulan:"}\NormalTok{)}
    \KeywordTok{cat}\NormalTok{(}\StringTok{"}\CharTok{\textbackslash{}n}\StringTok{nilai observasi sama dengan sebagian besar nilai yang ada"}\NormalTok{)}
\NormalTok{  \}}\ControlFlowTok{else} \ControlFlowTok{if}\NormalTok{(side}\OperatorTok{==}\StringTok{"Lower"} \OperatorTok{&}\StringTok{ }\NormalTok{obs}\OperatorTok{>}\NormalTok{pi_l1)\{}
    \KeywordTok{cat}\NormalTok{(}\StringTok{"PI Bawah ="}\NormalTok{,pi_l1,}\StringTok{",observasi baru="}\NormalTok{,obs)}
    \KeywordTok{cat}\NormalTok{(}\StringTok{"}\CharTok{\textbackslash{}n}\StringTok{----------------------------------------------------------------------"}\NormalTok{)}
    \KeywordTok{cat}\NormalTok{(}\StringTok{"}\CharTok{\textbackslash{}n}\StringTok{Kesimpulan:"}\NormalTok{)}
    \KeywordTok{cat}\NormalTok{(}\StringTok{"}\CharTok{\textbackslash{}n}\StringTok{nilai observasi sama dengan sebagian besar nilai yang ada"}\NormalTok{)}
\NormalTok{  \}}\ControlFlowTok{else}\NormalTok{\{}
    \KeywordTok{cat}\NormalTok{(}\StringTok{"PI Bawah ="}\NormalTok{,pi_l2, }\StringTok{",observasi baru="}\NormalTok{,obs,}\StringTok{",PI Atas ="}\NormalTok{,pi_u2)}
    \KeywordTok{cat}\NormalTok{(}\StringTok{"}\CharTok{\textbackslash{}n}\StringTok{---------------------------------------------------------"}\NormalTok{)}
    \KeywordTok{cat}\NormalTok{(}\StringTok{"}\CharTok{\textbackslash{}n}\StringTok{Kesimpulan:"}\NormalTok{)}
    \KeywordTok{cat}\NormalTok{(}\StringTok{"}\CharTok{\textbackslash{}n}\StringTok{nilai observasi sama dengan sebagian besar nilai yang ada"}\NormalTok{)}
\NormalTok{  \}}
\NormalTok{\}}
\end{Highlighting}
\end{Shaded}

\begin{Shaded}
\begin{Highlighting}[]
\CommentTok{# interval prediksi satu sisi}
\KeywordTok{PI_sim}\NormalTok{(}\DataTypeTok{x =}\NormalTok{ gwardat}\OperatorTok{$}\NormalTok{konsentrasi, }\DataTypeTok{obs =} \DecValTok{350}\NormalTok{, }\DataTypeTok{alpha=}\FloatTok{0.05}\NormalTok{, }\DataTypeTok{side=}\DecValTok{2}\NormalTok{)}
\end{Highlighting}
\end{Shaded}

\begin{verbatim}
## PI Bawah = -206.2 ,observasi baru= 350 ,PI Atas = 402.9
## ---------------------------------------------------------
## Kesimpulan:
## nilai observasi sama dengan sebagian besar nilai yang ada
\end{verbatim}

\bibliography{book.bib,packages.bib}


\end{document}
